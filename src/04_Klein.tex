\chapter{Moderne Formulierung}
Wir diskutieren in diesem Kapitel nun eine Formulierung der KK-Theorie in 
einer zeitgemäßen Formulierung. Die Lie-Gruppe $G$ sei kompakt und wirke auf
$E$ durch Isometrien. Wir diskutieren zunächst den Fall einer fündimensionalen
Manigfaltigkeit $E$.
Damit muss die Lie-Gruppe $\dim G=1$ eindimensional sein. Alle kompakten
eindimensionalem Manigfaltigkeiten sind diffeomorph zu $\Sphere^1$. Ohne
Einschränkung sei deshalb im Folgenden $E$ ein $\Sphere^1$-Faserbündel.
Die Wirkung der Gruppe $G$ erzeugt einen Fluss 
\begin{equation}
\Phi_t(p):=t.p
\end{equation}
sowie ein assoziertes Vektorfeld 
\begin{equation}
V(p):=\left[\dod{}{t}\Phi_t(p)\right]_{t=1}\,.
\end{equation}
weiter nehmen wir an das $g(V,V)\neq 0$. 
Die zu $V$ duale Form bezeichnen wir mit $\alpha$. 
Per Konstruktion gilt
\begin{equation}
\mathcal{L}_{V}g=0
\end{equation}
%TODO mit def von L
\begin{equation}
\pi^*\bar{g}=g
\end{equation}
Die physikalische 4 dimensionale Raumzeit ist damit $(E/G,\bar{g})$ gegeben.
Der eindimensionale Fall ist besonders schön da insbesondere nur eine kompakte 
Cinfty MF existiert. Jede Die Orbits sind dann zwingend periodisch. Dies muss 
schon für $d=2$ nicht mehr der Fall sein wie das Beispiel $\mathbb{T}^2$ zeigt.
\section{Horizontale und vertikale Räume}
Den Kern der Bündelprojektion nennen wir vertikalen Raum $\Verp
E:=\ker\{\pi_{*p}\}$. Anschaulich enthält $\Verp$ also gerade die Felder die
entlang der Faser verlaufen.
\begin{bemerkung}[$\Verp E$ ist $G$ invariant]
Zunächst bemerkt man, dass
$\pi\circ\Phi_g=\pi$ und damit $\pi_*\circ{\Phi_g}_*=\pi_*$. Sei $X_p\in \Verp$, dann gilt für 
$f\circ \Phi_g = f$ 
\begin{equation}
\begin{split}
{\Phi_g}_*\left(X_p(f)\right)&=X_{\Phi_g(p)}(f\circ\Phi_g)\\
&=X_{\Phi_g(p)}(f)
\end{split}
\end{equation}
Also ${\Phi_g}_*V_p=V_{g.p}$
\end{bemerkung}
Ein Zusammenhang auf $E$ ist eine Zerlegung 
\begin{equation}
\Tanp E=\Verp E\,\oplus\, \Horp E
\end{equation}
die stetig von $p$ abhängt und ${\Phi_g}_*\Horp=\mathrm{H}_{g.p}$.
Ist eine $G$-invariante Metrik auf $\Tan E$ gegeben, so ergibt sich durch  
\begin{equation}
\Horp:=\Verp^\perp=\left\{X\in \Tanp E\,\big|\, g_p(X,V)=0\,,\,\forall\, V\in
\Verp\right\}\,
\end{equation}
eine natürliche Definition des horizontalen Bündels.
Insegsammt erhalten wir eine Zerlegung des Tangentialbündels in ein vertikales
und ein horizontales Bündel
\begin{equation}
\Tan E=\Ver E\,\oplus\, \Hor E\,.
\end{equation}

\begin{equation}
\Ver E\cong \Tan G\cong M\times \mathfrak{g}\,,\quad
\Hor E\cong \Tan  (E/G)
\end{equation}
Die Horizontalen Felder sind diejenigen die physikalische Größen beschreiben. 
\section{Eigenschaften der Klein-Metrik}
Wie man leicht nachrechnet gilt
\begin{equation}
\begin{split}
\widehat{G}=\begin{pmatrix}B& C\\
C\transpose& H\end{pmatrix}
&=
\begin{pmatrix}1& A\\0& 1\end{pmatrix}
\begin{pmatrix}G& 0\\0& H
\end{pmatrix}
\begin{pmatrix}1& 0\\A\transpose& 1\end{pmatrix}\,.
\end{split}
\end{equation}
mit Blöcken 
\begin{equation}
G=B-CH^{-1}C\transpose\,,\quad A=CH^{-1}\,.
\end{equation}
Damit gilt weiter
\begin{equation}
\det(\fived{G})=\det(G)\det(H)
\end{equation}
Die Transformation lässt sich leicht umkehren
\begin{equation}
\begin{pmatrix}1& A\\0& 1\end{pmatrix}^{-1}=\begin{pmatrix}1& -A\\0&
1\end{pmatrix}\,.
\end{equation}
% TODO: damit misst das feld quasi das volumen der Zusatzdimension
Mit diesen Relationen ergibt sich eine alternative Darstellung des
Linienelements:
\begin{equation}
\begin{split}
\dif \fived{s}^2&=\dif \vec{x}\transpose G
\dif \vec{x}\\
&=\dif \vec{x}\transpose
\begin{pmatrix}1& -A\\0& 1\end{pmatrix}
\begin{pmatrix}G& 0\\0& H
\end{pmatrix}
\begin{pmatrix}1& 0\\-A\transpose& 1\end{pmatrix}
\dif \vec{x}\\
&=\tensor{g}{_\mu_\nu}\dif \tensor{x}{^\mu}\dif\tensor{x}{^\nu}
+\tensor{h}{_a_b}\left(\dif\tensor{x}{^a}-\tensor*{A}{^a_\mu}\dif\tensor{x}{^\mu}\right)
\left(\dif\tensor{x}{^b}-\tensor*{A}{^b_\mu}\dif\tensor{x}{^\mu}\right)\,.
\end{split}
\end{equation}
Da diese Darstellung im Wesentlichen vom Typ ist wie sie von Klein verwendet
wurde soll wird sie im Folgenden als Klein-Darstellung bzw. eine Metrik vom Typ
als Klein-Metrik bezeichnen.
%Berechnungen angelehnt an \cite{Coquereaux:1990qs} \cite{williams2015field}.
Im fünfdimensionalen Fall können wir die Metrik schreiben als
\begin{equation}
\tensor{\fived{g}}{_\mu_\nu}=\tensor{g}{_\mu_\nu}+\psi\tensor*{A}{_\mu}\tensor*{A}{_\nu}\,,\quad
\tensor{\fived{g}}{_4_\mu}=\psi\tensor*{A}{_\mu}\,,\quad
\tensor{\fived{g}}{_4_4}=\psi
\end{equation}
In Koordinatenfreier Darstellung
\begin{equation}
\fived{g}=g+\psi\,\omega\otimes\omega\,.
\end{equation} 
mit der 
1-Form $\omega$
\begin{equation}
\omega=\dif\tensor{x}{^4}+\tensor{A}{_\mu}\dif\tensor{x}{^\mu}\,.
\end{equation}
Damit ist $M=\ker \omega$, mit anderen Worten $\omega$ ist eine
Zusammenhangsform auf $E$
\subsection{Inverse}
Wie sich leicht nachprüfen lässt ist die Inverse der Klein-Metrik gegeben als
\begin{equation}
\tensor{\fived{g}}{^\mu^\nu}=\tensor{g}{^\mu^\nu}\,,\quad
\tensor{\fived{g}}{^4^\mu}=-\tensor{A}{^\mu}\,,\quad
\tensor{\fived{g}}{^4^4}=\tensor*{A}{_\mu}\tensor*{A}{^\mu}+\frac{1}{\psi}\,.
\end{equation}
% TODO damit kann entweder mit ghut oder g indices nach oben gezogen werden.
% Inkonsistent? ghut4mu?

% \subsection{Determinante}
% Die Determinante wird wie üblich zur Berechnung von Volumenelementen benötigt.
% Bei der Berechnung ist folgende Formel für Matrizen $A,B,C,D$ hilfreich:
% \begin{equation}
% \begin{split}
% \det\begin{pmatrix}A& B\\ C& D\end{pmatrix}&=\det\left[
% \begin{pmatrix}1& B\\0& D\end{pmatrix}\begin{pmatrix}A-BD^{-1}C& 0\\DC^{-1
% }& 1\end{pmatrix} \right]\\
% &= \det(D) \det\left(A - B D^{-1}
% C\right)\,.
% \end{split}
% \end{equation}
% %TODO evtl. Beweis
% Damit ergibt sich 
% \begin{equation}
% \begin{split}
%  \fived{g}&=
%  \det\begin{pmatrix}\tensor{g}{_\mu_\nu}+\psi\tensor{A}{_\mu}\tensor{A}{_\nu}
%  &\psi \tensor{A}{_\mu}\\
%  \psi \tensor{A}{_\mu}	 & \psi
%  \end{pmatrix}\\
%  &=\psi
%  \det\left(\tensor{g}{_\mu_\nu}+\psi\tensor{A}{_\mu}\tensor{A}{_\nu}
%  -\psi\tensor{A}{_\mu}\psi^{-1}\psi\tensor{A}{_\nu}\right)\\
%  &=\psi \det\left(\tensor{g}{_\mu_\nu}\right)\\
%  &=\psi g\,.
% \end{split}
% \end{equation}
% Die Determinante der fünfdimensionalen Metrik ist also proportional zur 4
% dimensionalen. 
% \begin{bemerkung}
% Die entsprechenden Ausdrücke für die Kaluza-Metrik sind deutlich komplizierter.
% \end{bemerkung}
% 
% 
% \section{Die fünfdimensionale Raumzeit}
% Konstruktion der MFG: Siehe "`Coquereau Geometry of Multidimensional
% Universes"'\cite{coquereaux1983geometry}.
% Die Metrik ist von der Form
\section{Identifikation der Größen}
Wir wollen die Objekte $A$ und $\psi$ als Eichpotential bzw. Skalarfeld auf
$M/G$ auffassen, dazu müssen wir zeigen das sie entsprechend transformieren.
Da die Lie-Ableitung der Metrik nach $V=\partial_4$ verschwindet wissen wir,
dass die Metrik invariant unter infinitesimalen Diffeomorphismen $\xi$ sein muss.
Betrachte die Abbildung $f:M\to M$ 
\begin{equation}
\tensor{x}{^\mu}\mapsto
\tensor{x}{^\mu}\,,\quad\tensor{x}{^4}\mapsto\tensor{x}{^4}+
\xi\left(\tensor{x}{^\mu}\right)
\end{equation}
mit $\xi\in C^\infty\left(\Reals\right)$ beliebig. 
%TODO kommt von killingvector del4
Pullback $\fived{g}^\prime:=f^*\fived{g}$
%TODO Pullback in Komponenten
\begin{equation}
\begin{split}
\tensor*{\fived{g}}{*^\prime_\mu_\nu}&=\tensor{\fived{g}}{_a_b}\dpd{f^a}{\tensor{x}{^\mu}}\dpd{f^b}{\tensor{x}{^\nu}}\\
&=\tensor{\fived{g}}{_\alpha_\beta}\dpd{f^\alpha}{\tensor{x}{^\mu}}\dpd{f^\beta}{\tensor{x}{^\nu}}
+\tensor{\fived{g}}{_\alpha_4}\dpd{f^\alpha}{\tensor{x}{^\mu}}\dpd{f^4}{\tensor{x}{^\nu}}
+\tensor{\fived{g}}{_4_\beta}\dpd{f^4}{\tensor{x}{^\mu}}\dpd{f^\beta}{\tensor{x}{^\nu}}
+\tensor{\fived{g}}{_4_4}\dpd{f^4}{\tensor{x}{^\mu}}\dpd{f^4}{\tensor{x}{^\nu}}
\\
&=\tensor{\fived{g}}{_\mu_\nu}
+\psi\tensor{A}{_\mu}\pdif{_\nu}\xi
+\psi\tensor{A}{_\nu}\pdif{_\mu}\xi
+\psi\pdif{_\mu}\xi\pdif{_\nu}\xi\\
&=\tensor{g}{_\mu_\nu}
+\psi\left(\tensor{A}{_\mu}+\pdif{_\mu}\xi\right)
\left(\tensor{A}{_\nu}+\pdif{_\nu}\xi\right)
\end{split}
\end{equation}
\begin{equation}
\begin{split}
\tensor*{\fived{g}}{*^\prime_\mu_4}&=\tensor{\fived{g}}{_a_b}\dpd{f^a}{\tensor{x}{^\mu}}\dpd{f^b}{\tensor{x}{^4}}\\
&=\tensor{\fived{g}}{_\alpha_\beta}\dpd{f^\alpha}{\tensor{x}{^\mu}}\dpd{f^\beta}{\tensor{x}{^4}}
+\tensor{\fived{g}}{_\alpha_4}\dpd{f^\alpha}{\tensor{x}{^\mu}}\dpd{f^4}{\tensor{x}{^4}}
+\tensor{\fived{g}}{_4_\beta}\dpd{f^4}{\tensor{x}{^\mu}}\dpd{f^\beta}{\tensor{x}{^4}}
+\tensor{\fived{g}}{_4_4}\dpd{f^4}{\tensor{x}{^\mu}}\dpd{f^4}{\tensor{x}{^4}}\\
&=\psi\left(\tensor{A}{_\mu}+\pdif{_\mu}\xi\right)
\end{split}
\end{equation}
\begin{equation}
\begin{split}
\tensor*{\fived{g}}{*^\prime_4_4}&=\tensor{\fived{g}}{_a_b}\dpd{f^a}{\tensor{x}{^4}}\dpd{f^b}{\tensor{x}{^4}}\\
&=\tensor{\fived{g}}{_\alpha_\beta}\dpd{f^\alpha}{\tensor{x}{^4}}\dpd{f^\beta}{\tensor{x}{^4}}
+\tensor{\fived{g}}{_\alpha_4}\dpd{f^\alpha}{\tensor{x}{^4}}\dpd{f^4}{\tensor{x}{^4}}
+\tensor{\fived{g}}{_4_\beta}\dpd{f^4}{\tensor{x}{^4}}\dpd{f^\beta}{\tensor{x}{^4}}
+\tensor{\fived{g}}{_4_4}\dpd{f^4}{\tensor{x}{^4}}\dpd{f^4}{\tensor{x}{^4}}\\
&=\psi
\end{split}
\end{equation}
Das Transformationsverhalten der Größen lässt sich zusammenfassen als 
\begin{equation}
\tensor{g}{_\mu_\nu}\to\tensor{g}{_\mu_\nu}\,,\quad
\tensor{A}{_\mu}\to\tensor{A}{_\mu}+\pdif{_\mu}\xi\,,\quad
\psi\to\psi
\end{equation}
Also entsprechen infinitesimalen Diffeomorphismen auf $M$ Eichtransformationen.
Weiter ist zu Zeigen das $A$ als Differentialform $M/G$ interpretiert werden,
um dies einzusehen studieren wir das Verhalten unter Koordinatenwechseln $f:M\to
M$
\begin{equation}
\tensor{x}{^\mu}\mapsto
h(\tensor{x}{^\mu})\,,\quad\tensor{x}{^4}\mapsto\tensor{x}{^4}
\end{equation}
mit einem Diffeomorphismus $h$
\begin{equation}
\begin{split}
\tensor*{\fived{g}}{*^\prime_\mu_\nu}&=\tensor{\fived{g}}{_a_b}\dpd{f^a}{\tensor{x}{^\mu}}\dpd{f^b}{\tensor{x}{^\nu}}\\
&=\tensor{\fived{g}}{_\alpha_\beta}\dpd{f^\alpha}{\tensor{x}{^\mu}}\dpd{f^\beta}{\tensor{x}{^\nu}}
+\tensor{\fived{g}}{_\alpha_4}\dpd{f^\alpha}{\tensor{x}{^\mu}}\dpd{f^4}{\tensor{x}{^\nu}}
+\tensor{\fived{g}}{_4_\beta}\dpd{f^4}{\tensor{x}{^\mu}}\dpd{f^\beta}{\tensor{x}{^\nu}}
+\tensor{\fived{g}}{_4_4}\dpd{f^4}{\tensor{x}{^\mu}}\dpd{f^4}{\tensor{x}{^\nu}}\\
&=\tensor{\fived{g}}{_\alpha_\beta}\dpd{h^\alpha}{\tensor{x}{^\mu}}\dpd{h^\beta}{\tensor{x}{^\nu}}\\
&=\tensor{g}{_\alpha_\beta}\dpd{h^\alpha}{\tensor{x}{^\mu}}\dpd{h^\beta}{\tensor{x}{^\nu}}
+\psi\tensor{A}{_\alpha}\dpd{h^\alpha}{\tensor{x}{^\mu}}
\tensor{A}{_\beta}\dpd{h^\beta}{\tensor{x}{^\nu}}
\end{split}
\end{equation}
\begin{equation}
\begin{split}
\tensor*{\fived{g}}{*^\prime_\mu_4}&=\tensor{\fived{g}}{_a_b}\dpd{f^a}{\tensor{x}{^\mu}}\dpd{f^b}{\tensor{x}{^4}}\\
&=\tensor{\fived{g}}{_\alpha_\beta}\dpd{f^\alpha}{\tensor{x}{^\mu}}\dpd{f^\beta}{\tensor{x}{^4}}
+\tensor{\fived{g}}{_\alpha_4}\dpd{f^\alpha}{\tensor{x}{^\mu}}\dpd{f^4}{\tensor{x}{^4}}
+\tensor{\fived{g}}{_4_\beta}\dpd{f^4}{\tensor{x}{^\mu}}\dpd{f^\beta}{\tensor{x}{^4}}
+\tensor{\fived{g}}{_4_4}\dpd{f^4}{\tensor{x}{^\mu}}\dpd{f^4}{\tensor{x}{^4}}\\
&=\psi\tensor{A}{_\alpha}\dpd{h^\alpha}{\tensor{x}{^\mu}}
\end{split}
\end{equation}
\begin{equation}
\begin{split}
\tensor*{\fived{g}}{*^\prime_4_4}&=\tensor{\fived{g}}{_a_b}\dpd{f^a}{\tensor{x}{^4}}\dpd{f^b}{\tensor{x}{^4}}\\
&=\tensor{\fived{g}}{_\alpha_\beta}\dpd{f^\alpha}{\tensor{x}{^4}}\dpd{f^\beta}{\tensor{x}{^4}}
+\tensor{\fived{g}}{_\alpha_4}\dpd{f^\alpha}{\tensor{x}{^4}}\dpd{f^4}{\tensor{x}{^4}}
+\tensor{\fived{g}}{_4_\beta}\dpd{f^4}{\tensor{x}{^4}}\dpd{f^\beta}{\tensor{x}{^4}}
+\tensor{\fived{g}}{_4_4}\dpd{f^4}{\tensor{x}{^4}}\dpd{f^4}{\tensor{x}{^4}}\\
&=\psi
\end{split}
\end{equation}
\begin{equation}
\tensor{g}{_\mu_\nu}\to\tensor{g}{_\alpha_\beta}\dpd{h^\alpha}{\tensor{x}{^\mu}}\dpd{h^\beta}{\tensor{x}{^\nu}}\,,\quad
\tensor{A}{_\mu}\to\tensor{A}{_\alpha}\dpd{h^\alpha}{\tensor{x}{^\mu}}\,,\quad
\psi\to\psi
\end{equation}
Diffeomorphismus-Invarianz der Klein Metrik impliziert also sowohl die
Eichinvarianz des Viererpotentials $\tensor{A}{_\mu}$, als auch das korrekte
Transformationsverhalten unter vierdimensionalen Koordinatenwechseln.
Dabei transformieren $\psi$, $\tensor{A}{_\mu}$ und $\tensor{g}{_\mu_\nu}$ als
Skalar, Vektor, bzw. Tensor.
%TODO sind noch weitere Trafos erlaubt?
\section{Krümmung}
\begin{equation}
\fived{R}=R-\frac{1}{4}\psi\tensor{F}{_\mu_\nu}\tensor{F}{^\mu^\nu}
+\frac{1}{2\psi^2}(\pdif{_\mu}\psi)(\pdif{^\mu}\psi)
-\frac{1}{\psi}\square\psi
\end{equation}
Wir führen zweckmäßigerweise ein Feld
$\sigma=\ln\frac{\psi}{2}\psi$ ein, womit sich der Term nochmals
vereinfacht \footnote{Dies stellt keine Einschränkung dar, da $\psi$ positiv sein muss, damit $\fived{g}$ das gleiche Vorzeichen hat wie $g$. (Bei der Zusatzdimension handelt es sich um
eine Raumdimension)}:
\begin{equation}
\fived{R}=R-\frac{1}{4}e^{2\sigma}\tensor{F}{_\mu_\nu}\tensor{F}{^\mu^\nu}
-2e^{-\sigma}\square e^{\sigma}
\end{equation}
\section{Die Wirkung in fünf Dimensionen}
Kleins Idee war es das Wirkungsprinzip nach Hilbert auf fünf Dimensionen zu
erweitern. Folglich lautet das Wirkungsintegral
\begin{equation}
\fived{S}=\int_{E}\sqrt{-\fived{g}}\fived{R}\dif{^5}
x\,.
\end{equation}
Da die Metrik $G$-invariant ist der Integrand offensichtlich ebenfalls
$G$-Invariant und wir können ??? anwenden. Damit gilt:
\begin{equation}
\begin{split}
\fived{S}&=2\pi
r\int_{M}\sqrt{-g}\psi^{1/2}\fived{R}\dif{^4}x\,,
\end{split}
\end{equation}
wobei wir mit $r=\frac{1}{2\pi}\vol(G)$ den Radius des Orbits bezeichnen.
Die Lagrangedichte auf $M$ ist also gegeben durch
\begin{equation}
\begin{split}
\mathcal{L}&=\sqrt{-g}e^{\sigma}\left(R-\frac{1}{4}e^{2\sigma}\tensor{F}{_\mu_\nu}\tensor{F}{^\mu^\nu}
-2e^{-\sigma}\square e^{\sigma}\right)\\
&=\sqrt{-g}e^{\sigma}\left(R-\frac{1}{4}e^{2\sigma}\tensor{F}{_\mu_\nu}\tensor{F}{^\mu^\nu}\right)+2\sqrt{-g}\square
e^{\sigma}\,.\label{eq:Lagrange1}
\end{split}
\end{equation}
Der letzte Term liefert als totale Divergenz keinen Beitrag.
\subsection{Konforme Transformation}
Da wir den Lagrangian gerne in einer Form vorliegen hätten, die dem
EH-Lagrangian enspricht müssen wir den Vorfaktor $e^{\sigma}$ loswerden. 
Wir führen dazu eine konforme Transformation durch
\begin{equation}
\tensor*{\fived{g}}{*^\star*_i*_j}=e^{2\tau}\tensor{\fived{g}}{_i_j}\,.
\end{equation}
Dies impliziert sofort
\begin{equation}
\tensor*{g}{*^\star*_\mu*_\nu}=e^{2\tau}\tensor{g}{_\mu_\nu}\,,\quad\sigma^\star
=\sigma+\tau\,,\quad \sqrt{-g}=e^{-4\tau}\sqrt{-g^\star}\,.
\end{equation}
Die Komponenten des elektrischen Feldstärketensors ist in der konformen Metrik
gegeben als
\begin{equation}
\tensor*{F}{*^\star_\mu_\nu}=\tensor{F}{_\mu_\nu}\,,\quad\tensor*{F}{*^\star^\mu^\nu}
=\tensor*{\fived{g}}{*^\star*^\mu*^\alpha}\tensor*{\fived{g}}{*^\star*^\nu*^\beta}\tensor{F}{_\alpha_\beta}
=e^{-4\tau}\tensor{F}{^\mu^\nu}\,.
\end{equation}
Wendet man die Formel für das Transformationsverhalten des Krümmungsskalars an,
so findet man schließlich
\begin{equation}
R=e^{2\tau}\left[R^\star-6(\tensor*{\partial}{^\star_\mu}\tau)(\tensor{\partial}{^\star^\mu}\tau)
-6\square^\star\tau\right]\,,
\end{equation}
beziehungsweise
\begin{equation}
\begin{split}
\sqrt{-g}\phi
\fived{R}
&=e^{\sigma-2\tau}\sqrt{-g^\star}\left[\fived{R}^\star
-6(\tensor*{\partial}{^\star_\mu}\tau)(\tensor{\partial}{^\star^\mu}\tau)
-6\square^\star\tau\right]\,.
\end{split}
\end{equation}
% TODO Einfluss konformer Trafos
Setzt man $\tau = \frac{1}{2}\sigma$\footnote{Die daraus resultierende
Relation $\sigma^\star=\frac{3}{2}\sigma$ kann durch eine Skalierung von
$\sigma$ behandelt werden.}, so ergibt sich
\begin{equation}
\begin{split}
\sqrt{-g}\phi
\fived{R}&=\sqrt{-g^\star}\left[R^\star
-\frac{3}{2}(\tensor*{\partial}{^\star_\mu}\sigma)(\tensor{\partial}{^\star^\mu}\sigma)
-3\square^\star\sigma\right]\,.
\end{split}
\end{equation}
Setzt man in \eqref{eq:Lagrange1} ein erhält man
\begin{equation}
\begin{split}
\mathcal{L}&=\sqrt{-g^\star}\left[R^\star
+\frac{3}{2}(\tensor*{\partial}{^\star_\mu}\sigma)(\tensor{\partial}{^\star^\mu}\sigma)
-3\square^\star\sigma\right]
+\sqrt{-g}e^{\sigma}\left(-\frac{1}{4}e^{2\sigma}\tensor{F}{_\mu_\nu}\tensor{F}{^\mu^\nu}\right)\\
&=\sqrt{-g^\star}\left[R^\star
-\frac{3}{2}(\tensor*{\partial}{^\star_\mu}\sigma)(\tensor{\partial}{^\star^\mu}\sigma)
-\frac{1}{4}e^{3\sigma}\tensor*{F}{^\star_\mu_\nu}\tensor*{F}{^\star^\mu^\nu}\right]-3\sqrt{-g^\star}\square^\star\sigma\,.
\end{split}
\end{equation}
Der Term mit der totalen Divergenz produziert einen nicht beitragenden Randterm
und kann deshalb fallen gelassen werden. 
%TODO Randterme
Da die konform Transformierte Metrik die selben Eigenschaften wie die Metrik
selbst besitzt sind die Probleme $S[g]\to\text{min}$ und
$S\left[g^\star\right]\to\text{min}$ äquivalent, wir lassen deshalb wir im
Folgenden die Sterne an den Größen weg. 
Zudem führen wir Bezeichnungen die den Lagrangian in eine gebräuchlich Form
bringen:
\begin{equation}
\begin{split}
\mathcal{L}
&=\sqrt{-g}\left[R
-\frac{1}{2}(\tensor{\partial}{_\mu}\sigma)(\tensor{\partial}{^\mu}\sigma)
-\frac{1}{4}\tensor*{H}{_\mu_\nu}\tensor*{F}{^\mu^\nu}\right]\,.
\label{eq:Lagrange2}
\end{split}
\end{equation}
Dabei bezeichnet die
Größe $H$ den elektromagnetischen Verschiebungstensor\footnote{In Analogie zu
den Verschiebungsfeldern $\vec{D},\vec{H}$ der klassischen Elektrodynamik.
}
%TODO Insbesondere handelt es sich nicht um das Nebenfeld
% $\tensor{H}{_\mu_\nu}$.
\begin{equation}
\tensor*{H}{_\mu_\nu}:=e^{\sqrt{3}\sigma}\tensor*{F}{_\mu_\nu}\,,
\end{equation}
wobei der Term $e^{\sqrt{3}\sigma}$ als variable Perimitivität $\mu(\sigma)$
interpretiert wird. 
% Ist es konsistent nach den Komponenten einzeln zu variieren?
\subsection{Bewegungsgleichungen}
In bekannter Manier impliziert die Form der Lagrangedichte 
Die Einsteingleichungen:
\begin{equation}
\tensor{G}{_\mu_\nu}=\tensor*{T}{*^{\sigma}_\mu_\nu}+\tensor*{T}{*^{A}*_\mu*_\nu}
\end{equation}
Weiter erhält man für die Variation nach den $A$ 
Das Feld $A$ ist zyklisch, taucht also nicht selbst in der Lagrangedichte auf.
Die resultierende Bewegungsgleichungen lautet
\begin{equation}
0=\tensor{\nabla}{_\alpha}
\left[\dpd{\mathcal{L}}{\left(\tensor{\nabla}{_\alpha}\tensor{A}{_\beta}\right)}\right]
=\tensor{\nabla}{_\alpha}\left[-\frac{1}{4}e^{\sqrt{3}\sigma}\dpd{\left(\tensor{F}{_\mu_\nu}\tensor{F}{^\mu^\nu}\right)}{\left(\tensor{\nabla}{_\alpha}\tensor{A}{_\beta}\right)}\right]
=-\tensor{\nabla}{_\alpha}\tensor{H}{^\alpha^\beta}
 \end{equation}
 Für das Vektorpotential $A$, sowie
 % Ableitung nach DA im MW teil herleinten
 \begin{equation}
0=\tensor{\nabla}{_\alpha}\left[\dpd{\mathcal{L}}{\left(\tensor{\partial}{_\alpha}\sigma\right)}\right]
-\dpd{\mathcal{L}}{\sigma}\\
=\square \sigma
-\frac{\sqrt{3}}{4}\tensor*{H}{_\mu_\nu}\tensor*{F}{^\mu^\nu}
\label{eq:dymdilat}
 \end{equation}
 für das Skalarfeld $\sigma$. 
 % Lässt sich das durch redifinition zur Klein Gordon Gleichung hinbiegen?
\section{Erhaltungrößen}
Da nach Konstruktion $\partial_4$ ein Killing Vektorfeld, damit erhalten wir
eine Erhaltungsgröße
\begin{equation}
\begin{split}
K&=\delta^\mu_4\fived{U}_\mu\\
&=\fived{g}_{4\mu}\fived{U}^\mu\\
&=\psi\left(A_\nu\fived{U}^\nu+\fived{U}^4\right)\\
&=\psi\left(A_\nu U^\nu+U^4\right)\dod{s}{\tau}\\
\end{split}
\end{equation}
Die vierergeschwindigkeit identifizert man mit
\begin{equation}
U^a=\dod{x^a}{\tau}=\dod{s}{\tau}\hat{U}^a
\end{equation}
Es ist günstig die Größen
\begin{equation}
\Delta=A_\nu U^\nu+U^4\,,\quad
\fived{\Delta}=\Delta\dod{s}{\tau}\,,
\end{equation}
einzuführen.
\begin{equation}
\begin{split}
K
&=\psi\hat{\Delta}\\
\end{split}
\end{equation}
Weiter gilt dann 
\begin{equation}
\left(\dod{\tau}{s}\right)^2=1+\psi\fived{\Delta}^2=1+K^2\psi^{-1}
\end{equation}
%TODO Tachionen
 \section{Geodäten und die Lorentzkraft}
In lokalen Koordinaten lautet die Geodätengleichung
\begin{equation}
\dod[2]{x^a}{s}+\fived{\Gamma}^{a}_{bc}\dod{x^b}{s}\dod{x^c}{s}=0
\end{equation}
Zunächst bemerken wir, dass das Linienelement  
$\dif s^2=\dif \tau^2+\psi(A_\nu\dif x^\nu+\dif x^4)^2$ 
verschieden von der Eigenzeit $\dif \tau^2$ ist.
Parametrisieren wir die Kurve nach der Eigenzeit $\tau$, so erhalten
wir 
\begin{equation}
\dod[2]{x^a}{\tau}+\fived{\Gamma}^{a}_{bc}\dod{x^b}{\tau}\dod{x^c}{\tau}
=f(\tau)\dod{x^a}{\tau}\,.
\end{equation}
Für die ersten vier Komponenten erhalten wir
\begin{equation}
\begin{split}
0=\dod[2]{x^\mu}{\tau}
+\fived{\Gamma}^{\mu}_{\nu\lambda}\dod{x^\nu}{\tau}\dod{x^\lambda}{\tau}
+2\fived{\Gamma}^{\mu}_{\nu 4}\dod{x^\nu}{\tau}\dod{x^4}{\tau}
+\fived{\Gamma}^{\mu}_{44}\dod{x^4}{\tau}\dod{x^4}{\tau}
-f(\tau)\dod{x^a}{\tau}\\
\end{split}
\end{equation}
Umstellen liefert wiederum
\begin{equation}
\begin{split}
\dod[2]{x^\mu}{\tau}
+\Gamma^{\mu}_{\nu\lambda}\dod{x^\nu}{\tau}\dod{x^\lambda}{\tau}
=&\left(\fived{\Gamma}^{\mu}_{\nu\lambda}-\Gamma^{\mu}_{\nu\lambda}\right)
\dod{x^\nu}{\tau}\dod{x^\lambda}{\tau}
-2\fived{\Gamma}^{\mu}_{\nu
4}\dod{x^\nu}{\tau}\dod{x^4}{\tau} \\
&-\fived{\Gamma}^{\mu}_{44}\left(\dod{x^4}{\tau}\right)^2
+f(\tau)\dod{x^\mu}{\tau}\,.\label{eq:geod4d}
\end{split}
\end{equation}
Geodäten der fünfdimensionalen Raumzeit sind also im Algemeinen nicht 
geodäten in der vierdimensionalen Raumzeit. Die
auftretenden Christoffelsymbole der Klein Metrik lauten
\begin{align}
\fived{\Gamma}^{\mu}_{\nu\lambda}-\Gamma^{\mu}_{\nu\lambda}
&=\frac{1}{2}g^{\mu\alpha}\left(\psi
A_{(\nu}F_{\lambda)\alpha}+A_{\nu}A_{\lambda}\partial_{\alpha}\psi \right)\,,
\\
\fived{\Gamma}^{\mu}_{\nu 4}
&=\frac{1}{2}g^{\mu\alpha}\left(\psi F_{\nu\alpha}-A_\nu\partial_\alpha
\psi\right)\,,\\
\fived{\Gamma}^{\mu}_{4 4}
&=-\frac{1}{2}g^{\mu\alpha}\partial_{\alpha}\psi\,.
\end{align}
Eine Berechnung findet sich beispielsweise in \cite{williams2015field} und wird
zweckmäßigerweise mithilfe eines Computers durchgeführt. In der
vierdimensionalen Raumzeit wirkt die  rechte Seite von
\eqref{eq:geod4d} wie eine Kraft $F^\mu$ 
\begin{equation}
\begin{split}
F^\mu=&\frac{1}{2}g^{\mu\alpha}\Bigg[\left(\psi
A_{\nu}F_{\lambda\alpha}+A_{\nu}A_{\lambda}\partial_{\alpha}\psi
\right)U^\nu U^\lambda\\
&+2\left(A_\nu\partial_\alpha\psi-\psi F_{\nu\alpha}\right)U^\nu
U^4+\partial_{\alpha}\psi \left(U^4\right)^2\Bigg]+f(\tau)U^\mu\\
=&\frac{1}{2}g^{\mu\alpha}\Bigg[\psi
(A_\nu U^\nu-2U^4) F_{\lambda\alpha}U^\lambda +
\Delta ^2\partial_\alpha\psi\Bigg]+f(\tau)U^\mu\,.
\end{split}
\end{equation}
Im \enquote{klassischen} Grenzfall
\begin{equation}
A_\nu U^\nu\to 0\,,\quad
\partial_\mu\phi\to 0\,\quad
\end{equation}
verbleibt nur ein Beitrag zur Kraft
\begin{equation}
\begin{split}
F^\mu=-\psi g^{\mu\alpha}F_{\nu\alpha}U^\nu U^4\,.
\end{split}
\end{equation}
Dieser Ausdruck motiviert die Identifikation $U^4=\frac{q}{m}$.
\begin{equation}
\begin{split}
F^\mu=F^\mu_{\text{L}}+\frac{1}{2}g^{\mu\alpha}\Bigg[\psi A_\nu
U^\nu F_{\lambda\alpha}U^\lambda + \Delta
^2\partial_\alpha\psi\Bigg]+h(\psi)U^\mu
\end{split}
\end{equation}
\subsection{Das Kaluza-Klein Wunder}
Setzt man in \eqref{eq:Lagrange2}, $\sigma=\const=0$  so erhält man 
\begin{equation}\label{eq:Lagrange2}
\begin{split}
\mathcal{L}
&=\sqrt{-g}\left[R-\frac{1}{4}\tensor*{F}{_\mu_\nu}\tensor*{F}{^\mu^\nu}\right]\,,
\end{split}
\end{equation}
die Lagrangedichte der klassischen
Maxwell-Einstein Theorie. 
Das Ergebnis ist beachtlich: Im fünfdimensionalen Vakuum sind die Einstein- und
Maxwell-Gleichungen enthalten.
Mit anderen Worten die Theorien Einsteins
und Maxwells lassen sich in einer einzigen Theorie vereinigen, die von 
überraschender Einfachheit ist. Dieser Umstand ist auch als
\emph{Kaluza-Klein Wunder} bekannt.
% 
% \begin{equation}  \begin{split}
%   C&=
%   \tensor{\fived{g}}{_\mu_\nu}\tensor{\fived{U}}{^\mu}\tensor{\fived{U}}{^\nu}
%   +\tensor{\fived{g}}{_4_4}\tensor{\fived{U}}{^4}\tensor{\fived{U}}{^4}\\\
%  &=\tensor{\fived{g}}{_\mu_\nu}\tensor{\fived{U}}{^\mu}\tensor{\fived{U}}{^\nu}
%  +\psi Q^2\\
%   &=\tensor{g}{_\mu_\nu}\tensor{\fived{U}}{^\mu}\tensor{\fived{U}}{^\nu}
%   -\psi\left(\tensor{A}{_\mu}\tensor{\fived{U}}{^\mu}\right)^2
%  +\psi Q^2\\
%  \end{split}
%  \end{equation}
%  
%  
%   Die Geodätengleichung hat die Form
%  \begin{equation}
%  0=\tensor{\fived{U}}{^m}\tensor{\fived{\nabla}}{_m} \tensor{\fived{U}}{_n}
%  \end{equation}
%  Betrachten wir speziell die vierdimensionalen Komponenten $n=\nu$ so finden wir
% \begin{equation}
% \begin{split}
% 0&=\tensor{\fived{U}}{^m}\tensor{\fived{\nabla}}{_m} \tensor{\fived{U}}{_\nu}\\
% &=\tensor{\fived{U}}{^m}\tensor{\partial}{_m} \tensor{\fived{U}}{_\nu}
%  +\tensor*{\fived{\Gamma}}{_m_\ell_\nu}
%  \tensor{\fived{U}}{^m}\tensor{\fived{U}}{^\ell}\\
%  &=\dod{}{\lambda} \tensor{\fived{U}}{_\nu}
% +\tensor*{\fived{\Gamma}}{_\mu_\lambda_\nu}
%  \tensor{\fived{U}}{^\mu}\tensor{\fived{U}}{^\lambda}
%  +2\tensor*{\fived{\Gamma}}{_\mu_4_\nu}
%  \tensor{\fived{U}}{^4}\tensor{\fived{U}}{^\mu}
%  +\tensor*{\fived{\Gamma}}{_4_4_\nu}
%  \tensor{\fived{U}}{^4}\tensor{\fived{U}}{^4}\\
%   &=\dod{}{\lambda} \tensor{\fived{U}}{_\nu}
% +\tensor*{\fived{\Gamma}}{_\mu_\lambda_\nu}
%  \tensor{\fived{U}}{^\mu}\tensor{\fived{U}}{^\lambda}
%  +2Q\tensor*{\fived{\Gamma}}{_\mu_4_\nu}
%  \tensor{\fived{U}}{^\mu}
%  +Q^2\tensor*{\fived{\Gamma}}{_4_4_\nu}
%  \\
%    &=\dod{}{\lambda} \tensor{\fived{U}}{_\nu}
% +\tensor*{\Gamma}{_\mu_\lambda_\nu}
%  \tensor{\fived{U}}{^\mu}\tensor{\fived{U}}{^\lambda}
%  +\psi Q\tensor*{F}{_\mu_\nu}
%  \tensor{\fived{U}}{^\mu}
%  +Q^2\partial_\nu\psi\\
% \end{split}
% \end{equation}
% Insegammt findet man 
% \begin{equation}
% m\tensor{\ddot{x}}{^\mu}+m\tensor*{\Gamma}{^\mu_\nu_\lambda}
% \tensor{\dot{x}}{^\nu}\tensor{\dot{x}}{^\lambda}=
% \psi mQ\tensor*{F}{_\mu_\nu}
%  \tensor{\dot{x}}{^\mu}
%  +mQ^2\partial_\nu\psi
% \end{equation}
% Es liegt nahe $mQ=q$ zu setzen
% \begin{equation}
% m\tensor{\ddot{x}}{^\mu}+m\tensor*{\Gamma}{^\mu_\nu_\lambda}
% \tensor{\dot{x}}{^\nu}\tensor{\dot{x}}{^\lambda}=
% \psi q\tensor*{F}{_\mu_\nu}
%  \tensor{\fived{U}}{^\mu}
%  \tensor{\fived{U}}{^\mu}
%  +\frac{q^2}{m}\partial_\nu\psi
% \end{equation}
% % Dies kann wie folgt interpretiert werden: 
% % \begin{itemize}
% %   \item $-Q\tensor*{F}{_\mu_\nu}
% %  \tensor{U}{^\mu}$ beschreibt die gewöhnliche Lorentzkraft modifiziert mit d
% %  \item  $-Q^2\partial_\nu\psi$ beschreibt eine Kraft die durch ein Potential
% %  $Q^2\psi$ ausgeübt wird.
% % \end{itemize}
% Setzt man $\psi=1$ so erhält man die bekannten Gleichungen der Maxwell Theorie.
% \begin{bemerkung}
% Will man die Normierung $\tensor{U}{_\mu}\tensor{U}{^\mu}={0,-1}$ aufrecht
% erhalten so muss
% $\tensor{\fived{U}}{_m}\tensor{\fived{U}}{^m}=\tensor{U}{_\mu}\tensor{U}{^\mu}+Q^2={Q^2,Q^2-1}$
% gelten. Die Vektoren in fünf Dimensionen sind also je nach Ladung raumartig!
% Dies führt zu fragen bezüglich Kausalität
% \end{bemerkung}
\section{Die Rolle des skalaren Felds}
% Radius des Internen space
  Klein maß dem zusätzlichen Freiheitsgrad der durch das Skalarfeld
  $\sigma$\footnote{Bzw. $\phi$ oder $\psi$.} gegeben ist, keine physikalische
  Bedeutung bei.
  Dementsprechend setzte er $\psi=\const=1$ .
 Dadurch vereinfacht sich die Lagrangedichte, zu der der Einstein-Maxwell
 Theorie.
 Allerdings impliziert die dynamische Gleichung
 \eqref{eq:dymdilat}
  \begin{equation}
\tensor*{F}{_\mu_\nu}\tensor*{F}{^\mu^\nu}=0\,,
 \end{equation}
 der Beitrag der elektrische Beitrag zu \eqref{eq:Lagrange2} verschwindet also.
 Dies führt die Konstruktion ad absurdum. Einziger Ausweg ist die $\tensor{g}{_4_4}$-Komponente
\emph{a priori} konstant zu setzen und nicht zu variieren. Dies trübt
die Allgemeinheit der Theorie, da dadurch diese Komponente gegenüber den anderen
ausgezeichnet ist.
Einziger Ausweg ist das Skalarfeld als physikalische Größe Zu betrachten, diese 
Interpretation wurde erstmals durch Yordan und Thiery vorgenommen.
Zusätzliche skalare Felder tauchen häufig in Theorien auf die kompakte
Zusatzdimensionen enthalten.
Sie werden häufig mit einem Teilchen, dem \emph{Dilaton} identifiziert. Zusätzliche Teilchen stellen an sich kein
Probleme dar, einige ungelöste Fragestellungen der Physik (dunkle
Materie/Energie) benötigen sie sogar um beantwortet zu werden. 
\section{Quantisierung der Ladung}
Die Kompaktheit von $\Sphere^1$ impliziert die Quantisierung der Ladung (Klein)
% \fived{R}^\star
% =e^{-\tau}\left(\fived{R}+3\tensor{\nabla}{_\mu}\left(\tensor{\fived{g}}{^\mu^\nu}\tensor{\partial}{_\nu}\tau\right)
% -\frac{3}{2}\tensor{\fived{g}}{^\mu^\nu}\tensor{\partial}{_\mu}\tau\tensor{\partial}{_\nu}\tau\right)\,,\quad\sqrt{-\fived{g}^\star}=e^{5\varphi}\sqrt{-\fived{g}}
% \end{equation}
% \begin{equation}
% \begin{split}
% \mathcal{L}^\star
% &=\fived{R}^\star\sqrt{-\fived{g}^\star}\\
% &=e^{3\varphi}\sqrt{-\fived{g}}\left(\fived{R}+\frac{16}{3}e^{-\frac{3}{2}\varphi}\square
% e^{\frac{3}{2}\varphi}\right)\\
% &=e^{3\varphi}\sqrt{-g}\phi\left(R-\frac{1}{4}\phi^3\tensor{F}{_\mu_\nu}\tensor{F}{^\mu^\nu}+\frac{16}{3}e^{-\frac{3}{2}\varphi}\square
% e^{\frac{3}{2}\varphi}\right)\\
% &=e^{3\varphi}\sqrt{-g}\phi\left(R-\frac{1}{4}\phi^3\tensor{F}{_\mu_\nu}\tensor{F}{^\mu^\nu}+12\pdif{_\mu}\varphi\pdif{^\mu}\varphi+8\square
% \varphi\right)\\
% &=e^{3\varphi}\sqrt{-g}\phi\left(R
% -\frac{1}{4}\phi^3\tensor{F}{_\mu_\nu}\tensor{F}{^\mu^\nu}
% +12\pdif{_\mu}\varphi\pdif{^\mu}\varphi
% +8e^{2\varphi}\square^\star\varphi+24e^{2\varphi}\pdif{_\mu}\varphi\pdif{^\mu}\varphi\right)
% \end{split}
% \end{equation}
% Es liegt nahe $\phi=e^{-3\varphi}$ zu setzen.
% \begin{equation}
% \begin{split}
% \fived{R}^\star\sqrt{-\fived{g}^\star}
% &=\sqrt{-g}\left(R+12\pdif{_\mu}\varphi\pdif{^\mu}\varphi+8\square
% \varphi\right)
% \end{split}
% \end{equation}
% $\sigma=\sqrt{24}\varphi$
% \begin{equation}
% \begin{split}
% \fived{R}^\star\sqrt{-\fived{g}^\star}
% &=\sqrt{-g}\left(R+\frac{1}{2}\pdif{_\mu}\sigma\pdif{^\mu}\sigma\right)
% \end{split}
% \end{equation}
% Coqueraux Jordan Thiery
% % Im Folgenden lassen wir den Term $2\pi
% % r$ weg, da er keinen Einfluss auf das Variationsprinzip hat. Die Lagrangedichte
% % ist damit
% % \begin{equation}
% % \begin{split}
% % \mathcal{L}=\sqrt{-g}\left(\phi
% % R-\frac{1}{4}\phi^3\tensor{F}{_\mu_\nu}\tensor{F}{^\mu^\nu}\right)
% % &=\sqrt{-g}\left(\phi
% % \mathcal{L}\textsubscript{g}+\phi^3\mathcal{L}\textsubscript{EM}\right)
% % \end{split}
% % \end{equation}
% % Die Tatsache dass sich die Lagrangedichte in einen Term der die Raumkrümmung
% % enthält und einen Elektromagnetischen Anteil aufspaltet ist auch als
% % Kaluza-Klein Wunder\footnote{"`Kaluza-Klein miracle"'} bekannt. Insbesondere
% % erhält man für $\phi=1$ die Lagrangedichte für ein System das sowohl einstein
% % als auch Maxwellgleichungen erfüllt. Da der Lagrangian keine Kinetischen Terme
% % in $\varphi$ enthält folgt
% % \begin{equation}
% % 0=\dpd{\mathcal{L}}{\phi}=\sqrt{-g}\left(R-\frac{3}{4}\phi^2\tensor{F}{_\mu_\nu}\tensor{F}{^\mu^\nu}\right)
% % \end{equation}
% % Weiter erhält man für die Variation nach den $A$ 
% % Das Feld $A$ ist Zyklisch, taucht als nicht selbst in der Lagrangedichte auf.
% % Die resultierende Erhaltungsgleichung lautet
% % \begin{equation}
% % 0=\tensor{\nabla}{_\alpha}\left(\dpd{\mathcal{L}}{\left(\tensor{\nabla}{_\alpha}\tensor{A}{_\beta}\right)}\right)
% =\tensor{\nabla}{_\alpha}\left(\phi^3\dpd{\tensor{F}{_\mu_\nu}\tensor{F}{^\mu^\nu}}{\left(\tensor{\nabla}{_\alpha}\tensor{A}{_\beta}\right)}\right)
% =4\tensor{\nabla}{_\alpha}\left(\phi^3\tensor{F}{^\alpha^\beta}\right)
% \end{equation}
% \begin{equation}
% \tensor{G}{_\mu_\nu}=\phi^2\tensor*{T}{*^{\text{M}}*_\mu*_\nu}
% \end{equation}
% Oder umformuliert
% \begin{equation}
% \tensor{\nabla}{_\alpha}\tensor{F}{^\alpha^\beta}
% =-\frac{3}{\phi}\tensor{F}{^\alpha^\beta}\tensor{\partial}{_\alpha}\phi
% \end{equation}
% Bzw:
% \begin{equation}
% \tensor{J}{^\beta}
% =-\frac{3}{\phi\sqrt{-g}}\tensor{F}{^\alpha^\beta}\tensor{\partial}{_\alpha}\phi
% =-3\left(-\fived{g}\right)^{-\nicefrac{1}{2}}\tensor{F}{^\alpha^\beta}\tensor{\partial}{_\alpha}\phi
% \end{equation}
% \begin{equation}
% \begin{split}
% \fived{R}&=e^{-2\varphi}\left(R+\frac{16}{3}e^{-\frac{3}{2}\varphi}\square
% e^{\frac{3}{2}\varphi}\right)\\
% &=e^{-2\varphi}\left(R+8\square\varphi+12\pdif{_\mu}\varphi\pdif{^\mu}\varphi\right)
% \end{split}
% \end{equation}
% \begin{equation}
% \begin{split}
% \phi\sqrt{-g}R
% &=\phi
% e^{-\varphi}\sqrt{\fived{g}}\left(\fived{R}+8\square\varphi+12\pdif{_\mu}\varphi\pdif{^\mu}\varphi\right)\\
% \end{split}
% \end{equation}
% $\varphi=\ln \phi$
% \begin{equation}
% \begin{split}
% \phi\sqrt{-g}R
% &=\sqrt{\fived{g}}\left(\fived{R}
% +8{\square}\ln\phi 
% +\frac{12}{\phi^2}\pdif{_\mu}\phi\pdif{^\mu}\phi\right)
% \end{split}
% \end{equation}
% \begin{equation}
% \begin{split}
% \phi\sqrt{-g}R
% &=\sqrt{\fived{g}}\left(\fived{R}
% +8\frac{1}{\phi}\square\phi 
% +\frac{4}{\phi^2}\pdif{_\mu}\phi\pdif{^\mu}\phi\right)
% \end{split}
% \end{equation}
% \begin{equation}
% \begin{split}
% \phi\sqrt{-g}R
% &=\sqrt{\fived{g}}\left(\fived{R}
% +8\frac{1}{\phi}\square\phi 
% +16\pdif{_\mu}\Lambda\pdif{^\mu}\Lambda\right)
% \end{split}
% \end{equation}
% \begin{equation}
% \begin{split}
% 0&=\frac{\delta\mathcal{L}}{\delta\tensor{g}{_\mu_\nu}}\\
% &=\phi\frac{\delta\mathcal{L}\textsubscript{EH}}{\delta\tensor{g}{_\mu_\nu}}
% +\phi^3\frac{\delta\mathcal{L}\textsubscript{EM}}{\delta\tensor{g}{_\mu_\nu}}\\
% &=
% \phi\sqrt{-g}\left[\frac{1}{2}\tensor{g}{^\mu^\nu}
% R+\tensor{R}{^\mu^\nu}\right]
% \end{split}
% \end{equation}
% \begin{split}
% \tensor*{\fived{g}}{*^\prime_\mu_\nu}&=\tensor{\fived{g}}{_a_b}\dpd{f^a}{\tensor{x}{^\mu}}\dpd{f^b}{\tensor{x}{^\nu}}\\
% &=\tensor{\fived{g}}{_\alpha_\beta}\dpd{f^\alpha}{\tensor{x}{^\mu}}\dpd{f^\beta}{\tensor{x}{^\nu}}
% +\tensor{\fived{g}}{_\alpha_4}\dpd{f^\alpha}{\tensor{x}{^\mu}}\dpd{f^4}{\tensor{x}{^\nu}}
% +\tensor{\fived{g}}{_4_\beta}\dpd{f^4}{\tensor{x}{^\mu}}\dpd{f^\beta}{\tensor{x}{^\nu}}
% +\tensor{\fived{g}}{_4_4}\dpd{f^4}{\tensor{x}{^\mu}}\dpd{f^4}{\tensor{x}{^\nu}}
% \\
% &=\tensor{\fived{g}}{_\alpha_\beta}\dpd{f^\alpha}{\tensor{x}{^\mu}}\dpd{f^\beta}{\tensor{x}{^\nu}}
% +\psi\tensor{A}{_\alpha}\dpd{g^\alpha}{\tensor{x}{^\mu}}\pdif{_\nu}h
% +\psi\tensor{A}{_\beta}\dpd{g^\beta}{\tensor{x}{^\nu}}\pdif{_\mu}h
% +\psi\pdif{_\mu}h\pdif{_\nu}h\\
% &=\tensor{g}{_\alpha_\beta}\dpd{g^\alpha}{\tensor{x}{^\mu}}\dpd{g^\beta}{\tensor{x}{^\nu}}
% +\psi\left(\tensor{A}{_\alpha}\dpd{g^\alpha}{\tensor{x}{^\mu}}+\pdif{_\mu}h\right)
% \left(\tensor{A}{_\alpha}\dpd{g^\alpha}{\tensor{x}{^\nu}}+\pdif{_\nu}h\right)
% \end{split}
% \end{equation}
% \begin{equation}
% \begin{split}
% \tensor*{\fived{g}}{*^\prime_\mu_4}&=\tensor{\fived{g}}{_a_b}\dpd{f^a}{\tensor{x}{^\mu}}\dpd{f^b}{\tensor{x}{^4}}\\
% &=\tensor{\fived{g}}{_\alpha_\beta}\dpd{f^\alpha}{\tensor{x}{^\mu}}\dpd{f^\beta}{\tensor{x}{^4}}
% +\tensor{\fived{g}}{_\alpha_4}\dpd{f^\alpha}{\tensor{x}{^\mu}}\dpd{f^4}{\tensor{x}{^4}}
% +\tensor{\fived{g}}{_4_\beta}\dpd{f^4}{\tensor{x}{^\mu}}\dpd{f^\beta}{\tensor{x}{^4}}
% +\tensor{\fived{g}}{_4_4}\dpd{f^4}{\tensor{x}{^\mu}}\dpd{f^4}{\tensor{x}{^4}}\\
% &=\psi\tensor{A}{_\alpha}\dpd{g^\alpha}{\tensor{x}{^\mu}}\pdif{_4}h+\psi\pdif{_\mu}h\pdif{_4}h\\
% &=\psi\pdif{_4}h\left(\tensor{A}{_\alpha}\dpd{g^\alpha}{\tensor{x}{^\mu}}+\pdif{_\mu}h\right)
% \end{split}
% \end{equation}
% \begin{equation}
% \begin{split}
% \tensor*{\fived{g}}{*^\prime_4_4}&=\tensor{\fived{g}}{_a_b}\dpd{f^a}{\tensor{x}{^4}}\dpd{f^b}{\tensor{x}{^4}}\\
% &=\tensor{\fived{g}}{_\alpha_\beta}\dpd{f^\alpha}{\tensor{x}{^4}}\dpd{f^\beta}{\tensor{x}{^4}}
% +\tensor{\fived{g}}{_\alpha_4}\dpd{f^\alpha}{\tensor{x}{^4}}\dpd{f^4}{\tensor{x}{^4}}
% +\tensor{\fived{g}}{_4_\beta}\dpd{f^4}{\tensor{x}{^4}}\dpd{f^\beta}{\tensor{x}{^4}}
% +\tensor{\fived{g}}{_4_4}\dpd{f^4}{\tensor{x}{^4}}\dpd{f^4}{\tensor{x}{^4}}\\
% &=\psi\left(\pdif{_4}h\right)^2
% \end{split}
% \end{equation}
% Um konsistent zu bleiben muss $\pdif{_4}h = 1$ also sind die erlaubten
% transformationen von der Form
% \begin{equation}
% \tensor{g}{_\mu_\nu}\to\tensor{g}{_\alpha_\beta}\dpd{g^\alpha}{\tensor{x}{^\mu}}\dpd{g^\beta}{\tensor{x}{^\nu}}\,,\quad
% \tensor{A}{_\alpha}\to\tensor{A}{_\alpha}\dpd{g^\alpha}{\tensor{x}{^\mu}}+\pdif{_\mu}h\\
% \psi\to\psi
% \end{equation}
%  
%  In Verallgemeinerung der vierdimensionalen Geodätischen
%  \begin{equation}
%  0=\tensor{\fived{U}}{^m}\tensor{\nabla}{_m}
%  \tensor{\fived{U}}{_n}=\dod{}{\lambda}
%  \tensor{\fived{U}}{_n}+\tensor{\fived{U}}{^m}\tensor*{\fived{\Gamma}}{^\ell_m_n}\tensor{\fived{U}}{_\ell}
%  \end{equation}
%  % Erhaltungsgröße zu killingtensor d4
%  \begin{equation}
%  \begin{split}
%   0&=\tensor{\fived{U}}{^m}\tensor{\nabla}{_m} \tensor{\fived{U}}{_5}\\
%  &=\dod{}{\lambda}\tensor{\fived{U}}{_5}
%  +\tensor{\fived{U}}{^m}\tensor*{\fived{\Gamma}}{*^n_5_m} \tensor{\fived{U}}{_n}\\
%  &=\dod{}{\lambda}\tensor{\fived{U}}{_5}
%  +\frac{1}{2}\tensor{\fived{U}}{^m}\tensor{\fived{g}}{^n^a}\left(\tensor{\fived{g}}{_a_m_{,5}}
%  +\tensor{\fived{g}}{_a_5_{,m}}
%  -\tensor{\fived{g}}{_m_5_{,a}}
%  \right)\tensor{\fived{U}}{_n}\\
%   &=\dod{}{\lambda}\tensor{\fived{U}}{_5}
%  +\frac{1}{2}\tensor{\fived{U}}{^m}\tensor{\fived{U}}{^a}\left(
%  \tensor{\fived{g}}{_a_5_{,m}}
%  -\tensor{\fived{g}}{_m_5_{,a}}
%  \right)\\
%   &=\dod{}{\lambda}\tensor{\fived{U}}{_5}\\
%  \end{split}
%  \end{equation}
%  Der zweite Term verschwindet als Spur eines Produkt eines symmetrischen mit
%  einem antisymmetrischen Tensors.