\chapter{Moderne Formulierung}
Auch wenn das Ergebnis sehr überzeugend ist, so ist das vorgehen nicht besonders
mathematisch. Annahmen:
\begin{enumerate}
\item Zylinderbedingung
\item keine Quellen, $\tensor{T}{_m_n}=0$.
\item die Metrik ist Extremum des EH-Funktionals
\item Testteilchen bewegen sich auf Geodäten
\item die Eigenzeit ist durch die Bogenlänge gegeben
\end{enumerate}
\section{Die fünfdimensionale Raumzeit}
Konstruktion der MFG: Siehe "`Coquereau Geometry of Multidimensional
Universes"'\cite{coquereaux1983geometry}.
Die Metrik ist von der Form
\section{Die fünfdimensionale Raumzeit}
Wir werden im folgenden nur den fünfdimensionalen Fall diskutieren. In den
meisten Fällen ergibt sich eine Verallgemeinerung natürlich. Sei $(E,\hat{g})$
eine fünfdimensionale pseudoriemannsche Mannigfaltigkeit. Die Liegruppe
kompakte Liegruppe $G$ operiere durch Isometrien auf $E$. 
Dann ergibt sich wie beschrieben eine Faserstruktur auf $E$ mit typischer Faser
$G.x x\in E$. Seien $T^{(a)}\in \mathfrak{g}$ die Generatoren von $G$
\begin{equation}
V^{(a)}(p):=\od{}{t}\bigg|_{t=0}\exp\left(tT^{(a)}\right).p\,,\quad p\in E
\end{equation}
Da $U(1)$ eindimensional ist existiert in unserem Fall nur ein solches
Vektorfeld. Wähle ein Koordinatensystem sodass 
\begin{equation}
V^{(1)}=V^4\partial_4\,,\quad \partial_4\tensor{\hat{g}}{_\mu_\nu}=0\,.
\end{equation}
% geht immer?
Die Metrik ist von der Form \footnote{$\tensor{\hat{g}}{_4_4}>0$ sonst
zusätzliche Zeitartige Vektoren}
\begin{equation}
\hat{g}=\tensor{\hat{g}}{_\mu_\nu}\dif\tensor{x}{^\mu}\dif\tensor{x}{^\nu}
+\tensor{\hat{g}}{_\mu_4}\dif\tensor{x}{^\mu}\dif\tensor{x}{^4}
+\tensor{\hat{g}}{_4_\mu}\dif\tensor{x}{^4}\dif\tensor{x}{^\mu}
+\tensor{\hat{g}}{_4_4}\dif\tensor{x}{^4}\dif\tensor{x}{^4}\,.
\end{equation}
Sortiert man die Terme in der Metrik etwas um, so erhält man die Klein-Metrik
\begin{equation}
\hat{g}=\left(\tensor{\hat{g}}{_\mu_\nu}-\frac{\tensor{\hat{g}}{_\mu_4}\tensor{\hat{g}}{_\nu_4}}{\tensor{\hat{g}}{_4_4}}\right)\dif\tensor{x}{^\mu}\dif\tensor{x}{^\nu}
+\tensor{\hat{g}}{_4_4}\left(\dif\tensor{x}{^4}+\frac{\tensor{\hat{g}}{_\mu_4}}{\tensor{\hat{g}}{_4_4}}\dif\tensor{x}{^\mu}\right)^2
\,.
\end{equation}
Man definiert 
\begin{equation}
\tensor{g}{_\mu_\nu}:=\tensor{\hat{g}}{_\mu_\nu}-\frac{\tensor{\hat{g}}{_\mu_4}\tensor{\hat{g}}{_\nu_4}}{\tensor{\hat{g}}{_4_4}}
\,,\quad
\tensor{A}{_\mu}:=\frac{\tensor{\hat{g}}{_\mu_4}}{\tensor{\hat{g}}{_4_4}}
\,,\quad
\psi:=\tensor{\hat{g}}{_4_4}\,,
\end{equation}
sowie eine 1-Form $\omega$ durch
\begin{equation}
\omega=\dif\tensor{x}{^4}+\tensor{A}{_\mu}\dif\tensor{x}{^\mu}\,.
\end{equation}
Damit lässt sich die Metrik in zwei Anteile Unterteilen:
\begin{equation}
\hat{g}=g+\psi\,\omega\otimes\omega\,.
\end{equation}
Es gilt:
\begin{equation}
\begin{split}
0&=\hat{g}(\partial_4,\partial_\mu)\\
&=\left(\tensor{\hat{g}}{_4_\nu}\dif\tensor{x}{^\nu}
+\tensor{\hat{g}}{_4_4}\dif\tensor{x}{^4}\right)(\partial_\mu)\\
&=\left(\tensor{\hat{g}}{_4_\mu}\dif\tensor{x}{^\mu}
+\tensor{\hat{g}}{_4_4}\dif\tensor{x}{^4}\right)(\partial_\mu)\\
&=: \tensor{\hat{g}}{_4_4}\omega(\partial_\mu)
\end{split}
\end{equation}
Die induzierte Metrik auf dem Bündel 
\begin{equation}
B=\bigsqcup_{p\in E/G}\mathrm{Span}\left\{\partial_\mu|_p\right\}
\end{equation}
ist also gegeben durch die Metrik $g$. $B$ soll im Folgenden als Modell für die
(bekannte) vierdimensionale Raumzeit stehen.
Es handelt sich zunächst lediglich
um eine Umbenennung, die Bedeutung dieser Größen wird im Folgenden klar werden. Die von Klein verwendet Metrik ist also verschieden von der Kaluza
Metrik, beide sind aber vom Informationsgehalt äquivalent\footnote{Verschiedene
Autoren verwenden zu $\hat{g}$ konform äquivalente Metriken, 
bzw. Metriken die noch einen Zusätzlichen Freiheitsgrad $\kappa$ enthalten.
Dieser lässt sich aber in die Größen $\tensor{A}{_\mu}$ bzw. $\phi$ absorbieren.}
Die bekannten Vierervektoren $X^\mu$ entsprechen Fünfervektoren mit
\begin{equation}
\tensor{X}{^5}=-\tensor{A}{_\mu}\tensor{X}{^\mu}
\end{equation}
%(=Projektion?)
Da $\tensor{A}{_\mu}$ nur von $\tensor{X}{^\mu}$ abhängt, ist $\tensor{X}{^5}$
durch die ersten vier Komponenten eindeutig gegeben und enthält keine
Zusatzinformation.
\section{Identifikation der Größen}
Wir wollen die Objekte $A$ und $\psi$ als Eichpotential bzw. Skalarfeld auf $M/G$
auffassen, dazu müssen wir zeigen das sie entsprechend transformieren.
Da die Lieableitung der Metrik nach $V=\partial_4$ verschwindet wissen wir, dass die
Metrik invariant unter infinitesimalen Diffeomorphismsen $\xi$ sein muss.
Betrachte die Abbildung $f:M\to M$ 
\begin{equation}
\tensor{x}{^\mu}\mapsto
\tensor{x}{^\mu}\,,\quad\tensor{x}{^4}\mapsto\tensor{x}{^4}+
\xi\left(\tensor{x}{^\mu}\right)
\end{equation}
mit $\xi\in C^\infty\left(\Reals\right)$ beliebig. 
%TODO kommt von killingvector del4
Pullback $\hat{g}^\prime:=f^*\hat{g}$
%TODO Pullback in Komponenten
\begin{equation}
\begin{split}
\tensor*{\hat{g}}{*^\prime_\mu_\nu}&=\tensor{\hat{g}}{_a_b}\dpd{f^a}{\tensor{x}{^\mu}}\dpd{f^b}{\tensor{x}{^\nu}}\\
&=\tensor{\hat{g}}{_\alpha_\beta}\dpd{f^\alpha}{\tensor{x}{^\mu}}\dpd{f^\beta}{\tensor{x}{^\nu}}
+\tensor{\hat{g}}{_\alpha_4}\dpd{f^\alpha}{\tensor{x}{^\mu}}\dpd{f^4}{\tensor{x}{^\nu}}
+\tensor{\hat{g}}{_4_\beta}\dpd{f^4}{\tensor{x}{^\mu}}\dpd{f^\beta}{\tensor{x}{^\nu}}
+\tensor{\hat{g}}{_4_4}\dpd{f^4}{\tensor{x}{^\mu}}\dpd{f^4}{\tensor{x}{^\nu}}
\\
&=\tensor{\hat{g}}{_\mu_\nu}
+\psi\tensor{A}{_\mu}\pdif{_\nu}\xi
+\psi\tensor{A}{_\nu}\pdif{_\mu}\xi
+\psi\pdif{_\mu}\xi\pdif{_\nu}\xi\\
&=\tensor{g}{_\mu_\nu}
+\psi\left(\tensor{A}{_\mu}+\pdif{_\mu}\xi\right)
\left(\tensor{A}{_\nu}+\pdif{_\nu}\xi\right)
\end{split}
\end{equation}
\begin{equation}
\begin{split}
\tensor*{\hat{g}}{*^\prime_\mu_4}&=\tensor{\hat{g}}{_a_b}\dpd{f^a}{\tensor{x}{^\mu}}\dpd{f^b}{\tensor{x}{^4}}\\
&=\tensor{\hat{g}}{_\alpha_\beta}\dpd{f^\alpha}{\tensor{x}{^\mu}}\dpd{f^\beta}{\tensor{x}{^4}}
+\tensor{\hat{g}}{_\alpha_4}\dpd{f^\alpha}{\tensor{x}{^\mu}}\dpd{f^4}{\tensor{x}{^4}}
+\tensor{\hat{g}}{_4_\beta}\dpd{f^4}{\tensor{x}{^\mu}}\dpd{f^\beta}{\tensor{x}{^4}}
+\tensor{\hat{g}}{_4_4}\dpd{f^4}{\tensor{x}{^\mu}}\dpd{f^4}{\tensor{x}{^4}}\\
&=\psi\left(\tensor{A}{_\mu}+\pdif{_\mu}\xi\right)
\end{split}
\end{equation}
\begin{equation}
\begin{split}
\tensor*{\hat{g}}{*^\prime_4_4}&=\tensor{\hat{g}}{_a_b}\dpd{f^a}{\tensor{x}{^4}}\dpd{f^b}{\tensor{x}{^4}}\\
&=\tensor{\hat{g}}{_\alpha_\beta}\dpd{f^\alpha}{\tensor{x}{^4}}\dpd{f^\beta}{\tensor{x}{^4}}
+\tensor{\hat{g}}{_\alpha_4}\dpd{f^\alpha}{\tensor{x}{^4}}\dpd{f^4}{\tensor{x}{^4}}
+\tensor{\hat{g}}{_4_\beta}\dpd{f^4}{\tensor{x}{^4}}\dpd{f^\beta}{\tensor{x}{^4}}
+\tensor{\hat{g}}{_4_4}\dpd{f^4}{\tensor{x}{^4}}\dpd{f^4}{\tensor{x}{^4}}\\
&=\psi
\end{split}
\end{equation}
Das Transformationsverhalten der Größen lässt sich zusammenfassen als 
\begin{equation}
\tensor{g}{_\mu_\nu}\to\tensor{g}{_\mu_\nu}\,,\quad
\tensor{A}{_\mu}\to\tensor{A}{_\mu}+\pdif{_\mu}\xi\,,\quad
\psi\to\psi
\end{equation}
Also entsprechen Infinitisimale Diffeomorphismen auf $M$ Eichtransformationen.
Weiter ist zu Zeigen das $A$ als Differentialform $M/G$ interpretiert werden,
um dies einzusehen studieren wir das Verhalten unter Koordinatenwechseln $f:M\to
M$
\begin{equation}
\tensor{x}{^\mu}\mapsto
h(\tensor{x}{^\mu})\,,\quad\tensor{x}{^4}\mapsto\tensor{x}{^4}
\end{equation}
mit einem Diffeomorphismus $h$
\begin{equation}
\begin{split}
\tensor*{\hat{g}}{*^\prime_\mu_\nu}&=\tensor{\hat{g}}{_a_b}\dpd{f^a}{\tensor{x}{^\mu}}\dpd{f^b}{\tensor{x}{^\nu}}\\
&=\tensor{\hat{g}}{_\alpha_\beta}\dpd{f^\alpha}{\tensor{x}{^\mu}}\dpd{f^\beta}{\tensor{x}{^\nu}}
+\tensor{\hat{g}}{_\alpha_4}\dpd{f^\alpha}{\tensor{x}{^\mu}}\dpd{f^4}{\tensor{x}{^\nu}}
+\tensor{\hat{g}}{_4_\beta}\dpd{f^4}{\tensor{x}{^\mu}}\dpd{f^\beta}{\tensor{x}{^\nu}}
+\tensor{\hat{g}}{_4_4}\dpd{f^4}{\tensor{x}{^\mu}}\dpd{f^4}{\tensor{x}{^\nu}}\\
&=\tensor{\hat{g}}{_\alpha_\beta}\dpd{h^\alpha}{\tensor{x}{^\mu}}\dpd{h^\beta}{\tensor{x}{^\nu}}\\
&=\tensor{g}{_\alpha_\beta}\dpd{h^\alpha}{\tensor{x}{^\mu}}\dpd{h^\beta}{\tensor{x}{^\nu}}
+\psi\tensor{A}{_\alpha}\dpd{h^\alpha}{\tensor{x}{^\mu}}
\tensor{A}{_\beta}\dpd{h^\beta}{\tensor{x}{^\nu}}
\end{split}
\end{equation}
\begin{equation}
\begin{split}
\tensor*{\hat{g}}{*^\prime_\mu_4}&=\tensor{\hat{g}}{_a_b}\dpd{f^a}{\tensor{x}{^\mu}}\dpd{f^b}{\tensor{x}{^4}}\\
&=\tensor{\hat{g}}{_\alpha_\beta}\dpd{f^\alpha}{\tensor{x}{^\mu}}\dpd{f^\beta}{\tensor{x}{^4}}
+\tensor{\hat{g}}{_\alpha_4}\dpd{f^\alpha}{\tensor{x}{^\mu}}\dpd{f^4}{\tensor{x}{^4}}
+\tensor{\hat{g}}{_4_\beta}\dpd{f^4}{\tensor{x}{^\mu}}\dpd{f^\beta}{\tensor{x}{^4}}
+\tensor{\hat{g}}{_4_4}\dpd{f^4}{\tensor{x}{^\mu}}\dpd{f^4}{\tensor{x}{^4}}\\
&=\psi\tensor{A}{_\alpha}\dpd{h^\alpha}{\tensor{x}{^\mu}}
\end{split}
\end{equation}
\begin{equation}
\begin{split}
\tensor*{\hat{g}}{*^\prime_4_4}&=\tensor{\hat{g}}{_a_b}\dpd{f^a}{\tensor{x}{^4}}\dpd{f^b}{\tensor{x}{^4}}\\
&=\tensor{\hat{g}}{_\alpha_\beta}\dpd{f^\alpha}{\tensor{x}{^4}}\dpd{f^\beta}{\tensor{x}{^4}}
+\tensor{\hat{g}}{_\alpha_4}\dpd{f^\alpha}{\tensor{x}{^4}}\dpd{f^4}{\tensor{x}{^4}}
+\tensor{\hat{g}}{_4_\beta}\dpd{f^4}{\tensor{x}{^4}}\dpd{f^\beta}{\tensor{x}{^4}}
+\tensor{\hat{g}}{_4_4}\dpd{f^4}{\tensor{x}{^4}}\dpd{f^4}{\tensor{x}{^4}}\\
&=\psi
\end{split}
\end{equation}
\begin{equation}
\tensor{g}{_\mu_\nu}\to\tensor{g}{_\alpha_\beta}\dpd{h^\alpha}{\tensor{x}{^\mu}}\dpd{h^\beta}{\tensor{x}{^\nu}}\,,\quad
\tensor{A}{_\mu}\to\tensor{A}{_\alpha}\dpd{h^\alpha}{\tensor{x}{^\mu}}\,,\quad
\psi\to\psi
\end{equation}
Diffeomorphismeninvarianz der Klein Metrik impliziert also sowohl die
Eichinvarianz des Viererpotentials $\tensor{A}{_\mu}$, als auch das korrekte
Transformationsverhalten unter vierdimensionalen Koordinatenwechseln.
Dabei transformieren $\psi$, $\tensor{A}{_\mu}$ und $\tensor{g}{_\mu_\nu}$ als
Skalar, Vektor, bzw. Tensor.
%TODO sind noch weitere Trafos erlaubt?
\section{Eigenschaften der Klein-Metrik}
An jedem Punkt lässt sich die Metrik $\hat{g}$ durch eine symmetrische
(Block-)Matrix
\begin{equation}
\widehat{G}=\begin{pmatrix}A& B\\B\transpose&
C\end{pmatrix}
\end{equation}
beschreiben, wobei
$A\in\operatorname{Sym}_4$
\footnote{$\operatorname{Sym}_k$ bezeichnet die Menge aller symmetrischen
$k\times k$-Matritzen.} , $C\in\operatorname{Sym}_n$.
Im folgenden sei zudem die Matrix $C$ invertierbar. 
Da die Matrix reell symmetrisch ist, ist sie (block-) diagonalisierbar
\begin{equation}
\begin{split}
\widehat{G}
&=
\begin{pmatrix}1& D\\0& 1\end{pmatrix}
\begin{pmatrix}G& 0\\0& C
\end{pmatrix}
\begin{pmatrix}1& 0\\D\transpose& 1\end{pmatrix}\,.
\end{split}
\end{equation}
Die Blöcke sind gegeben als $G=A-BC^{-1}B\transpose$, $D=BC^{-1}$.
Damit lässt sich direkt die Determinante angeben
\begin{equation}
\det(\hat{g})=\det(C)\det(g)
\end{equation}
Die Transformation lässt sich leicht umkehren
\begin{equation}
\begin{pmatrix}1& D\\0& 1\end{pmatrix}^{-1}=\begin{pmatrix}1& -D\\0&
1\end{pmatrix}\,.
\end{equation}
Mit diesen Relationen ergibt sich eine alternative Darstellung des
Linienelements
\begin{equation}
\begin{split}
\dif \hat{s}^2&=\dif \vec{x}\transpose G
\dif \vec{x}\\
&=\dif \vec{x}\transpose
\begin{pmatrix}1& -D\\0& 1\end{pmatrix}
\begin{pmatrix}g& 0\\0& C
\end{pmatrix}
\begin{pmatrix}1& 0\\-D\transpose& 1\end{pmatrix}
\dif \vec{x}\\
&=\tensor{g}{_\mu_\nu}\dif \tensor{x}{^\mu}\dif\tensor{x}{^\nu}
+\tensor{C}{_a_b}\left(\dif\tensor{x}{^a}-\tensor*{D}{^a_\mu}\dif\tensor{x}{^\mu}\right)
\left(\dif\tensor{x}{^b}-\tensor*{D}{^b_\mu}\dif\tensor{x}{^\mu}\right)
\end{split}
\end{equation}
Da diese Darstellung im Wesentlichen vom Typ ist wie sie von Klein verwendet
wurde soll wird sie im Folgenden als Klein-Metrik bezeichnet. Berechnungen
angelehnt an \cite{Coquereaux:1990qs} \cite{williams2015field}.
\subsection{Darstellung in lokalen Koordinaten}
\begin{equation}
\tensor{\hat{g}}{_\mu_\nu}=\tensor{g}{_\mu_\nu}-\psi\tensor{A}{_\mu}\tensor{A}{_\nu}\,,\quad
\tensor{\hat{g}}{_4_\mu}=\psi\tensor{A}{_\mu}\,,\quad
\tensor{\hat{g}}{_4_4}=\psi
\end{equation}
\subsection{Inverse}
Wie sich leicht nachprüfen lässt ist die Inverse der Klein-Metrik gegeben als
\begin{equation}
\tensor{\hat{g}}{^\mu^\nu}=\tensor{g}{^\mu^\nu}\,,\quad
\tensor{\hat{g}}{^4^\mu}=-\tensor{A}{^\mu}\,,\quad
\tensor{\hat{g}}{^4^4}=\tensor{A}{_\mu}\tensor{A}{^\mu}+\frac{1}{\psi}\,.
\end{equation}
\subsection{Determinante}
Die Determinante wird wie üblich zur Berechnung von Volumenelementen benötigt.
Bei der Berechnung ist folgende Formel für Matrizen $A,B,C,D$ hilfreich:
\begin{equation}
\begin{split}
\det\begin{pmatrix}A& B\\ C& D\end{pmatrix}&=\det\left[
\begin{pmatrix}1& B\\0& D\end{pmatrix}\begin{pmatrix}A-BD^{-1}C& 0\\DC^{-1
}& 1\end{pmatrix} \right]\\
&= \det(D) \det\left(A - B D^{-1}
C\right)\,.
\end{split}
\end{equation}
%TODO evtl. Beweis
Damit ergibt sich 
\begin{equation}
\begin{split}
 \hat{g}&=
 \det\begin{pmatrix}\tensor{g}{_\mu_\nu}+\psi\tensor{A}{_\mu}\tensor{A}{_\nu}
 &\psi \tensor{A}{_\mu}\\
 \psi \tensor{A}{_\mu}	 & \psi
 \end{pmatrix}\\
 &=\psi
 \det\left(\tensor{g}{_\mu_\nu}+\psi\tensor{A}{_\mu}\tensor{A}{_\nu}
 -\psi\tensor{A}{_\mu}\psi^{-1}\psi\tensor{A}{_\nu}\right)\\
 &=\psi \det\left(\tensor{g}{_\mu_\nu}\right)\\
 &=\psi g\,.
\end{split}
\end{equation}
Die Determinante der fünfdimensionalen Metrik ist also proportional zur 4
dimensionalen. 
\begin{bemerkung}
Die entsprechenden Ausdrücke für die Kaluza-Metrik sind deutlich komplizierter.
\end{bemerkung}
\subsection{Christoffelsymbole}
Siehe Williams: "`Field Equations and Lagrangian for the Kaluza Metric
Evaluated with Tensor Algebra Software"'\cite{williams2015field}
\subsection{Skalarkrümmung}
\begin{equation}
\hat{R}=R-\frac{1}{4}\psi\tensor{F}{_\mu_\nu}\tensor{F}{^\mu^\nu}
+\frac{1}{2\psi^2}(\pdif{_\mu}\psi)(\pdif{^\mu}\psi)
-\frac{1}{\psi}\square\psi
\end{equation}
Wir führen zwei weitere Felder $\psi=:\phi^2$ und $\psi=:e^{2\sigma}$ ein womit
sich der Term nochmals vereinfacht \footnote{Dies stellt keine Einschränkung
dar, da $\psi$ positiv sein muss, damit $\hat{g}$ das gleiche Vorzeichen hat wie $g$. (Bei der Zusatzdimension handelt es sich um
eine Raumdimension)}:
\begin{equation}
\hat{R}=R-\frac{1}{4}\phi^2\tensor{F}{_\mu_\nu}\tensor{F}{^\mu^\nu}
-\frac{2}{\phi}\square \phi
\end{equation}

\begin{equation}
\hat{R}=R-\frac{1}{4}e^{2\sigma}\tensor{F}{_\mu_\nu}\tensor{F}{^\mu^\nu}
-2e^{-\sigma}\square e^{\sigma}
\end{equation}
\section{Lagrangedichte}
%Karte auf $M/G$?
Kleins Idee war es das Wirkungsprinzip nach Hilbert auf fünf Dimensionen zu
erweitern. Folglich lautet das Wirkungsintegral
\begin{equation}
\hat{S}=\int_{M_5}\sqrt{-\hat{g}}\hat{R}\dif{^5}
x\,.
\end{equation}
% TODO d.h. Vakuum
Hierbei tritt ein ernsthaftes Problem auf. Der Integrand ist nicht von der
Zusatzdimension abhängig. Würde man also $\tensor{x}{^4}$ wie die verbleibenden
Raumkoordinaten behandeln, wie dies beispielsweise bei Kaluza der Fall war, so
ist $\hat{S}$ nicht wohldefiniert. Klein erkannte das er das Problem umgehen
konnte wenn die Zusatzkoordinate zylindrisch wäre. Nimmt man für 
$\tensor{x}{^4}$ einen Radius $r$ an so ergibt sich
\begin{equation}
\begin{split}
\hat{S}&=\int_{M_5}\sqrt{-g}\phi\hat{R}\dif{^5}x\\
&=2\pi
r\int_{M_4}\sqrt{-g}\phi\hat{R}\dif{^4}x\,.
\end{split}
\end{equation}
%TODO dies ist die minimale skalare Theorie
%TODO wenn faserbündel korrekt dann mit trafoformel argumentieren (überdeckung
% etc\ldots)
Die Lagrangedichte ist also gegeben als
\begin{equation}
\begin{split}
\mathcal{L}&=\sqrt{-g}e^{\sigma}\left(R-\frac{1}{4}e^{2\sigma}\tensor{F}{_\mu_\nu}\tensor{F}{^\mu^\nu}
-2e^{-\sigma}\square e^{\sigma}\right)\\
&=\sqrt{-g}e^{\sigma}\left(R-\frac{1}{4}e^{2\sigma}\tensor{F}{_\mu_\nu}\tensor{F}{^\mu^\nu}\right)+2\sqrt{-g}\square
e^{\sigma}\label{eq:Lagrange1}
\end{split}
\end{equation}
Der letzte Term liefert als totale Divergenz keinen Beitrag.
Der Term der die Krümmung enthält unterscheidet sich um einen Faktor $\phi$ von
der klassischen Einstein-Hilbert Lagrangedichte.
\subsection{Konforme Transformation}
Da wir den Lagrangian gerne in einer Form vorliegen hätten, die dem
EH-Lagrangian enspricht müssen wir den Vorfaktor $e^{\sigma}$ loswerden. 
Wir führen dazu eine konforme Transformation durch
\begin{equation}
\tensor*{\hat{g}}{*^\star*_i*_j}=e^{2\tau}\tensor{\hat{g}}{_i_j}\,.
\end{equation}
Dies impliziert sofort
\begin{equation}
\tensor*{g}{*^\star*_\mu*_\nu}=e^{2\tau}\tensor{g}{_\mu_\nu}\,,\quad\sigma^\star
=\sigma+\tau\,,\quad \sqrt{-g}=e^{-4\tau}\sqrt{-g^\star}\,.
\end{equation}
Die Komponenten des elektrischen Feldstärketensors ist in der konformen Metrik
gegeben als
\begin{equation}
\tensor*{F}{*^\star_\mu_\nu}=\tensor{F}{_\mu_\nu}\,,\quad\tensor*{F}{*^\star^\mu^\nu}
=\tensor*{\hat{g}}{*^\star*^\mu*^\alpha}\tensor*{\hat{g}}{*^\star*^\nu*^\beta}\tensor{F}{_\alpha_\beta}
=e^{-4\tau}\tensor{F}{^\mu^\nu}
\end{equation}
Wendet man die Formel für das Transformationsverhalten des Krümmungsskalars an,
so findet man schließlich
\begin{equation}
R=e^{2\tau}\left[R^\star-6(\tensor*{\partial}{^\star_\mu}\tau)(\tensor{\partial}{^\star^\mu}\tau)
-6\square^\star\tau\right]\,,
\end{equation}
beziehungsweise
\begin{equation}
\begin{split}
\sqrt{-g}\phi
\hat{R}
&=e^{\sigma-2\tau}\sqrt{-g^\star}\left[\hat{R}^\star
-6(\tensor*{\partial}{^\star_\mu}\tau)(\tensor{\partial}{^\star^\mu}\tau)
-6\square^\star\tau\right]\,.
\end{split}
\end{equation}
% TODO Einfluss konformer Trafos
Setzt man $\tau = \frac{1}{2}\sigma$\footnote{Die daraus resultierende
Relation $\sigma^\star=\frac{3}{2}\sigma$ kann durch eine Skalierung von
$\sigma$ behandelt werden.}, so ergibt sich
\begin{equation}
\begin{split}
\sqrt{-g}\phi
\hat{R}&=\sqrt{-g^\star}\left[R^\star
-\frac{3}{2}(\tensor*{\partial}{^\star_\mu}\sigma)(\tensor{\partial}{^\star^\mu}\sigma)
-3\square^\star\sigma\right]\,.
\end{split}
\end{equation}
Setzt man in \eqref{eq:Lagrange1} ein erhält man
\begin{equation}
\begin{split}
\mathcal{L}&=\sqrt{-g^\star}\left[R^\star
+\frac{3}{2}(\tensor*{\partial}{^\star_\mu}\sigma)(\tensor{\partial}{^\star^\mu}\sigma)
-3\square^\star\sigma\right]
+\sqrt{-g}e^{\sigma}\left(-\frac{1}{4}e^{2\sigma}\tensor{F}{_\mu_\nu}\tensor{F}{^\mu^\nu}\right)\\
&=\sqrt{-g^\star}\left[R^\star
-\frac{3}{2}(\tensor*{\partial}{^\star_\mu}\sigma)(\tensor{\partial}{^\star^\mu}\sigma)
-\frac{1}{4}e^{3\sigma}\tensor*{F}{^\star_\mu_\nu}\tensor*{F}{^\star^\mu^\nu}\right]-3\sqrt{-g^\star}\square^\star\sigma\,.
\end{split}
\end{equation}
Der Term mit der totalen Divergenz produziert einen nicht beitragenden Randterm
und kann deshalb fallen gelassen werden. 
%TODO Randterme
Da eine Variation der Konformen Metrik auf die
gleichen Gleichungen führt, lassen wir im folgenden die Sterne an den Größen weg. Zudem führen wir
Bezeichnungen ein die zu einer üblichen Form führen
\begin{equation}
\begin{split}
\mathcal{L}
&=\sqrt{-g}\left[R
-\frac{1}{2}(\tensor{\partial}{_\mu}\sigma)(\tensor{\partial}{^\mu}\sigma)
-\frac{1}{4}\tensor*{H}{_\mu_\nu}\tensor*{F}{^\mu^\nu}\right]\,.
\label{eq:Lagrange2}
\end{split}
\end{equation}
Dabei bezeichnet die
Größe $H$ die elektromagnetische Verschiebungsfelddichte\footnote{In Analogie zu
den Verschiebungsfeldern $\vec{D},\vec{H}$ der klassischen Elektrodynamik.
}
%TODO Insbesondere handelt es sich nicht um das Nebenfeld
% $\tensor{H}{_\mu_\nu}$.
\begin{equation}
\tensor*{H}{_\mu_\nu}:=e^{\sqrt{3}\sigma}\tensor*{F}{_\mu_\nu}\,,
\end{equation}
wobei der Term $e^{\sqrt{3}\sigma}$ als variable Perimitivität $\mu(\sigma)$
interpretiert wird. Dies ist ein beachtliches Ergebnis, dass falls
$\sigma=\text{const.}=0$ exakt der Lagrangedichte der klassischen
Maxwell-Einstein Theorie enstpricht. Dieser Umstand ist auch als
\emph{Kaluza-Klein Wunder} bekannt.
% Ist es konsistent nach den Komponenten einzeln zu variieren?
\subsection{Bewegungsgleichungen}
In bekannter Manier impliziert die Form der Lagrangedichte 
Die Einsteingleichungen:
\begin{equation}
\tensor{G}{_\mu_\nu}=\tensor*{T}{*^{\sigma}_\mu_\nu}+\tensor*{T}{*^{A}*_\mu*_\nu}
\end{equation}
Weiter erhält man für die Variation nach den $A$ 
Das Feld $A$ ist Zyklisch, taucht als nicht selbst in der Lagrangedichte auf.
Die resultierende Bewgungsgleichungen lautet
\begin{equation}
0=\tensor{\nabla}{_\alpha}\left(\dpd{\mathcal{L}}{\left(\tensor{\nabla}{_\alpha}\tensor{A}{_\beta}\right)}\right)
=\tensor{\nabla}{_\alpha}\left[e^{\sqrt{3}\sigma}\dpd{\left(\tensor{F}{_\mu_\nu}\tensor{F}{^\mu^\nu}\right)}{\left(\tensor{\nabla}{_\alpha}\tensor{A}{_\beta}\right)}\right]
=4\tensor{\nabla}{_\alpha}\tensor{H}{^\alpha^\beta}
 \end{equation}
 Für das Vektorpotential $A$, sowie
 % Ableitung nach DA im MW teil herleinten
 \begin{equation}
0=\tensor{\nabla}{_\alpha}\left(\dpd{\mathcal{L}}{\left(\tensor{\partial}{_\alpha}\sigma\right)}\right)
-\dpd{\mathcal{L}}{\sigma}\\
=\square \sigma
-\frac{\sqrt{3}}{4}\tensor*{H}{_\mu_\nu}\tensor*{F}{^\mu^\nu}
\label{eq:dymdilat}
 \end{equation}
 für das Skalarfeld $\sigma$. 
 % Lässt sich das durch redifinition zur Klein Gordon Gleichung hinbiegen?
\section{Erhaltungrößen}
Per Konstruktion ist $K=\partial_4$ ein Killing-Vektorfeld, die entsprechende
Erhaltungsgröße ist 
\begin{equation}
\tensor{\hat{K}}{_n}\tensor{\hat{U}}{^n}=\tensor*{\delta}{^4_n}\tensor{\hat{U}}{^n}
=\tensor{\hat{U}}{^4}:=Q\,.
\end{equation}
Da wir Vakuumlösungen $\tensor{T}{_n_m}=0$ betrachten, lässt sich keine erhalte
Energie 
\begin{equation}
E=\int\dif{}^3x \sqrt{-g} \tensor{T}{_0_n}\tensor{K}{^n}
\end{equation}
definieren. Wir wollen die erhaltene vierte Komponente der Geschwindigkeit mit 
der Ladung $Q$ eines Teilchens identifizieren.
 \section{Geodäten}
 Wir nehmen an das sich (geladene) Teilchen entlang von fünfdimensionalen
 Geodäten bewegen. 
 Zunächst gilt weiterhin, dass diese proportional zur Bogelänge parametrisiert
 sind, d.h.
 \begin{equation}
 \begin{split}
  C&=
  \tensor{\hat{g}}{_\mu_\nu}\tensor{\hat{U}}{^\mu}\tensor{\hat{U}}{^\nu}
  +\tensor{\hat{g}}{_4_4}\tensor{\hat{U}}{^4}\tensor{\hat{U}}{^4}\\\
 &=\tensor{\hat{g}}{_\mu_\nu}\tensor{\hat{U}}{^\mu}\tensor{\hat{U}}{^\nu}
 +\psi Q^2\\
  &=\tensor{g}{_\mu_\nu}\tensor{\hat{U}}{^\mu}\tensor{\hat{U}}{^\nu}
  -\psi\left(\tensor{A}{_\mu}\tensor{\hat{U}}{^\mu}\right)^2
 +\psi Q^2\\
 \end{split}
 \end{equation}
%   Die Geodätengleichung hat die Form
%  \begin{equation}
%  0=\tensor{\hat{U}}{^m}\tensor{\hat{\nabla}}{_m} \tensor{\hat{U}}{_n}
%  \end{equation}
%  Betrachten wir speziell die vierdimensionalen Komponenten $n=\nu$ so finden wir
% \begin{equation}
% \begin{split}
% 0&=\tensor{\hat{U}}{^m}\tensor{\hat{\nabla}}{_m} \tensor{\hat{U}}{_\nu}\\
% &=\tensor{\hat{U}}{^m}\tensor{\partial}{_m} \tensor{\hat{U}}{_\nu}
%  +\tensor*{\hat{\Gamma}}{_m_\ell_\nu}
%  \tensor{\hat{U}}{^m}\tensor{\hat{U}}{^\ell}\\
%  &=\dod{}{\lambda} \tensor{\hat{U}}{_\nu}
% +\tensor*{\hat{\Gamma}}{_\mu_\lambda_\nu}
%  \tensor{\hat{U}}{^\mu}\tensor{\hat{U}}{^\lambda}
%  +2\tensor*{\hat{\Gamma}}{_\mu_4_\nu}
%  \tensor{\hat{U}}{^4}\tensor{\hat{U}}{^\mu}
%  +\tensor*{\hat{\Gamma}}{_4_4_\nu}
%  \tensor{\hat{U}}{^4}\tensor{\hat{U}}{^4}\\
%   &=\dod{}{\lambda} \tensor{\hat{U}}{_\nu}
% +\tensor*{\hat{\Gamma}}{_\mu_\lambda_\nu}
%  \tensor{\hat{U}}{^\mu}\tensor{\hat{U}}{^\lambda}
%  +2Q\tensor*{\hat{\Gamma}}{_\mu_4_\nu}
%  \tensor{\hat{U}}{^\mu}
%  +Q^2\tensor*{\hat{\Gamma}}{_4_4_\nu}
%  \\
%    &=\dod{}{\lambda} \tensor{\hat{U}}{_\nu}
% +\tensor*{\Gamma}{_\mu_\lambda_\nu}
%  \tensor{\hat{U}}{^\mu}\tensor{\hat{U}}{^\lambda}
%  +\psi Q\tensor*{F}{_\mu_\nu}
%  \tensor{\hat{U}}{^\mu}
%  +Q^2\partial_\nu\psi\\
% \end{split}
% \end{equation}
% Insegammt findet man 
% \begin{equation}
% m\tensor{\ddot{x}}{^\mu}+m\tensor*{\Gamma}{^\mu_\nu_\lambda}
% \tensor{\dot{x}}{^\nu}\tensor{\dot{x}}{^\lambda}=
% \psi mQ\tensor*{F}{_\mu_\nu}
%  \tensor{\dot{x}}{^\mu}
%  +mQ^2\partial_\nu\psi
% \end{equation}
% Es liegt nahe $mQ=q$ zu setzen
% \begin{equation}
% m\tensor{\ddot{x}}{^\mu}+m\tensor*{\Gamma}{^\mu_\nu_\lambda}
% \tensor{\dot{x}}{^\nu}\tensor{\dot{x}}{^\lambda}=
% \psi q\tensor*{F}{_\mu_\nu}
%  \tensor{\hat{U}}{^\mu}
%  \tensor{\hat{U}}{^\mu}
%  +\frac{q^2}{m}\partial_\nu\psi
% \end{equation}
% % Dies kann wie folgt interpretiert werden: 
% % \begin{itemize}
% %   \item $-Q\tensor*{F}{_\mu_\nu}
% %  \tensor{U}{^\mu}$ beschreibt die gewöhnliche Lorentzkraft modifiziert mit d
% %  \item  $-Q^2\partial_\nu\psi$ beschreibt eine Kraft die durch ein Potential
% %  $Q^2\psi$ ausgeübt wird.
% % \end{itemize}
% Setzt man $\psi=1$ so erhält man die bekannten Gleichungen der Maxwell Theorie.
% \begin{bemerkung}
% Will man die Normierung $\tensor{U}{_\mu}\tensor{U}{^\mu}={0,-1}$ aufrecht
% erhalten so muss
% $\tensor{\hat{U}}{_m}\tensor{\hat{U}}{^m}=\tensor{U}{_\mu}\tensor{U}{^\mu}+Q^2={Q^2,Q^2-1}$
% gelten. Die Vektoren in fünf Dimensionen sind also je nach Ladung raumartig!
% Dies führt zu fragen bezüglich Kausalität
% \end{bemerkung}
\section{Die Rolle des skalaren Felds}
  Klein maß dem zusätzlichen Freiheitsgrad der durch das skalaarfeld
  $\sigma$\footnote{Bzw. $\phi$ oder $\psi$.} gegeben ist, keine physikalische
  Bedeutung bei.
  Demensprechend setzte er $\psi=\mathrm{const.}=1$ .
 Dadurch vereinfacht sich die Lagrangedichte zur Lagrangedichte der
 Einstein-Maxwell Theorie. Allerdings impliziert die dynamische Gleichung
 \eqref{eq:dymdilat}
  \begin{equation}
\tensor*{F}{_\mu_\nu}\tensor*{F}{^\mu^\nu}=0\,,
 \end{equation}
 der Beitrag der elektrische Beitrag zu \eqref{eq:Lagrange2} verschwindet also.
 Dies führt die Konstruktion ad absurdum. Einziger Ausweg ist die $\tensor{g}{_4_4}$-Komponente
\emph{a priori} konstant zu setzen und nicht zu variieren. Dies trübt
die Allgemeinheit der Theorie, da dadurch diese Komponente gegenüber den anderen
ausgezeichnet ist. Zusätzliche skalare Felder tauchen häufig in Theorien auf die
sich der dimensionalen Reduktion bedienen. Sie werden häufig mit einem Teilchen,
dem \emph{Dilaton} identifiziert. Zusätzliche Teilchen stellen an sich kein
Probleme dar, einige ungelöste Fragestellungen der Physik (dunkle
Materie/Energie) benötigen sie sogar um beantwortet zu werden. 
\section{Quantisierung der Ladung}
Die Kompaktheit von $\Sphere^1$ impliziert die Quantisierung der Ladung (Klein)
% \hat{R}^\star
% =e^{-\tau}\left(\hat{R}+3\tensor{\nabla}{_\mu}\left(\tensor{\hat{g}}{^\mu^\nu}\tensor{\partial}{_\nu}\tau\right)
% -\frac{3}{2}\tensor{\hat{g}}{^\mu^\nu}\tensor{\partial}{_\mu}\tau\tensor{\partial}{_\nu}\tau\right)\,,\quad\sqrt{-\hat{g}^\star}=e^{5\varphi}\sqrt{-\hat{g}}
% \end{equation}
% \begin{equation}
% \begin{split}
% \mathcal{L}^\star
% &=\hat{R}^\star\sqrt{-\hat{g}^\star}\\
% &=e^{3\varphi}\sqrt{-\hat{g}}\left(\hat{R}+\frac{16}{3}e^{-\frac{3}{2}\varphi}\square
% e^{\frac{3}{2}\varphi}\right)\\
% &=e^{3\varphi}\sqrt{-g}\phi\left(R-\frac{1}{4}\phi^3\tensor{F}{_\mu_\nu}\tensor{F}{^\mu^\nu}+\frac{16}{3}e^{-\frac{3}{2}\varphi}\square
% e^{\frac{3}{2}\varphi}\right)\\
% &=e^{3\varphi}\sqrt{-g}\phi\left(R-\frac{1}{4}\phi^3\tensor{F}{_\mu_\nu}\tensor{F}{^\mu^\nu}+12\pdif{_\mu}\varphi\pdif{^\mu}\varphi+8\square
% \varphi\right)\\
% &=e^{3\varphi}\sqrt{-g}\phi\left(R
% -\frac{1}{4}\phi^3\tensor{F}{_\mu_\nu}\tensor{F}{^\mu^\nu}
% +12\pdif{_\mu}\varphi\pdif{^\mu}\varphi
% +8e^{2\varphi}\square^\star\varphi+24e^{2\varphi}\pdif{_\mu}\varphi\pdif{^\mu}\varphi\right)
% \end{split}
% \end{equation}
% Es liegt nahe $\phi=e^{-3\varphi}$ zu setzen.
% \begin{equation}
% \begin{split}
% \hat{R}^\star\sqrt{-\hat{g}^\star}
% &=\sqrt{-g}\left(R+12\pdif{_\mu}\varphi\pdif{^\mu}\varphi+8\square
% \varphi\right)
% \end{split}
% \end{equation}
% $\sigma=\sqrt{24}\varphi$
% \begin{equation}
% \begin{split}
% \hat{R}^\star\sqrt{-\hat{g}^\star}
% &=\sqrt{-g}\left(R+\frac{1}{2}\pdif{_\mu}\sigma\pdif{^\mu}\sigma\right)
% \end{split}
% \end{equation}
% Coqueraux Jordan Thiery
% % Im Folgenden lassen wir den Term $2\pi
% % r$ weg, da er keinen Einfluss auf das Variationsprinzip hat. Die Lagrangedichte
% % ist damit
% % \begin{equation}
% % \begin{split}
% % \mathcal{L}=\sqrt{-g}\left(\phi
% % R-\frac{1}{4}\phi^3\tensor{F}{_\mu_\nu}\tensor{F}{^\mu^\nu}\right)
% % &=\sqrt{-g}\left(\phi
% % \mathcal{L}\textsubscript{g}+\phi^3\mathcal{L}\textsubscript{EM}\right)
% % \end{split}
% % \end{equation}
% % Die Tatsache dass sich die Lagrangedichte in einen Term der die Raumkrümmung
% % enthält und einen Elektromagnetischen Anteil aufspaltet ist auch als
% % Kaluza-Klein Wunder\footnote{"`Kaluza-Klein miracle"'} bekannt. Insbesondere
% % erhält man für $\phi=1$ die Lagrangedichte für ein System das sowohl einstein
% % als auch Maxwellgleichungen erfüllt. Da der Lagrangian keine Kinetischen Terme
% % in $\varphi$ enthält folgt
% % \begin{equation}
% % 0=\dpd{\mathcal{L}}{\phi}=\sqrt{-g}\left(R-\frac{3}{4}\phi^2\tensor{F}{_\mu_\nu}\tensor{F}{^\mu^\nu}\right)
% % \end{equation}
% % Weiter erhält man für die Variation nach den $A$ 
% % Das Feld $A$ ist Zyklisch, taucht als nicht selbst in der Lagrangedichte auf.
% % Die resultierende Erhaltungsgleichung lautet
% % \begin{equation}
% % 0=\tensor{\nabla}{_\alpha}\left(\dpd{\mathcal{L}}{\left(\tensor{\nabla}{_\alpha}\tensor{A}{_\beta}\right)}\right)
% =\tensor{\nabla}{_\alpha}\left(\phi^3\dpd{\tensor{F}{_\mu_\nu}\tensor{F}{^\mu^\nu}}{\left(\tensor{\nabla}{_\alpha}\tensor{A}{_\beta}\right)}\right)
% =4\tensor{\nabla}{_\alpha}\left(\phi^3\tensor{F}{^\alpha^\beta}\right)
% \end{equation}
% \begin{equation}
% \tensor{G}{_\mu_\nu}=\phi^2\tensor*{T}{*^{\text{M}}*_\mu*_\nu}
% \end{equation}
% Oder umformuliert
% \begin{equation}
% \tensor{\nabla}{_\alpha}\tensor{F}{^\alpha^\beta}
% =-\frac{3}{\phi}\tensor{F}{^\alpha^\beta}\tensor{\partial}{_\alpha}\phi
% \end{equation}
% Bzw:
% \begin{equation}
% \tensor{J}{^\beta}
% =-\frac{3}{\phi\sqrt{-g}}\tensor{F}{^\alpha^\beta}\tensor{\partial}{_\alpha}\phi
% =-3\left(-\hat{g}\right)^{-\nicefrac{1}{2}}\tensor{F}{^\alpha^\beta}\tensor{\partial}{_\alpha}\phi
% \end{equation}
% \begin{equation}
% \begin{split}
% \tilde{R}&=e^{-2\varphi}\left(R+\frac{16}{3}e^{-\frac{3}{2}\varphi}\square
% e^{\frac{3}{2}\varphi}\right)\\
% &=e^{-2\varphi}\left(R+8\square\varphi+12\pdif{_\mu}\varphi\pdif{^\mu}\varphi\right)
% \end{split}
% \end{equation}
% \begin{equation}
% \begin{split}
% \phi\sqrt{-g}R
% &=\phi
% e^{-\varphi}\sqrt{\tilde{g}}\left(\tilde{R}+8\square\varphi+12\pdif{_\mu}\varphi\pdif{^\mu}\varphi\right)\\
% \end{split}
% \end{equation}
% $\varphi=\ln \phi$
% \begin{equation}
% \begin{split}
% \phi\sqrt{-g}R
% &=\sqrt{\tilde{g}}\left(\tilde{R}
% +8{\square}\ln\phi 
% +\frac{12}{\phi^2}\pdif{_\mu}\phi\pdif{^\mu}\phi\right)
% \end{split}
% \end{equation}
% \begin{equation}
% \begin{split}
% \phi\sqrt{-g}R
% &=\sqrt{\tilde{g}}\left(\tilde{R}
% +8\frac{1}{\phi}\square\phi 
% +\frac{4}{\phi^2}\pdif{_\mu}\phi\pdif{^\mu}\phi\right)
% \end{split}
% \end{equation}
% \begin{equation}
% \begin{split}
% \phi\sqrt{-g}R
% &=\sqrt{\tilde{g}}\left(\tilde{R}
% +8\frac{1}{\phi}\square\phi 
% +16\pdif{_\mu}\Lambda\pdif{^\mu}\Lambda\right)
% \end{split}
% \end{equation}
% \begin{equation}
% \begin{split}
% 0&=\frac{\delta\mathcal{L}}{\delta\tensor{g}{_\mu_\nu}}\\
% &=\phi\frac{\delta\mathcal{L}\textsubscript{EH}}{\delta\tensor{g}{_\mu_\nu}}
% +\phi^3\frac{\delta\mathcal{L}\textsubscript{EM}}{\delta\tensor{g}{_\mu_\nu}}\\
% &=
% \phi\sqrt{-g}\left[\frac{1}{2}\tensor{g}{^\mu^\nu}
% R+\tensor{R}{^\mu^\nu}\right]
% \end{split}
% \end{equation}
% \begin{split}
% \tensor*{\hat{g}}{*^\prime_\mu_\nu}&=\tensor{\hat{g}}{_a_b}\dpd{f^a}{\tensor{x}{^\mu}}\dpd{f^b}{\tensor{x}{^\nu}}\\
% &=\tensor{\hat{g}}{_\alpha_\beta}\dpd{f^\alpha}{\tensor{x}{^\mu}}\dpd{f^\beta}{\tensor{x}{^\nu}}
% +\tensor{\hat{g}}{_\alpha_4}\dpd{f^\alpha}{\tensor{x}{^\mu}}\dpd{f^4}{\tensor{x}{^\nu}}
% +\tensor{\hat{g}}{_4_\beta}\dpd{f^4}{\tensor{x}{^\mu}}\dpd{f^\beta}{\tensor{x}{^\nu}}
% +\tensor{\hat{g}}{_4_4}\dpd{f^4}{\tensor{x}{^\mu}}\dpd{f^4}{\tensor{x}{^\nu}}
% \\
% &=\tensor{\hat{g}}{_\alpha_\beta}\dpd{f^\alpha}{\tensor{x}{^\mu}}\dpd{f^\beta}{\tensor{x}{^\nu}}
% +\psi\tensor{A}{_\alpha}\dpd{g^\alpha}{\tensor{x}{^\mu}}\pdif{_\nu}h
% +\psi\tensor{A}{_\beta}\dpd{g^\beta}{\tensor{x}{^\nu}}\pdif{_\mu}h
% +\psi\pdif{_\mu}h\pdif{_\nu}h\\
% &=\tensor{g}{_\alpha_\beta}\dpd{g^\alpha}{\tensor{x}{^\mu}}\dpd{g^\beta}{\tensor{x}{^\nu}}
% +\psi\left(\tensor{A}{_\alpha}\dpd{g^\alpha}{\tensor{x}{^\mu}}+\pdif{_\mu}h\right)
% \left(\tensor{A}{_\alpha}\dpd{g^\alpha}{\tensor{x}{^\nu}}+\pdif{_\nu}h\right)
% \end{split}
% \end{equation}
% \begin{equation}
% \begin{split}
% \tensor*{\hat{g}}{*^\prime_\mu_4}&=\tensor{\hat{g}}{_a_b}\dpd{f^a}{\tensor{x}{^\mu}}\dpd{f^b}{\tensor{x}{^4}}\\
% &=\tensor{\hat{g}}{_\alpha_\beta}\dpd{f^\alpha}{\tensor{x}{^\mu}}\dpd{f^\beta}{\tensor{x}{^4}}
% +\tensor{\hat{g}}{_\alpha_4}\dpd{f^\alpha}{\tensor{x}{^\mu}}\dpd{f^4}{\tensor{x}{^4}}
% +\tensor{\hat{g}}{_4_\beta}\dpd{f^4}{\tensor{x}{^\mu}}\dpd{f^\beta}{\tensor{x}{^4}}
% +\tensor{\hat{g}}{_4_4}\dpd{f^4}{\tensor{x}{^\mu}}\dpd{f^4}{\tensor{x}{^4}}\\
% &=\psi\tensor{A}{_\alpha}\dpd{g^\alpha}{\tensor{x}{^\mu}}\pdif{_4}h+\psi\pdif{_\mu}h\pdif{_4}h\\
% &=\psi\pdif{_4}h\left(\tensor{A}{_\alpha}\dpd{g^\alpha}{\tensor{x}{^\mu}}+\pdif{_\mu}h\right)
% \end{split}
% \end{equation}
% \begin{equation}
% \begin{split}
% \tensor*{\hat{g}}{*^\prime_4_4}&=\tensor{\hat{g}}{_a_b}\dpd{f^a}{\tensor{x}{^4}}\dpd{f^b}{\tensor{x}{^4}}\\
% &=\tensor{\hat{g}}{_\alpha_\beta}\dpd{f^\alpha}{\tensor{x}{^4}}\dpd{f^\beta}{\tensor{x}{^4}}
% +\tensor{\hat{g}}{_\alpha_4}\dpd{f^\alpha}{\tensor{x}{^4}}\dpd{f^4}{\tensor{x}{^4}}
% +\tensor{\hat{g}}{_4_\beta}\dpd{f^4}{\tensor{x}{^4}}\dpd{f^\beta}{\tensor{x}{^4}}
% +\tensor{\hat{g}}{_4_4}\dpd{f^4}{\tensor{x}{^4}}\dpd{f^4}{\tensor{x}{^4}}\\
% &=\psi\left(\pdif{_4}h\right)^2
% \end{split}
% \end{equation}
% Um konsistent zu bleiben muss $\pdif{_4}h = 1$ also sind die erlaubten
% transformationen von der Form
% \begin{equation}
% \tensor{g}{_\mu_\nu}\to\tensor{g}{_\alpha_\beta}\dpd{g^\alpha}{\tensor{x}{^\mu}}\dpd{g^\beta}{\tensor{x}{^\nu}}\,,\quad
% \tensor{A}{_\alpha}\to\tensor{A}{_\alpha}\dpd{g^\alpha}{\tensor{x}{^\mu}}+\pdif{_\mu}h\\
% \psi\to\psi
% \end{equation}
%  
%  In Verallgemeinerung der vierdimensionalen Geodätischen
%  \begin{equation}
%  0=\tensor{\hat{U}}{^m}\tensor{\nabla}{_m}
%  \tensor{\hat{U}}{_n}=\dod{}{\lambda}
%  \tensor{\hat{U}}{_n}+\tensor{\hat{U}}{^m}\tensor*{\hat{\Gamma}}{^\ell_m_n}\tensor{\hat{U}}{_\ell}
%  \end{equation}
%  % Erhaltungsgröße zu killingtensor d4
%  \begin{equation}
%  \begin{split}
%   0&=\tensor{\hat{U}}{^m}\tensor{\nabla}{_m} \tensor{\hat{U}}{_5}\\
%  &=\dod{}{\lambda}\tensor{\hat{U}}{_5}
%  +\tensor{\hat{U}}{^m}\tensor*{\hat{\Gamma}}{*^n_5_m} \tensor{\hat{U}}{_n}\\
%  &=\dod{}{\lambda}\tensor{\hat{U}}{_5}
%  +\frac{1}{2}\tensor{\hat{U}}{^m}\tensor{\hat{g}}{^n^a}\left(\tensor{\hat{g}}{_a_m_{,5}}
%  +\tensor{\hat{g}}{_a_5_{,m}}
%  -\tensor{\hat{g}}{_m_5_{,a}}
%  \right)\tensor{\hat{U}}{_n}\\
%   &=\dod{}{\lambda}\tensor{\hat{U}}{_5}
%  +\frac{1}{2}\tensor{\hat{U}}{^m}\tensor{\hat{U}}{^a}\left(
%  \tensor{\hat{g}}{_a_5_{,m}}
%  -\tensor{\hat{g}}{_m_5_{,a}}
%  \right)\\
%   &=\dod{}{\lambda}\tensor{\hat{U}}{_5}\\
%  \end{split}
%  \end{equation}
%  Der zweite Term verschwindet als Spur eines Produkt eines symmetrischen mit
%  einem antisymmetrischen Tensors.