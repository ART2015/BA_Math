\chapter{Kaluzas erste Schritte}
Wie bereits in den vorherigen Kapiteln klar wurde, weißt die Struktur von
Elektromagnetismus und allgemeiner Relativitätstheorie deutliche Parallelen auf.
Dies wurde bereits in den ersten Jahren nach der Entwicklung der
allgemeinen Relativitätstheorie untersucht und mündete
unter anderem in Theorien von \name{H. Weyl}\cite{weyl1918gravitation} und
\name{G. Nordström} \cite{nordstrom1914moglichkeit}.

Dem Finnen \name{Nordström} gebührt dabei die Ehre, als Erster den
Elektromagnetismus als Phänomen einer fünfdimensionalen Raumzeit zu deuten.
Dabei baute seine Theorie auf der \name{Nordström}schen Gravitationstheorie
auf, welche allerdings verworfen wurde, da Voraussagen macht, welche nicht in
Einklang mit den Beobachtungen stehen\footnote{Beispielsweise drastische
Abweichungen in der gravitative Zeitverschiebung und der Periheldrehung des
Merkur.} .
Nicht zuletzt deshalb fand die Theorie seinerzeit wenig Beachtung, ist heute
allerdings weiterhin von theoretischem Interesse.
\name{Kaluza} bezieht sich vor allem auf \name{Weyl}s Arbeit, die aber ihrem
Charakter nach deutlich verschieden von der \name{Kaluza}-Theorie ist.
\section{Kaluza-Theorie}
\name{Kaluza} erkennt die Verwandtschaft der Theorien an der Form der
Christoffelsymbolen. Dazu bemerkt er\cite{kaluza1921unitatsproblem}:
%zu welcher er bei dem Vortrag \enquote{zum Unitätsproblem der Physik}
\begin{quote}
Die Rotorform der elektromagnetischen Feldkomponenten
$\tensor{F}{_\mu_\nu}$, noch mehr aber das
unverkennbare formale Entsprechen im Bau der Gravitations- 
und der elektromagnetischen Gleichungen fordern förmlich
die Vermutung heraus, die Feldkomponenten könnten so etwas wie
verstümmelte Dreizeigergrößen [Christoffelsymbole] sein.
\end{quote}
Er bezieht sich dabei darauf das die Gestalt der Christoffelsymbole erster Art
\begin{equation}
\tensor*{\Gamma}{_\lambda_\mu_\nu}=\frac{1}{2}
\left(\tensor{\partial}{_\nu} \tensor{g}{_\lambda_\mu} +
\tensor{\partial}{_\mu}
\tensor{g}{_\lambda_\nu} - \tensor{\partial}{_\lambda} 
\tensor{g}{_\mu_\nu}\right)\,,
\end{equation}
welche formal identisch zu 
\begin{equation}
\tensor{F}{_\mu_\nu}=\partial_\mu\tensor{A}{_\nu}-\partial_\nu\tensor{A}{_\mu}\,.
\end{equation}
wären, falls einer der Terme vom Typ $\tensor{\partial}{_\nu}
\tensor{g}{_\lambda_\mu}$ wegfiele.

Um die Gravitation mit der Elektrodynamik zu vereinigen werden aber zunächst 
zusätzliche Freiheitsgrade benötigt, die bereits in $\tensor{g}{_\mu_\nu}$
enthaltenen sind für die Gravitation verbraucht.
Statt des üblichen Minkowski Raums $\Reals^4$ (bzw. einer entsprechenden
Riemannschen Manigfaltigkeit), führen wir eine zusätzliche fünfte Raumdimension
ein sodass wir lokal homöomorph zu $\Reals^5$ sind. 
Die neue Koordinate benennen wir $\tensor{x}{^4}$\footnote{Kaluza
bezeichnet die Zusatzkomponente mit $x^0$, was bei uns für die Zeitkomponente
vorbehalten ist.}.
Dadurch stehen zusätzlich fünf
unabhängige Freiheitsgrade $\tensor{\hat{g}}{_4_i}$ zur Verfügung. Da eine
fünften Raumdimension nicht beobachtet wird, fordert Kaluza zusätzlich das die
Metrik nicht von der neuen Koordinate abhängt
\begin{equation}
\tensor{\partial}{_4} \tensor{\hat{g}}{_i_j}=0\,,\label{eq:Zylinderbed}
\end{equation}
die so genannte \emph{Zylinderbedingung}.
%TODO Erklären wieso die so heißt
Es mag jetzt zunächst so erscheinen als ob die Zusatzdimension dadurch überhaupt
keinen Einfluss mehr auf die Physik hätte. Das dem nicht so ist zeigt sich im
Folgenden.

Kommen wir zu den Christoffelsymbolen zurück, die 
Ausgangspunkt unserer Überlegungen waren. In fünf Dimensionen haben diese die
Gestalt
\begin{equation}
\tensor*{\hat{\Gamma}}{_i_j_k}=\frac{1}{2}
\left(\tensor{\partial}{_i}
\tensor{\hat{g}}{_k_j}+\tensor{\partial}{_j} \tensor{\hat{g}}{_k_i} -
\tensor{\partial}{_k} \tensor{\hat{g}}{_i_j}\right)\, .
\end{equation}
% \begin{equation}
% \tensor*{\Gamma}{_\lambda_\mu_\nu}=\frac{1}{2}
% \left(\tensor{\partial}{_\lambda} \tensor{g}{_\mu_\nu} + \tensor{\partial}{_\mu} \tensor{g}{_\lambda_\nu} -
% \tensor{\partial}{_\nu} \tensor{g}{_\lambda_\mu}\right)
% \end{equation}
Mit der Zylinderbedingung \eqref{eq:Zylinderbed} findet man 
\begin{equation}
\begin{split}
\tensor*{\hat{\Gamma}}{_4_\mu_\nu}&=\frac{1}{2}
	\left(\tensor{\partial}{_4}\tensor{\hat{g}}{_\mu_\nu} 
+ 	\tensor{\partial}{_\mu}\tensor{\hat{g}}{_\nu_4} 
- 	\tensor{\partial}{_\nu}\tensor{\hat{g}}{_4_\mu}\right)\\
&=\frac{1}{2}
\left(\tensor{\partial}{_\mu}
\tensor{\hat{g}}{_4_\nu} - \tensor{\partial}{_\nu}
\tensor{\hat{g}}{_4_\mu}\right)\,,
\end{split}
\end{equation}
also tatsächlich die gewünschte \enquote{Rotorform}. 
Für die verbleibenden unabhängigen Komponenten ergibt sich analog
\begin{align}
\tensor*{\hat{\Gamma}}{_\mu_\nu_4}&=\frac{1}{2}
\left(\tensor{\partial}{_\mu}
\tensor{\hat{g}}{_4_\nu}+\tensor{\partial}{_\nu}
\tensor{\hat{g}}{_4_\mu}\right)\,,\\
\tensor*{\hat{\Gamma}}{_4_\mu_4}&=-\tensor*{\hat{\Gamma}}{_\mu_4_4}=\frac{1}{2}\tensor{\partial}{_\mu}
\tensor{\hat{g}}{_4_4}\,,\\
\tensor*{\hat{\Gamma}}{_4_4_4}&= 0\,.
\end{align}
Es liegt nahe die neue
Feldkomponente
$\tensor{\hat{g}}{_4_\mu}$ mit
dem Viererpotential $ $ zu verknüpfen
\begin{equation}
\tensor{\hat{g}}{_4_\mu}=2\alpha\tensor{A}{_\mu}\,.
\end{equation}
Die Konstante $\alpha$ wird dabei erst später angepasst.
 Weiter benötigen wir neben der antisymmetrisieren
Ableitung $\tensor{F}{_\mu_\nu}=\tensor{\partial}{_{[\mu}}\tensor{A}{_{\nu]}}$, auch die
symmetrisierte Form $\tensor{\partial}{_{(\mu}}\tensor{A}{_{\nu)}}$, um alle
auftretenden Terme abzudecken. Dazu führen wir das \emph{Nebenfeld}
\begin{equation}
\tensor{H}{_\mu_\nu}:=\tensor{\partial}{_{(\mu}}\tensor{A}{_{\nu)}}=\partial_\mu\tensor{A}{_\nu}+\partial_\nu\tensor{A}{_\mu}
\end{equation}
ein. Zuletzt identifizieren wir die Größe $\tensor{\hat{g}}{_4_4}$  mit einem
Skalar\footnote{Bei Kaluza als \emph{Eckpotential} bezeichnet.}
\begin{equation}
\phi:=\frac{1}{2}\tensor{\hat{g}}{_4_4}\,.
\end{equation}
Die unabhängigen Christoffelsymbole sind damit von der Gestalt
% \begin{align}
% \tensor*{\hat{\Gamma}}{_\lambda_\mu_\nu}&=\tensor*{\Gamma}{_\lambda_\mu_\nu}\,,\\
% \tensor*{\hat{\Gamma}}{_\mu_\nu_4}&=-\alpha \tensor{H}{_\mu_\nu}\, ,\\
% \tensor*{\hat{\Gamma}}{_4_\mu_\nu}&=\alpha\tensor{F}{_\mu_\nu}\,,\\
% \tensor*{\hat{\Gamma}}{_4_4_\mu}&=-\tensor*{\hat{\Gamma}}{_4_\mu_4}
% =\tensor{\partial}{_\mu}\phi\,.
% \end{align}
% \begin{equation}
%     \tensor*{\hat{\Gamma}}{_\lambda_\mu_\nu}=\tensor*{\Gamma}{_\lambda_\mu_\nu}\,,
%     \quad\tensor*{\hat{\Gamma}}{_\mu_\nu_4}=-\alpha \tensor{H}{_\mu_\nu}\,,
%     \quad\tensor*{\hat{\Gamma}}{_4_\mu_\nu}=\alpha\tensor{F}{_\mu_\nu}\,,
%    \quad\tensor*{\hat{\Gamma}}{_4_4_\mu}=-\tensor*{\hat{\Gamma}}{_4_\mu_4}=\tensor{\partial}{_\mu}\phi\,.
% \end{equation}
\begin{equation}
  \begin{alignedat}{2}
    \tensor*{\hat{\Gamma}}{_\lambda_\mu_\nu}&=\tensor*{\Gamma}{_\lambda_\mu_\nu}\,,
    & \qquad \tensor*{\hat{\Gamma}}{_4_\mu_\nu}&=\alpha\tensor{F}{_\mu_\nu},\\
    \tensor*{\hat{\Gamma}}{_\mu_\nu_4}&=-\alpha \tensor{H}{_\mu_\nu}\,,&
    \qquad\quad\tensor*{\hat{\Gamma}}{_4_4_\mu}&=-\tensor*{\hat{\Gamma}}{_4_\mu_4}=\tensor{\partial}{_\mu}\phi\,.
  \end{alignedat}
\end{equation}
\begin{lemma}
\label{lemma:Christrel}
Die Christoffelsymbole erfüllen die Relation
\begin{equation}
\pdif{_m}\left(\tensor*{\hat{\Gamma}}{_i_k_\ell}+\tensor*{\hat{\Gamma}}{_k_\ell_i}+\tensor*{\hat{\Gamma}}{_\ell_i_k}\right)
=\tensor*{\hat{\Gamma}}{_m_i_k_{,\ell}}+\tensor*{\hat{\Gamma}}{_m_k_\ell_{,i}}+\tensor*{\hat{\Gamma}}{_m_\ell_i_{,k}}\,.
\end{equation}
\end{lemma}
\begin{proof}
\begin{equation*}
\begin{split}
\pdif{_m}\left(\tensor*{\hat{\Gamma}}{_i_k_\ell}+\tensor*{\hat{\Gamma}}{_k_\ell_i}+\tensor*{\hat{\Gamma}}{_\ell_i_k}\right)
&=\tensor*{\hat{\Gamma}}{_i_k_\ell_{,m}}+\tensor*{\hat{\Gamma}}{_k_\ell_i_{,m}}+\tensor*{\hat{\Gamma}}{_\ell_i_k_{,m}}\\
&=\frac{1}{2}
\left( \tensor{\hat{g}}{_\ell_i_{,km}} + \tensor{\hat{g}}{_\ell_k_{,im}}
-\tensor{\hat{g}}{_i_k_{,\ell m}}\right)\\
&\phantom{=}+\frac{1}{2}\left( \tensor{\hat{g}}{_i_\ell_{,km}} +
\tensor{\hat{g}}{_i_k_{,\ell m}} -\tensor{\hat{g}}{_k_\ell_{,im}}\right)\\
&\phantom{=}+\frac{1}{2}\left( \tensor{\hat{g}}{_k_\ell_{,im}} +
\tensor{\hat{g}}{_k_i_{,\ell m}} -\tensor{\hat{g}}{_i_\ell_{,km}}\right)\\
&=\frac{1}{2}
\left( \tensor{\hat{g}}{_\ell_i_{,km}} + \tensor{\hat{g}}{_\ell_k_{,im}}
+\tensor{\hat{g}}{_i_k_{,\ell m}}\right)\\
\end{split}
\end{equation*}
Durch Nulladdition von $\tensor{\hat{g}}{_k_m_{,i\ell}}$, $
\tensor{\hat{g}}{_i_m_{,\ell k}}$ und
$\tensor{\hat{g}}{_i_m_{,\ell k}}$ erhält man
\begin{equation}
\begin{split}
\pdif{_m}\left(\tensor*{\hat{\Gamma}}{_i_k_\ell}+\tensor*{\hat{\Gamma}}{_k_\ell_i}+\tensor*{\hat{\Gamma}}{_\ell_i_k}\right)
&=\frac{1}{2}
\left( \tensor{\hat{g}}{_k_m_{,i\ell}} + \tensor{\hat{g}}{_k_i_{,m\ell}}
-\tensor{\hat{g}}{_m_i_{,k\ell}}\right)\\
&\phantom{=}+\frac{1}{2}\left( \tensor{\hat{g}}{_\ell_m_{,ki}} +
\tensor{\hat{g}}{_\ell_k_{,mi}} -\tensor{\hat{g}}{_m_k_{,\ell i}}\right)\\
&\phantom{=}+\frac{1}{2}\left( \tensor{\hat{g}}{_i_m_{,\ell k}} +
\tensor{\hat{g}}{_i_\ell_{,mk}} -\tensor{\hat{g}}{_m_\ell_{,ik}}\right)\\
&=\tensor*{\hat{\Gamma}}{_m_i_k_{,\ell}}+\tensor*{\hat{\Gamma}}{_m_k_\ell_{,i}}+\tensor*{\hat{\Gamma}}{_m_\ell_i_{,k}}
\,. \qedhere
\end{split}
\end{equation}
\end{proof}
Lemma~\ref{lemma:Christrel} impliziert dann für $m=4$, zusammen mit der
Zylinderbedingung
\begin{equation}
0=\tensor*{\hat{\Gamma}}{_4_i_k_{,\ell}}+\tensor*{\hat{\Gamma}}{_4_k_\ell_{,i}}+\tensor*{\hat{\Gamma}}{_4_\ell_i_{,k}}\,.
\label{eq:fdchristrel}
\end{equation}
Setzt man $\ell = \lambda$, $i=\nu$, $k = \mu$, erhält man weiter 
\begin{equation}
\begin{split}
0&=\tensor*{\hat{\Gamma}}{_4_\nu_\mu_{,\lambda}}
+\tensor*{\hat{\Gamma}}{_4_\mu_\lambda_{,\nu}}
+\tensor*{\hat{\Gamma}}{_4_\lambda_\nu_{,\mu}}\\
&=\alpha\left(\tensor{F}{_\nu_\mu_{,\lambda}}
+\tensor{F}{_\mu_\lambda_{,\nu}}
+\tensor{F}{_\lambda_\nu_{,\mu}}\right)\,,
\end{split}
\end{equation}
die homogenen Maxwell-Gleichungen \eqref{eq:MaxwellHom}. Alle weiteren
Relationen, welche sich aus \eqref{eq:fdchristrel} ableiten lassen sind
trivial
\footnote{Zum einen erhält man falls zwei Indizes gleich vier sind
$0=\tensor{\phi}{_{,\nu\mu}}-\tensor{\phi}{_{,\mu\nu}}$, was aber nach dem
Satz von Schwarz für beliebige glatte $\phi$ erfüllt ist. Falls alle
Indizes gleich vier sind, verschwindet auch die rechte Seite der Gleichung.}.
%für
% $\ell = 4$, $i=\nu$, $k = \mu$ erhält man weiter
% \begin{equation}
% \begin{split}
% 0&=\tensor*{\hat{\Gamma}}{_4_\nu_\mu_{,4}}
% +\tensor*{\hat{\Gamma}}{_4_\mu_4_{,\nu}}
% +\tensor*{\hat{\Gamma}}{_4_4_\nu_{,\mu}}\\
% &=\tensor{\phi}{_{,\nu\mu}}-\tensor{\phi}{_{,\mu\nu}}
% \end{split}
% \end{equation}
% Was nach dem Satz von Schwartz erfüllt ist. Wenn mindestens zwei Indizes 4 sind
% ist die Gleichung mit der Zylinderbedingung ref erfüllt.

Wir wollen nun weiter überprüfen, welche Gleichungen sich aus den
fünfdimensionalen Analogien der Feld- und Geodätengleichungen ergeben.
Kaluza führte seine Berechnungen in der linearisierten Theorie durch, d.h. 
die auftretenden Energien sind klein.
%TODO wie klein
In der linearisierten Theorie gilt insbesondere
\begin{align}
\Gamma^2&\approx 0\tag{N1}\label{eq:N1}\,,\\
R&\approx 0\tag{N2}\label{eq:N2}\,,
\end{align}
wobei $\Gamma^2$ einen beliebigen, quadratischen Ausdruck in den
Christoffelsymbolen bezeichne.
Zur Berechnung des Ricci Tensors
$\tensor{\hat{R}}{_i_j}$ ist die folgende Formel nützlich:
\begin{equation}
\tensor{\hat{R}}{_i_j}=\tensor{\partial}{_l}\tensor*{\hat{\Gamma}}{^l_i_j}
-\tensor*{\hat{\Gamma}}{^m_i_l}\tensor*{\hat{\Gamma}}{^l_j_m}
-\tensor{\nabla}{_j}\left[\tensor{\partial}{_i}\left(\log\sqrt{-\hat{g}}\right)\right]\,
.\end{equation}
Insbesondere gilt aufgrund der Zylinderbedingung und der Näherungen
\eqref{eq:N1}`
%TODO wie schauts mit RNKS aus sind die resultierenden gleichungen tensoriell?
\begin{equation}
\tensor{\hat{R}}{_4_j}=\tensor{\partial}{_\ell}\tensor*{\hat{\Gamma}}{^\ell_4_j}
=\tensor{\partial}{^\ell}\tensor*{\hat{\Gamma}}{_4_j_\ell}
=\tensor{\partial}{^\lambda}\tensor*{\hat{\Gamma}}{_4_j_\lambda}\,.
\end{equation}
% \begin{equation}
% \tensor{\hat{R}}{_4_j}=\tensor{\partial}{_\lambda}\tensor*{\hat{\Gamma}}{^\lambda_4_j}
% -\tensor*{\hat{\Gamma}}{^m_4_l}\tensor*{\hat{\Gamma}}{^l_j_m}\,
% .\end{equation}
Damit berechnen wir für die zusätzlichen Komponenten von $\tensor{\hat{R}}{_i_j}$ 
% \begin{equation}
% \begin{split}
% \tensor{\hat{R}}{_4_4}
% &=\tensor{\partial}{_\lambda}\tensor*{\hat{\Gamma}}{^\lambda_4_4}
% -\tensor*{\hat{\Gamma}}{^m_4_l}\tensor*{\hat{\Gamma}}{^l_4_m}\\
% &=\tensor{\partial}{_\lambda}\tensor*{\hat{\Gamma}}{^\lambda_4_4}
% -\tensor*{\hat{\Gamma}}{^\mu_4_\lambda}\tensor*{\hat{\Gamma}}{^\lambda_4_\mu}
% -\tensor*{\hat{\Gamma}}{^\mu_4_4}\tensor*{\hat{\Gamma}}{^4_4_\mu}
% -\tensor*{\hat{\Gamma}}{^4_4_\lambda}\tensor*{\hat{\Gamma}}{^\lambda_4_4}
% \\
% &=-\tensor{\partial}{_\lambda}\tensor{\partial}{^\lambda}\phi
% -\alpha^2\tensor{F}{^\mu_\lambda}\tensor{F}{^\lambda_\mu}
% +\tensor{\partial}{^\mu}\phi\tensor{\partial}{_\mu}\phi
% +\tensor{\partial}{_\lambda}\phi\tensor{\partial}{^\lambda}\phi\\
% &=-\alpha^2\tensor{F}{_\mu_\nu}\tensor{F}{^\mu^\nu}-\square\phi
% +2\tensor{\partial}{^\mu}\phi\tensor{\partial}{_\mu}\phi
% \end{split}
% \end{equation}
% \begin{equation}
% \begin{split}
% \tensor{\hat{R}}{_4_\nu}
% &=\tensor{\partial}{_\lambda}\tensor*{\hat{\Gamma}}{^\lambda_4_\nu}
% -\tensor*{\hat{\Gamma}}{^m_4_l}\tensor*{\hat{\Gamma}}{^l_\nu_m}\\
% &=\tensor{\partial}{_\lambda}\tensor*{\hat{\Gamma}}{^\lambda_4_\nu}
% -\tensor*{\hat{\Gamma}}{^\mu_4_\lambda}\tensor*{\hat{\Gamma}}{^\lambda_\nu_\mu}
% -\tensor*{\hat{\Gamma}}{^\mu_4_4}\tensor*{\hat{\Gamma}}{^4_\nu_\mu}
% -\tensor*{\hat{\Gamma}}{^4_4_\lambda}\tensor*{\hat{\Gamma}}{^\lambda_\nu_4}\\
% &=-\alpha\tensor{\partial}{_\lambda}\tensor{F}{^\lambda_\nu}
% +\alpha\tensor{F}{^\mu_\lambda}\tensor*{\Gamma}{^\lambda_\nu_\mu}
% +\alpha\tensor{\partial}{^\mu}\phi\tensor{H}{_\nu_\mu}
% -\alpha\tensor{\partial}{_\lambda}\phi\tensor{F}{^\lambda_\nu}\\
% &=-\alpha\left(\tensor{\partial}{^\mu}\tensor{F}{_\mu_\nu}
% +\tensor*{\Gamma}{_\lambda_\mu_\nu}\tensor{F}{^\mu^\lambda}
% +2\tensor{\partial}{^\mu}\phi\tensor{\partial}{_\nu}\tensor{A}{_\mu}\right)
% \end{split}
% \end{equation}
\begin{equation}
\tensor{\hat{R}}{_4_4}
=\tensor{\partial}{^\lambda}\tensor*{\hat{\Gamma}}{_4_4_\lambda}
=\tensor{\partial}{^\lambda}\tensor{\partial}{_\lambda}\phi=\square\phi\,, 
\end{equation}
für die 44 Komponente, sowie 
\begin{equation}
\tensor{\hat{R}}{_4_\nu}
=\tensor{\partial}{^\lambda}\tensor*{\hat{\Gamma}}{_\lambda_4_\nu}
=-\alpha\tensor{\partial}{^\lambda}\tensor{F}{_\lambda_\nu}
=:-\alpha\tensor{J}{_\nu}\,.
\end{equation}
Trivialerweise gilt zudem $\tensor{\hat{R}}{_\mu_\nu}=\tensor{R}{_\mu_\nu}$. Die
, auf 5 Dimensionen verallgemeinerten, Feldgleichungen sind unter Näherung
\eqref{eq:N2} und mit Annahme verschwindender Kosmologischer Konstante
\begin{equation}
\kappa\tensor{\hat{T}}{_i_j}=\tensor{\hat{R}}{_i_j}\,.
\end{equation}
Damit ergeben sich zusätzlich zu den vierdimensionalen Einsteinschen
Feldgleichungen
\begin{equation}
\kappa\tensor{\hat{T}}{_4_4}=-\square\phi, \quad
\kappa\tensor{\hat{T}}{_4_\mu}=-\alpha\tensor{J}{_\mu}\,.\label{eq:ZusatzFG}
\end{equation}
Die $\tensor{\hat{T}}{_4_4}$ Komponente entspricht also der Masse des Feldes
$\phi$, $\tensor{\hat{T}}{_4_\mu}$ ist proportional zur Stromdichte.
 Wir definieren die Fünfergeschwindigkeit $\tensor{\hat{u}}{_i}$
\begin{equation}
\tensor{\hat{u}}{_i}:=\dod{\tensor{x}{_i}}{\hat{\tau}}\,,
\quad\dif\hat{\tau}^2=\tensor{\hat{g}}{_i_j}\dif\tensor{x}{^i}\dif\tensor{x}{^j}\,.
\end{equation}
Im Weiteren machen wir zusätzlich die Näherung, dass alle
Geschwindigkeiten\footnote{Im Fall von $\tensor{u}{_4}$ spricht man
besser von kleiner spezifischer Ladung.} klein sind, sprich
\begin{equation}
\tensor{\hat{u}}{_1},\tensor{\hat{u}}{_2},\tensor{\hat{u}}{_3},\tensor{\hat{u}}{_4}\ll 1
\,,\quad\tensor{\hat{u}}{_0}\approx 1\, .\tag{N3}\label{eq:N3}
\end{equation}
Wir beschreiben unser System durch Staub, d.h. wir vernachlässigen den Druck. 
Den zugehörigen Energie-Impuls-Tensor $\tensor{T}{_\mu_\nu}$
verallgemeinern wir auf fünf Dimensionen
\begin{equation}
\tensor{\hat{T}}{_i_j}=\mu_0\tensor{\hat{u}}{_i}\tensor{\hat{u}}{_j}\, ,
\end{equation}
wobei $\mu_0$ die Ruhemassendichte des Staubs bezeichnet. Für die
Zusatzkomponenten findet man 
\begin{equation}
\tensor{\hat{T}}{_4_\mu}
=\mu_0\tensor{\hat{u}}{_4}\tensor{\hat{u}}{_\mu}\approx\mu_0\tensor{\hat{u}}{_4}\tensor{u}{_\mu}\,,
\label{eq:StaubEMTens}
\end{equation}
da wegen \eqref{eq:N3} auch $\dif\hat{\tau}\approx\dif\tau$ gilt.
Dabei ist $\tensor{u}{_\mu}$ die gewöhnliche Vierergeschwindigkeit.
Setzt man jetzt \eqref{eq:ZusatzFG} und \eqref{eq:StaubEMTens} gleich so erhält
man
\begin{equation}
-\alpha\tensor{J}{_\mu}\approx\kappa\mu_0\tensor{\hat{u}}{_4}\tensor{u}{_\mu}
\,,
\end{equation}
und damit mit $\tensor{J}{_\mu}=\varrho_0\tensor{u}{_\mu}$
\begin{equation}
\varrho_0=-\frac{\kappa\mu_0}{\alpha}\tensor{\hat{u}}{_4}\,.\label{eq:murhorel}
\end{equation}
Wenden wir uns den Geodäten zu. Die verallgemeinerte Geodätengleichung
lautet
\begin{equation}
\od{\tensor{\hat{u}}{^\ell}}{\hat{\tau}}=-\tensor*{\hat{\Gamma}}{^\ell_m_n}\tensor{\hat{u}}{^m}\tensor{\hat{u}}{^n}
\, ,
\end{equation}
Da wir uns für die Bahnen in den uns zugänglichen vier Raumzeitdimensionen
interessieren ist, sind natürlich vor allem diese Komponenten interessant.
Hierbei gilt
\begin{equation}
\begin{split}
\dod{\tensor{\hat{u}}{^\lambda}}{\hat{\tau}}
+\tensor*{\hat{\Gamma}}{^\lambda_\mu_\nu}\tensor{\hat{u}}{^\mu}\tensor{\hat{u}}{^\nu}
&=
-\tensor*{\hat{\Gamma}}{^\lambda_m_n}\tensor{\hat{u}}{^m}\tensor{\hat{u}}{^n}
+\tensor*{\hat{\Gamma}}{^\lambda_\mu_\nu}\tensor{\hat{u}}{^\mu}\tensor{\hat{u}}{^\nu}\\
&=
-\tensor*{\hat{\Gamma}}{^\lambda_4_4}\tensor{\hat{u}}{^4}\tensor{\hat{u}}{^4}
-\tensor*{\hat{\Gamma}}{^\lambda_4_\nu}\tensor{\hat{u}}{^4}\tensor{\hat{u}}{^\nu}
-\tensor*{\hat{\Gamma}}{^\lambda_\mu_4}\tensor{\hat{u}}{^\mu}\tensor{\hat{u}}{^4}
-\tensor*{\hat{\Gamma}}{^\lambda_\mu_\nu}\tensor{\hat{u}}{^\mu}\tensor{\hat{u}}{^\nu}
+\tensor*{\hat{\Gamma}}{^\lambda_\mu_\nu}\tensor{\hat{u}}{^\mu}\tensor{\hat{u}}{^\nu}
\\
&=
\tensor{\partial}{_\lambda}\phi\left(\tensor{\hat{u}}{^4}\right)^2
+\alpha\tensor{F}{^\lambda_\nu}\tensor{\hat{u}}{^4}\tensor{\hat{u}}{^\nu}
+\alpha\tensor{F}{^\lambda_\mu}\tensor{\hat{u}}{^\mu}\tensor{\hat{u}}{^4}\\
&=
\tensor{\partial}{_\lambda}\phi\left(\tensor{\hat{u}}{^4}\right)^2
+2\alpha\tensor{F}{^\lambda_\nu}\tensor{\hat{u}}{^4}\tensor{\hat{u}}{^\nu}\,.
\end{split}
\end{equation}
Nach Voraussetzung ist $\tensor{\hat{u}}{^4}$ klein und damit
näherungsweise
\begin{equation}
\dod{\tensor{u}{^\lambda}}{\tau}
+\tensor*{\Gamma}{^\lambda_\mu_\nu}\tensor{u}{^\mu}\tensor{u}{^\nu}
=2\alpha\tensor{F}{^\lambda^\nu}\tensor{\hat{u}}{^4}\tensor{\hat{u}}{_\nu}\,.
\label{eq:KaluzaGeo}
\end{equation}
Damit sind wir am Ziel, denn der Parameter $\alpha$ ist weiterhin eine
freie Größe, wir setzen 
 $\alpha=\sqrt{\frac{\kappa}{2}}\approx 3,06\cdot 10^{-14}$ und
erhalten vermöge \eqref{eq:murhorel}
\begin{equation}
2\alpha\tensor{\hat{u}}{^4}=-
\frac{\varrho_0}{\mu_0}\,.
\end{equation}
Die Tatsache, dass $\alpha$ klein ist, legitimiert
die zuvor gemachten Näherungen im Nachhinein, denn Terme die quadratisch in den
Christoffelsymbolen sind sind auch quadratisch in $\alpha$.
Setzt man nun in \eqref{eq:KaluzaGeo} ein, so erhält man
\begin{equation}
\dod{\tensor{u}{^\lambda}}{\tau}
+\tensor*{\Gamma}{^\lambda_\mu_\nu}\tensor{u}{^\mu}\tensor{u}{^\nu}
=-\frac{\varrho_0}{\mu_0}\tensor{F}{^\lambda^\nu}\tensor{u}{_\nu}\, ,
\end{equation}
die Formel für die Geodäten mit Massendichte $\mu_0$ und Ladungsdichte
$\varrho_0$, unter Einwirkung der Lorentzkraft! Es verbleibt eine
Untersuchung der 4. Komponente
\begin{equation}
\begin{split}
\dod{\tensor{\hat{u}}{^4}}{\hat{\tau}}
&=
-\tensor*{\hat{\Gamma}}{^4_m_n}\tensor{\hat{u}}{^m}\tensor{\hat{u}}{^n}\\
&=
-\tensor*{\hat{\Gamma}}{^4_4_\nu}\tensor{\hat{u}}{^4}\tensor{\hat{u}}{^\nu}
-\tensor*{\hat{\Gamma}}{^4_\mu_4}\tensor{\hat{u}}{^\mu}\tensor{\hat{u}}{^4}
-\tensor*{\hat{\Gamma}}{^4_\mu_\nu}\tensor{\hat{u}}{^\mu}\tensor{\hat{u}}{^\nu}\\
&=
-\tensor*{\hat{\Gamma}}{^4_4_\nu}\tensor{\hat{u}}{^4}\tensor{\hat{u}}{^\nu}
-\tensor*{\hat{\Gamma}}{^4_\mu_4}\tensor{\hat{u}}{^\mu}\tensor{\hat{u}}{^4}
-\alpha\tensor{H}{_\mu_\nu}\tensor{\hat{u}}{^\mu}\tensor{\hat{u}}{^\nu}\\
&\approx
-\alpha\tensor{H}{_0_0}\\
&=
-2\alpha\tensor{A}{_0}\\
\end{split}
\end{equation}
aufgrund der Näherung ist $\dod{\tensor{\hat{u}}{^4}}{\hat{\tau}}\approx 0$ und
damit $\alpha$ konstant
\begin{equation}
\begin{split}
\tensor{\partial}{_0}\left(\frac{\varrho_0}{\mu_0}\right)
=-2\alpha\tensor{\partial}{_0}u_4=
4\alpha^2\tensor{A}{_0}
\end{split}
\end{equation}
\begin{bemerkung}
In den physikalisch relevanten Gleichungen taucht das Nebenfeld
$\tensor{H}{_\mu_\nu}$ nicht mehr auf. Eine Abhängigkeit der Gleichungen stünde
allerdings auch im Widerspruch zur Eichinvarianz.
\end{bemerkung}
\begin{bemerkung}
Wir verzichten darauf Indices mit der Metrik zu heben bzw. zu senken, da dies
nicht Wohldefiniert ist:
\begin{equation}
\tensor*{\delta}{*_\mu^\nu}=\tensor{\hat{g}}{_\mu_a}\tensor{\hat{g}}{^a^\nu}
=\tensor{g}{_\mu_\sigma}\tensor{g}{^\sigma^\nu}+\tensor{\hat{g}}{_\mu_5}\tensor{\hat{g}}{^5^\nu}
\end{equation}
\begin{equation}
\tensor{g}{_\mu_\sigma}\tensor{g}{^\sigma^\nu}
=\tensor*{\delta}{*_\mu^\nu}-\tensor{\hat{g}}{_\mu_5}\tensor{\hat{g}}{^5^\nu}\neq\tensor*{\delta}{*_\mu^\nu}
\end{equation}
\end{bemerkung}
\subsection{Erfolg}
Die Ableitung Kaluzas hat Charme. Insbesondere, dass er die
Lorentzkraft den fünfdimensionalen Einsteingleichungen folgern kann ist sehr 
befriedigend, da weniger Annahmen gemacht werden müssen als in der klassischen
Theorie. Die Annahme das die Spezifische Ladung klein ist ist schon für relativ
generische Objekte wie beispielsweise ein Elektron verletzt: die spezifische
Elementarladung, d.h. das Verhältnis eines Elektrons zu seiner Masse, beträgt nämlich
$\frac{e}{m}\approx1.76\cdot10^{11}\unitfrac{C}{kg}$.
Letztlich ist die Theorie damit mehr ein Spezialfall für "`schwach"' geladene
Objekte. Auch ist das Prozedere sehr konstruktiv und an vielen Stellen müssen
Näherungen gemacht werden.
Insgesamt ist durch Kaluzas Arbeit allerdings der Grundstein für die
Vereinheitlichung der Kräfte gelegt. Eine modernere Formulierung die sich unter
anderem an der Arbeiten von Klein, Jordan und Thiry orientiert wird im nächsten
Kapitel vorgestellt.

%
%
% Um weitere Freiheitsgrade zu erhalten führen wir eine weitere Raumdimension ein
% (Weitere Zeitdimensionen bereiten Probleme), da nur 3
% raumdimensionen beobachtet werden muss es einen Grund dafür geben das diese
% nicht sichtbar sind. Wir erweitern die Theorie deshalb um eine zusätzliche
% \emph{kompakte} Raumdimension mit ``Ausdehnung'' auf einer uns nicht
% zugänglichen Skala ($r\sim l\textsubscript{Plank}$ = Welche Länge? = Welche
% Energie Vgl. mit zugänglicher Energieskala am LHC).
% Wir nehmen an das die kompakte Dimension durch $\Sphere^1$ beschrieben wird, die
% Raumzeitmanigfaltigkeit als lokal(?) homoömorph zu $\Reals^4\times\Sphere^1$
% ist.
% Wir nehmen an das auch in dieser 5 dimensionalen Raumzeit die
% Einsteingleichungen ihre gültigkeit behalten. Allerdings erhalten wir nun
% anstatt der 10 Gleichungen 15 also 5 zusätzliche Freiheitsgrade (???)

% \section{Kleins Ansatz}
% 
% 
% 
% \section{Kaluzas Erweiterung}
% \section{Probleme}
% \section{Faserbündel}
% \section{Yang-Mills}
% \section{Flache Raumzeit}
% Falls die 4-Raumzeit flach ist also die betrachtete Manigfaltigkeit
% $M=\Reals^4\times\Sphere^1$ (Achtung hier braucht man eigendlich Faserbündel!!!)
% ergibt sich als Metrik mit $z=\left(x,\theta\right)$
% \begin{equation}
% \tensor{g}{_i_j}\dif\tensor{z}{^i}\dif\tensor{z}{^j}
% =\tensor{\eta}{_\mu_\nu}\dif\tensor{x}{^\mu}\dif\tensor{x}{^\nu}+r^2\dif\theta^2
% \end{equation}
% Untersuchung von Störungen dieser Metrik
% \begin{equation}
% \tensor{\hat{g}}{_i_j}\dif\tensor{z}{^i}\dif\tensor{z}{^j}
% =e^{-\phi/3}\left[e^{\phi}\left(\dif\theta+\kappa\tensor{A}{_\mu}\dif\tensor{x}{^\mu}\right)^2
% +\tensor{g}{_\mu_\nu}\dif\tensor{x}{^\mu}\dif\tensor{x}{^\nu}\right]
% \end{equation}

%\begin{equation}
%\tensor{\hat{g}}{_{MN}}\dif\tensor{z}{^M}\dif\tensor{z}{^N}
%=e^{-\phi/3}\left[e^{\phi}\left(\dif\theta+\kappa\tensor{A}{_\mu}\dif\tensor{x}{^\mu}\right)^2
% +\tensor{g}{_\mu_\nu}\dif\tensor{x}{^\mu}\dif\tensor{x}{^\nu}\right]
%\end{equation}
% Entwicklung in Fourrierreihen
% \begin{align*}
% \tensor{g}{_\mu_\nu}(z)&=\sum_{n=-\infty}^\infty
% \tensor*{g}{*^{(n)}*_\mu*_\nu}\left(x\right)e^{\imI n \theta}\\
% \tensor{A}{_\mu}(z)&=\sum_{n=-\infty}^\infty
% \tensor*{A}{*^{(n)}*_\mu}\left(x\right)e^{\imI n \theta}\\
% \phi(z)&=\sum_{n=-\infty}^\infty
% \phi^{(n)}\left(x\right)e^{\imI n \theta}
% \end{align*}
% \section{Zylinderbedingung}
% \section{Nordströms Theorie} 
% Bereits vor Einstein formulierte Gunnar Nordström 1912, eine rein geometrische Theorie der Gravitation (Zusammenhang?)
% 
% TODO ist dtheta ein killing vektorfeld?