\chapter{Allgemeine Begriffe}

\section{Konventionen}
Bereits Rechnungen in der "`gewöhnlichen"' Relativitätstheorie haben die Tendenz
unübersichtlich zu werden, das Einführen zusätzlicher Dimensionen hilft dabei
natürlich kaum.
Es ist deshalb an dieser Stelle nützlich einige Konventionen festzulegen. Im
Folgenden bezeichnet $n$ die Anzahl der Zusatzdimensionen.\footnote{Im Weiteren
meist $n=1$}

Die Metrik soll stets die Signatur $(1,3)$, bzw. $(1,3+n)$, insbesondere sind
die Zusatzdimensionen. Die meisten Rechnungen werden in lokalen Koordinaten
durchgeführt.
Wir beginnen die Indizierung bei 0, wobei die 0. Komponente mit der Zeit identifiziert wird.
Griechische Indices laufen über ersten vier Raum-Zeit Koordinaten
$\mu=0,\ldots,3\,$, lateinische über alle inklusive der
Zusatzkomponenten \footnote{Verwirrung bezüglich der gängigen Konvention
mit lateinischen Indices die Raumkomponenten zu benennen sollte dabei nicht
aufkommen.}, $i=0,\ldots,3+n\,$. Weiter verwenden wir die Einsteinsche
Summenkonvention, d.h. über paare von indices wird implizit summiert.

Im Zusammenhang mit Zusatzdimensionen tauchen Größen auf, die sowohl ein 4, als
auch ein $4+n$ dimensionales Pendant besitzen. Um diese von einander zu
unterscheiden, kennzeichnen wir die $4+n$ dimensionale Version mit einem
Zirkumflex. 
Beispielsweise bezeichnen wir den $4+n$ dimensionalen Ricci-Tensor mit
$\tensor{\hat{R}}{_i_j}$. Mit $g$ bezeichnen wir die Determinante der Metrik.
\subsection*{Geometrisierte Einheiten}
Wir verwenden geometrisierte Einheiten, d.h. Einheiten in denen
$8\pi G=c=1$. Es verbleibt nur noch eine Längendimension.
\section{Differentialgeometrie}
\subsection{Vektoren und Tensoren}
Riemann-Tensor, Ricci-Tensor, Geodätische.
\subsection{Lie-Gruppen}
\subsection{Quotientenmanigfaltigkeiten}
\subsection[G-Räume]{$G$-Räume}
Gruppen können auf Mengen operieren. Ein einfaches Beispiel ist die Wirkung
der Drehgruppe $SO(2)$ auf $\Reals^2$ durch Multiplikation von
Vektoren mit Drehmatritzen. Um die Diskussion allgemeiner zu gestalten definieren wir
Operationen allgemeiner Gruppen. Die Hauptforderung an die Wirkung der Gruppe
auf der Menge ist, dass die Operation mit der Gruppenoperation kompatibel ist.
 \begin{definition}[Gruppenwirkung]
Sei $G$ eine Gruppe, $X$ eine Menge. Eine Abbildung
\begin{equation}
\phi:G\times X\to X\,,\quad (g,x)\mapsto\phi_g(x)\,\,.
\end{equation}
heißt \emph{Gruppenwirkung} von $G$ auf $X$, falls die folgende Eigenschaften
erfüllt sind
\begin{enumerate}
  \item \emph{Identität} Für das neutrale Element $e\in G$ gilt
  $\displaystyle\phi_e=\id_X$
  \item \emph{Verträglichkeit} $\displaystyle\phi_{gh}=\phi_g\circ\phi_h$
\end{enumerate}
$X$ heißt dann auch $G$-Menge.
\end{definition}
Statt $\phi_g(x)$ schreiben wir im folgenden kurz $g\gmal x$.
Ist die Gruppe $G$ eine Lie-Gruppe und $X=M$ eine Mannigfaltigkeit und $\phi_g$
 differenzierbar, so spricht man von einer \emph{Lie-Gruppenwirkuqng}. $M$ heißt dann auch $G$-Mannigfaltigkeit.
\begin{definition}[Orbit]
Sei $X$ eine $G$-Menge, $x\in X$, dann heißt die Menge
\begin{equation}
G \gmal x=\{g \gmal x\,|\,g\in G\}
\end{equation}
Orbit von $x$.
\end{definition}
Die Orbits stellen eine Partition der Menge $X$ dar, somit ist der Quotient 
$X/G$ wohldefiniert.
% Definieren
Wir wollen im folgenden Quotienten der Form $M/G$ betrachten, für
$G$-Mannigfaltigkeit $M$ betrachten. Dieser Quotient muss allerdings im
Allgemeinen keine Mannigfaltigkeit sein. 
\begin{beispiel}[Ein nicht Hausdorffer Quotient]
% Let Z/2 act on R×{0}∪R×{1} by g⋅(t,0)=(t,1) and g⋅(t,1)=(t,0) if t≠0 and
% g⋅(0,0)=(0,0) and g⋅(1,1)=1. Then the quotient space is the line with 
% two origins which is certainly not Hausdorff.
\end{beispiel}
% Bsp nicht M/G nicht Hausdorf?
Das Slice Theorem liefert, das falls $G$ kompakt und fixpunktfrei ist $M/G$ eine
Manigfaltigkeitstruktur besitzt, bzw. sogar $M\to M/G$ ein $G$-Hauptfaserbündel
ist.
% %http://math.stackexchange.com/questions/1315445/quotient-manifold-theorem-provides-a-fibrations
Da die Gruppenwirkung differenzierbar ist lasst sich auch die Orbitkarte 
\begin{equation}
\sigma_x:G\to M\,,\quad x\mapsto g \gmal x
\end{equation}
bei in der Identität differenzierbar. Man erhält so zu jedem $x\in M$ einen
Vektor insgesammt erhält man ein Vektorfeld $V$ auf $M$.
%Gefaserte Manigfaltigkeiten.
\begin{theorem}[Principal Orbit Theorem]
\end{theorem}
\section{Variationsprinzip}
Feldtheorien lassen sich meist über 
Variationsprinzipien formulieren. Dies gilt insbesondere für
Elektromagnetismus und allgemeine Relativitätstheorie.
\subsection{Lagrange Dichte}
Wald\cite{wald2010general} S.454 ff.
\begin{definition}[Tensordichte]
Eine Tensordichte $\mathcal{T}$ vom Gewicht $w$
\end{definition}
Wirkungsintegral, Lagrangedichte, Lokalität
\begin{beispiel}[Levi-Civita-Symbol]
\end{beispiel}
\begin{definition}[Variationsableitung]
% es wird nach einem Feld gesucht nicht nach einem "`optimalen"' Weg, quasi
% unendlichdimensionale Form
% https://en.wikipedia.org/wiki/Lagrangian_system
\end{definition}
\subsubsection{Skalarfelder (Spin-0)}
\begin{equation}
\mathcal{L}\left[\phi(x),\nabla_\alpha\phi(x),x\right]=
\end{equation}
\subsubsection{Vektorfelder (Spin-1)}
\begin{equation}
\mathcal{L}\left[A_\alpha(x),\nabla_\beta
A_\alpha(x),x\right]=-\frac{1}{4}\tensor{F}{_\mu_\nu}\tensor{F}{^\mu^\nu}\,.
\end{equation}
%https://en.wikipedia.org/wiki/Euler%E2%80%93Lagrange_equation
%https://en.wikipedia.org/wiki/Functional_derivative#cite_note-2

% Fibred manifold vs principal bundle (spezialfall\ldots)
% In topology, the words fiber (Faser in German) and fiber space (gefaserter Raum)
% appeared for the first time in a paper by Seifert in 1932
% Seifert, H. (1932). "Topologie dreidimensionaler geschlossener Räume". Acta
% Math. (in French) 60: 147–238. doi:10.1007/bf02398271.
