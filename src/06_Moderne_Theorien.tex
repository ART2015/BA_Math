\chapter{Im Kontext moderner Theorien}
Yang-Mills
%Duff\footnote{M. J. Duff, GR11 Konferenz, Stockholm 1989.}
%\begin{quote}
% Kaluza-Klein is dead, long
% live Kaluza-Klein!
%\end{quote}
% Sowohl Kaluza als auch klein setzten $\phi=const.$, als zusätzliches feld
% interpretiert zunächst durch Jordan und Thiery.
% 
% Probleme: Fermionmassen zu groß . quantized mass
% Besides that, E. Witten has
% proved the so-called Witten no-go theorem, telling that the Kaluza-Klein(like)
% theories have severe difculties obtaining massless fermions chirally coupled
% to the Kaluza-Klein-type gauge elds in (1+3) dimension, as required by the
% standard model.
% 
% M = M/Sx which inherits a metric g since the bilinear form gjj defined on M by
% gh = g - a otimes a satisfìes Cv9h = 0. Of course, if p denotes the projection
% from M to M, gn = p*g.
% As a result, one can view M as an S1-principal bundle over M on which a is a
% connection form.
\section{b}
Wir beginnen mit einer $n+4$ dimensionalen $G$ Manigfaltikeit $E$. 
Diese stellt ein Modell für die Umgebende höherdimensionale Raumzeit dar. Die
$4$ dimensionale Raumzeit sei durch eine Manigfaltikeit $M$ gegeben. Lokal sieht
$E$ aus wie das Produkt $M\times G$. $M$ ist gegeben durch $E/G$ .$G.u$  die
Typische Fasern des Faserbündels $E$. Eichtransformationen werden durch
Abbildungen $\beta:G\to G$ induziert. Eine Eichung Entspricht einem Schnitt durch die Faser.
