\chapter{Maxwell-Einstein Theorie \label{chap:EMW}}
Dieser Teil soll die physikalischen Theorien vorstellen, die der
Kaluza-Klein-Theorie zu Grunde liegen. Dazu zählt neben der
in ihrer heutigen Formulierung auf Maxwell zurückgehenden
Elektrodynamik, Einsteins allgemeine Relativitätstheorie.
Deutlich älter ist die Elektrodynamik, deren fundamentales Gesetz, die Konstanz
der Lichtgeschwindigkeit, die Grundlage der speziellen Relativitätstheorie
darstellt.
Die Ähnlichkeit beider Theorien lässt hoffen, dass eine
übergeordnete Theorie beide als Spezialfälle enthält.
\section{Elektrodynamik}
Die Elektrodynamik beschreibt das Wechselspiel von elektrischen und
magnetischen Feldern. Die Phänomenologie lässt sich mittels zweier glatter Vektorfelder
${\vec{E},\vec{B}:\Reals^3\times\Reals\to \Reals^3}$ beschreiben.
Diese Felder erfüllen die auf Maxwell zurückgehenden Gleichungen
\begin{equation}
\begin{alignedat}{2}
\Div\vec{B} &= 0    \,,  & \qquad \Div\vec{E} &= \rho
\,,\\
\Rot\vec{E}+\dpd{\vec{B}}{t}&=0\,,& \qquad\Rot\vec{B}-\dpd{\vec{E}}{t}&
=\vec{j}\,.
\end{alignedat}
\end{equation}
Die links stehenden Gleichungen werden als \emph{homogene}, die rechten als
\emph{inhomogene} Maxwell-Gleichungen bezeichnet. Die Ladungsdichte $\rho$,
sowie die Stromdichte $\vec{j}$ können als "`Quellen"' der Felder angesehen
werden. Zusätzlich muss eine weitere Gleichung hinzugefügt werden, die die
Dynamik von Testteilchen\footnote{Ein Testteilchen ist ein idealisiertes Objekt, welches selbst die wirkenden Felder nicht beeinflusst.} beschreibt
\begin{equation}
m\ddot{\vec{x}}=\vec{F}(\vec{x},\dot{\vec{x}},t)
=q\left[\vec{E}(\vec{x},t)+\dot{\vec{x}}\times\vec{B}(\vec{x},t)\right]\,,
\end{equation}
dabei ist $\vec{F}$ die so genannte \emph{Lorentzkraft} und $q$, $m$ die
Ladung bzw. Masse des Teilchens.
Es lässt sich mittels des Helmholtz-Theorems und der Form der Maxwell-Gleichung
die Existenz von \emph{Potentialen} $\vec{A}$, $\Phi$ folgern, sodass
\begin{equation}
\vec{B}=\Rot \vec{A}\,,\quad
\vec{E}=-\Grad\Phi+\dpd{\vec{A}}{t}\,.\label{eq:Vecpot}
\end{equation}
Es ist von Vorteil, eine Beschreibung mit Hilfe der Differentialgeometrie
zu wählen.
Die Potentiale werden als Komponenten
$\tensor{A}{^\mu}=(\Phi,\vec{A})\transpose$ einer 1-Form
$A=\tensor{A}{_\mu}\dif\tensor{x}{^\mu}$ aufgefasst.
Der Elektromagnetische Feldstärketensor ist definiert als
\begin{equation}
F:=\dif
A\,,
\end{equation}
bzw. in lokalen Koordinaten
\begin{equation}
\tensor{F}{_\mu_\nu}=\partial_\mu\tensor{A}{_\nu}-\partial_\nu\tensor{A}{_\mu}\,.
\end{equation}
Die physikalischen Felder lassen sich aus dem Feldstärketensor wiederum durch
\begin{equation}
\vec{E}_{i}=\tensor{F}{_0_i}\,,\quad
\vec{B}_i=\frac{1}{2}\tensor{\varepsilon}{_i_j_k}\tensor{F}{^j^k}
\end{equation}
zurückgewinnen, wie man leicht mit Hilfe der Definition von $F$ und
\eqref{eq:Vecpot} nachrechnet. Führt man auch eine Viererstromdichte
$\tensor{J}{^\nu}=(\rho,\vec{j})\transpose$ ein,
reduzieren sich die Maxwell-Gleichungen zu zwei
Gleichungen
\begin{equation}
\tensor{\varepsilon}{^\alpha^\beta^\gamma^\delta}
\pdif{_{\alpha}}\tensor{F}{_\gamma_{\delta}}=0\,,\quad
\pdif{_\mu}\tensor{F}{^\mu^\nu}=\tensor{J}{^\nu}\,.
\end{equation}
Die homogene Gleichung ist dabei trivial, da sie bereits aus der Form des
Feldstärketensors folgt.
%d^2=0 oder 
% https://en.wikipedia.org/wiki/Yang%E2%80%93Mills_theory
% ^ Zusammenhang Jakobi und Maxwellgleichung\ldots
In koordinatenfreier Notation lauten die Gleichungen schließlich 
\begin{equation}
\dif F = 0\,,\quad\quad \dif\,(\star F)= J\,.
\end{equation}
% MW2= Jakobi?
% Die Lagrange Dichte für das Elektromagnetische Feld im Vakuum $\tensor{J}{^\mu}=0$
% \begin{equation}
% \mathcal{L}\left[\partial_\mu
% A(x),x\right]=-\frac{1}{4}\tensor{F}{_\mu_\nu}\tensor{F}{^\mu^\nu} \,.
% \end{equation}
% Wobei Beiträge die die Dynamik von Spin-$\nicefrac{1}{2}$ beschreiben nicht
% berücksichtigt wurden.
Wie in \autoref{bsp:Spinone} gezeigt wurde, erfüllen die Felder die (Vakuum-)
Maxwell-Gleichungen\footnote{Der Fall $J^\mu\neq 0$ erfordert zusätzlich das
Einführen des Elektronfeldes, worauf wir der Einfachheit halber hier
verzichten.}, sofern sie das zum Wirkungsintegral
\begin{equation}
S[A]=-\frac{1}{4}\int_M\tr\left(\star F\wedge F\right)\dif x
\end{equation}
minimieren.
\subsection{Eichtransformationen}
In den Maxwell-Gleichungen taucht das Feld $A$ nie explizit, sondern stets in
der durch $F$ vorgegebenen Kombination auf. Man macht sich leicht klar, das $F$
invariant unter Transformationen $A\to A+\dif\Lambda$ mit $\Lambda\in
C^\infty(\Reals^4)$ einer glatten Funktion. Diese zusätzliche Unbestimmtheit
bezeichnet man als \emph{Eichfreibeit}, bzw. Transformationen der Form als
Eichtransformationen.
%TODO U1 symmetrie
%TODO relativistische Lorentzkraft
%TODO QED in gekrümmtem Raum
\section{Allgemeine Relativitätstheorie}
\subsection{Einsteinsche Feldgleichungen}
Wie die Elektrodynamik, wird auch die allgemeine Relativitätstheorie durch einen
Satz von Differentialgleichungen beschrieben. Die so genannten
\emph{Einsteinschen Feldgleichungen} lauten in lokalen Koordinaten\footnote{Der Term proportional
zur kosmologischen Konstante $\Lambda$ hat über kleine Distanzen vernachlässigbaren Einfluss und wird im Folgenden nicht berücksichtigt.}
\begin{equation}
\tensor{R}{_\mu_\nu} - \frac{R}{2}\, \tensor{g}{_\mu_\nu}
+\Lambda\, \tensor{g}{_\mu_\nu}
=\tensor{T}{_\mu_\nu}\,.
\end{equation}
Der Tensor $\tensor{T}{_\mu_\nu}$ heißt \emph{Energie-Impuls-Tensor} und
enthält Information über die Materie, d.h. über die Felder im Raum. Eine
besonders einfache Form nehmen die Gleichungen im Vakuum an, denn dann gilt
$\tensor{T}{_\mu_\nu}=0$.
Bilden wir die Spur der Einsteingleichungen, so finden wir
\begin{equation}
0=R- 2 R =-R\,,
\end{equation}
die Skalarkrümmung $R$ verschwindet.
Einsetzen in die Einsteingleichungen liefert wiederum die Vakuumgleichung
\begin{equation}
\tensor{R}{_\mu_\nu}=0\,.
\end{equation}
\subsection{Variationsprinzip}
% TODO So formulieren Das S: Rank2sym->R problem S-> Max und
% |_\alphS(g+\alphah)=0 für alle Testfunktionen. Rechenregeln
Eine elegante Ableitung der Vakuumgleichungen ergibt sich mithilfe eines auf
Hilbert zurückgehenden Variationsprinzips.
Die Idee lautet: wähle die Metrik $g$ als kritischen
Punkt eines lokalen Funktionales, welches nur vom Riemannschen Krümmungstensor
abhängt.
Das einfachste solche Funktional ist die so genannte
\emph{Einstein-Hilbert-Wirkung}
\begin{equation}
S[g]=\int_{M}R[g]\,\Omega = \int_{M}\sqrt{-g}R[g] \dif{}^4x \,.
\end{equation}
Die zugehörige Lagrange Funktion ist gegeben als
\begin{equation}
L(g)=\sqrt{-g}R[g]\,.
\end{equation}
Bei der Variation dieses Funktionales treten verschiedene Terme auf. Wir
betrachten zunächst den Volumenfaktor $\sqrt{-g}$, mit Jacobis Formel und der
Kettenregel
%TODO REF
\begin{equation}
\begin{split}
\delta \sqrt{-g}
&= -\frac{1}{2\sqrt{-g}}  \left(\sqrt{-g}\tensor{g}{_\mu_\nu}
\delta\tensor{g}{^\mu^\nu} \right)\\
&= -\frac{1}{2}\sqrt{-g}\tensor{g}{_\mu_\nu}
\delta\tensor{g}{^\mu^\nu} \\
\end{split}
\end{equation}
Um die Variation des Krümmungskalar zu berechnen, wählen wir zweckmäßigerweise
ein Riemannsches Normalkoordinatensystem\footnote{Ein Koordinatensystem in dem
$\cSym{\rho}{\mu}{\sigma}=0$ im betrachteten Punkt}, sodass gilt
\begin{equation}
\tensor{R}{^\rho_\mu_\nu_\sigma}=\tensor{\partial}{_\nu}\cSym{\rho}{\mu}{\sigma}
-\tensor{\partial}{_\mu}\cSym{\rho}{\nu}{\sigma}\,.\\
\end{equation}
Betrachtet die Variation des Ricci-Tensors nach der Metrik so gilt
\begin{equation}
\begin{split}
\delta \tensor{R}{_\mu_\nu}
&=\delta \tensor{R}{^\rho_\mu_\rho_\nu}\\
&=\delta\tensor{\partial}{_\rho} \cSym{\rho}{\mu}{\nu}
-\delta\tensor{\partial}{_\mu} \cSym{\rho}{\rho}{\nu}\\
&=\tensor{\partial}{_\rho}\delta \cSym{\rho}{\mu}{\nu}
-\tensor{\partial}{_\mu}\delta \cSym{\rho}{\rho}{\nu}\\
&=\tensor{\nabla}{_\rho}\delta \cSym{\rho}{\mu}{\nu}
-\tensor{\nabla}{_\mu}\delta \cSym{\rho}{\rho}{\nu}\,,
\end{split}
\end{equation}
d.h. die Variation liefert eine totale Divergenz.
Weiter berechnen wir
\begin{equation}
\begin{split}
\delta R &=\delta \left(\tensor{g}{^\mu^\nu}\tensor{R}{_\mu_\nu}\right)\\
&=\tensor{R}{_\mu_\nu}\delta\tensor{g}{^\mu^\nu}
+\tensor{g}{^\mu^\nu}\delta\tensor{R}{_\mu_\nu}\,.
\end{split}
\end{equation}
Damit ist die Variation des EH-Funktionales gegeben als
\begin{equation}
\begin{split}
\delta S
&=\int\left(
R\delta\sqrt{-g}+\sqrt{-g}\delta R\right)\dif{}^4x\\
&=\int \left[
\frac{1}{2}\sqrt{-g}\tensor{g}{_\mu_\nu}\delta\tensor{g}{^\mu^\nu}
R+\sqrt{-g}\left(\tensor{R}{_\mu_\nu}\delta\tensor{g}{^\mu^\nu}
+\tensor{g}{^\mu^\nu}\delta\tensor{R}{_\mu_\nu}\right)\right]\dif{}^4x\\
&=\int \sqrt{-g}\left(
\frac{R}{2}\tensor{g}{_\mu_\nu}
+\tensor{R}{_\mu_\nu}\right)\delta\tensor{g}{^\mu^\nu}\dif{}^4x
+\int \sqrt{-g}\tensor{g}{^\mu^\nu}\delta\tensor{R}{_\mu_\nu}\dif{}^4x
\,.
\end{split}
\end{equation}
Das zweite Integral liefert als totale Divergenz nur einen Oberflächenterm.
Damit verbleibt
\begin{equation}
\begin{split}
\delta S
&=\int \sqrt{-g}\left(
\frac{R}{2}\tensor{g}{_\mu_\nu}
+\tensor{R}{_\mu_\nu}\right)\delta\tensor{g}{^\mu^\nu}\dif{}^4x\,.
\end{split}
\end{equation}
Die Extremalbedingung $\delta S=0$ für beliebige $\delta\tensor{g}{^\mu^\nu}$
impliziert
\begin{equation}
\tensor{R}{_\mu_\nu}+\frac{R}{2}\tensor{g}{_\mu_\nu}=0\, .
\end{equation}
Die Einsteingleichungen lassen sich also aus dem Variationsprinzip herleiten.
\subsection{Skalar-Tensor Theorien}
Brans-Dicke-Theorie
%TODO Overduin and Wesson 1997a).
\section{Gemeinsamkeiten und Unterschiede}
Unterschiede:
\begin{itemize}
  \item Lorentz-Kraft muss postuliert werden. Reine ART enthält keine Kräfte,
  alle Teilchen bewegen sich frei auf Geodäten.
  \item Ladung hat zwei Vorzeichen, d.h. es kann mittels umgekehrter Ladung
  entschieden werden ob man frei fällt oder nicht
  \item Die Einsteingleichungen sind nichtlinear und zweiter Ordnung die
  Maxwell-Gleichungen linear und erster Ordnung
  \item Kraft vs Geometrie
\end{itemize}
Gemeinsamkeiten:
\begin{itemize}
  \item Diffeomorphismen-Invarianz/Eichinvarianz
  \item In der nichtrelativistischen Näherung $1/r^2$ Gesetz.
  \item Masseloses bosonisches Vermittlerteilchen.
  \item Lagrangeformulierung
\end{itemize}
Insgesamt lässt sich hoffen das eine vereinheitlichte Theorie so formuliert
werden kann das sie einerseits die Ähnlichkeiten erklärt und auf gemeinsame
Ursachen zurückführt, andererseits aber auch Gründe für die Unterschiede gibt
bzw. diese möglicherweise ausräumt.
