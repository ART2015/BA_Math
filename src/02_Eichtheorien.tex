\chapter{Maxwell-Einstein Theorie}
Dieser Teil soll die physikalischen Theorien vorstellen, die der
Kaluza-Klein-Theorie zu Grunde liegen. Dazu zählt neben der 
in ihrer heutigen Formulierung auf Maxwell zurückgehenden 
Elektrodynamik, Einsteins allgemeine Relativitätstheorie. 
Deutlich älter ist die Elektrodynamik, deren fundamentales Gesetz, die Konstanz
der Lichtgeschwindigkeit, die Grundlage der speziellen Relativitätstheorie
darstellt.
Die Ähnlichkeit beider Theorien lässt hoffen, dass eine 
übergeordnete Theorie beide als Spezialfälle enthält. 
\section{Elektrodynamik}
Die Elektrodynamik beschreibt das Wechselspiel von elektrischen und 
magnetischen Feldern. Die Phänomenologie lässt sich mittels zweier glatter Vektorfelder
$\vec{E},\vec{B}:\Reals^3\times\Reals\to \Reals^3$ beschreiben.
Diese Felder erfüllen die auf Maxwell zurückgehenden Gleichungen 
\begin{equation}
  \begin{alignedat}{2}
    \Div\vec{E} &= \rho\,,  & \qquad \Rot\vec{E}+\dpd{\vec{B}}{t}&
    =0\,,\\
    \Div\vec{B} &= 0\,,& \qquad\Rot\vec{B}-\dpd{\vec{E}}{t}&
    =\vec{j}\,.
  \end{alignedat}
\end{equation}
Zusätzlich muss eine weitere Gleichung hinzugefügt werden, die die Dynamik von
Testteilchen\footnote{Ein Testteilchen ist ein idealisiertes Objekt, welches
selbst die wirkenden Felder nicht beeinflusst.} beschreibt
\begin{equation}
m\ddot{\vec{x}}=\vec{F}(\vec{x},\dot{\vec{x}},t)
=q\left[\vec{E}(\vec{x},t)+\dot{\vec{x}}\times\vec{B}(\vec{x},t)\right]\,,
\end{equation}
dabei ist $\vec{F}$ die so genannte \emph{Lorentzkraft} und $q$, $m$ die
Ladung bzw. Masse des Teilchens.
Es lässt sich mittels des Helmholtz-Theorems und der Form der Maxwell-Gleichung
die Existenz von Potentialen $\vec{A}$, $\Phi$ folgern, sodass
\begin{equation}
\vec{B}=\Rot \vec{A}\,,\quad
\vec{E}=-\Grad\Phi+\dpd{\vec{A}}{t}\,.\label{eq:Vecpot}
\end{equation}
Die Elektrodynamik ist eng mit der speziellen Relativitätstheorie verbunden.
Diese lässt sich am besten mit Objekten der Differentialgeometrie beschreiben.
Die Potentiale werden als Komponenten
$\tensor{A}{^\mu}=(\Phi,\vec{A})\transpose$ einer 1-Form
$A=\tensor{A}{_\mu}\dif\tensor{x}{^\mu}$ aufgefasst. Der Elektromagnetische Feldstärketensor ist definiert als
\begin{equation}
F:=\dif
A\,,
\end{equation}
bzw. in lokalen Koordinaten
\begin{equation}
\tensor{F}{_\mu_\nu}:=\partial_\mu\tensor{A}{_\nu}-\partial_\nu\tensor{A}{_\mu}\,.
\end{equation}
Die physikalischen Felder lassen sich aus dem Feldstärketensor wiederum durch 
\begin{equation}
\vec{E}_{i}=\tensor{F}{_0_i}\,,\quad
\vec{B}_i=\frac{1}{2}\tensor{\varepsilon}{^i^j^k}\tensor{F}{^j^k}
\end{equation}
zurückgewinnen, wie man sich leicht mit Hilfe der Definition von $F$ und
\eqref{eq:Vecpot} nachrechnet.
Damit reduzieren sich die Maxwell-Gleichungen zu zwei
Gleichungen
% \begin{align} 
% \pdif{_\mu}\tensor{F}{^\mu^\nu}&=\tensor{J}{^\nu}
% \label{eq:MaxwellInhom}\\
% \pdif{_{\mu}}\tensor{F}{_\nu_{\rho}}+
% \pdif{_{\mu}}\tensor{F}{_\rho_{\nu}}+
% \pdif{_{\nu}}\tensor{F}{_\mu_{\rho}}&=0
% \label{eq:MaxwellHom}
% \end{align}
\begin{align}
\pdif{_\mu}\tensor{F}{^\mu^\nu}&=\tensor{J}{^\nu}
\label{eq:MaxwellInhom}\,,\\
\tensor{\varepsilon}{^\alpha^\beta^\gamma^\delta}
\pdif{_{\alpha}}\tensor{F}{_\gamma_{\delta}}&=0\,.
\label{eq:MaxwellHom}
\end{align}
Bzw in indexfreier Notation
\begin{equation}
\quad \dif\star F= J\,,\quad\dif F = 0\,.
\end{equation}
% MW2= Jakobi?
% Die Lagrange Dichte für das Elektromagnetische Feld im Vakuum $\tensor{J}{^\mu}=0$
% \begin{equation}
% \mathcal{L}\left[\partial_\mu
% A(x),x\right]=-\frac{1}{4}\tensor{F}{_\mu_\nu}\tensor{F}{^\mu^\nu} \,.
% \end{equation}
% Wobei Beiträge die die Dynamik von Spin-$\nicefrac{1}{2}$ beschreiben nicht
% berücksichtigt wurden. 
\subsection{Eichtransformationen}
\section{Allgemeine Relativitätstheorie}
\subsection{Einsteinsche Feldgleichungen}
Die Einsteinschen Feldgleichungen lauten in lokalen Koordinaten\footnote{Der
Term proportional zur kosmologischen Konstante $\Lambda$ hat über kleine Distanzen
vernachlässigbaren Einfluss und wird im Folgenden nicht berücksichtigt.}
\begin{equation}
\tensor{R}{_\mu_\nu} - \frac{R}{2}\, \tensor{g}{_\mu_\nu}
+\Lambda\, \tensor{g}{_\mu_\nu}
=\tensor{T}{_\mu_\nu}\,.
\end{equation}
Der Tensor $\tensor{T}{_\mu_\nu}$ heißt \emph{Energie-Impuls-Tensor} und
enthält Information über die Materie, d.h. über die Felder im Raum. Eine
besonders einfache Form nehmen die Gleichungen im Vakuum an, denn dann gilt 
$\tensor{T}{_\mu_\nu}=0$.
Bilden wir die Spur der Einsteingleichungen, so finden wir 
\begin{equation}
0=R- \frac{R}{2}\, \tensor{g}{^\mu_\mu}= -R\,,
\end{equation}
die Skalarkrümmung $R$ verschwindet.
Einsetzen in die Einsteingleichungen liefert wiederum
\begin{equation}
\tensor{R}{_\mu_\nu}=0\,.
\end{equation}
\subsection{Variationsprinzip}
% TODO So formulieren Das S: Rank2sym->R problem S-> Max und
% |_\alphS(g+\alphah)=0 für alle Testfunktionen. Rechenregeln
Eine elegante Ableitung der Einsteingleichungen ergibt sich mithilfe eines auf
Hilbert zurückgehenden Variationsprinzips. 
Die Idee lautet: wähle die Metrik $g$ als kritischen
Punkt eines Funktionales, dass nur von der Krümmung abhängt. Das
einfachste solche Funktional ist die so genannte
\emph{Einstein-Hilbert-Wirkung}
\begin{equation}
S[g]=\int_{M}\sqrt{-g}R[g] \dif{}^4x \,.
\end{equation}
Bei der Variation dieses Funktionales treten verschiedene Terme auf. Wir
betrachten zunächst den Volumenfaktor $\sqrt{-g}$
\begin{equation}
\delta g = g  \tensor{g}{^\mu^\nu}\delta\tensor{g}{_\mu_\nu}
\end{equation}
\begin{equation}
\begin{split}
\delta \sqrt{-g} 
&= -\frac{1}{2\sqrt{-g}}\delta g \\
&= \frac{1}{2} \sqrt{-g} \left(\tensor{g}{^\mu^\nu} \delta
\tensor{g}{_\mu_\nu}\right)\\
&= -\frac{1}{2} \sqrt{-g} \left(\tensor{g}{_\mu_\nu} \delta
\tensor{g}{^\mu^\nu}\right)
\end{split}
\end{equation}
Um die Variation des Krümmungskalar zu berechnen wähle man zweckmäßigerweise
ein Riemannsches Normalkoordinatensystem\footnote{Ein Koordinatensystem in dem
$\cSym{\rho}{\mu}{\sigma}=0$ im betrachteten Punkt},
\begin{equation}
\begin{split}
\delta \tensor{R}{^\rho_\mu_\nu_\sigma}
&=\delta \tensor{\partial}{_\nu}\cSym{\rho}{\mu}{\sigma}
-\delta \tensor{\partial}{_\mu}\cSym{\rho}{\nu}{\sigma}\\
&=\tensor{\partial}{_\nu}\delta \cSym{\rho}{\mu}{\sigma}
-\tensor{\partial}{_\mu}\delta \cSym{\rho}{\nu}{\sigma}\\
\end{split}
\end{equation}
\begin{equation}
\begin{split}
\delta \tensor{R}{_\mu_\nu}
&=\delta \tensor{R}{^\rho_\mu_\rho_\nu}\\
&=\tensor{\partial}{_\rho}\delta \cSym{\rho}{\mu}{\nu}
-\tensor{\partial}{_\mu}\delta \cSym{\rho}{\rho}{\nu}\\
&=\tensor{\nabla}{_\rho}\delta \cSym{\rho}{\mu}{\nu}
-\tensor{\nabla}{_\mu}\delta \cSym{\rho}{\rho}{\nu}\\
\end{split}
\end{equation}
Die Gleichung ist koordinatenunabhängig, da sie eine Tensor Gleichung ist.
Weiter berechnet man 
\begin{equation}
\begin{split}
\delta R &=\delta \left(\tensor{g}{^\mu^\nu}\tensor{R}{_\mu_\nu}\right)\\
&=\tensor{R}{_\mu_\nu}\delta\tensor{g}{^\mu^\nu}
+\tensor{g}{^\mu^\nu}\delta\tensor{R}{_\mu_\nu}\,.
\end{split}
\end{equation}
Damit ist die Variation des EH-Funktionales gegeben als
\begin{equation}
\begin{split}
\delta S
&=\fourint \left[
R\delta\sqrt{-g}+\sqrt{-g}\delta R\right]\\
&=\fourint \left[
\frac{1}{2}\sqrt{-g}\tensor{g}{^\mu^\nu}\delta\tensor{g}{_\mu_\nu}
R+\sqrt{-g}\left(\tensor{R}{_\mu_\nu}\delta\tensor{g}{^\mu^\nu}
+\tensor{g}{^\mu^\nu}\delta\tensor{R}{_\mu_\nu}\right)\right]\\
&=\fourint \sqrt{-g}\left(
\frac{1}{2}\tensor{g}{^\mu^\nu}
R+\tensor{R}{^\mu^\nu}\right)\delta\tensor{g}{_\mu_\nu}
+\fourint \sqrt{-g}\tensor{g}{^\mu^\nu}\delta\tensor{R}{_\mu_\nu}
\,.
\end{split}
\end{equation}
Das zweite Integral liefert nur einen Oberflächenterm:
\begin{equation}
\begin{split}
\fourint \sqrt{-g}\tensor{g}{^\mu^\nu}\delta\tensor{R}{_\mu_\nu}
&=\fourint \sqrt{-g}\tensor{g}{^\mu^\nu}\left(\tensor{\nabla}{_\rho}\delta
\cSym{\rho}{\mu}{\nu} -\tensor{\nabla}{_\mu}\delta \cSym{\rho}{\rho}{\nu}\right)
\\
&=\fourint \tensor{\nabla}{_\rho}\left(\sqrt{-g}\tensor{g}{^\mu^\nu}\delta
\cSym{\rho}{\mu}{\nu}\right)-\int\dif{}^4 x \tensor{\nabla}{_\mu}\left(\sqrt{-g}\tensor{g}{^\mu^\nu}\delta
\cSym{\rho}{\rho}{\nu}\right)\,.
\end{split}
\end{equation}
Damit verbleibt 
\begin{equation}
\begin{split}
\delta S\textsubscript{EH}
&=\fourint \sqrt{-g}\left(
\frac{1}{2}\tensor{g}{^\mu^\nu}
R+\tensor{R}{^\mu^\nu}\right)\delta\tensor{g}{_\mu_\nu}\,.
\end{split}
\end{equation}
Wir können also die Variation nach der Metrik berechnen 
\begin{equation}
\frac{\delta
S\textsubscript{EH}[\tensor{g}{_\mu_\nu}(x)]}{\delta\tensor{g}{_\mu_\nu}(x')}
=\sqrt{-g}\left(\frac{1}{2}\tensor{g}{^\mu^\nu}
R+\tensor{R}{^\mu^\nu}\right)
\end{equation}
Die Extremalforderung impliziert
\begin{equation}
\tensor{R}{^\mu^\nu}+\frac{R}{2}\tensor{g}{^\mu^\nu}=0\, .
\end{equation}
Die Einsteingleichungen lassen sich also aus dem Variationsprinzip herleiten.
\subsection{Skalar-Tensor Theorien}
Brans-Dicke-Theorie
\section{Gemeinsamkeiten und Unterschiede}
Unterschiede:
\begin{itemize}
  	\item Lorentz-Kraft muss postuliert werden. Reine ART enthält keine Kräfte,
 	alle Teilchen bewegen sich frei auf Geodätischen.
	\item Ladung hat zwei Vorzeichen, d.h. es kann mittels umgekehrter Ladung
	entschieden werden ob man frei fällt oder nicht
	\item Die Einsteingleichungen sind nichtlinear und zweiter Ordnung die
	Maxwellgleichungen linear und erster Ordnung
	\item Kraft vs Geometrie
\end{itemize}
Gemeinsamkeiten:
\begin{itemize}
  	\item Diffeomorphismeninvarianz/Eichinvarianz
	\item In der nichtrelativistischen Näherung $1/r^2$ Gesetz.
	\item Massenloses bosonisches Vermittlerteilchen.
	\item Lagrangeformulierung
\end{itemize}
Insgesamt lässt sich hoffen das eine vereinheitlichte Theorie so formuliert
werden kann das sie einerseits die Ähnlichkeiten erklärt und auf gemeinsame
Ursachen zurückführt, andererseits aber auch Gründe für die Unterschiede gibt
bzw. diese möglicherweise ausräumt.

