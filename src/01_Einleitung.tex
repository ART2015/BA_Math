\chapter{Einleitung}
% Anfang des 20. Jahrhunderts sah man sich mit der Situation konfrontieret das
% es mit der Einsteinschen ART und der Maxwellschen Elektrodynamik, Zwei
% Theorien gab die durch das Experimentell verifiziert und von einem Großteil
% der Wissenschaftsgemeinde akzeptiert waren. Das Problem das sich Stelle war
% ein zunächst rein philosophisches
% Die Klassischen Theorien der der Elektrodynamik und der Gravitation, wie sie
% von Newton und Colulomb (?) waren in ihren Gesetzen ausgesprochen symetrisch,
% größter Unterschied stellten nicht vorhandene \enquote{negative}
% Gravitationsladungen dar
% Die ART und Maxwell-Theorie unterscheiden sich allerdings in verschiedenen
% Punkten, deshalb:  Gibt es eine Theorie, bzw. einen Formalsismus der beide
% Theorien als Teil eines Ganzen beinhaltet?
%https://www.google.de/search?q=Randall%E2%80%93Sundrum+model+kaluza+klein&ie=utf-8&oe=utf-8&gws_rd=cr&ei=fW7wVsnJAuWBywP2v6GACA
\section{Historischer Hintergrund}
In der Geschichte der Physik gab es häufig Punkte an denen eine \enquote{theory
of everything} (TOE) greifbar erschien. Anfang des 20.\ Jahrhunderts, zur
Zeit ihrer Formulierung, schien die Theorie des deutschen Physikers \name{Theodor Kaluza} 
ein viel versprechender Kandidat für eine solche Theorie zu sein.

Sie vereinigt die Gravitation und die Elektromagnetische Kraft und damit alle zu
dieser Zeit bekannten Wechselwirkungen in einem bestechend einfachen
Formalismus. Weiterentwicklungen durch \name{Oskar Klein} um 1921 führten dazu
das es heute üblich ist von von der \emph{Kaluza-Klein(KK)-Theorie} zu sprechen.
Daneben stammen bedeutende Arbeiten von Jordan und Thiry sodass manche Autoren
von der Kaluza-Klein-Jordan-Thiry-Theorie sprechen.

In ihrem ursprünglichen Ziel der Vereinigung von Elektromagnetismus und
Gravitation ist die Theorie erfolgreich, 
allerdings zeichnete sich mit der Entdeckung weiterer
fundamentaler Wechselwirkungen Mitte des 20.\ Jahrhunderts ab, dass 
sie zumindest erweitert werden muss.
Im Standardmodell der Teichenphysik (SM) gelang es die neuen Kräfte mit dem
Elektromagnetismus zu vereinigen und bis auf wenige, wenn auch
bedeutsame Ausnahmen, alle nicht-gravitative Interaktionen zu erklären. 

In seinem für das
Feld der KK-Theorien bedeutenden Artikel \enquote{Search for a Realistic
Kaluza-Klein Theory} beschreibt
\name{E.\ Witten} Probleme, die auftreten wenn man versucht 
von der Eichgruppe $O(1)$ des EM auf die SM-Eichgruppe $SU(3)\times SU(2) \times
U(1)$ überzugehen.
% Chiral Fermions
%
% Historically, the main reason the modern KK program was rejected
% (or at least slowed down) was this paper by Witten. One of the generic
% difficulties in constructing any model of fundamental physics is
% that the standard model has chiral fermions -- fermions of different chirality
% have different couplings. This is hard to achieve because fermions
% tend to come and go in pairs of opposite chirality whose couplings are exactly
% the same. If you do manage to somehow construct chirally asymmetric
% models, these models have many more possibilities to be inconsistent
% (anomalous) and therefore many more consistency checks to pass.
% This is therefore one of the best, most stringent, tests to subject any claim
% for beyond the standard model physics.
% What Witten has shown is that there is no way to get chiral
% fermions starting with higher dimensional (super)gravity theory on any smooth
% manifold. This caused a general loss of interest in this research
% direction. Ironically, it was Witten and various collaborators that
% demonstrated, about 15 years later, that the problem can be solved (in
% string theory, using singular manifolds). Turns out that String theory has exactly the right ingredients to make the physics of the required singularities regular, and to pass all the non-trivial consistency checks that accompany any chiral theory.

In den folgenden Jahren wurde die ursprüngliche KK-Theorie von vielen
Autoren für tot erklärt.
Sie ist allerdings Mutter einer ganzen Klasse von Theorien, die häufig auch
Kalzua-Klein-Theorien genannt werden. Als Beispiel kann die 1954 formulierte 
\emph{Yang-Mills-Theorie} angesehen werden.
% TODO: Check bzgl. Yang Mills
Verschiedene Konzepte, die ursprünglich im
Rahmen der KK-Theorie entwickelt wurden, insbesondere kompakte Dimensionen,
haben auch Eingang in den Formalismus des jüngsten
TOE-Aspiranten, der Stringtheorie, gefunden.
Das Studium der KK-Theorie ist deshalb nicht nur als eine der ersten
vereinheitlichen Theorien von historischem Interesse, sondern bietet auch eine
Einführung. Nicht zuletzt stehen wir heute vor einer ähnlichen Situation
wie Kaluza, in der es gilt, zwei überaus erfolgreiche Theorien miteinander zu vereinen. 

Das Kapitel \autoref{} dient des mathematischen Begriffsaparats der zur
Formulierung der Theorie notwendig ist. In Kapitel gehen wir auf die die
Maxwell-Einstein Theorie ein, die ihrerseits die Grundlage der KK-Theorie
bildet.

zumindest die Gravitation als eine von vier elementaren Kräften durch Geometrie
zu beschreiben. Die verbleibenden Kräfte, die schwache, die starke und die elektromagnetische Kraft
blieben grundsätzlich unangetastet. Es liegt Nahe, besonders da die Theorien
viele Ähnlichkeiten aufweisen, dass auch die anderen Kräfte durch eine ähnliche
Theorie beschreiben werden können.

Die neue Theorie soll dabei der alten 
möglichst ähnlich sein, also eine minimale Erweiterung der allgemeinen
Relativitätstheorie darstellen.
Da die Einsteingleichungen keine
zusätzlichen Freiheitsgrade enthalten, führt man heuristisch zusätzliche
Dimensionen ein, behält aber die Struktur der Gleichungen.
Da makroskopische Zusatzdimensionen aus verschiedenen Gründen ausgeschlossen
sind müssen die Zusatzdimensionen kompakt sein.
Zusammenfassend zeichnet sich eine solche Theorie durch folgende Punkte aus:
\begin{enumerate} 
\item Die Physik lässt sich vollständig durch Geometrie beschreiben.
\item Die uns zugängliche, vierdimensionale Raumzeit ist eingebettet in einen
höherdimensionalen Raum, welcher kompakte Zusatzdimensionen enthält.
\item Die Theorie ist eine minimale Erweiterung der allgemeinen
Relativitätstheorie.
\item Die Projektion auf vier Dimensionen liefert die uns bekannten Gesetze, mit
möglicherweise kleinen Abweichungen, welche sich auf den von beobachteten Skalen
nicht zeigen.
\end{enumerate}
Was genau unter Kaluza-Klein Theorien zu verstehen ist variiert von Autor zu
Autor. Kern ist allerdings immer die ursprüngliche fünfdimensionale Theorie,
von der es allerdings ebenfalls verschiedene Varianten gibt
% TODO Klein, Thiery Jordan\ldots
Die Arbeit ist in ihrer Behandlung des Komplexes keineswegs vollständig. Im
wesentlichen handelt es sich um allgemeine Relativitätstheorie in fünf
Dimensionen, allerdings mit einigen Zusatzannahmen.
Klein steuerte die Kompaktifizierung der Zusatzdimensionen bei.
%TODO Kompakt=> Alle orbits  periodisch? in 1d? in d>1?
%TODO Arbeiten von Jordan (1947, 1955). Bergmann (1948), Thiry (1948), Lessner (1982), and Liu and
%Wesson (1997). 

Die Idee Kaluzas ist dabei einfach: statt vierdimensional wie bei Einstein ist
die Raumzeit fünfdimensional. 

Da bereits vier Dimensionen die Vorstellungskraft der meisten Menschen
übersteigt (Einstein wird zugeschrieben sich diese Tatsächlich geometrisch
vorstellen zu können), war es für viele Physiker nicht weiter schwierig sich mit 
einer fünfdimensionalen Theorie abzufinden. 
