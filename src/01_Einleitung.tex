\chapter{Einleitung}
\section{Historischer Hintergrund}
Die Idee, dass die Welt in ihren Grundfesten geometrischer Natur ist
existiert ist nicht neu.
Beispielsweise führte Keppler die Abstände der Planetenbahnen auf
verschachtelte platonische Körper zurück. Auch manche Schulen der griechischen
Philosophie glaubten, dass Geometrie neben den Zahlen, das sei was das Universum
im wesentlichen ausmacht.
Selbsversändlich hatten sie dabei weder Differentialgeometrie noch die 
Relatitätstheorie im Sinn.  
Im laufe der Jahre löste sich die physikalischen Theorie immer weiter von dieser Idee. 
Zu Zeiten Kaluzas alle Kräfte durch Geometrie beschrieben
 

Mit der allgemeinen Relativitätstheorie gelang es Albert Einstein 1915
zumindest die Gravitation als eine von vier elementaren Kräften durch Geometrie
zu beschreiben. Die verbleibenden Kräfte, die schwache, die starke und die elektromagnetische Kraft
blieben grundsätzlich unangetastet. Es liegt Nahe, besonders da die Theorien
viele Ähnlichkeiten aufweisen, dass auch die anderen Kräfte durch eine ähnliche
Theorie beschreiben werden können.

Die neue Theorie soll dabei der alten 
möglichst ähnlich sein, also eine minimale Erweiterung der allgemeinen
Relativitätstheorie darstellen.
Da die Einsteingleichungen keine
zusätzlichen Freiheitsgrade enthalten, führt man heuristisch zusätzliche
Dimensionen ein, behält aber die Struktur der Gleichungen.
Da makroskopische Zusatzdimensionen aus verschiedenen Gründen ausgeschlossen
sind müssen die Zusatzdimensionen kompakt sein.
Zusammenfassend zeichnet sich eine solche Theorie durch folgende Punkte aus:
\begin{enumerate}
\item Die Physik lässt sich vollständig durch Geometrie beschreiben.
\item Die uns zugängliche, vierdimensionale Raumzeit ist eingebettet in einen
höherdimensionalen Raum, welcher kompakte Zusatzdimensionen enthält.
\item Die Theorie ist eine minimale Erweiterung der allgemeinen
Relativitätstheorie.
\item Die Projektion auf vier Dimensionen liefert die uns bekannten Gesetze, mit
möglicherweise kleinen Abweichungen, welche sich auf den von beobachteten Skalen
nicht zeigen.
\end{enumerate}