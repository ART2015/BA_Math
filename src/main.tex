\documentclass[a4paper,oneside]{scrreprt}

%%% PACKAGES + MODIFICATIONS%%%
 
%language and encoding
\usepackage[utf8]{inputenc}
\usepackage[T1]{fontenc}
\usepackage[english,ngerman]{babel}
\usepackage[autostyle=true,german=quotes]{csquotes}

%math
\usepackage{amsmath}
\usepackage{amsthm,thmtools}
\usepackage{amssymb}
\usepackage{amsfonts}
\usepackage{mathtools}
\usepackage{commath}
\usepackage{dsfont}		% double stroke characters
\usepackage{braket}
\usepackage{mdframed}
\usepackage{eufrak}
\usepackage{tensor}

%science
\usepackage{units}
\usepackage{bpchem}	% type­set chem­i­cal names, for­mu­lae, etc

%layout + style
\usepackage{sectsty}
\usepackage[automark,headsepline]{scrlayer-scrpage} 						% headings
\renewcommand*{\headfont}{\normalfont}										% nicht kursive Kopfzeile
\usepackage{setspace}														% set space be­tween lines
\usepackage[font=small,labelfont=bf,labelsep=endash,format=plain]{caption}	% change style of captions
\usepackage[section]{placeins}												% de­fines a \FloatBar­rier com­mand, be­yond which floats may not pass.
\usepackage[protrusion=true,expansion=true]{microtype}						% sublim­i­nal re­fine­ments to­wards ty­po­graph­i­cal per­fec­tion
\usepackage{enumerate} % better way to config enumerates
\usepackage{pdflscape} % landscape mode
\usepackage{afterpage}\addtokomafont{caption}{\small\linespread{1}\selectfont}					% Ändert Schriftgröße und Zeilenabstand bei captions

%graphics
\usepackage{graphicx}
\usepackage{xcolor}
\usepackage{subfig}		% subfigures
\usepackage{wrapfig}	% pro­duces fig­ures which text can flow around

%tikz
\usepackage{tikz}
\usetikzlibrary{trees}
\usetikzlibrary{decorations.pathmorphing}
\usetikzlibrary{decorations.markings}
\usetikzlibrary{positioning,arrows,shapes}

\tikzset{
    partial ellipse/.style args={#1:#2:#3}{
        insert path={+ (#1:#3) arc (#1:#2:#3)}
    }
}

%tables
\usepackage{tabularx}	% tab­u­lars with ad­justable-width columns
\usepackage{multirow}	% create tab­u­lar cells span­ning mul­ti­ple rows
\usepackage{booktabs}	% publi­ca­tion qual­ity ta­bles in LaTeX
\usepackage{array}
\usepackage{dcolumn}	% align on the dec­i­mal point of num­bers in tab­u­lar columns

%hyperref
\usepackage[pdftex]{hyperref}
\hypersetup{
	pdftitle={Kaluza-Klein-Theorie},
	pdfsubject={Cosmology},
	pdfkeywords={Kaluza, Klein, Differential Geometrie},
	pdfauthor={Michael Ruf},
	pdfcreator={Michael Ruf},
	pdfproducer={Michael Ruf},
	bookmarksnumbered=true, % bookmarks are numbered
	bookmarksopen=true,     % show bookmarks at start of pdf viewer
	bookmarksopenlevel=2,   % level to which the bookmarks are opened
	bookmarksdepth=2,       % depth of bookmarks 
	unicode=true,           % non-Latin characters in pdf viewer's bookmarks
	pdftoolbar=false,       % show pdf viewer's toolbar?
	pdfmenubar=true,        % show pdf viewer's menu?
	pdffitwindow=false,     % window fit to page when opened
	pdfstartview={FitH},    % fits the width of the page to the window
	pdfnewwindow=true,      % links in new window
	pdfborder={0 0 1},		% no border for links
	colorlinks=false,		% false: black links; true: colored links
	linkcolor=section_color,         % color of internal links (change box color with
	% linkbordercolor)
	citecolor=green,        % color of links to bibliography
	filecolor=magenta,      % color of file links
	urlcolor=blue			% color of external links
}
\usepackage{nameref}

%general stuff
\usepackage{cite}
\usepackage{glossaries}	% create glos­saries and lists of acronyms
\usepackage[nohyperlinks]{acronym}
\setcounter{tocdepth}{1} 

\newcommand{\HRule}{\rule{\linewidth}{0.5mm}}			% line for title page
\newcommand*\EmptyPage{\newpage\null\thispagestyle{empty}\newpage}

\addto\extrasenglish{
	\renewcommand{\chapterautorefname}{Chapter}
	\renewcommand{\sectionautorefname}{Chapter}
	\renewcommand{\subsectionautorefname}{Chapter}
 	\renewcommand{\subsubsectionautorefname}{Chapter}
}

\addto\extrasngerman{
	\renewcommand{\sectionautorefname}{Kapitel}
	\renewcommand{\subsectionautorefname}{Kapitel}
 	\renewcommand{\subsubsectionautorefname}{Kapitel}
}
\newcommand{\const}{const.}		


% general math

\newcommand{\transpose}{^\top}
\DeclareMathOperator{\tr}{Tr}
\DeclareMathOperator{\difD}{D}
\DeclareMathOperator{\diag}{diag}
\DeclareMathOperator{\re}{Re}
\DeclareMathOperator{\im}{Im}
\DeclareMathOperator{\id}{id}
\DeclareMathOperator{\vol}{vol}

% vector calculus

\DeclareMathOperator{\Div}{div}
\DeclareMathOperator{\Rot}{rot}
\DeclareMathOperator{\Grad}{grad}

% sets

\newcommand{\Reals}{\ensuremath{\mathbb{R}}}
\newcommand{\Complex}{\ensuremath{\mathbb{C}}}
\newcommand{\Integers}{\ensuremath{\mathbb{Z}}}
\newcommand{\Sphere}{\ensuremath{\mathbb{S}}}

% relativity

%\newcommand{\cSym}[3]{\ensuremath{\begin{Bmatrix} #1 \\ #2 #3 \end{Bmatrix}}}
\newcommand{\cSym}[3]{\ensuremath{\tensor*{\Gamma}{^{#1}_{#2}_{#3}}}}
\newcommand{\csym}[3]{\ensuremath{[#1 #2,\, #3]}}
\newcommand{\affin}[3]{\ensuremath{\Gamma^{#1}_{#2 #3}}}
\newcommand{\fourint}{\int\dif{}^4 x\,}
\newcommand{\lagrangian}{\mathcal{L}}
\newcommand{\liedif}[2]{\ensuremath{\mathcal{L}_{#1}#2}}
\newcommand{\pdif}[1]{\tensor{\partial}{#1}}
\newcommand{\cdif}[1]{\tensor{\partial}{#1}}

% area functions
\DeclareMathOperator{\artanh}{artanh}
\DeclareMathOperator{\arsinh}{arsinh}
\DeclareMathOperator{\arcosh}{arcosh}

% miscellaneous

\newcommand{\imI}{\ensuremath{\mathrm{i}}}
\newcommand{\landauO}{\mathcal{O}}
\newcommand{\name}[1]{\textsc{#1}}
\newcommand\grad{\ensuremath{^\circ}}
\let\originalleft\left  %fix bracket spacing when using \left( \right)
\let\originalright\right
\renewcommand{\left}{\mathopen{}\mathclose\bgroup\originalleft}
\renewcommand{\right}{\aftergroup\egroup\originalright}
\renewcommand{\vec}{\mathbf}
\newcommand\gmal{.}

%new environments

\newtheorem{theorem}{Theorem}
\newtheorem{lemma}{Lemma}
\newtheorem{satz}{Satz}
\newtheorem{proposition}{Proposition}
\newtheorem{korollar}{Korollar}

\theoremstyle{definition}
\newtheorem{definition}{Definition}

\theoremstyle{remark}
\newtheorem{bemerkung}{Bemerkung}
\newtheorem{beispiel}{Beispiel}

\newenvironment{tabulars}[1]{\renewcommand*{\arraystretch}{2}\tabular{#1}}{\endtabular}		% stretched table

%%% DOCUMENT %%%

\begin{document}
\selectlanguage{ngerman}
\onehalfspacing
\begin{titlepage}
\begin{center}
	\HRule \\[0.4cm]
	{ \huge \bfseries Kaluza Klein Theorie}\\
	\HRule \\[0.5cm]
	\begin{flushright}
  		\Large \textbf{Michael Ruf}\\[1cm]
  	\end{flushright}
  \vspace*{\fill}	
%  \includegraphics{logo.eps}\\
  \vspace*{\fill}
  \normalsize
  \textsc{Mathematisches Institut} \\
  \textsc{Albert-Ludwigs-Universität} \\
  \textsc{Freiburg im Breisgau} \\[1cm]
  
  \large Freiburg im Breisgau \\
  Mai 2016
\end{center}
\end{titlepage} 


\EmptyPage

\hypersetup{pageanchor=false} %stop page numbering (hyperref) to prevent for double page numers


\begin{titlepage}
\begin{center}
	\HRule \\[0.4cm]
	{ \huge \bfseries Kaluza Klein Theorie}\\
	\HRule \\[0.5cm]
	\begin{flushright}
  		\Large \textbf{Michael Ruf}\\[1cm]
  	\end{flushright}
  \vspace*{\fill}	
%  \includegraphics{logo.eps}\\
  \vspace*{\fill}
  \normalsize
  \textsc{Mathematisches Institut} \\
  \textsc{Albert-Ludwigs-Universität} \\
  \textsc{Freiburg im Breisgau} \\[1cm]
  
  \large Freiburg im Breisgau \\
  Mai 2016
\end{center}
\end{titlepage} 


\EmptyPage

\begin{titlepage}
\begin{center}
	\HRule \\[0.4cm]
	{ \huge \bfseries Kaluza-Klein-Theorie}\\
	\HRule \\[2cm]
	
	\textsc{\LARGE Bachelorarbeit}\\[1.5cm]
	
	\large Autor \\
  	\Large Michael \textsc{Ruf}\\[1.5cm]
  	
  	\large Betreuerin\\
  	\Large Prof. Dr. Katrin \textsc{Wendland}
  	\vfill
	\normalsize
	\textsc{Mathematisches Institut} \\
	\textsc{Albert-Ludwigs-Universität} \\
	\textsc{Freiburg im Breisgau} \\[2cm]
  
	\large Freiburg im Breisgau \\
	Mai 2016
\end{center}
\end{titlepage} 


\thispagestyle{empty}

\EmptyPage

\chapter*{Erklärung}
\vspace{2.5cm}
Hiermit versichere ich, die eingereichte Bachelorarbeit selbständig verfasst und 
keine anderen als die von mir angegebenen Quellen und Hilfsmittel benutzt zu haben. 
Wörtlich oder inhaltlich verwendete Quellen wurden entsprechend den anerkannten Regeln 
wissenschaftlichen Arbeitens (lege artis) zitiert. Ich erkläre weiterhin, dass die 
vorliegende Arbeit noch nicht anderweitig als Bachelorarbeit eingereicht wurde. 
\\[3.5cm]
\begin{tabular}{p{7cm}p{.5cm}l}
\dotfill \\ 
Ort, Datum
\end{tabular}
\vspace{1,5 cm} 
\begin{tabular}{p{7cm}p{.5cm}l}
\dotfill \\ 
Michael Ruf 
\end{tabular}
\thispagestyle{empty}

\pagenumbering{roman}
\setcounter{page}{1}
\cfoot[- \textit{\pagemark} -]{- \textit{\pagemark} -}

\tableofcontents
\newpage

\pagenumbering{arabic} 
\hypersetup{pageanchor=true} %start page numbering again
\setcounter{page}{1}
\cfoot[\pagemark]{\pagemark}


\chapter{Einleitung}
\section{Historischer Hintergrund}
Die Idee, dass die Welt in ihren Grundfesten geometrischer Natur ist
existiert ist nicht neu.
Beispielsweise führte Keppler die Abstände der Planetenbahnen auf
verschachtelte platonische Körper zurück. Auch manche Schulen der griechischen
Philosophie glaubten, dass Geometrie neben den Zahlen, das sei was das Universum
im wesentlichen ausmacht.
Selbsversändlich hatten sie dabei weder Differentialgeometrie noch die 
Relatitätstheorie im Sinn.  
Im laufe der Jahre löste sich die physikalischen Theorie immer weiter von dieser Idee. 
Zu Zeiten Kaluzas alle Kräfte durch Geometrie beschrieben
 

Mit der allgemeinen Relativitätstheorie gelang es Albert Einstein 1915
zumindest die Gravitation als eine von vier elementaren Kräften durch Geometrie
zu beschreiben. Die verbleibenden Kräfte, die schwache, die starke und die elektromagnetische Kraft
blieben grundsätzlich unangetastet. Es liegt Nahe, besonders da die Theorien
viele Ähnlichkeiten aufweisen, dass auch die anderen Kräfte durch eine ähnliche
Theorie beschreiben werden können.

Die neue Theorie soll dabei der alten 
möglichst ähnlich sein, also eine minimale Erweiterung der allgemeinen
Relativitätstheorie darstellen.
Da die Einsteingleichungen keine
zusätzlichen Freiheitsgrade enthalten, führt man heuristisch zusätzliche
Dimensionen ein, behält aber die Struktur der Gleichungen.
Da makroskopische Zusatzdimensionen aus verschiedenen Gründen ausgeschlossen
sind müssen die Zusatzdimensionen kompakt sein.
Zusammenfassend zeichnet sich eine solche Theorie durch folgende Punkte aus:
\begin{enumerate} 
\item Die Physik lässt sich vollständig durch Geometrie beschreiben.
\item Die uns zugängliche, vierdimensionale Raumzeit ist eingebettet in einen
höherdimensionalen Raum, welcher kompakte Zusatzdimensionen enthält.
\item Die Theorie ist eine minimale Erweiterung der allgemeinen
Relativitätstheorie.
\item Die Projektion auf vier Dimensionen liefert die uns bekannten Gesetze, mit
möglicherweise kleinen Abweichungen, welche sich auf den von beobachteten Skalen
nicht zeigen.
\end{enumerate}
\chapter{Allgemeine Begriffe}
 
\section{Konventionen}
Bereits Rechnungen in der "`gewöhnlichen"' Relativitätstheorie haben die Tendenz
unübersichtlich zu werden, das Einführen zusätzlicher Dimensionen hilft dabei
natürlich kaum.
Es ist deshalb an dieser Stelle nützlich einige Konventionen festzulegen. Im
Folgenden bezeichnet $n$ die Anzahl der Zusatzdimensionen.\footnote{Im Weiteren
meist $n=1$}

Die Metrik soll stets die Signatur $(1,3)$, bzw. $(1,3+n)$, insbesondere sind
die Zusatzdimensionen. Die meisten Rechnungen werden in lokalen Koordinaten
durchgeführt.
Wir beginnen die Indizierung bei 0, wobei die 0. Komponente mit der Zeit identifiziert wird.
Griechische Indices laufen über ersten vier Raum-Zeit Koordinaten
$\mu=0,\ldots,3\,$, lateinische über alle inklusive der
Zusatzkomponenten \footnote{Verwirrung bezüglich der gängigen Konvention
mit lateinischen Indices die Raumkomponenten zu benennen sollte dabei nicht
aufkommen.}, $i=0,\ldots,3+n\,$. Weiter verwenden wir die Einsteinsche
Summenkonvention, d.h. über paare von indices wird implizit summiert.

Im Zusammenhang mit Zusatzdimensionen tauchen Größen auf, die sowohl ein 4, als
auch ein $4+n$ dimensionales Pendant besitzen. Um diese von einander zu
unterscheiden, kennzeichnen wir die $4+n$ dimensionale Version mit einem
Zirkumflex. 
Beispielsweise bezeichnen wir den $4+n$ dimensionalen Ricci-Tensor mit
$\tensor{\hat{R}}{_i_j}$. Mit $g$ bezeichnen wir die Determinante der Metrik.
\subsection*{Geometrisierte Einheiten}
Wir verwenden geometrisierte Einheiten, d.h. Einheiten in denen
$8\pi G=c=1$. Es verbleibt nur noch eine Längendimension.
\section{Differentialgeometrie}
Wir wollen Manigfaltigkeiten betrachten, welche gewisse Symmetrien besitzen.
Anschaulich beschreibt eine Symmetrie die Invarianz eines Objektes unter einer
Operation.
Ein einfaches Beispiel für eine solche Symmetrie stellt der Zylinder
\begin{equation}
Z=\Reals\times\Sphere^1=\left\{(x,y,z)\in\Reals^3\,\Big|\,x^2+y^2=1\right\} 
\end{equation}
dar. Offensichtlich kann wird dieser durch Drehung um die $z$-Achse in sich
selbst überführt. 
Wir werden uns auf bestimmte Symmetrien beschränken, die
beispielsweise infinitisimal erzeugt werden können, dies schließt Spiegelungen
und andere diskrete Symmetrien aus.
Wir wollen dieses Konzept nun etwas formalisieren. Der Zylinder dient dabei als
Anschauungsobjekt um die abstrakten Begriffe zu veranschaulichen.
\subsection{Vektoren und Tensoren}
Riemann-Tensor, Ricci-Tensor, Geodätische.
\subsection{Gruppenwirkungen}
\subsection{Lie-Gruppen}
Eine Lie-Gruppe ist eine Gruppe, die zusätzlich eine Differenzierbare Struktur
besitzt.
In unserem Fall ist die Lie-Gruppe die die Operationen beschreibt die Gruppe der
Drehungen um die $z$-Achse. Diese wird mit $SO(2)$ bezeichnet, eine Darstellung
ergit sich beispielsweise durch Matrizen der Form
\begin{equation}
R(\alpha)=
\begin{pmatrix}
\cos\alpha&\sin\alpha&0\\
-\sin\alpha&\cos\alpha&0\\
0&0&1
\end{pmatrix}
\end{equation}
Das diese Gruppe eine differenzierbare Struktur besitz ist klar da die
Komponenten der Matrizen differenzierbar sind.
\begin{definition}[Lie Gruppe]

\end{definition}

Gruppen können auf Mengen operieren.  Um die Diskussion allgemeiner zu gestalten definieren wir
Operationen allgemeiner Gruppen. Die Hauptforderung an die Wirkung der Gruppe
auf der Menge ist, dass die Operation mit der Gruppenoperation kompatibel ist.
 \begin{definition}[Gruppenwirkung]
Sei $G$ eine Gruppe, $X$ eine Menge. Eine Abbildung
\begin{equation}
\Phi:G\times X\to X\,,\quad (g,x)\mapsto\Phi_g(x)
\end{equation}
heißt \emph{Gruppenwirkung} von $G$ auf $X$, falls die folgende Eigenschaften
erfüllt sind
\begin{enumerate}
  \item \emph{Identität}: für das neutrale Element $e\in G$ gilt
  $\displaystyle\Phi_e=\id_X$
  \item \emph{Verträglichkeit}: $\displaystyle\Phi_{gh}=\Phi_g\circ\Phi_h$
\end{enumerate}
$X$ heißt dann auch $G$-Menge.
\end{definition}
 Statt $\Phi_g(x)$ schreiben wir im folgenden kurz $g\gmal x$.
Ist die Gruppe $G$ eine Lie-Gruppe, $X=M$ eine Mannigfaltigkeit und $\Phi_g$
 glatt, so spricht man von einer \emph{Lie-Gruppenwirkuqng}. $M$ heißt dann auch
 $G$-Mannigfaltigkeit.
 \begin{definition}
 Sei $X$ eine $G$-Menge. Die Wirkung von $G$ heißt:
 \begin{enumerate}
   \item \emph{eigentlich}, falls unter der Abbildung $\Gamma_G:(g,x)\mapsto (g
   \gmal x,x)$, Urbilder kompakter Mengen kompakt sind
   \item \emph{(fixpunkt-)frei}, falls nur die Identität Fixpunkte besitz, d.h.
   $g\gmal x=x$, impliziert $g=e$.
 \end{enumerate}
 \end{definition}
 \begin{beispiel}[Wirkung von $\Reals/\mathbb{Z}$ auf einem Zylinder]
Sei $G=(\Reals/\mathbb{Z},+)$, $Z$ wie oben. Eine Wirkung von $G$ auf $Z$ ist
erklärt durch
\begin{equation}
t.\vec{x}= \begin{pmatrix}
\cos\left( 2\pi t\right)&\sin\left( 2\pi t\right)&0\\
-\sin\left( 2\pi t\right)&\cos\left( 2\pi t\right)&0\\
0&0&1
\end{pmatrix}\vec{x}\,.
\end{equation}
Wie man sich leicht klar macht ist die Wirkung glatt, frei und eigentlich sowie 
$Z$ eine $G$-Manigfaltigkeit.
%TODO eigentlich?
\end{beispiel}
\begin{figure}
\centering
\begin{tikzpicture}
\draw[-latex] (2,1)-- (-2,-1)  ;
\draw[-latex]  (-2,1)--(2,-1)  ;

 \draw[fill=white,draw=none] (-1,4) -- (-1,0) arc (180:360:1cm and 0.5cm) -- (1,4) arc (-180:0:-1cm and 0.5cm) ;
\draw[fill=white,draw=none] (0,4) ellipse (1cm and 0.5cm);
\draw[-,thick, dashed] (0,-0.5) -- (0,4) ;
\draw[dashed] (2,1)-- (-2,-1)  ;
\draw[dashed]  (-2,1)--(2,-1)  ;
\draw[-,thick] (0,-1) -- (0,-0.5) ;
\node at (0,0) {\tiny\textbullet};
\node at (0,4) {\tiny\textbullet};
  \draw[densely dashed] (-1,2) arc (180:0:1cm and 0.5cm);

  \draw[fill=gray,fill opacity = 0.1,draw=none] (0,4) ellipse (1cm and 0.5cm);
 
  \draw[fill=gray,fill opacity = 0.2] (-1,4) -- (-1,0) arc (180:360:1cm and 0.5cm) -- (1,4) arc (-180:0:-1cm and 0.5cm) ;
  \draw[densely dashed] (-1,0) arc (180:0:1cm and 0.5cm);
  \draw(-1,4) arc (180:0:1cm and 0.5cm);

  \draw[densely dashed]  (-1,2) arc (-180:0:1cm and 0.5cm);
  \draw[thick, red, -latex] (0,2) [partial ellipse=-60:-140:1cm and 0.5cm];
 %\draw[thick, blue!80!black,-latex] (0.51,1.57)-- (0.51,3);
  \node at (2,4) {$\mathbb{R}\times\mathbb{S}^1$};
  \node at (2+0.3,-1-0.2) {$y$};
  \node at (-2+0.3,-1-0.2) {$x$};
  \node at (0.3,5) {$z$};
   \node at (0.5,1.25) {$p$};
  \draw[-latex,thick] (0,4) -- (0,5) ;
  \node at (0.51,1.57){\textbullet};
;\end{tikzpicture}
\caption{Wirkung von $\Reals/\mathbb{Z}$ auf dem Zylinder $Z$.}
\end{figure}

 \subsection{Orbiträume}
\begin{definition}[Orbit]
Sei $X$ eine $G$-Menge, $x\in X$, dann heißt die Menge
\begin{equation}
G \gmal x=\{g \gmal x\,|\,g\in G\}
\end{equation}
Orbit von $x$.
\end{definition}
Als \emph{Orbitraum} bezeichnen wir den Quotienten $M/G$ einer
$G$-Mannigfaltigkeit $M$
\begin{equation}
M/G=\{[x]=G.x|x\in M\}\,.
\end{equation}
Dieser lässt sich mit einer natürlichen Quotiententopologie $\tau$ ausstatten,
die die gröbste Topologie ist, für die die kanonische Projektion $x\mapsto [x]$
stetig ist.
\begin{beispiel}
Wenn wir wieder das Beispiel $G=\Reals/\mathbb{Z}$, $M=Z$ heranziehen, so sind die Orbits gegeben durch 
\begin{equation}
G.(0,0,z)=\left\{(\sin (2\pi t),\sin (2\pi t),z)\,,t \in
[0,1)\right\}\cong\Sphere^1\,.
\end{equation}
Der Orbitraum $M/G$ ist diffeomorph zu $\mathrm{R}$. Der Zylinder lässt sich
also lokal als Produkt von Orbitraum und Orbit darstellen.
\end{beispiel}
\begin{figure}
\centering
\begin{tikzpicture}
\draw[-latex] (2,1)-- (-2,-1)  ;
\draw[-latex]  (-2,1)--(2,-1)  ;

 \draw[fill=white,draw=none] (-1,4) -- (-1,0) arc (180:360:1cm and 0.5cm) -- (1,4) arc (-180:0:-1cm and 0.5cm) ;
\draw[fill=white,draw=none] (0,4) ellipse (1cm and 0.5cm);
\draw[-,thick, dashed] (0,-0.5) -- (0,4) ;
\draw[dashed] (2,1)-- (-2,-1)  ;
\draw[dashed]  (-2,1)--(2,-1)  ;
\draw[-,thick] (0,-1) -- (0,-0.5) ;
\node at (0,0) {\tiny\textbullet};
\node at (0,4) {\tiny\textbullet};
  \draw[densely dashed,red,thick] (-1,2) arc (180:0:1cm and 0.5cm);

  \draw[fill=gray,fill opacity = 0.1,draw=none] (0,4) ellipse (1cm and 0.5cm);
 
  \draw[fill=gray,fill opacity = 0.2] (-1,4) -- (-1,0) arc (180:360:1cm and 0.5cm) -- (1,4) arc (-180:0:-1cm and 0.5cm) ;
  \draw[densely dashed] (-1,0) arc (180:0:1cm and 0.5cm);
  \draw(-1,4) arc (180:0:1cm and 0.5cm);

  \draw[red,thick]  (-1,2) arc (-180:0:1cm and 0.5cm);
 % \draw[thick, red, -latex] (0,2) [partial ellipse=-60:-140:1cm and 0.5cm];
 \draw[thick, blue!80!black] (0.51,-0.4)-- (0.51,3.55);
  \node at (2,4) {$\mathbb{R}\times\mathbb{S}^1$};
  \node[red] at (1.7,2) {$G.p$};
  \node[ blue!80!black] at (0.6,-0.9) {$M/G$};
  \node at (2+0.3,-1-0.2) {$y$};
  \node at (-2+0.3,-1-0.2) {$x$};
  \node at (0.3,5) {$z$};
   \node at (0.75,1.25) {$p$};
  \draw[-latex,thick] (0,4) -- (0,5) ;
  \node at (0.51,1.57){\textbullet};
;\end{tikzpicture}
\end{figure}
% \begin{figure}
% \centering
% \begin{tikzpicture}
% 
%    \draw[dashed] (1.3,-1.33) [partial ellipse= 90:270:0.5cm and 1cm];
%    \draw[dashed] (-1.3,-1.33) [partial ellipse=90:-90:0.5cm and 1cm];
%    \draw[thick, red,dashed] (-0,-1.5) [partial ellipse=270:90:0.4cm and 1cm];
%    \node[fill=white] at (0.7,-1.2) {$p$};
%   \node at (0.4,-1.5){\textbullet};
%    \node at (1.8,-1.3){\textbullet};
%    \node at (-1.8,-1.3){\textbullet};
% \fill[fill=gray,fill opacity = 0.2]  (-3.5,0) -- (0, 2.5)  -- (3.5,0);
% \fill[fill=gray,fill opacity = 0.2]  (-3.5,0) -- (0, -2.5)  -- (3.5,0);
% \draw[fill=gray,fill opacity = 0.2] (-3.5,0) .. controls (-3.5,2) and (-1.5,2.5) .. (0,2.5);
% \draw[xscale=-1,fill=gray,fill opacity = 0.2] (-3.5,0) .. controls (-3.5,2) and (-1.5,2.5) .. (0,2.5);
% \draw[rotate=180,fill=gray,fill opacity = 0.2] (-3.5,0) .. controls (-3.5,2) and (-1.5,2.5) .. (0,2.5);
% \draw[yscale=-1,fill=gray,fill opacity = 0.2] (-3.5,0) .. controls (-3.5,2) and (-1.5,2.5) .. (0,2.5);
% 
% \draw (-2,.2) .. controls (-1.5,-0.3) and (-1,-0.5) .. (0,-.5) .. controls (1,-0.5) and (1.5,-0.3) .. (2,0.2);
% 
% \draw[fill=white] (-1.75,0) .. controls (-1.5,0.3) and (-1,0.5) .. (0,.5) .. controls (1,0.5) and (1.5,0.3) .. (1.75,0);
% \draw[fill=white] (-1.75,0) .. controls (-1.5,-0.3) and (-1,-0.5) .. (0,-.5) .. controls (1,-0.5) and (1.5,-0.3) .. (1.75,0);
%   \draw[thick, red] (-0,-1.5) [partial ellipse=90:-90:0.4cm and 1cm];
%    \draw (1.3,-1.33) [partial ellipse=90:-90:0.5cm and 1.cm];
% 
%     \draw (-1.3,-1.33)  [partial ellipse=270:90:0.5cm and 1cm];
%     
%     \draw[dashed,  blue!80!black,thick] (-0,0) [partial ellipse=-180:0:3.5cm and 1.5cm];
%     \draw[thick,  blue!80!black] (-0,0) [partial ellipse=-110:-70:3.5cm and 1.5cm];
%   \node at (1,3) {$\mathbb{S}^1\times\mathbb{S}^1$};
% \end{tikzpicture}
% \end{figure}
Das dabei auch Probleme auftreten können zeigt folgendes 
\begin{beispiel}[Ein nicht Hausdorfscher Quotient]
Sei $G=(\Reals,+)$, $M=\Reals$ und $G$ wirke auf $M$ durch $t.x=e^tx$.
Der Quotient $M/G$ enthält die drei Äquivalenzklassen $[-1],[0],[1]$, da die
Abbildung das Vorzeichen nicht ändert.
Die Quotiententopologie $\tau$ lässt sich explizit angeben:
 \begin{equation}
 \tau =\big\{\emptyset,\{[-1]\},\{[1]\},\{[-1],[1]\},M/G\big\}\,.
 \end{equation} 
 Offensichtlich ist die einzige Menge die $[0]$ enthält $M/G$. Die Elemente
 $[0],[1]$ lassen sich also nicht durch offene Mengen trennen, d.h. $(M/G,\tau)$
 ist nicht Hausdorff.
 \begin{figure}
\centering
\begin{tikzpicture}

\draw [dashed](-3,0)--(-2,0);
\draw [dashed](3,0)--(2,0);
\draw (-2,0)--(-0.3,0);
\draw (2,0)--(0.3,0);
\node at (-1,-2){\textbullet};
\node at (0,-2){\textbullet};
\node at (1,-2){\textbullet};
\node at (4,0){$M=\mathbb{R}$};
\node at (4,-2){$M/G$};
%\draw[<-] (-1,-3)--(-0.8,0);
%\draw (-1,-3)--(-2.5,-2.5);
%\draw (1,-3)--(0,0) --(2,0) --cycle;
%\draw (0,-3)--(0,0)  --cycle;
\node at (0,0){\textbullet};
\node at (-1,-2.5){$[-1]$};
\node at (0,-2.5){$[0]$};
\node at (1,-2.5){$[1]$};
\node at (0,.5){$G.0$};
\node at (-1.3,.5){$G.(-1)$};
\node at (1.3,.5){$G.1$};
\node at (-0.3,0){$)$};
\node at (0.3,0){$($};
\end{tikzpicture}
\caption{Konstruktion eines nicht Hausdorffschen Quotients.}
\end{figure}
\end{beispiel} 
Tatsächlich scheitert die Konstruktion daran, dass die Wirkung nicht eigentlich
ist, beispielsweise ist das Urbild $\Gamma_G^{-1}\big([0,1]\times
\{1\}\big)=(-\infty,0]\times \{1\}$ nicht kompakt.
Tatsächlich ist die Forderung notwendig dafür das der Quotient Hausdorff ist.
%TODO REF

Wir geben nun eine Charakterisierung, die solche pathologische
Fälle auschließt. 
% \begin{proposition}
% Sei $M$ eine $G$-Manigfaltigkeit, $R:=\big\{(g,g.x)|g\in G x\in M\big\}$ dann
% ist $R$ genau dann eine abgeschlossene Untermanigfaltigkeit von $M\times M$,
% wenn $M/G$ eine glatte Maigfaltigkeitsstruktur besitzt, sodass $\pi:M\to M/G$
% eine Submersion ist.
% \end{proposition}
% \begin{proof}
% Siehe R.Abraham "`Foundations of Mechanics"' \cite{abraham1978foundations}.
% \end{proof}
\begin{theorem}[Slice Theorem]
Ist $G$ eigentlich, so besitzt $M/G$ eine  
\end{theorem}
% Das Slice Theorem liefert, das falls $G$ kompakt und fixpunktfrei ist $M/G$ eine
% Manigfaltigkeitstruktur besitzt, bzw. sogar $M\to M/G$ ein $G$-Hauptfaserbündel
% ist.
% %http://math.stackexchange.com/questions/1315445/quotient-manifold-theorem-provides-a-fibrations
% Da die Gruppenwirkung differenzierbar ist lasst sich auch die Orbitkarte 
% \begin{equation}
% \sigma_x:G\to M\,,\quad x\mapsto g \gmal x
% \end{equation}
% bei in der Identität differenzierbar. Man erhält so zu jedem $x\in M$ einen
% Vektor insgesammt erhält man ein Vektorfeld $V$ auf $M$.
Typen von Orbits. Maximale Orbits.
\begin{theorem}[Principal Orbit Theorem]
\end{theorem}
\begin{theorem}[Quotient Manifold Theorem]
\end{theorem}
\subsection{\ldots}
Im folgenden sei $G$ eine $d$-dimensionale kompakte Liegruppe, $E$ eine
$(4+d)$-dimensionale (pseudo-)riemannsche $G$-Manigfaltigkeit mit $G$
invarianter Metrik $g$. Weiter operiere $G$ frei auf $E$. Mit $M=E/G$ bezeichen
wir die $4$-dimensionale Untermanigfaltigkeit von $E$ die durch identifikation
der Orbits gegeben ist. Die kanonische Projektion $\pi:E\to M$ macht $E$ ein
$G$-Hauptfaserbündel mit typischer Faser $G$.
\subsection[Integration inv Fkt]{Integration $G$ invarianter Funktionen}
\begin{definition}
Sei $X$ eine $G$-Menge, $Y$-Menge, $f:X\to Y$ mit
\begin{equation}
f(g.p)=f(p)\,,\quad \forall g\in G\,,p\in E\,,
\end{equation}
dann heißt $f$ $G$-invariant.
\end{definition}
Sei $\pi: E\to M$ ein $G$-Hauptfaserbündel, ausgestattet mit einer
$G$-invarianten (pseudo-)riemanschen Metrik $g$, weiter sei
$f:E\to \mathrm{R}$, $G$-invariant und integrierbar. Eine Tivialisierung einer
offenen Menge $U\subset M=E/G$, ist ein Homoemorphismus
\begin{equation}
\Psi:\pi^{-1}(U)\to U\times G\,.
\end{equation}
Wir definieren eine Abbildung $\tilde{f}: M\to\mathrm{R}$ 
\begin{equation}
\tilde{f}\left(x\right):=\left(f\circ \Psi^{-1}\right)(x,e)\,.
\end{equation}
% Diese ist wohldefiniert, denn 
% \begin{equation}
% \begin{split}
% f\left(\Psi^{-1}(x,h)\right)&=f\left(\Psi^{-1}(x,h.e)\right)\\
% &=f\left(h.\Psi^{-1}(x,e)\right)\\
% &=f\left(\Psi^{-1}(x,e)\right)\,.
% \end{split}
% \end{equation}
Damit gilt 
\begin{equation}
\begin{split}
\int_{\pi^{-1}(U)}f(x)\dif x&=
\int_{\Psi\left(\pi^{-1}(U)\right)}\left[f\circ\Psi^{-1}\right](y,h)\dif y
\dif h\\
&=
\int_{U\times G}\left[f\circ\Psi^{-1}\right](y,h)\dif y
\dif h\\
&=\int_G\int_{U}\tilde{f}\left(y\right)\dif y
\dif h\\ 
&=\vol (G)\int_{U}\tilde{f}\left(y\right)\dif y\,,\label{eq:Intprod}
\end{split}
\end{equation}
wobei im zweiten Schritt der Satz von Fubini-Tonelli genutzt wurde, und
$\vol(G):=\int\dif h<\infty$ Eine anschauliche Betrachtung $\tilde{f}$ als
Mittelung der makroskopischen Funktion $f$ über die Zusatzdimension auffassen.
Phsikalisch liegt ein solcher Fall vor, falls die Auflösung einer Messung zu
gering ist um die kompakte Dimension zu beobachten. Bisher konnten kompakte
Zusatzdimensionen nicht beobachtet werden, neue Erkentnisse bringt
möglicherweise der 2015 gestarteten Lauf des Large Hadron Coliders am CERN. Die
lokalen Trivialisierungen überdecken $M$, damit lassen sich auch Integrale auf dem gesammten Raum auf
Integrale von der Form \eqref{eq:Intprod} zurückführen insbesondere gilt für die 
 $G$-invariante Funktion $f\sqrt{g}$ 
\begin{equation}
\int_{E}f(x)\sqrt{g(x)}\dif x=\vol
(G)\int_{M}\tilde{f}\left(y\right)\sqrt{\tilde{g}(y)}\dif y
\end{equation}
wobei $\tilde{g}(y)=\left[g\circ\Psi^{-1}\right](y,h)$. Im folgenden lassen wir
die Tilden weg und identifizieren die Ausdrücke stillschweigend miteinander.
\section{Variationsprinzip}
Feldtheorien lassen sich meist über 
Variationsprinzipien formulieren. Dies gilt insbesondere für
Elektromagnetismus und allgemeine Relativitätstheorie.
\subsection{Lagrange Dichte}
Wald\cite{wald2010general} S.454 ff.
\begin{definition}[Tensordichte]
Eine Tensordichte $\mathcal{T}$ vom Gewicht $w$
\end{definition}
Wirkungsintegral, Lagrangedichte, Lokalität
\begin{beispiel}[Levi-Civita-Symbol]
\end{beispiel}
\begin{definition}[Variationsableitung]
% es wird nach einem Feld gesucht nicht nach einem "`optimalen"' Weg, quasi
% unendlichdimensionale Form
% https://en.wikipedia.org/wiki/Lagrangian_system
\end{definition}
\subsubsection{Skalarfelder (Spin-0)}
\begin{equation}
\mathcal{L}\left[\phi(x),\nabla_\alpha\phi(x),x\right]=
\end{equation}
\subsubsection{Vektorfelder (Spin-1)}
\begin{equation}
\mathcal{L}\left[A_\alpha(x),\nabla_\beta
A_\alpha(x),x\right]=-\frac{1}{4}\tensor{F}{_\mu_\nu}\tensor{F}{^\mu^\nu}\,.
\end{equation}
%https://en.wikipedia.org/wiki/Euler%E2%80%93Lagrange_equation
%https://en.wikipedia.org/wiki/Functional_derivative#cite_note-2

% Fibred manifold vs principal bundle (spezialfall\ldots)
% In topology, the words fiber (Faser in German) and fiber space (gefaserter Raum)
% appeared for the first time in a paper by Seifert in 1932
% Seifert, H. (1932). "Topologie dreidimensionaler geschlossener Räume". Acta
% Math. (in French) 60: 147–238. doi:10.1007/bf02398271.

\chapter{Maxwell-Einstein Theorie}
Dieser Teil soll die physikalischen Theorien vorstellen, die der
Kaluza-Klein-Theorie zu Grunde liegen. Dazu zählt neben der 
in ihrer heutigen Formulierung auf Maxwell zurückgehenden 
Elektrodynamik, Einsteins allgemeine Relativitätstheorie. 
Deutlich älter ist die Elektrodynamik, deren fundamentales Gesetz, die Konstanz
der Lichtgeschwindigkeit, die Grundlage der speziellen Relativitätstheorie
darstellt.
Die Ähnlichkeit beider Theorien lässt hoffen, dass eine 
übergeordnete Theorie beide als Spezialfälle enthält. 
\section{Elektrodynamik}
Die Elektrodynamik beschreibt das Wechselspiel von elektrischen und 
magnetischen Feldern. Die Phänomenologie lässt sich mittels zweier glatter Vektorfelder
$\vec{E},\vec{B}:\Reals^3\times\Reals\to \Reals^3$ beschreiben.
Diese Felder erfüllen die auf Maxwell zurückgehenden Gleichungen 
\begin{equation}
  \begin{alignedat}{2}
    \Div\vec{E} &= \rho\,,  & \qquad \Rot\vec{E}+\dpd{\vec{B}}{t}&
    =0\,,\\
    \Div\vec{B} &= 0\,,& \qquad\Rot\vec{B}-\dpd{\vec{E}}{t}&
    =\vec{j}\,.
  \end{alignedat}
\end{equation}
Zusätzlich muss eine weitere Gleichung hinzugefügt werden, die die Dynamik von
Testteilchen\footnote{Ein Testteilchen ist ein idealisiertes Objekt, welches
selbst die wirkenden Felder nicht beeinflusst.} beschreibt
\begin{equation}
m\ddot{\vec{x}}=\vec{F}(\vec{x},\dot{\vec{x}},t)
=q\left[\vec{E}(\vec{x},t)+\dot{\vec{x}}\times\vec{B}(\vec{x},t)\right]\,,
\end{equation}
dabei ist $\vec{F}$ die so genannte \emph{Lorentzkraft} und $q$, $m$ die
Ladung bzw. Masse des Teilchens.
Es lässt sich mittels des Helmholtz-Theorems und der Form der Maxwell-Gleichung
die Existenz von Potentialen $\vec{A}$, $\Phi$ folgern, sodass
\begin{equation}
\vec{B}=\Rot \vec{A}\,,\quad
\vec{E}=-\Grad\Phi+\dpd{\vec{A}}{t}\,.\label{eq:Vecpot}
\end{equation}
Die Elektrodynamik ist eng mit der speziellen Relativitätstheorie verbunden.
Diese lässt sich am besten mit Objekten der Differentialgeometrie beschreiben.
Die Potentiale werden als Komponenten
$\tensor{A}{^\mu}=(\Phi,\vec{A})\transpose$ einer 1-Form
$A=\tensor{A}{_\mu}\dif\tensor{x}{^\mu}$ aufgefasst. Der Elektromagnetische Feldstärketensor ist definiert als
\begin{equation}
F:=\dif
A\,,
\end{equation}
bzw. in lokalen Koordinaten
\begin{equation}
\tensor{F}{_\mu_\nu}:=\partial_\mu\tensor{A}{_\nu}-\partial_\nu\tensor{A}{_\mu}\,.
\end{equation}
Die physikalischen Felder lassen sich aus dem Feldstärketensor wiederum durch 
\begin{equation}
\vec{E}_{i}=\tensor{F}{_0_i}\,,\quad
\vec{B}_i=\frac{1}{2}\tensor{\varepsilon}{^i^j^k}\tensor{F}{^j^k}
\end{equation}
zurückgewinnen, wie man sich leicht mit Hilfe der Definition von $F$ und
\eqref{eq:Vecpot} nachrechnet.
Damit reduzieren sich die Maxwell-Gleichungen zu zwei
Gleichungen
% \begin{align} 
% \pdif{_\mu}\tensor{F}{^\mu^\nu}&=\tensor{J}{^\nu}
% \label{eq:MaxwellInhom}\\
% \pdif{_{\mu}}\tensor{F}{_\nu_{\rho}}+
% \pdif{_{\mu}}\tensor{F}{_\rho_{\nu}}+
% \pdif{_{\nu}}\tensor{F}{_\mu_{\rho}}&=0
% \label{eq:MaxwellHom}
% \end{align}
\begin{align}
\pdif{_\mu}\tensor{F}{^\mu^\nu}&=\tensor{J}{^\nu}
\label{eq:MaxwellInhom}\,,\\
\tensor{\varepsilon}{^\alpha^\beta^\gamma^\delta}
\pdif{_{\alpha}}\tensor{F}{_\gamma_{\delta}}&=0\,.
\label{eq:MaxwellHom}
\end{align}
Bzw in indexfreier Notation
\begin{equation}
\quad \dif\star F= J\,,\quad\dif F = 0\,.
\end{equation}
% MW2= Jakobi?
% Die Lagrange Dichte für das Elektromagnetische Feld im Vakuum $\tensor{J}{^\mu}=0$
% \begin{equation}
% \mathcal{L}\left[\partial_\mu
% A(x),x\right]=-\frac{1}{4}\tensor{F}{_\mu_\nu}\tensor{F}{^\mu^\nu} \,.
% \end{equation}
% Wobei Beiträge die die Dynamik von Spin-$\nicefrac{1}{2}$ beschreiben nicht
% berücksichtigt wurden. 
\subsection{Eichtransformationen}
\section{Allgemeine Relativitätstheorie}
\subsection{Einsteinsche Feldgleichungen}
Die Einsteinschen Feldgleichungen lauten in lokalen Koordinaten\footnote{Der
Term proportional zur kosmologischen Konstante $\Lambda$ hat über kleine Distanzen
vernachlässigbaren Einfluss und wird im Folgenden nicht berücksichtigt.}
\begin{equation}
\tensor{R}{_\mu_\nu} - \frac{R}{2}\, \tensor{g}{_\mu_\nu}
+\Lambda\, \tensor{g}{_\mu_\nu}
=\tensor{T}{_\mu_\nu}\,.
\end{equation}
Der Tensor $\tensor{T}{_\mu_\nu}$ heißt \emph{Energie-Impuls-Tensor} und
enthält Information über die Materie, d.h. über die Felder im Raum. Eine
besonders einfache Form nehmen die Gleichungen im Vakuum an, denn dann gilt 
$\tensor{T}{_\mu_\nu}=0$.
Bilden wir die Spur der Einsteingleichungen, so finden wir 
\begin{equation}
0=R- \frac{R}{2}\, \tensor{g}{^\mu_\mu}= -R\,,
\end{equation}
die Skalarkrümmung $R$ verschwindet.
Einsetzen in die Einsteingleichungen liefert wiederum
\begin{equation}
\tensor{R}{_\mu_\nu}=0\,.
\end{equation}
\subsection{Variationsprinzip}
% TODO So formulieren Das S: Rank2sym->R problem S-> Max und
% |_\alphS(g+\alphah)=0 für alle Testfunktionen. Rechenregeln
Eine elegante Ableitung der Einsteingleichungen ergibt sich mithilfe eines auf
Hilbert zurückgehenden Variationsprinzips. 
Die Idee lautet: wähle die Metrik $g$ als kritischen
Punkt eines Funktionales, dass nur von der Krümmung abhängt. Das
einfachste solche Funktional ist die so genannte
\emph{Einstein-Hilbert-Wirkung}
\begin{equation}
S[g]=\int_{M}\sqrt{-g}R[g] \dif{}^4x \,.
\end{equation}
Bei der Variation dieses Funktionales treten verschiedene Terme auf. Wir
betrachten zunächst den Volumenfaktor $\sqrt{-g}$
\begin{equation}
\delta g = g  \tensor{g}{^\mu^\nu}\delta\tensor{g}{_\mu_\nu}
\end{equation}
\begin{equation}
\begin{split}
\delta \sqrt{-g} 
&= -\frac{1}{2\sqrt{-g}}\delta g \\
&= \frac{1}{2} \sqrt{-g} \left(\tensor{g}{^\mu^\nu} \delta
\tensor{g}{_\mu_\nu}\right)\\
&= -\frac{1}{2} \sqrt{-g} \left(\tensor{g}{_\mu_\nu} \delta
\tensor{g}{^\mu^\nu}\right)
\end{split}
\end{equation}
Um die Variation des Krümmungskalar zu berechnen wähle man zweckmäßigerweise
ein Riemannsches Normalkoordinatensystem\footnote{Ein Koordinatensystem in dem
$\cSym{\rho}{\mu}{\sigma}=0$ im betrachteten Punkt},
\begin{equation}
\begin{split}
\delta \tensor{R}{^\rho_\mu_\nu_\sigma}
&=\delta \tensor{\partial}{_\nu}\cSym{\rho}{\mu}{\sigma}
-\delta \tensor{\partial}{_\mu}\cSym{\rho}{\nu}{\sigma}\\
&=\tensor{\partial}{_\nu}\delta \cSym{\rho}{\mu}{\sigma}
-\tensor{\partial}{_\mu}\delta \cSym{\rho}{\nu}{\sigma}\\
\end{split}
\end{equation}
\begin{equation}
\begin{split}
\delta \tensor{R}{_\mu_\nu}
&=\delta \tensor{R}{^\rho_\mu_\rho_\nu}\\
&=\tensor{\partial}{_\rho}\delta \cSym{\rho}{\mu}{\nu}
-\tensor{\partial}{_\mu}\delta \cSym{\rho}{\rho}{\nu}\\
&=\tensor{\nabla}{_\rho}\delta \cSym{\rho}{\mu}{\nu}
-\tensor{\nabla}{_\mu}\delta \cSym{\rho}{\rho}{\nu}\\
\end{split}
\end{equation}
Die Gleichung ist koordinatenunabhängig, da sie eine Tensor Gleichung ist.
Weiter berechnet man 
\begin{equation}
\begin{split}
\delta R &=\delta \left(\tensor{g}{^\mu^\nu}\tensor{R}{_\mu_\nu}\right)\\
&=\tensor{R}{_\mu_\nu}\delta\tensor{g}{^\mu^\nu}
+\tensor{g}{^\mu^\nu}\delta\tensor{R}{_\mu_\nu}\,.
\end{split}
\end{equation}
Damit ist die Variation des EH-Funktionales gegeben als
\begin{equation}
\begin{split}
\delta S
&=\fourint \left[
R\delta\sqrt{-g}+\sqrt{-g}\delta R\right]\\
&=\fourint \left[
\frac{1}{2}\sqrt{-g}\tensor{g}{^\mu^\nu}\delta\tensor{g}{_\mu_\nu}
R+\sqrt{-g}\left(\tensor{R}{_\mu_\nu}\delta\tensor{g}{^\mu^\nu}
+\tensor{g}{^\mu^\nu}\delta\tensor{R}{_\mu_\nu}\right)\right]\\
&=\fourint \sqrt{-g}\left(
\frac{1}{2}\tensor{g}{^\mu^\nu}
R+\tensor{R}{^\mu^\nu}\right)\delta\tensor{g}{_\mu_\nu}
+\fourint \sqrt{-g}\tensor{g}{^\mu^\nu}\delta\tensor{R}{_\mu_\nu}
\,.
\end{split}
\end{equation}
Das zweite Integral liefert nur einen Oberflächenterm:
\begin{equation}
\begin{split}
\fourint \sqrt{-g}\tensor{g}{^\mu^\nu}\delta\tensor{R}{_\mu_\nu}
&=\fourint \sqrt{-g}\tensor{g}{^\mu^\nu}\left(\tensor{\nabla}{_\rho}\delta
\cSym{\rho}{\mu}{\nu} -\tensor{\nabla}{_\mu}\delta \cSym{\rho}{\rho}{\nu}\right)
\\
&=\fourint \tensor{\nabla}{_\rho}\left(\sqrt{-g}\tensor{g}{^\mu^\nu}\delta
\cSym{\rho}{\mu}{\nu}\right)-\int\dif{}^4 x \tensor{\nabla}{_\mu}\left(\sqrt{-g}\tensor{g}{^\mu^\nu}\delta
\cSym{\rho}{\rho}{\nu}\right)\,.
\end{split}
\end{equation}
Damit verbleibt 
\begin{equation}
\begin{split}
\delta S\textsubscript{EH}
&=\fourint \sqrt{-g}\left(
\frac{1}{2}\tensor{g}{^\mu^\nu}
R+\tensor{R}{^\mu^\nu}\right)\delta\tensor{g}{_\mu_\nu}\,.
\end{split}
\end{equation}
Wir können also die Variation nach der Metrik berechnen 
\begin{equation}
\frac{\delta
S\textsubscript{EH}[\tensor{g}{_\mu_\nu}(x)]}{\delta\tensor{g}{_\mu_\nu}(x')}
=\sqrt{-g}\left(\frac{1}{2}\tensor{g}{^\mu^\nu}
R+\tensor{R}{^\mu^\nu}\right)
\end{equation}
Die Extremalforderung impliziert
\begin{equation}
\tensor{R}{^\mu^\nu}+\frac{R}{2}\tensor{g}{^\mu^\nu}=0\, .
\end{equation}
Die Einsteingleichungen lassen sich also aus dem Variationsprinzip herleiten.
\subsection{Skalar-Tensor Theorien}
Brans-Dicke-Theorie
\section{Gemeinsamkeiten und Unterschiede}
Unterschiede:
\begin{itemize}
  	\item Lorentz-Kraft muss postuliert werden. Reine ART enthält keine Kräfte,
 	alle Teilchen bewegen sich frei auf Geodätischen.
	\item Ladung hat zwei Vorzeichen, d.h. es kann mittels umgekehrter Ladung
	entschieden werden ob man frei fällt oder nicht
	\item Die Einsteingleichungen sind nichtlinear und zweiter Ordnung die
	Maxwellgleichungen linear und erster Ordnung
	\item Kraft vs Geometrie
\end{itemize}
Gemeinsamkeiten:
\begin{itemize}
  	\item Diffeomorphismeninvarianz/Eichinvarianz
	\item In der nichtrelativistischen Näherung $1/r^2$ Gesetz.
	\item Massenloses bosonisches Vermittlerteilchen.
	\item Lagrangeformulierung
\end{itemize}
Insgesamt lässt sich hoffen das eine vereinheitlichte Theorie so formuliert
werden kann das sie einerseits die Ähnlichkeiten erklärt und auf gemeinsame
Ursachen zurückführt, andererseits aber auch Gründe für die Unterschiede gibt
bzw. diese möglicherweise ausräumt.


\chapter{Kaluzas erste Schritte}
Wie bereits in den vorherigen Kapiteln klar wurde, weißt die Struktur von
Elektromagnetismus und allgemeiner Relativitätstheorie deutliche Parallelen auf.
Dies wurde bereits in den ersten Jahren nach der Entwicklung der
allgemeinen Relativitätstheorie untersucht und mündete
unter anderem in Theorien von \name{H. Weyl}\cite{weyl1918gravitation} und
\name{G. Nordström} \cite{nordstrom1914moglichkeit}.

Dem Finnen \name{Nordström} gebührt dabei die Ehre, als Erster den
Elektromagnetismus als Phänomen einer fünfdimensionalen Raumzeit zu deuten.
Dabei baute seine Theorie auf der \name{Nordström}schen Gravitationstheorie
auf, welche allerdings verworfen wurde, da Voraussagen macht, welche nicht in
Einklang mit den Beobachtungen stehen\footnote{Beispielsweise drastische
Abweichungen in der gravitative Zeitverschiebung und der Periheldrehung des
Merkur.} .
Nicht zuletzt deshalb fand die Theorie seinerzeit wenig Beachtung, ist heute
allerdings weiterhin von theoretischem Interesse.
\name{Kaluza} bezieht sich vor allem auf \name{Weyl}s Arbeit, die aber ihrem
Charakter nach deutlich verschieden von der \name{Kaluza}-Theorie ist.
\section{Kaluza-Theorie}
\name{Kaluza} erkennt die Verwandtschaft der Theorien an der Form der
Christoffelsymbolen. Dazu bemerkt er\cite{kaluza1921unitatsproblem}:
%zu welcher er bei dem Vortrag \enquote{zum Unitätsproblem der Physik}
\begin{quote}
Die Rotorform der elektromagnetischen Feldkomponenten
$\tensor{F}{_\mu_\nu}$, noch mehr aber das
unverkennbare formale Entsprechen im Bau der Gravitations- 
und der elektromagnetischen Gleichungen fordern förmlich
die Vermutung heraus, die Feldkomponenten könnten so etwas wie
verstümmelte Dreizeigergrößen [Christoffelsymbole] sein.
\end{quote}
Er bezieht sich dabei darauf das die Gestalt der Christoffelsymbole erster Art
\begin{equation}
\tensor*{\Gamma}{_\lambda_\mu_\nu}=\frac{1}{2}
\left(\tensor{\partial}{_\nu} \tensor{g}{_\lambda_\mu} +
\tensor{\partial}{_\mu}
\tensor{g}{_\lambda_\nu} - \tensor{\partial}{_\lambda} 
\tensor{g}{_\mu_\nu}\right)\,,
\end{equation}
welche formal identisch zu 
\begin{equation}
\tensor{F}{_\mu_\nu}=\partial_\mu\tensor{A}{_\nu}-\partial_\nu\tensor{A}{_\mu}\,.
\end{equation}
wären, falls einer der Terme vom Typ $\tensor{\partial}{_\nu}
\tensor{g}{_\lambda_\mu}$ wegfiele.

Um die Gravitation mit der Elektrodynamik zu vereinigen werden aber zunächst 
zusätzliche Freiheitsgrade benötigt, die bereits in $\tensor{g}{_\mu_\nu}$
enthaltenen sind für die Gravitation verbraucht.
Statt des üblichen Minkowski Raums $\Reals^4$ (bzw. einer entsprechenden
Riemannschen Manigfaltigkeit), führen wir eine zusätzliche fünfte Raumdimension
ein sodass wir lokal homöomorph zu $\Reals^5$ sind. 
Die neue Koordinate benennen wir $\tensor{x}{^4}$\footnote{Kaluza
bezeichnet die Zusatzkomponente mit $x^0$, was bei uns für die Zeitkomponente
vorbehalten ist.}.
Dadurch stehen zusätzlich fünf
unabhängige Freiheitsgrade $\tensor{\hat{g}}{_4_i}$ zur Verfügung. Da eine
fünften Raumdimension nicht beobachtet wird, fordert Kaluza zusätzlich das die
Metrik nicht von der neuen Koordinate abhängt
\begin{equation}
\tensor{\partial}{_4} \tensor{\hat{g}}{_i_j}=0\,,\label{eq:Zylinderbed}
\end{equation}
die so genannte \emph{Zylinderbedingung}.
%TODO Erklären wieso die so heißt
Es mag jetzt zunächst so erscheinen als ob die Zusatzdimension dadurch überhaupt
keinen Einfluss mehr auf die Physik hätte. Das dem nicht so ist zeigt sich im
Folgenden.

Kommen wir zu den Christoffelsymbolen zurück, die 
Ausgangspunkt unserer Überlegungen waren. In fünf Dimensionen haben diese die
Gestalt
\begin{equation}
\tensor*{\hat{\Gamma}}{_i_j_k}=\frac{1}{2}
\left(\tensor{\partial}{_i}
\tensor{\hat{g}}{_k_j}+\tensor{\partial}{_j} \tensor{\hat{g}}{_k_i} -
\tensor{\partial}{_k} \tensor{\hat{g}}{_i_j}\right)\, .
\end{equation}
% \begin{equation}
% \tensor*{\Gamma}{_\lambda_\mu_\nu}=\frac{1}{2}
% \left(\tensor{\partial}{_\lambda} \tensor{g}{_\mu_\nu} + \tensor{\partial}{_\mu} \tensor{g}{_\lambda_\nu} -
% \tensor{\partial}{_\nu} \tensor{g}{_\lambda_\mu}\right)
% \end{equation}
Mit der Zylinderbedingung \eqref{eq:Zylinderbed} findet man 
\begin{equation}
\begin{split}
\tensor*{\hat{\Gamma}}{_4_\mu_\nu}&=\frac{1}{2}
	\left(\tensor{\partial}{_4}\tensor{\hat{g}}{_\mu_\nu} 
+ 	\tensor{\partial}{_\mu}\tensor{\hat{g}}{_\nu_4} 
- 	\tensor{\partial}{_\nu}\tensor{\hat{g}}{_4_\mu}\right)\\
&=\frac{1}{2}
\left(\tensor{\partial}{_\mu}
\tensor{\hat{g}}{_4_\nu} - \tensor{\partial}{_\nu}
\tensor{\hat{g}}{_4_\mu}\right)\,,
\end{split}
\end{equation}
also tatsächlich die gewünschte \enquote{Rotorform}. 
Für die verbleibenden unabhängigen Komponenten ergibt sich analog
\begin{align}
\tensor*{\hat{\Gamma}}{_\mu_\nu_4}&=\frac{1}{2}
\left(\tensor{\partial}{_\mu}
\tensor{\hat{g}}{_4_\nu}+\tensor{\partial}{_\nu}
\tensor{\hat{g}}{_4_\mu}\right)\,,\\
\tensor*{\hat{\Gamma}}{_4_\mu_4}&=-\tensor*{\hat{\Gamma}}{_\mu_4_4}=\frac{1}{2}\tensor{\partial}{_\mu}
\tensor{\hat{g}}{_4_4}\,,\\
\tensor*{\hat{\Gamma}}{_4_4_4}&= 0\,.
\end{align}
Es liegt nahe die neue
Feldkomponente
$\tensor{\hat{g}}{_4_\mu}$ mit
dem Viererpotential $ $ zu verknüpfen
\begin{equation}
\tensor{\hat{g}}{_4_\mu}=2\alpha\tensor{A}{_\mu}\,.
\end{equation}
Die Konstante $\alpha$ wird dabei erst später angepasst.
 Weiter benötigen wir neben der antisymmetrisieren
Ableitung $\tensor{F}{_\mu_\nu}=\tensor{\partial}{_{[\mu}}\tensor{A}{_{\nu]}}$, auch die
symmetrisierte Form $\tensor{\partial}{_{(\mu}}\tensor{A}{_{\nu)}}$, um alle
auftretenden Terme abzudecken. Dazu führen wir das \emph{Nebenfeld}
\begin{equation}
\tensor{H}{_\mu_\nu}:=\tensor{\partial}{_{(\mu}}\tensor{A}{_{\nu)}}=\partial_\mu\tensor{A}{_\nu}+\partial_\nu\tensor{A}{_\mu}
\end{equation}
ein. Zuletzt identifizieren wir die Größe $\tensor{\hat{g}}{_4_4}$  mit einem
Skalar\footnote{Bei Kaluza als \emph{Eckpotential} bezeichnet.}
\begin{equation}
\phi:=\frac{1}{2}\tensor{\hat{g}}{_4_4}\,.
\end{equation}
Die unabhängigen Christoffelsymbole sind damit von der Gestalt
% \begin{align}
% \tensor*{\hat{\Gamma}}{_\lambda_\mu_\nu}&=\tensor*{\Gamma}{_\lambda_\mu_\nu}\,,\\
% \tensor*{\hat{\Gamma}}{_\mu_\nu_4}&=-\alpha \tensor{H}{_\mu_\nu}\, ,\\
% \tensor*{\hat{\Gamma}}{_4_\mu_\nu}&=\alpha\tensor{F}{_\mu_\nu}\,,\\
% \tensor*{\hat{\Gamma}}{_4_4_\mu}&=-\tensor*{\hat{\Gamma}}{_4_\mu_4}
% =\tensor{\partial}{_\mu}\phi\,.
% \end{align}
% \begin{equation}
%     \tensor*{\hat{\Gamma}}{_\lambda_\mu_\nu}=\tensor*{\Gamma}{_\lambda_\mu_\nu}\,,
%     \quad\tensor*{\hat{\Gamma}}{_\mu_\nu_4}=-\alpha \tensor{H}{_\mu_\nu}\,,
%     \quad\tensor*{\hat{\Gamma}}{_4_\mu_\nu}=\alpha\tensor{F}{_\mu_\nu}\,,
%    \quad\tensor*{\hat{\Gamma}}{_4_4_\mu}=-\tensor*{\hat{\Gamma}}{_4_\mu_4}=\tensor{\partial}{_\mu}\phi\,.
% \end{equation}
\begin{equation}
  \begin{alignedat}{2}
    \tensor*{\hat{\Gamma}}{_\lambda_\mu_\nu}&=\tensor*{\Gamma}{_\lambda_\mu_\nu}\,,
    & \qquad \tensor*{\hat{\Gamma}}{_4_\mu_\nu}&=\alpha\tensor{F}{_\mu_\nu},\\
    \tensor*{\hat{\Gamma}}{_\mu_\nu_4}&=-\alpha \tensor{H}{_\mu_\nu}\,,&
    \qquad\quad\tensor*{\hat{\Gamma}}{_4_4_\mu}&=-\tensor*{\hat{\Gamma}}{_4_\mu_4}=\tensor{\partial}{_\mu}\phi\,.
  \end{alignedat}
\end{equation}
\begin{lemma}
\label{lemma:Christrel}
Die Christoffelsymbole erfüllen die Relation
\begin{equation}
\pdif{_m}\left(\tensor*{\hat{\Gamma}}{_i_k_\ell}+\tensor*{\hat{\Gamma}}{_k_\ell_i}+\tensor*{\hat{\Gamma}}{_\ell_i_k}\right)
=\tensor*{\hat{\Gamma}}{_m_i_k_{,\ell}}+\tensor*{\hat{\Gamma}}{_m_k_\ell_{,i}}+\tensor*{\hat{\Gamma}}{_m_\ell_i_{,k}}\,.
\end{equation}
\end{lemma}
\begin{proof}
\begin{equation*}
\begin{split}
\pdif{_m}\left(\tensor*{\hat{\Gamma}}{_i_k_\ell}+\tensor*{\hat{\Gamma}}{_k_\ell_i}+\tensor*{\hat{\Gamma}}{_\ell_i_k}\right)
&=\tensor*{\hat{\Gamma}}{_i_k_\ell_{,m}}+\tensor*{\hat{\Gamma}}{_k_\ell_i_{,m}}+\tensor*{\hat{\Gamma}}{_\ell_i_k_{,m}}\\
&=\frac{1}{2}
\left( \tensor{\hat{g}}{_\ell_i_{,km}} + \tensor{\hat{g}}{_\ell_k_{,im}}
-\tensor{\hat{g}}{_i_k_{,\ell m}}\right)\\
&\phantom{=}+\frac{1}{2}\left( \tensor{\hat{g}}{_i_\ell_{,km}} +
\tensor{\hat{g}}{_i_k_{,\ell m}} -\tensor{\hat{g}}{_k_\ell_{,im}}\right)\\
&\phantom{=}+\frac{1}{2}\left( \tensor{\hat{g}}{_k_\ell_{,im}} +
\tensor{\hat{g}}{_k_i_{,\ell m}} -\tensor{\hat{g}}{_i_\ell_{,km}}\right)\\
&=\frac{1}{2}
\left( \tensor{\hat{g}}{_\ell_i_{,km}} + \tensor{\hat{g}}{_\ell_k_{,im}}
+\tensor{\hat{g}}{_i_k_{,\ell m}}\right)\\
\end{split}
\end{equation*}
Durch Nulladdition von $\tensor{\hat{g}}{_k_m_{,i\ell}}$, $
\tensor{\hat{g}}{_i_m_{,\ell k}}$ und
$\tensor{\hat{g}}{_i_m_{,\ell k}}$ erhält man
\begin{equation}
\begin{split}
\pdif{_m}\left(\tensor*{\hat{\Gamma}}{_i_k_\ell}+\tensor*{\hat{\Gamma}}{_k_\ell_i}+\tensor*{\hat{\Gamma}}{_\ell_i_k}\right)
&=\frac{1}{2}
\left( \tensor{\hat{g}}{_k_m_{,i\ell}} + \tensor{\hat{g}}{_k_i_{,m\ell}}
-\tensor{\hat{g}}{_m_i_{,k\ell}}\right)\\
&\phantom{=}+\frac{1}{2}\left( \tensor{\hat{g}}{_\ell_m_{,ki}} +
\tensor{\hat{g}}{_\ell_k_{,mi}} -\tensor{\hat{g}}{_m_k_{,\ell i}}\right)\\
&\phantom{=}+\frac{1}{2}\left( \tensor{\hat{g}}{_i_m_{,\ell k}} +
\tensor{\hat{g}}{_i_\ell_{,mk}} -\tensor{\hat{g}}{_m_\ell_{,ik}}\right)\\
&=\tensor*{\hat{\Gamma}}{_m_i_k_{,\ell}}+\tensor*{\hat{\Gamma}}{_m_k_\ell_{,i}}+\tensor*{\hat{\Gamma}}{_m_\ell_i_{,k}}
\,. \qedhere
\end{split}
\end{equation}
\end{proof}
Lemma~\ref{lemma:Christrel} impliziert dann für $m=4$, zusammen mit der
Zylinderbedingung
\begin{equation}
0=\tensor*{\hat{\Gamma}}{_4_i_k_{,\ell}}+\tensor*{\hat{\Gamma}}{_4_k_\ell_{,i}}+\tensor*{\hat{\Gamma}}{_4_\ell_i_{,k}}\,.
\label{eq:fdchristrel}
\end{equation}
Setzt man $\ell = \lambda$, $i=\nu$, $k = \mu$, erhält man weiter 
\begin{equation}
\begin{split}
0&=\tensor*{\hat{\Gamma}}{_4_\nu_\mu_{,\lambda}}
+\tensor*{\hat{\Gamma}}{_4_\mu_\lambda_{,\nu}}
+\tensor*{\hat{\Gamma}}{_4_\lambda_\nu_{,\mu}}\\
&=\alpha\left(\tensor{F}{_\nu_\mu_{,\lambda}}
+\tensor{F}{_\mu_\lambda_{,\nu}}
+\tensor{F}{_\lambda_\nu_{,\mu}}\right)\,,
\end{split}
\end{equation}
die homogenen Maxwell-Gleichungen \eqref{eq:MaxwellHom}. Alle weiteren
Relationen, welche sich aus \eqref{eq:fdchristrel} ableiten lassen sind
trivial
\footnote{Zum einen erhält man falls zwei Indizes gleich vier sind
$0=\tensor{\phi}{_{,\nu\mu}}-\tensor{\phi}{_{,\mu\nu}}$, was aber nach dem
Satz von Schwarz für beliebige glatte $\phi$ erfüllt ist. Falls alle
Indizes gleich vier sind, verschwindet auch die rechte Seite der Gleichung.}.
%für
% $\ell = 4$, $i=\nu$, $k = \mu$ erhält man weiter
% \begin{equation}
% \begin{split}
% 0&=\tensor*{\hat{\Gamma}}{_4_\nu_\mu_{,4}}
% +\tensor*{\hat{\Gamma}}{_4_\mu_4_{,\nu}}
% +\tensor*{\hat{\Gamma}}{_4_4_\nu_{,\mu}}\\
% &=\tensor{\phi}{_{,\nu\mu}}-\tensor{\phi}{_{,\mu\nu}}
% \end{split}
% \end{equation}
% Was nach dem Satz von Schwartz erfüllt ist. Wenn mindestens zwei Indizes 4 sind
% ist die Gleichung mit der Zylinderbedingung ref erfüllt.

Wir wollen nun weiter überprüfen, welche Gleichungen sich aus den
fünfdimensionalen Analogien der Feld- und Geodätengleichungen ergeben.
Kaluza führte seine Berechnungen in der linearisierten Theorie durch, d.h. 
die auftretenden Energien sind klein.
%TODO wie klein
In der linearisierten Theorie gilt insbesondere
\begin{align}
\Gamma^2&\approx 0\tag{N1}\label{eq:N1}\,,\\
R&\approx 0\tag{N2}\label{eq:N2}\,,
\end{align}
wobei $\Gamma^2$ einen beliebigen, quadratischen Ausdruck in den
Christoffelsymbolen bezeichne.
Zur Berechnung des Ricci Tensors
$\tensor{\hat{R}}{_i_j}$ ist die folgende Formel nützlich:
\begin{equation}
\tensor{\hat{R}}{_i_j}=\tensor{\partial}{_l}\tensor*{\hat{\Gamma}}{^l_i_j}
-\tensor*{\hat{\Gamma}}{^m_i_l}\tensor*{\hat{\Gamma}}{^l_j_m}
-\tensor{\nabla}{_j}\left[\tensor{\partial}{_i}\left(\log\sqrt{-\hat{g}}\right)\right]\,
.\end{equation}
Insbesondere gilt aufgrund der Zylinderbedingung und der Näherungen
\eqref{eq:N1}`
%TODO wie schauts mit RNKS aus sind die resultierenden gleichungen tensoriell?
\begin{equation}
\tensor{\hat{R}}{_4_j}=\tensor{\partial}{_\ell}\tensor*{\hat{\Gamma}}{^\ell_4_j}
=\tensor{\partial}{^\ell}\tensor*{\hat{\Gamma}}{_4_j_\ell}
=\tensor{\partial}{^\lambda}\tensor*{\hat{\Gamma}}{_4_j_\lambda}\,.
\end{equation}
% \begin{equation}
% \tensor{\hat{R}}{_4_j}=\tensor{\partial}{_\lambda}\tensor*{\hat{\Gamma}}{^\lambda_4_j}
% -\tensor*{\hat{\Gamma}}{^m_4_l}\tensor*{\hat{\Gamma}}{^l_j_m}\,
% .\end{equation}
Damit berechnen wir für die zusätzlichen Komponenten von $\tensor{\hat{R}}{_i_j}$ 
% \begin{equation}
% \begin{split}
% \tensor{\hat{R}}{_4_4}
% &=\tensor{\partial}{_\lambda}\tensor*{\hat{\Gamma}}{^\lambda_4_4}
% -\tensor*{\hat{\Gamma}}{^m_4_l}\tensor*{\hat{\Gamma}}{^l_4_m}\\
% &=\tensor{\partial}{_\lambda}\tensor*{\hat{\Gamma}}{^\lambda_4_4}
% -\tensor*{\hat{\Gamma}}{^\mu_4_\lambda}\tensor*{\hat{\Gamma}}{^\lambda_4_\mu}
% -\tensor*{\hat{\Gamma}}{^\mu_4_4}\tensor*{\hat{\Gamma}}{^4_4_\mu}
% -\tensor*{\hat{\Gamma}}{^4_4_\lambda}\tensor*{\hat{\Gamma}}{^\lambda_4_4}
% \\
% &=-\tensor{\partial}{_\lambda}\tensor{\partial}{^\lambda}\phi
% -\alpha^2\tensor{F}{^\mu_\lambda}\tensor{F}{^\lambda_\mu}
% +\tensor{\partial}{^\mu}\phi\tensor{\partial}{_\mu}\phi
% +\tensor{\partial}{_\lambda}\phi\tensor{\partial}{^\lambda}\phi\\
% &=-\alpha^2\tensor{F}{_\mu_\nu}\tensor{F}{^\mu^\nu}-\square\phi
% +2\tensor{\partial}{^\mu}\phi\tensor{\partial}{_\mu}\phi
% \end{split}
% \end{equation}
% \begin{equation}
% \begin{split}
% \tensor{\hat{R}}{_4_\nu}
% &=\tensor{\partial}{_\lambda}\tensor*{\hat{\Gamma}}{^\lambda_4_\nu}
% -\tensor*{\hat{\Gamma}}{^m_4_l}\tensor*{\hat{\Gamma}}{^l_\nu_m}\\
% &=\tensor{\partial}{_\lambda}\tensor*{\hat{\Gamma}}{^\lambda_4_\nu}
% -\tensor*{\hat{\Gamma}}{^\mu_4_\lambda}\tensor*{\hat{\Gamma}}{^\lambda_\nu_\mu}
% -\tensor*{\hat{\Gamma}}{^\mu_4_4}\tensor*{\hat{\Gamma}}{^4_\nu_\mu}
% -\tensor*{\hat{\Gamma}}{^4_4_\lambda}\tensor*{\hat{\Gamma}}{^\lambda_\nu_4}\\
% &=-\alpha\tensor{\partial}{_\lambda}\tensor{F}{^\lambda_\nu}
% +\alpha\tensor{F}{^\mu_\lambda}\tensor*{\Gamma}{^\lambda_\nu_\mu}
% +\alpha\tensor{\partial}{^\mu}\phi\tensor{H}{_\nu_\mu}
% -\alpha\tensor{\partial}{_\lambda}\phi\tensor{F}{^\lambda_\nu}\\
% &=-\alpha\left(\tensor{\partial}{^\mu}\tensor{F}{_\mu_\nu}
% +\tensor*{\Gamma}{_\lambda_\mu_\nu}\tensor{F}{^\mu^\lambda}
% +2\tensor{\partial}{^\mu}\phi\tensor{\partial}{_\nu}\tensor{A}{_\mu}\right)
% \end{split}
% \end{equation}
\begin{equation}
\tensor{\hat{R}}{_4_4}
=\tensor{\partial}{^\lambda}\tensor*{\hat{\Gamma}}{_4_4_\lambda}
=\tensor{\partial}{^\lambda}\tensor{\partial}{_\lambda}\phi=\square\phi\,, 
\end{equation}
für die 44 Komponente, sowie 
\begin{equation}
\tensor{\hat{R}}{_4_\nu}
=\tensor{\partial}{^\lambda}\tensor*{\hat{\Gamma}}{_\lambda_4_\nu}
=-\alpha\tensor{\partial}{^\lambda}\tensor{F}{_\lambda_\nu}
=:-\alpha\tensor{J}{_\nu}\,.
\end{equation}
Trivialerweise gilt zudem $\tensor{\hat{R}}{_\mu_\nu}=\tensor{R}{_\mu_\nu}$. Die
, auf 5 Dimensionen verallgemeinerten, Feldgleichungen sind unter Näherung
\eqref{eq:N2} und mit Annahme verschwindender Kosmologischer Konstante
\begin{equation}
\kappa\tensor{\hat{T}}{_i_j}=\tensor{\hat{R}}{_i_j}\,.
\end{equation}
Damit ergeben sich zusätzlich zu den vierdimensionalen Einsteinschen
Feldgleichungen
\begin{equation}
\kappa\tensor{\hat{T}}{_4_4}=-\square\phi, \quad
\kappa\tensor{\hat{T}}{_4_\mu}=-\alpha\tensor{J}{_\mu}\,.\label{eq:ZusatzFG}
\end{equation}
Die $\tensor{\hat{T}}{_4_4}$ Komponente entspricht also der Masse des Feldes
$\phi$, $\tensor{\hat{T}}{_4_\mu}$ ist proportional zur Stromdichte.
 Wir definieren die Fünfergeschwindigkeit $\tensor{\hat{u}}{_i}$
\begin{equation}
\tensor{\hat{u}}{_i}:=\dod{\tensor{x}{_i}}{\hat{\tau}}\,,
\quad\dif\hat{\tau}^2=\tensor{\hat{g}}{_i_j}\dif\tensor{x}{^i}\dif\tensor{x}{^j}\,.
\end{equation}
Im Weiteren machen wir zusätzlich die Näherung, dass alle
Geschwindigkeiten\footnote{Im Fall von $\tensor{u}{_4}$ spricht man
besser von kleiner spezifischer Ladung.} klein sind, sprich
\begin{equation}
\tensor{\hat{u}}{_1},\tensor{\hat{u}}{_2},\tensor{\hat{u}}{_3},\tensor{\hat{u}}{_4}\ll 1
\,,\quad\tensor{\hat{u}}{_0}\approx 1\, .\tag{N3}\label{eq:N3}
\end{equation}
Wir beschreiben unser System durch Staub, d.h. wir vernachlässigen den Druck. 
Den zugehörigen Energie-Impuls-Tensor $\tensor{T}{_\mu_\nu}$
verallgemeinern wir auf fünf Dimensionen
\begin{equation}
\tensor{\hat{T}}{_i_j}=\mu_0\tensor{\hat{u}}{_i}\tensor{\hat{u}}{_j}\, ,
\end{equation}
wobei $\mu_0$ die Ruhemassendichte des Staubs bezeichnet. Für die
Zusatzkomponenten findet man 
\begin{equation}
\tensor{\hat{T}}{_4_\mu}
=\mu_0\tensor{\hat{u}}{_4}\tensor{\hat{u}}{_\mu}\approx\mu_0\tensor{\hat{u}}{_4}\tensor{u}{_\mu}\,,
\label{eq:StaubEMTens}
\end{equation}
da wegen \eqref{eq:N3} auch $\dif\hat{\tau}\approx\dif\tau$ gilt.
Dabei ist $\tensor{u}{_\mu}$ die gewöhnliche Vierergeschwindigkeit.
Setzt man jetzt \eqref{eq:ZusatzFG} und \eqref{eq:StaubEMTens} gleich so erhält
man
\begin{equation}
-\alpha\tensor{J}{_\mu}\approx\kappa\mu_0\tensor{\hat{u}}{_4}\tensor{u}{_\mu}
\,,
\end{equation}
und damit mit $\tensor{J}{_\mu}=\varrho_0\tensor{u}{_\mu}$
\begin{equation}
\varrho_0=-\frac{\kappa\mu_0}{\alpha}\tensor{\hat{u}}{_4}\,.\label{eq:murhorel}
\end{equation}
Wenden wir uns den Geodäten zu. Die verallgemeinerte Geodätengleichung
lautet
\begin{equation}
\od{\tensor{\hat{u}}{^\ell}}{\hat{\tau}}=-\tensor*{\hat{\Gamma}}{^\ell_m_n}\tensor{\hat{u}}{^m}\tensor{\hat{u}}{^n}
\, ,
\end{equation}
Da wir uns für die Bahnen in den uns zugänglichen vier Raumzeitdimensionen
interessieren ist, sind natürlich vor allem diese Komponenten interessant.
Hierbei gilt
\begin{equation}
\begin{split}
\dod{\tensor{\hat{u}}{^\lambda}}{\hat{\tau}}
+\tensor*{\hat{\Gamma}}{^\lambda_\mu_\nu}\tensor{\hat{u}}{^\mu}\tensor{\hat{u}}{^\nu}
&=
-\tensor*{\hat{\Gamma}}{^\lambda_m_n}\tensor{\hat{u}}{^m}\tensor{\hat{u}}{^n}
+\tensor*{\hat{\Gamma}}{^\lambda_\mu_\nu}\tensor{\hat{u}}{^\mu}\tensor{\hat{u}}{^\nu}\\
&=
-\tensor*{\hat{\Gamma}}{^\lambda_4_4}\tensor{\hat{u}}{^4}\tensor{\hat{u}}{^4}
-\tensor*{\hat{\Gamma}}{^\lambda_4_\nu}\tensor{\hat{u}}{^4}\tensor{\hat{u}}{^\nu}
-\tensor*{\hat{\Gamma}}{^\lambda_\mu_4}\tensor{\hat{u}}{^\mu}\tensor{\hat{u}}{^4}
-\tensor*{\hat{\Gamma}}{^\lambda_\mu_\nu}\tensor{\hat{u}}{^\mu}\tensor{\hat{u}}{^\nu}
+\tensor*{\hat{\Gamma}}{^\lambda_\mu_\nu}\tensor{\hat{u}}{^\mu}\tensor{\hat{u}}{^\nu}
\\
&=
\tensor{\partial}{_\lambda}\phi\left(\tensor{\hat{u}}{^4}\right)^2
+\alpha\tensor{F}{^\lambda_\nu}\tensor{\hat{u}}{^4}\tensor{\hat{u}}{^\nu}
+\alpha\tensor{F}{^\lambda_\mu}\tensor{\hat{u}}{^\mu}\tensor{\hat{u}}{^4}\\
&=
\tensor{\partial}{_\lambda}\phi\left(\tensor{\hat{u}}{^4}\right)^2
+2\alpha\tensor{F}{^\lambda_\nu}\tensor{\hat{u}}{^4}\tensor{\hat{u}}{^\nu}\,.
\end{split}
\end{equation}
Nach Voraussetzung ist $\tensor{\hat{u}}{^4}$ klein und damit
näherungsweise
\begin{equation}
\dod{\tensor{u}{^\lambda}}{\tau}
+\tensor*{\Gamma}{^\lambda_\mu_\nu}\tensor{u}{^\mu}\tensor{u}{^\nu}
=2\alpha\tensor{F}{^\lambda^\nu}\tensor{\hat{u}}{^4}\tensor{\hat{u}}{_\nu}\,.
\label{eq:KaluzaGeo}
\end{equation}
Damit sind wir am Ziel, denn der Parameter $\alpha$ ist weiterhin eine
freie Größe, wir setzen 
 $\alpha=\sqrt{\frac{\kappa}{2}}\approx 3,06\cdot 10^{-14}$ und
erhalten vermöge \eqref{eq:murhorel}
\begin{equation}
2\alpha\tensor{\hat{u}}{^4}=-
\frac{\varrho_0}{\mu_0}\,.
\end{equation}
Die Tatsache, dass $\alpha$ klein ist, legitimiert
die zuvor gemachten Näherungen im Nachhinein, denn Terme die quadratisch in den
Christoffelsymbolen sind sind auch quadratisch in $\alpha$.
Setzt man nun in \eqref{eq:KaluzaGeo} ein, so erhält man
\begin{equation}
\dod{\tensor{u}{^\lambda}}{\tau}
+\tensor*{\Gamma}{^\lambda_\mu_\nu}\tensor{u}{^\mu}\tensor{u}{^\nu}
=-\frac{\varrho_0}{\mu_0}\tensor{F}{^\lambda^\nu}\tensor{u}{_\nu}\, ,
\end{equation}
die Formel für die Geodäten mit Massendichte $\mu_0$ und Ladungsdichte
$\varrho_0$, unter Einwirkung der Lorentzkraft! Es verbleibt eine
Untersuchung der 4. Komponente
\begin{equation}
\begin{split}
\dod{\tensor{\hat{u}}{^4}}{\hat{\tau}}
&=
-\tensor*{\hat{\Gamma}}{^4_m_n}\tensor{\hat{u}}{^m}\tensor{\hat{u}}{^n}\\
&=
-\tensor*{\hat{\Gamma}}{^4_4_\nu}\tensor{\hat{u}}{^4}\tensor{\hat{u}}{^\nu}
-\tensor*{\hat{\Gamma}}{^4_\mu_4}\tensor{\hat{u}}{^\mu}\tensor{\hat{u}}{^4}
-\tensor*{\hat{\Gamma}}{^4_\mu_\nu}\tensor{\hat{u}}{^\mu}\tensor{\hat{u}}{^\nu}\\
&=
-\tensor*{\hat{\Gamma}}{^4_4_\nu}\tensor{\hat{u}}{^4}\tensor{\hat{u}}{^\nu}
-\tensor*{\hat{\Gamma}}{^4_\mu_4}\tensor{\hat{u}}{^\mu}\tensor{\hat{u}}{^4}
-\alpha\tensor{H}{_\mu_\nu}\tensor{\hat{u}}{^\mu}\tensor{\hat{u}}{^\nu}\\
&\approx
-\alpha\tensor{H}{_0_0}\\
&=
-2\alpha\tensor{A}{_0}\\
\end{split}
\end{equation}
aufgrund der Näherung ist $\dod{\tensor{\hat{u}}{^4}}{\hat{\tau}}\approx 0$ und
damit $\alpha$ konstant
\begin{equation}
\begin{split}
\tensor{\partial}{_0}\left(\frac{\varrho_0}{\mu_0}\right)
=-2\alpha\tensor{\partial}{_0}u_4=
4\alpha^2\tensor{A}{_0}
\end{split}
\end{equation}
\begin{bemerkung}
In den physikalisch relevanten Gleichungen taucht das Nebenfeld
$\tensor{H}{_\mu_\nu}$ nicht mehr auf. Eine Abhängigkeit der Gleichungen stünde
allerdings auch im Widerspruch zur Eichinvarianz.
\end{bemerkung}
\begin{bemerkung}
Wir verzichten darauf Indices mit der Metrik zu heben bzw. zu senken, da dies
nicht Wohldefiniert ist:
\begin{equation}
\tensor*{\delta}{*_\mu^\nu}=\tensor{\hat{g}}{_\mu_a}\tensor{\hat{g}}{^a^\nu}
=\tensor{g}{_\mu_\sigma}\tensor{g}{^\sigma^\nu}+\tensor{\hat{g}}{_\mu_5}\tensor{\hat{g}}{^5^\nu}
\end{equation}
\begin{equation}
\tensor{g}{_\mu_\sigma}\tensor{g}{^\sigma^\nu}
=\tensor*{\delta}{*_\mu^\nu}-\tensor{\hat{g}}{_\mu_5}\tensor{\hat{g}}{^5^\nu}\neq\tensor*{\delta}{*_\mu^\nu}
\end{equation}
\end{bemerkung}
\subsection{Erfolg}
Die Ableitung Kaluzas hat Charme. Insbesondere, dass er die
Lorentzkraft den fünfdimensionalen Einsteingleichungen folgern kann ist sehr 
befriedigend, da weniger Annahmen gemacht werden müssen als in der klassischen
Theorie. Die Annahme das die Spezifische Ladung klein ist ist schon für relativ
generische Objekte wie beispielsweise ein Elektron verletzt: die spezifische
Elementarladung, d.h. das Verhältnis eines Elektrons zu seiner Masse, beträgt nämlich
$\frac{e}{m}\approx1.76\cdot10^{11}\unitfrac{C}{kg}$.
Letztlich ist die Theorie damit mehr ein Spezialfall für "`schwach"' geladene
Objekte. Auch ist das Prozedere sehr konstruktiv und an vielen Stellen müssen
Näherungen gemacht werden.
Insgesamt ist durch Kaluzas Arbeit allerdings der Grundstein für die
Vereinheitlichung der Kräfte gelegt. Eine modernere Formulierung die sich unter
anderem an der Arbeiten von Klein, Jordan und Thiry orientiert wird im nächsten
Kapitel vorgestellt.

%
%
% Um weitere Freiheitsgrade zu erhalten führen wir eine weitere Raumdimension ein
% (Weitere Zeitdimensionen bereiten Probleme), da nur 3
% raumdimensionen beobachtet werden muss es einen Grund dafür geben das diese
% nicht sichtbar sind. Wir erweitern die Theorie deshalb um eine zusätzliche
% \emph{kompakte} Raumdimension mit ``Ausdehnung'' auf einer uns nicht
% zugänglichen Skala ($r\sim l\textsubscript{Plank}$ = Welche Länge? = Welche
% Energie Vgl. mit zugänglicher Energieskala am LHC).
% Wir nehmen an das die kompakte Dimension durch $\Sphere^1$ beschrieben wird, die
% Raumzeitmanigfaltigkeit als lokal(?) homoömorph zu $\Reals^4\times\Sphere^1$
% ist.
% Wir nehmen an das auch in dieser 5 dimensionalen Raumzeit die
% Einsteingleichungen ihre gültigkeit behalten. Allerdings erhalten wir nun
% anstatt der 10 Gleichungen 15 also 5 zusätzliche Freiheitsgrade (???)

% \section{Kleins Ansatz}
% 
% 
% 
% \section{Kaluzas Erweiterung}
% \section{Probleme}
% \section{Faserbündel}
% \section{Yang-Mills}
% \section{Flache Raumzeit}
% Falls die 4-Raumzeit flach ist also die betrachtete Manigfaltigkeit
% $M=\Reals^4\times\Sphere^1$ (Achtung hier braucht man eigendlich Faserbündel!!!)
% ergibt sich als Metrik mit $z=\left(x,\theta\right)$
% \begin{equation}
% \tensor{g}{_i_j}\dif\tensor{z}{^i}\dif\tensor{z}{^j}
% =\tensor{\eta}{_\mu_\nu}\dif\tensor{x}{^\mu}\dif\tensor{x}{^\nu}+r^2\dif\theta^2
% \end{equation}
% Untersuchung von Störungen dieser Metrik
% \begin{equation}
% \tensor{\hat{g}}{_i_j}\dif\tensor{z}{^i}\dif\tensor{z}{^j}
% =e^{-\phi/3}\left[e^{\phi}\left(\dif\theta+\kappa\tensor{A}{_\mu}\dif\tensor{x}{^\mu}\right)^2
% +\tensor{g}{_\mu_\nu}\dif\tensor{x}{^\mu}\dif\tensor{x}{^\nu}\right]
% \end{equation}

%\begin{equation}
%\tensor{\hat{g}}{_{MN}}\dif\tensor{z}{^M}\dif\tensor{z}{^N}
%=e^{-\phi/3}\left[e^{\phi}\left(\dif\theta+\kappa\tensor{A}{_\mu}\dif\tensor{x}{^\mu}\right)^2
% +\tensor{g}{_\mu_\nu}\dif\tensor{x}{^\mu}\dif\tensor{x}{^\nu}\right]
%\end{equation}
% Entwicklung in Fourrierreihen
% \begin{align*}
% \tensor{g}{_\mu_\nu}(z)&=\sum_{n=-\infty}^\infty
% \tensor*{g}{*^{(n)}*_\mu*_\nu}\left(x\right)e^{\imI n \theta}\\
% \tensor{A}{_\mu}(z)&=\sum_{n=-\infty}^\infty
% \tensor*{A}{*^{(n)}*_\mu}\left(x\right)e^{\imI n \theta}\\
% \phi(z)&=\sum_{n=-\infty}^\infty
% \phi^{(n)}\left(x\right)e^{\imI n \theta}
% \end{align*}
% \section{Zylinderbedingung}
% \section{Nordströms Theorie} 
% Bereits vor Einstein formulierte Gunnar Nordström 1912, eine rein geometrische Theorie der Gravitation (Zusammenhang?)
% 
% TODO ist dtheta ein killing vektorfeld?
\chapter{Moderne Formulierung}
Auch wenn das Ergebnis sehr überzeugend ist, so ist das vorgehen nicht besonders
mathematisch. Annahmen:
\begin{enumerate}
\item Zylinderbedingung
\item keine Quellen, $\tensor{T}{_m_n}=0$.
\item die Metrik ist Extremum des EH-Funktionals
\item Testteilchen bewegen sich auf Geodäten
\item die Eigenzeit ist durch die Bogenlänge gegeben
\end{enumerate}
\section{Alternative Darstellung $n+4$ dimensionaler Biliearformen}
\section{Eigenschaften der Klein-Metrik}
An jedem Punkt lässt sich die Metrik $\hat{g}$ durch eine symmetrische
(Block-)Matrix
\begin{equation}
\widehat{G}=\begin{pmatrix}B& C\\
C\transpose& H\end{pmatrix}
\end{equation}
beschreiben, wobei
$B\in\operatorname{Sym}_4$
\footnote{$\operatorname{Sym}_k$ bezeichnet die Menge aller symmetrischen
$k\times k$-Matritzen.} , $H\in\operatorname{Sym}_n$.
$H$ ist nicht entartet, $g$ nicht entartet ist. 
Da die Matrix reell symmetrisch ist, ist sie (block-) diagonalisierbar
\begin{equation}
\begin{split}
\widehat{G}
&=
\begin{pmatrix}1& A\\0& 1\end{pmatrix}
\begin{pmatrix}G& 0\\0& H
\end{pmatrix}
\begin{pmatrix}1& 0\\A\transpose& 1\end{pmatrix}\,.
\end{split}
\end{equation}
Die Blöcke sind gegeben als
\begin{equation}
G=B-CH^{-1}C\transpose\,,\quad A=CH^{-1}\,.
\end{equation}
Damit lässt sich direkt die Determinante angeben
\begin{equation}
\det(\hat{G})=\det(G)\det(H)
\end{equation}
Die Transformation lässt sich leicht umkehren
\begin{equation}
\begin{pmatrix}1& A\\0& 1\end{pmatrix}^{-1}=\begin{pmatrix}1& -A\\0&
1\end{pmatrix}\,.
\end{equation}
Mit diesen Relationen ergibt sich eine alternative Darstellung des
Linienelements
\begin{equation}
\begin{split}
\dif \hat{s}^2&=\dif \vec{x}\transpose G
\dif \vec{x}\\
&=\dif \vec{x}\transpose
\begin{pmatrix}1& -A\\0& 1\end{pmatrix}
\begin{pmatrix}G& 0\\0& H
\end{pmatrix}
\begin{pmatrix}1& 0\\-A\transpose& 1\end{pmatrix}
\dif \vec{x}\\
&=\tensor{g}{_\mu_\nu}\dif \tensor{x}{^\mu}\dif\tensor{x}{^\nu}
+\tensor{h}{_a_b}\left(\dif\tensor{x}{^a}-\tensor*{A}{^a_\mu}\dif\tensor{x}{^\mu}\right)
\left(\dif\tensor{x}{^b}-\tensor*{A}{^b_\mu}\dif\tensor{x}{^\mu}\right)
\end{split}
\end{equation}
Da diese Darstellung im Wesentlichen vom Typ ist wie sie von Klein verwendet
wurde soll wird sie im Folgenden als Klein-Darstellung bzw. eine Metrik vom Typ
als Klein-Metrik bezeichnen.
Berechnungen angelehnt an \cite{Coquereaux:1990qs} \cite{williams2015field}.
\subsection{Darstellung in lokalen Koordinaten}
\begin{equation}
\tensor{\hat{g}}{_\mu_\nu}=\tensor{g}{_\mu_\nu}+\tensor{h}{_a_b}\tensor{A}*{^a_\mu}\tensor*{A}{^b_\nu}\,,\quad
\tensor{\hat{g}}{_a_\mu}=\tensor{h}{_a_b}\tensor*{A}{^b_\mu}\,,\quad
\tensor{\hat{g}}{_a_b}=\tensor{h}{_a_b}
\end{equation}
\subsection{Inverse}
Wie sich leicht nachprüfen lässt ist die Inverse der Klein-Metrik gegeben als
\begin{equation}
\tensor{\hat{g}}{^\mu^\nu}=\tensor{g}{^\mu^\nu}\,,\quad
\tensor{\hat{g}}{^a^\mu}=-\tensor{A}{^a^\mu}\,,\quad
\tensor{\hat{g}}{^a^b}=\tensor*{A}{^a_\mu}\tensor*{A}{^b^\mu}+\tensor{h}{^a^b}\,.
\end{equation}
% \subsection{Determinante}
% Die Determinante wird wie üblich zur Berechnung von Volumenelementen benötigt.
% Bei der Berechnung ist folgende Formel für Matrizen $A,B,C,D$ hilfreich:
% \begin{equation}
% \begin{split}
% \det\begin{pmatrix}A& B\\ C& D\end{pmatrix}&=\det\left[
% \begin{pmatrix}1& B\\0& D\end{pmatrix}\begin{pmatrix}A-BD^{-1}C& 0\\DC^{-1
% }& 1\end{pmatrix} \right]\\
% &= \det(D) \det\left(A - B D^{-1}
% C\right)\,.
% \end{split}
% \end{equation}
% %TODO evtl. Beweis
% Damit ergibt sich 
% \begin{equation}
% \begin{split}
%  \hat{g}&=
%  \det\begin{pmatrix}\tensor{g}{_\mu_\nu}+\psi\tensor{A}{_\mu}\tensor{A}{_\nu}
%  &\psi \tensor{A}{_\mu}\\
%  \psi \tensor{A}{_\mu}	 & \psi
%  \end{pmatrix}\\
%  &=\psi
%  \det\left(\tensor{g}{_\mu_\nu}+\psi\tensor{A}{_\mu}\tensor{A}{_\nu}
%  -\psi\tensor{A}{_\mu}\psi^{-1}\psi\tensor{A}{_\nu}\right)\\
%  &=\psi \det\left(\tensor{g}{_\mu_\nu}\right)\\
%  &=\psi g\,.
% \end{split}
% \end{equation}
% Die Determinante der fünfdimensionalen Metrik ist also proportional zur 4
% dimensionalen. 
% \begin{bemerkung}
% Die entsprechenden Ausdrücke für die Kaluza-Metrik sind deutlich komplizierter.
% \end{bemerkung}
\subsection{Christoffelsymbole}
Siehe Williams: "`Field Equations and Lagrangian for the Kaluza Metric
Evaluated with Tensor Algebra Software"'\cite{williams2015field}
\subsection{Skalarkrümmung}
\begin{equation}
\hat{R}=R-\frac{1}{4}\psi\tensor{F}{_\mu_\nu}\tensor{F}{^\mu^\nu}
+\frac{1}{2\psi^2}(\pdif{_\mu}\psi)(\pdif{^\mu}\psi)
-\frac{1}{\psi}\square\psi
\end{equation}
Wir führen zwei weitere Felder $\psi=:\phi^2$ und $\psi=:e^{2\sigma}$ ein womit
sich der Term nochmals vereinfacht \footnote{Dies stellt keine Einschränkung
dar, da $\psi$ positiv sein muss, damit $\hat{g}$ das gleiche Vorzeichen hat wie $g$. (Bei der Zusatzdimension handelt es sich um
eine Raumdimension)}:
\begin{equation}
\hat{R}=R-\frac{1}{4}\phi^2\tensor{F}{_\mu_\nu}\tensor{F}{^\mu^\nu}
-\frac{2}{\phi}\square \phi
\end{equation}

\begin{equation}
\hat{R}=R-\frac{1}{4}e^{2\sigma}\tensor{F}{_\mu_\nu}\tensor{F}{^\mu^\nu}
-2e^{-\sigma}\square e^{\sigma}
\end{equation}

\section{Die fünfdimensionale Raumzeit}
Konstruktion der MFG: Siehe "`Coquereau Geometry of Multidimensional
Universes"'\cite{coquereaux1983geometry}.
Die Metrik ist von der Form
\section{Die fünfdimensionale Raumzeit}
Wir werden im folgenden nur den fünfdimensionalen Fall diskutieren. In den
meisten Fällen ergibt sich eine Verallgemeinerung natürlich. Sei $(E,\hat{g})$
eine fünfdimensionale pseudoriemannsche Mannigfaltigkeit. Die Liegruppe
kompakte Liegruppe $G$ operiere durch Isometrien auf $E$. 
Dann ergibt sich wie beschrieben eine Faserstruktur auf $E$ mit typischer Faser
$G.x x\in E$. Seien $T^{(a)}\in \mathfrak{g}$ die Generatoren von $G$
\begin{equation}
V^{(a)}(p):=\od{}{t}\bigg|_{t=0}\exp\left(tT^{(a)}\right).p\,,\quad p\in E
\end{equation}
Da $U(1)$ eindimensional ist existiert in unserem Fall nur ein solches
Vektorfeld. Wähle ein Koordinatensystem sodass 
\begin{equation}
V^{(1)}=V^4\partial_4\,,\quad \partial_4\tensor{\hat{g}}{_\mu_\nu}=0\,.
\end{equation}
% geht immer?
Die Metrik ist von der Form \footnote{$\tensor{\hat{g}}{_4_4}>0$ sonst
zusätzliche Zeitartige Vektoren}
\begin{equation}
\hat{g}=\tensor{\hat{g}}{_\mu_\nu}\dif\tensor{x}{^\mu}\dif\tensor{x}{^\nu}
+\tensor{\hat{g}}{_\mu_4}\dif\tensor{x}{^\mu}\dif\tensor{x}{^4}
+\tensor{\hat{g}}{_4_\mu}\dif\tensor{x}{^4}\dif\tensor{x}{^\mu}
+\tensor{\hat{g}}{_4_4}\dif\tensor{x}{^4}\dif\tensor{x}{^4}\,.
\end{equation}
Sortiert man die Terme in der Metrik etwas um, so erhält man die Klein-Metrik
\begin{equation}
\hat{g}=\left(\tensor{\hat{g}}{_\mu_\nu}-\frac{\tensor{\hat{g}}{_\mu_4}\tensor{\hat{g}}{_\nu_4}}{\tensor{\hat{g}}{_4_4}}\right)\dif\tensor{x}{^\mu}\dif\tensor{x}{^\nu}
+\tensor{\hat{g}}{_4_4}\left(\dif\tensor{x}{^4}+\frac{\tensor{\hat{g}}{_\mu_4}}{\tensor{\hat{g}}{_4_4}}\dif\tensor{x}{^\mu}\right)^2
\,.
\end{equation}
Man definiert 
\begin{equation}
\tensor{g}{_\mu_\nu}:=\tensor{\hat{g}}{_\mu_\nu}-\frac{\tensor{\hat{g}}{_\mu_4}\tensor{\hat{g}}{_\nu_4}}{\tensor{\hat{g}}{_4_4}}
\,,\quad
\tensor{A}{_\mu}:=\frac{\tensor{\hat{g}}{_\mu_4}}{\tensor{\hat{g}}{_4_4}}
\,,\quad
\psi:=\tensor{\hat{g}}{_4_4}\,,
\end{equation}
sowie eine 1-Form $\omega$ durch
\begin{equation}
\omega=\dif\tensor{x}{^4}+\tensor{A}{_\mu}\dif\tensor{x}{^\mu}\,.
\end{equation}
Damit lässt sich die Metrik in zwei Anteile Unterteilen:
\begin{equation}
\hat{g}=g+\psi\,\omega\otimes\omega\,.
\end{equation}
Es gilt:
\begin{equation}
\begin{split}
0&=\hat{g}(\partial_4,\partial_\mu)\\
&=\left(\tensor{\hat{g}}{_4_\nu}\dif\tensor{x}{^\nu}
+\tensor{\hat{g}}{_4_4}\dif\tensor{x}{^4}\right)(\partial_\mu)\\
&=\left(\tensor{\hat{g}}{_4_\mu}\dif\tensor{x}{^\mu}
+\tensor{\hat{g}}{_4_4}\dif\tensor{x}{^4}\right)(\partial_\mu)\\
&=: \tensor{\hat{g}}{_4_4}\omega(\partial_\mu)
\end{split}
\end{equation}
Die induzierte Metrik auf dem Bündel 
\begin{equation}
B=\bigsqcup_{p\in E/G}\mathrm{Span}\left\{\partial_\mu|_p\right\}
\end{equation}
ist also gegeben durch die Metrik $g$. $B$ soll im Folgenden als Modell für die
(bekannte) vierdimensionale Raumzeit stehen.
Es handelt sich zunächst lediglich
um eine Umbenennung, die Bedeutung dieser Größen wird im Folgenden klar werden. Die von Klein verwendet Metrik ist also verschieden von der Kaluza
Metrik, beide sind aber vom Informationsgehalt äquivalent\footnote{Verschiedene
Autoren verwenden zu $\hat{g}$ konform äquivalente Metriken, 
bzw. Metriken die noch einen Zusätzlichen Freiheitsgrad $\kappa$ enthalten.
Dieser lässt sich aber in die Größen $\tensor{A}{_\mu}$ bzw. $\phi$ absorbieren.}
Die bekannten Vierervektoren $X^\mu$ entsprechen Fünfervektoren mit
\begin{equation}
\tensor{X}{^5}=-\tensor{A}{_\mu}\tensor{X}{^\mu}
\end{equation}
%(=Projektion?)
Da $\tensor{A}{_\mu}$ nur von $\tensor{X}{^\mu}$ abhängt, ist $\tensor{X}{^5}$
durch die ersten vier Komponenten eindeutig gegeben und enthält keine
Zusatzinformation.
\section{Identifikation der Größen}
Wir wollen die Objekte $A$ und $\psi$ als Eichpotential bzw. Skalarfeld auf $M/G$
auffassen, dazu müssen wir zeigen das sie entsprechend transformieren.
Da die Lieableitung der Metrik nach $V=\partial_4$ verschwindet wissen wir, dass die
Metrik invariant unter infinitesimalen Diffeomorphismsen $\xi$ sein muss.
Betrachte die Abbildung $f:M\to M$ 
\begin{equation}
\tensor{x}{^\mu}\mapsto
\tensor{x}{^\mu}\,,\quad\tensor{x}{^4}\mapsto\tensor{x}{^4}+
\xi\left(\tensor{x}{^\mu}\right)
\end{equation}
mit $\xi\in C^\infty\left(\Reals\right)$ beliebig. 
%TODO kommt von killingvector del4
Pullback $\hat{g}^\prime:=f^*\hat{g}$
%TODO Pullback in Komponenten
\begin{equation}
\begin{split}
\tensor*{\hat{g}}{*^\prime_\mu_\nu}&=\tensor{\hat{g}}{_a_b}\dpd{f^a}{\tensor{x}{^\mu}}\dpd{f^b}{\tensor{x}{^\nu}}\\
&=\tensor{\hat{g}}{_\alpha_\beta}\dpd{f^\alpha}{\tensor{x}{^\mu}}\dpd{f^\beta}{\tensor{x}{^\nu}}
+\tensor{\hat{g}}{_\alpha_4}\dpd{f^\alpha}{\tensor{x}{^\mu}}\dpd{f^4}{\tensor{x}{^\nu}}
+\tensor{\hat{g}}{_4_\beta}\dpd{f^4}{\tensor{x}{^\mu}}\dpd{f^\beta}{\tensor{x}{^\nu}}
+\tensor{\hat{g}}{_4_4}\dpd{f^4}{\tensor{x}{^\mu}}\dpd{f^4}{\tensor{x}{^\nu}}
\\
&=\tensor{\hat{g}}{_\mu_\nu}
+\psi\tensor{A}{_\mu}\pdif{_\nu}\xi
+\psi\tensor{A}{_\nu}\pdif{_\mu}\xi
+\psi\pdif{_\mu}\xi\pdif{_\nu}\xi\\
&=\tensor{g}{_\mu_\nu}
+\psi\left(\tensor{A}{_\mu}+\pdif{_\mu}\xi\right)
\left(\tensor{A}{_\nu}+\pdif{_\nu}\xi\right)
\end{split}
\end{equation}
\begin{equation}
\begin{split}
\tensor*{\hat{g}}{*^\prime_\mu_4}&=\tensor{\hat{g}}{_a_b}\dpd{f^a}{\tensor{x}{^\mu}}\dpd{f^b}{\tensor{x}{^4}}\\
&=\tensor{\hat{g}}{_\alpha_\beta}\dpd{f^\alpha}{\tensor{x}{^\mu}}\dpd{f^\beta}{\tensor{x}{^4}}
+\tensor{\hat{g}}{_\alpha_4}\dpd{f^\alpha}{\tensor{x}{^\mu}}\dpd{f^4}{\tensor{x}{^4}}
+\tensor{\hat{g}}{_4_\beta}\dpd{f^4}{\tensor{x}{^\mu}}\dpd{f^\beta}{\tensor{x}{^4}}
+\tensor{\hat{g}}{_4_4}\dpd{f^4}{\tensor{x}{^\mu}}\dpd{f^4}{\tensor{x}{^4}}\\
&=\psi\left(\tensor{A}{_\mu}+\pdif{_\mu}\xi\right)
\end{split}
\end{equation}
\begin{equation}
\begin{split}
\tensor*{\hat{g}}{*^\prime_4_4}&=\tensor{\hat{g}}{_a_b}\dpd{f^a}{\tensor{x}{^4}}\dpd{f^b}{\tensor{x}{^4}}\\
&=\tensor{\hat{g}}{_\alpha_\beta}\dpd{f^\alpha}{\tensor{x}{^4}}\dpd{f^\beta}{\tensor{x}{^4}}
+\tensor{\hat{g}}{_\alpha_4}\dpd{f^\alpha}{\tensor{x}{^4}}\dpd{f^4}{\tensor{x}{^4}}
+\tensor{\hat{g}}{_4_\beta}\dpd{f^4}{\tensor{x}{^4}}\dpd{f^\beta}{\tensor{x}{^4}}
+\tensor{\hat{g}}{_4_4}\dpd{f^4}{\tensor{x}{^4}}\dpd{f^4}{\tensor{x}{^4}}\\
&=\psi
\end{split}
\end{equation}
Das Transformationsverhalten der Größen lässt sich zusammenfassen als 
\begin{equation}
\tensor{g}{_\mu_\nu}\to\tensor{g}{_\mu_\nu}\,,\quad
\tensor{A}{_\mu}\to\tensor{A}{_\mu}+\pdif{_\mu}\xi\,,\quad
\psi\to\psi
\end{equation}
Also entsprechen Infinitisimale Diffeomorphismen auf $M$ Eichtransformationen.
Weiter ist zu Zeigen das $A$ als Differentialform $M/G$ interpretiert werden,
um dies einzusehen studieren wir das Verhalten unter Koordinatenwechseln $f:M\to
M$
\begin{equation}
\tensor{x}{^\mu}\mapsto
h(\tensor{x}{^\mu})\,,\quad\tensor{x}{^4}\mapsto\tensor{x}{^4}
\end{equation}
mit einem Diffeomorphismus $h$
\begin{equation}
\begin{split}
\tensor*{\hat{g}}{*^\prime_\mu_\nu}&=\tensor{\hat{g}}{_a_b}\dpd{f^a}{\tensor{x}{^\mu}}\dpd{f^b}{\tensor{x}{^\nu}}\\
&=\tensor{\hat{g}}{_\alpha_\beta}\dpd{f^\alpha}{\tensor{x}{^\mu}}\dpd{f^\beta}{\tensor{x}{^\nu}}
+\tensor{\hat{g}}{_\alpha_4}\dpd{f^\alpha}{\tensor{x}{^\mu}}\dpd{f^4}{\tensor{x}{^\nu}}
+\tensor{\hat{g}}{_4_\beta}\dpd{f^4}{\tensor{x}{^\mu}}\dpd{f^\beta}{\tensor{x}{^\nu}}
+\tensor{\hat{g}}{_4_4}\dpd{f^4}{\tensor{x}{^\mu}}\dpd{f^4}{\tensor{x}{^\nu}}\\
&=\tensor{\hat{g}}{_\alpha_\beta}\dpd{h^\alpha}{\tensor{x}{^\mu}}\dpd{h^\beta}{\tensor{x}{^\nu}}\\
&=\tensor{g}{_\alpha_\beta}\dpd{h^\alpha}{\tensor{x}{^\mu}}\dpd{h^\beta}{\tensor{x}{^\nu}}
+\psi\tensor{A}{_\alpha}\dpd{h^\alpha}{\tensor{x}{^\mu}}
\tensor{A}{_\beta}\dpd{h^\beta}{\tensor{x}{^\nu}}
\end{split}
\end{equation}
\begin{equation}
\begin{split}
\tensor*{\hat{g}}{*^\prime_\mu_4}&=\tensor{\hat{g}}{_a_b}\dpd{f^a}{\tensor{x}{^\mu}}\dpd{f^b}{\tensor{x}{^4}}\\
&=\tensor{\hat{g}}{_\alpha_\beta}\dpd{f^\alpha}{\tensor{x}{^\mu}}\dpd{f^\beta}{\tensor{x}{^4}}
+\tensor{\hat{g}}{_\alpha_4}\dpd{f^\alpha}{\tensor{x}{^\mu}}\dpd{f^4}{\tensor{x}{^4}}
+\tensor{\hat{g}}{_4_\beta}\dpd{f^4}{\tensor{x}{^\mu}}\dpd{f^\beta}{\tensor{x}{^4}}
+\tensor{\hat{g}}{_4_4}\dpd{f^4}{\tensor{x}{^\mu}}\dpd{f^4}{\tensor{x}{^4}}\\
&=\psi\tensor{A}{_\alpha}\dpd{h^\alpha}{\tensor{x}{^\mu}}
\end{split}
\end{equation}
\begin{equation}
\begin{split}
\tensor*{\hat{g}}{*^\prime_4_4}&=\tensor{\hat{g}}{_a_b}\dpd{f^a}{\tensor{x}{^4}}\dpd{f^b}{\tensor{x}{^4}}\\
&=\tensor{\hat{g}}{_\alpha_\beta}\dpd{f^\alpha}{\tensor{x}{^4}}\dpd{f^\beta}{\tensor{x}{^4}}
+\tensor{\hat{g}}{_\alpha_4}\dpd{f^\alpha}{\tensor{x}{^4}}\dpd{f^4}{\tensor{x}{^4}}
+\tensor{\hat{g}}{_4_\beta}\dpd{f^4}{\tensor{x}{^4}}\dpd{f^\beta}{\tensor{x}{^4}}
+\tensor{\hat{g}}{_4_4}\dpd{f^4}{\tensor{x}{^4}}\dpd{f^4}{\tensor{x}{^4}}\\
&=\psi
\end{split}
\end{equation}
\begin{equation}
\tensor{g}{_\mu_\nu}\to\tensor{g}{_\alpha_\beta}\dpd{h^\alpha}{\tensor{x}{^\mu}}\dpd{h^\beta}{\tensor{x}{^\nu}}\,,\quad
\tensor{A}{_\mu}\to\tensor{A}{_\alpha}\dpd{h^\alpha}{\tensor{x}{^\mu}}\,,\quad
\psi\to\psi
\end{equation}
Diffeomorphismeninvarianz der Klein Metrik impliziert also sowohl die
Eichinvarianz des Viererpotentials $\tensor{A}{_\mu}$, als auch das korrekte
Transformationsverhalten unter vierdimensionalen Koordinatenwechseln.
Dabei transformieren $\psi$, $\tensor{A}{_\mu}$ und $\tensor{g}{_\mu_\nu}$ als
Skalar, Vektor, bzw. Tensor.
%TODO sind noch weitere Trafos erlaubt?

\section{Lagrangedichte}
%Karte auf $M/G$?
Kleins Idee war es das Wirkungsprinzip nach Hilbert auf fünf Dimensionen zu
erweitern. Folglich lautet das Wirkungsintegral
\begin{equation}
\hat{S}=\int_{M_5}\sqrt{-\hat{g}}\hat{R}\dif{^5}
x\,.
\end{equation}
% TODO d.h. Vakuum
Hierbei tritt ein ernsthaftes Problem auf. Der Integrand ist nicht von der
Zusatzdimension abhängig. Würde man also $\tensor{x}{^4}$ wie die verbleibenden
Raumkoordinaten behandeln, wie dies beispielsweise bei Kaluza der Fall war, so
ist $\hat{S}$ nicht wohldefiniert. Klein erkannte das er das Problem umgehen
konnte wenn die Zusatzkoordinate zylindrisch wäre. Nimmt man für 
$\tensor{x}{^4}$ einen Radius $r$ an so ergibt sich
\begin{equation}
\begin{split}
\hat{S}&=\int_{M_5}\sqrt{-g}\phi\hat{R}\dif{^5}x\\
&=2\pi
r\int_{M_4}\sqrt{-g}\phi\hat{R}\dif{^4}x\,.
\end{split}
\end{equation}
%TODO dies ist die minimale skalare Theorie
%TODO wenn faserbündel korrekt dann mit trafoformel argumentieren (überdeckung
% etc\ldots)
Die Lagrangedichte ist also gegeben als
\begin{equation}
\begin{split}
\mathcal{L}&=\sqrt{-g}e^{\sigma}\left(R-\frac{1}{4}e^{2\sigma}\tensor{F}{_\mu_\nu}\tensor{F}{^\mu^\nu}
-2e^{-\sigma}\square e^{\sigma}\right)\\
&=\sqrt{-g}e^{\sigma}\left(R-\frac{1}{4}e^{2\sigma}\tensor{F}{_\mu_\nu}\tensor{F}{^\mu^\nu}\right)+2\sqrt{-g}\square
e^{\sigma}\label{eq:Lagrange1}
\end{split}
\end{equation}
Der letzte Term liefert als totale Divergenz keinen Beitrag.
Der Term der die Krümmung enthält unterscheidet sich um einen Faktor $\phi$ von
der klassischen Einstein-Hilbert Lagrangedichte.
\subsection{Konforme Transformation}
Da wir den Lagrangian gerne in einer Form vorliegen hätten, die dem
EH-Lagrangian enspricht müssen wir den Vorfaktor $e^{\sigma}$ loswerden. 
Wir führen dazu eine konforme Transformation durch
\begin{equation}
\tensor*{\hat{g}}{*^\star*_i*_j}=e^{2\tau}\tensor{\hat{g}}{_i_j}\,.
\end{equation}
Dies impliziert sofort
\begin{equation}
\tensor*{g}{*^\star*_\mu*_\nu}=e^{2\tau}\tensor{g}{_\mu_\nu}\,,\quad\sigma^\star
=\sigma+\tau\,,\quad \sqrt{-g}=e^{-4\tau}\sqrt{-g^\star}\,.
\end{equation}
Die Komponenten des elektrischen Feldstärketensors ist in der konformen Metrik
gegeben als
\begin{equation}
\tensor*{F}{*^\star_\mu_\nu}=\tensor{F}{_\mu_\nu}\,,\quad\tensor*{F}{*^\star^\mu^\nu}
=\tensor*{\hat{g}}{*^\star*^\mu*^\alpha}\tensor*{\hat{g}}{*^\star*^\nu*^\beta}\tensor{F}{_\alpha_\beta}
=e^{-4\tau}\tensor{F}{^\mu^\nu}
\end{equation}
Wendet man die Formel für das Transformationsverhalten des Krümmungsskalars an,
so findet man schließlich
\begin{equation}
R=e^{2\tau}\left[R^\star-6(\tensor*{\partial}{^\star_\mu}\tau)(\tensor{\partial}{^\star^\mu}\tau)
-6\square^\star\tau\right]\,,
\end{equation}
beziehungsweise
\begin{equation}
\begin{split}
\sqrt{-g}\phi
\hat{R}
&=e^{\sigma-2\tau}\sqrt{-g^\star}\left[\hat{R}^\star
-6(\tensor*{\partial}{^\star_\mu}\tau)(\tensor{\partial}{^\star^\mu}\tau)
-6\square^\star\tau\right]\,.
\end{split}
\end{equation}
% TODO Einfluss konformer Trafos
Setzt man $\tau = \frac{1}{2}\sigma$\footnote{Die daraus resultierende
Relation $\sigma^\star=\frac{3}{2}\sigma$ kann durch eine Skalierung von
$\sigma$ behandelt werden.}, so ergibt sich
\begin{equation}
\begin{split}
\sqrt{-g}\phi
\hat{R}&=\sqrt{-g^\star}\left[R^\star
-\frac{3}{2}(\tensor*{\partial}{^\star_\mu}\sigma)(\tensor{\partial}{^\star^\mu}\sigma)
-3\square^\star\sigma\right]\,.
\end{split}
\end{equation}
Setzt man in \eqref{eq:Lagrange1} ein erhält man
\begin{equation}
\begin{split}
\mathcal{L}&=\sqrt{-g^\star}\left[R^\star
+\frac{3}{2}(\tensor*{\partial}{^\star_\mu}\sigma)(\tensor{\partial}{^\star^\mu}\sigma)
-3\square^\star\sigma\right]
+\sqrt{-g}e^{\sigma}\left(-\frac{1}{4}e^{2\sigma}\tensor{F}{_\mu_\nu}\tensor{F}{^\mu^\nu}\right)\\
&=\sqrt{-g^\star}\left[R^\star
-\frac{3}{2}(\tensor*{\partial}{^\star_\mu}\sigma)(\tensor{\partial}{^\star^\mu}\sigma)
-\frac{1}{4}e^{3\sigma}\tensor*{F}{^\star_\mu_\nu}\tensor*{F}{^\star^\mu^\nu}\right]-3\sqrt{-g^\star}\square^\star\sigma\,.
\end{split}
\end{equation}
Der Term mit der totalen Divergenz produziert einen nicht beitragenden Randterm
und kann deshalb fallen gelassen werden. 
%TODO Randterme
Da eine Variation der Konformen Metrik auf die
gleichen Gleichungen führt, lassen wir im folgenden die Sterne an den Größen weg. Zudem führen wir
Bezeichnungen ein die zu einer üblichen Form führen
\begin{equation}
\begin{split}
\mathcal{L}
&=\sqrt{-g}\left[R
-\frac{1}{2}(\tensor{\partial}{_\mu}\sigma)(\tensor{\partial}{^\mu}\sigma)
-\frac{1}{4}\tensor*{H}{_\mu_\nu}\tensor*{F}{^\mu^\nu}\right]\,.
\label{eq:Lagrange2}
\end{split}
\end{equation}
Dabei bezeichnet die
Größe $H$ die elektromagnetische Verschiebungsfelddichte\footnote{In Analogie zu
den Verschiebungsfeldern $\vec{D},\vec{H}$ der klassischen Elektrodynamik.
}
%TODO Insbesondere handelt es sich nicht um das Nebenfeld
% $\tensor{H}{_\mu_\nu}$.
\begin{equation}
\tensor*{H}{_\mu_\nu}:=e^{\sqrt{3}\sigma}\tensor*{F}{_\mu_\nu}\,,
\end{equation}
wobei der Term $e^{\sqrt{3}\sigma}$ als variable Perimitivität $\mu(\sigma)$
interpretiert wird. Dies ist ein beachtliches Ergebnis, dass falls
$\sigma=\text{const.}=0$ exakt der Lagrangedichte der klassischen
Maxwell-Einstein Theorie enstpricht. Dieser Umstand ist auch als
\emph{Kaluza-Klein Wunder} bekannt.
% Ist es konsistent nach den Komponenten einzeln zu variieren?
\subsection{Bewegungsgleichungen}
In bekannter Manier impliziert die Form der Lagrangedichte 
Die Einsteingleichungen:
\begin{equation}
\tensor{G}{_\mu_\nu}=\tensor*{T}{*^{\sigma}_\mu_\nu}+\tensor*{T}{*^{A}*_\mu*_\nu}
\end{equation}
Weiter erhält man für die Variation nach den $A$ 
Das Feld $A$ ist Zyklisch, taucht als nicht selbst in der Lagrangedichte auf.
Die resultierende Bewgungsgleichungen lautet
\begin{equation}
0=\tensor{\nabla}{_\alpha}\left(\dpd{\mathcal{L}}{\left(\tensor{\nabla}{_\alpha}\tensor{A}{_\beta}\right)}\right)
=\tensor{\nabla}{_\alpha}\left[e^{\sqrt{3}\sigma}\dpd{\left(\tensor{F}{_\mu_\nu}\tensor{F}{^\mu^\nu}\right)}{\left(\tensor{\nabla}{_\alpha}\tensor{A}{_\beta}\right)}\right]
=4\tensor{\nabla}{_\alpha}\tensor{H}{^\alpha^\beta}
 \end{equation}
 Für das Vektorpotential $A$, sowie
 % Ableitung nach DA im MW teil herleinten
 \begin{equation}
0=\tensor{\nabla}{_\alpha}\left(\dpd{\mathcal{L}}{\left(\tensor{\partial}{_\alpha}\sigma\right)}\right)
-\dpd{\mathcal{L}}{\sigma}\\
=\square \sigma
-\frac{\sqrt{3}}{4}\tensor*{H}{_\mu_\nu}\tensor*{F}{^\mu^\nu}
\label{eq:dymdilat}
 \end{equation}
 für das Skalarfeld $\sigma$. 
 % Lässt sich das durch redifinition zur Klein Gordon Gleichung hinbiegen?
\section{Erhaltungrößen}
Per Konstruktion ist $K=\partial_4$ ein Killing-Vektorfeld, die entsprechende
Erhaltungsgröße ist 
\begin{equation}
\tensor{\hat{K}}{_n}\tensor{\hat{U}}{^n}=\tensor*{\delta}{^4_n}\tensor{\hat{U}}{^n}
=\tensor{\hat{U}}{^4}:=Q\,.
\end{equation}
Da wir Vakuumlösungen $\tensor{T}{_n_m}=0$ betrachten, lässt sich keine erhalte
Energie 
\begin{equation}
E=\int\dif{}^3x \sqrt{-g} \tensor{T}{_0_n}\tensor{K}{^n}
\end{equation}
definieren. Wir wollen die erhaltene vierte Komponente der Geschwindigkeit mit 
der Ladung $Q$ eines Teilchens identifizieren.
 \section{Geodäten}
 Wir nehmen an das sich (geladene) Teilchen entlang von fünfdimensionalen
 Geodäten bewegen. 
 Zunächst gilt weiterhin, dass diese proportional zur Bogelänge parametrisiert
 sind, d.h.
 \begin{equation}
 \begin{split}
  C&=
  \tensor{\hat{g}}{_\mu_\nu}\tensor{\hat{U}}{^\mu}\tensor{\hat{U}}{^\nu}
  +\tensor{\hat{g}}{_4_4}\tensor{\hat{U}}{^4}\tensor{\hat{U}}{^4}\\\
 &=\tensor{\hat{g}}{_\mu_\nu}\tensor{\hat{U}}{^\mu}\tensor{\hat{U}}{^\nu}
 +\psi Q^2\\
  &=\tensor{g}{_\mu_\nu}\tensor{\hat{U}}{^\mu}\tensor{\hat{U}}{^\nu}
  -\psi\left(\tensor{A}{_\mu}\tensor{\hat{U}}{^\mu}\right)^2
 +\psi Q^2\\
 \end{split}
 \end{equation}
%   Die Geodätengleichung hat die Form
%  \begin{equation}
%  0=\tensor{\hat{U}}{^m}\tensor{\hat{\nabla}}{_m} \tensor{\hat{U}}{_n}
%  \end{equation}
%  Betrachten wir speziell die vierdimensionalen Komponenten $n=\nu$ so finden wir
% \begin{equation}
% \begin{split}
% 0&=\tensor{\hat{U}}{^m}\tensor{\hat{\nabla}}{_m} \tensor{\hat{U}}{_\nu}\\
% &=\tensor{\hat{U}}{^m}\tensor{\partial}{_m} \tensor{\hat{U}}{_\nu}
%  +\tensor*{\hat{\Gamma}}{_m_\ell_\nu}
%  \tensor{\hat{U}}{^m}\tensor{\hat{U}}{^\ell}\\
%  &=\dod{}{\lambda} \tensor{\hat{U}}{_\nu}
% +\tensor*{\hat{\Gamma}}{_\mu_\lambda_\nu}
%  \tensor{\hat{U}}{^\mu}\tensor{\hat{U}}{^\lambda}
%  +2\tensor*{\hat{\Gamma}}{_\mu_4_\nu}
%  \tensor{\hat{U}}{^4}\tensor{\hat{U}}{^\mu}
%  +\tensor*{\hat{\Gamma}}{_4_4_\nu}
%  \tensor{\hat{U}}{^4}\tensor{\hat{U}}{^4}\\
%   &=\dod{}{\lambda} \tensor{\hat{U}}{_\nu}
% +\tensor*{\hat{\Gamma}}{_\mu_\lambda_\nu}
%  \tensor{\hat{U}}{^\mu}\tensor{\hat{U}}{^\lambda}
%  +2Q\tensor*{\hat{\Gamma}}{_\mu_4_\nu}
%  \tensor{\hat{U}}{^\mu}
%  +Q^2\tensor*{\hat{\Gamma}}{_4_4_\nu}
%  \\
%    &=\dod{}{\lambda} \tensor{\hat{U}}{_\nu}
% +\tensor*{\Gamma}{_\mu_\lambda_\nu}
%  \tensor{\hat{U}}{^\mu}\tensor{\hat{U}}{^\lambda}
%  +\psi Q\tensor*{F}{_\mu_\nu}
%  \tensor{\hat{U}}{^\mu}
%  +Q^2\partial_\nu\psi\\
% \end{split}
% \end{equation}
% Insegammt findet man 
% \begin{equation}
% m\tensor{\ddot{x}}{^\mu}+m\tensor*{\Gamma}{^\mu_\nu_\lambda}
% \tensor{\dot{x}}{^\nu}\tensor{\dot{x}}{^\lambda}=
% \psi mQ\tensor*{F}{_\mu_\nu}
%  \tensor{\dot{x}}{^\mu}
%  +mQ^2\partial_\nu\psi
% \end{equation}
% Es liegt nahe $mQ=q$ zu setzen
% \begin{equation}
% m\tensor{\ddot{x}}{^\mu}+m\tensor*{\Gamma}{^\mu_\nu_\lambda}
% \tensor{\dot{x}}{^\nu}\tensor{\dot{x}}{^\lambda}=
% \psi q\tensor*{F}{_\mu_\nu}
%  \tensor{\hat{U}}{^\mu}
%  \tensor{\hat{U}}{^\mu}
%  +\frac{q^2}{m}\partial_\nu\psi
% \end{equation}
% % Dies kann wie folgt interpretiert werden: 
% % \begin{itemize}
% %   \item $-Q\tensor*{F}{_\mu_\nu}
% %  \tensor{U}{^\mu}$ beschreibt die gewöhnliche Lorentzkraft modifiziert mit d
% %  \item  $-Q^2\partial_\nu\psi$ beschreibt eine Kraft die durch ein Potential
% %  $Q^2\psi$ ausgeübt wird.
% % \end{itemize}
% Setzt man $\psi=1$ so erhält man die bekannten Gleichungen der Maxwell Theorie.
% \begin{bemerkung}
% Will man die Normierung $\tensor{U}{_\mu}\tensor{U}{^\mu}={0,-1}$ aufrecht
% erhalten so muss
% $\tensor{\hat{U}}{_m}\tensor{\hat{U}}{^m}=\tensor{U}{_\mu}\tensor{U}{^\mu}+Q^2={Q^2,Q^2-1}$
% gelten. Die Vektoren in fünf Dimensionen sind also je nach Ladung raumartig!
% Dies führt zu fragen bezüglich Kausalität
% \end{bemerkung}
\section{Die Rolle des skalaren Felds}
  Klein maß dem zusätzlichen Freiheitsgrad der durch das skalaarfeld
  $\sigma$\footnote{Bzw. $\phi$ oder $\psi$.} gegeben ist, keine physikalische
  Bedeutung bei.
  Demensprechend setzte er $\psi=\mathrm{const.}=1$ .
 Dadurch vereinfacht sich die Lagrangedichte zur Lagrangedichte der
 Einstein-Maxwell Theorie. Allerdings impliziert die dynamische Gleichung
 \eqref{eq:dymdilat}
  \begin{equation}
\tensor*{F}{_\mu_\nu}\tensor*{F}{^\mu^\nu}=0\,,
 \end{equation}
 der Beitrag der elektrische Beitrag zu \eqref{eq:Lagrange2} verschwindet also.
 Dies führt die Konstruktion ad absurdum. Einziger Ausweg ist die $\tensor{g}{_4_4}$-Komponente
\emph{a priori} konstant zu setzen und nicht zu variieren. Dies trübt
die Allgemeinheit der Theorie, da dadurch diese Komponente gegenüber den anderen
ausgezeichnet ist. Zusätzliche skalare Felder tauchen häufig in Theorien auf die
sich der dimensionalen Reduktion bedienen. Sie werden häufig mit einem Teilchen,
dem \emph{Dilaton} identifiziert. Zusätzliche Teilchen stellen an sich kein
Probleme dar, einige ungelöste Fragestellungen der Physik (dunkle
Materie/Energie) benötigen sie sogar um beantwortet zu werden. 
\section{Quantisierung der Ladung}
Die Kompaktheit von $\Sphere^1$ impliziert die Quantisierung der Ladung (Klein)
% \hat{R}^\star
% =e^{-\tau}\left(\hat{R}+3\tensor{\nabla}{_\mu}\left(\tensor{\hat{g}}{^\mu^\nu}\tensor{\partial}{_\nu}\tau\right)
% -\frac{3}{2}\tensor{\hat{g}}{^\mu^\nu}\tensor{\partial}{_\mu}\tau\tensor{\partial}{_\nu}\tau\right)\,,\quad\sqrt{-\hat{g}^\star}=e^{5\varphi}\sqrt{-\hat{g}}
% \end{equation}
% \begin{equation}
% \begin{split}
% \mathcal{L}^\star
% &=\hat{R}^\star\sqrt{-\hat{g}^\star}\\
% &=e^{3\varphi}\sqrt{-\hat{g}}\left(\hat{R}+\frac{16}{3}e^{-\frac{3}{2}\varphi}\square
% e^{\frac{3}{2}\varphi}\right)\\
% &=e^{3\varphi}\sqrt{-g}\phi\left(R-\frac{1}{4}\phi^3\tensor{F}{_\mu_\nu}\tensor{F}{^\mu^\nu}+\frac{16}{3}e^{-\frac{3}{2}\varphi}\square
% e^{\frac{3}{2}\varphi}\right)\\
% &=e^{3\varphi}\sqrt{-g}\phi\left(R-\frac{1}{4}\phi^3\tensor{F}{_\mu_\nu}\tensor{F}{^\mu^\nu}+12\pdif{_\mu}\varphi\pdif{^\mu}\varphi+8\square
% \varphi\right)\\
% &=e^{3\varphi}\sqrt{-g}\phi\left(R
% -\frac{1}{4}\phi^3\tensor{F}{_\mu_\nu}\tensor{F}{^\mu^\nu}
% +12\pdif{_\mu}\varphi\pdif{^\mu}\varphi
% +8e^{2\varphi}\square^\star\varphi+24e^{2\varphi}\pdif{_\mu}\varphi\pdif{^\mu}\varphi\right)
% \end{split}
% \end{equation}
% Es liegt nahe $\phi=e^{-3\varphi}$ zu setzen.
% \begin{equation}
% \begin{split}
% \hat{R}^\star\sqrt{-\hat{g}^\star}
% &=\sqrt{-g}\left(R+12\pdif{_\mu}\varphi\pdif{^\mu}\varphi+8\square
% \varphi\right)
% \end{split}
% \end{equation}
% $\sigma=\sqrt{24}\varphi$
% \begin{equation}
% \begin{split}
% \hat{R}^\star\sqrt{-\hat{g}^\star}
% &=\sqrt{-g}\left(R+\frac{1}{2}\pdif{_\mu}\sigma\pdif{^\mu}\sigma\right)
% \end{split}
% \end{equation}
% Coqueraux Jordan Thiery
% % Im Folgenden lassen wir den Term $2\pi
% % r$ weg, da er keinen Einfluss auf das Variationsprinzip hat. Die Lagrangedichte
% % ist damit
% % \begin{equation}
% % \begin{split}
% % \mathcal{L}=\sqrt{-g}\left(\phi
% % R-\frac{1}{4}\phi^3\tensor{F}{_\mu_\nu}\tensor{F}{^\mu^\nu}\right)
% % &=\sqrt{-g}\left(\phi
% % \mathcal{L}\textsubscript{g}+\phi^3\mathcal{L}\textsubscript{EM}\right)
% % \end{split}
% % \end{equation}
% % Die Tatsache dass sich die Lagrangedichte in einen Term der die Raumkrümmung
% % enthält und einen Elektromagnetischen Anteil aufspaltet ist auch als
% % Kaluza-Klein Wunder\footnote{"`Kaluza-Klein miracle"'} bekannt. Insbesondere
% % erhält man für $\phi=1$ die Lagrangedichte für ein System das sowohl einstein
% % als auch Maxwellgleichungen erfüllt. Da der Lagrangian keine Kinetischen Terme
% % in $\varphi$ enthält folgt
% % \begin{equation}
% % 0=\dpd{\mathcal{L}}{\phi}=\sqrt{-g}\left(R-\frac{3}{4}\phi^2\tensor{F}{_\mu_\nu}\tensor{F}{^\mu^\nu}\right)
% % \end{equation}
% % Weiter erhält man für die Variation nach den $A$ 
% % Das Feld $A$ ist Zyklisch, taucht als nicht selbst in der Lagrangedichte auf.
% % Die resultierende Erhaltungsgleichung lautet
% % \begin{equation}
% % 0=\tensor{\nabla}{_\alpha}\left(\dpd{\mathcal{L}}{\left(\tensor{\nabla}{_\alpha}\tensor{A}{_\beta}\right)}\right)
% =\tensor{\nabla}{_\alpha}\left(\phi^3\dpd{\tensor{F}{_\mu_\nu}\tensor{F}{^\mu^\nu}}{\left(\tensor{\nabla}{_\alpha}\tensor{A}{_\beta}\right)}\right)
% =4\tensor{\nabla}{_\alpha}\left(\phi^3\tensor{F}{^\alpha^\beta}\right)
% \end{equation}
% \begin{equation}
% \tensor{G}{_\mu_\nu}=\phi^2\tensor*{T}{*^{\text{M}}*_\mu*_\nu}
% \end{equation}
% Oder umformuliert
% \begin{equation}
% \tensor{\nabla}{_\alpha}\tensor{F}{^\alpha^\beta}
% =-\frac{3}{\phi}\tensor{F}{^\alpha^\beta}\tensor{\partial}{_\alpha}\phi
% \end{equation}
% Bzw:
% \begin{equation}
% \tensor{J}{^\beta}
% =-\frac{3}{\phi\sqrt{-g}}\tensor{F}{^\alpha^\beta}\tensor{\partial}{_\alpha}\phi
% =-3\left(-\hat{g}\right)^{-\nicefrac{1}{2}}\tensor{F}{^\alpha^\beta}\tensor{\partial}{_\alpha}\phi
% \end{equation}
% \begin{equation}
% \begin{split}
% \tilde{R}&=e^{-2\varphi}\left(R+\frac{16}{3}e^{-\frac{3}{2}\varphi}\square
% e^{\frac{3}{2}\varphi}\right)\\
% &=e^{-2\varphi}\left(R+8\square\varphi+12\pdif{_\mu}\varphi\pdif{^\mu}\varphi\right)
% \end{split}
% \end{equation}
% \begin{equation}
% \begin{split}
% \phi\sqrt{-g}R
% &=\phi
% e^{-\varphi}\sqrt{\tilde{g}}\left(\tilde{R}+8\square\varphi+12\pdif{_\mu}\varphi\pdif{^\mu}\varphi\right)\\
% \end{split}
% \end{equation}
% $\varphi=\ln \phi$
% \begin{equation}
% \begin{split}
% \phi\sqrt{-g}R
% &=\sqrt{\tilde{g}}\left(\tilde{R}
% +8{\square}\ln\phi 
% +\frac{12}{\phi^2}\pdif{_\mu}\phi\pdif{^\mu}\phi\right)
% \end{split}
% \end{equation}
% \begin{equation}
% \begin{split}
% \phi\sqrt{-g}R
% &=\sqrt{\tilde{g}}\left(\tilde{R}
% +8\frac{1}{\phi}\square\phi 
% +\frac{4}{\phi^2}\pdif{_\mu}\phi\pdif{^\mu}\phi\right)
% \end{split}
% \end{equation}
% \begin{equation}
% \begin{split}
% \phi\sqrt{-g}R
% &=\sqrt{\tilde{g}}\left(\tilde{R}
% +8\frac{1}{\phi}\square\phi 
% +16\pdif{_\mu}\Lambda\pdif{^\mu}\Lambda\right)
% \end{split}
% \end{equation}
% \begin{equation}
% \begin{split}
% 0&=\frac{\delta\mathcal{L}}{\delta\tensor{g}{_\mu_\nu}}\\
% &=\phi\frac{\delta\mathcal{L}\textsubscript{EH}}{\delta\tensor{g}{_\mu_\nu}}
% +\phi^3\frac{\delta\mathcal{L}\textsubscript{EM}}{\delta\tensor{g}{_\mu_\nu}}\\
% &=
% \phi\sqrt{-g}\left[\frac{1}{2}\tensor{g}{^\mu^\nu}
% R+\tensor{R}{^\mu^\nu}\right]
% \end{split}
% \end{equation}
% \begin{split}
% \tensor*{\hat{g}}{*^\prime_\mu_\nu}&=\tensor{\hat{g}}{_a_b}\dpd{f^a}{\tensor{x}{^\mu}}\dpd{f^b}{\tensor{x}{^\nu}}\\
% &=\tensor{\hat{g}}{_\alpha_\beta}\dpd{f^\alpha}{\tensor{x}{^\mu}}\dpd{f^\beta}{\tensor{x}{^\nu}}
% +\tensor{\hat{g}}{_\alpha_4}\dpd{f^\alpha}{\tensor{x}{^\mu}}\dpd{f^4}{\tensor{x}{^\nu}}
% +\tensor{\hat{g}}{_4_\beta}\dpd{f^4}{\tensor{x}{^\mu}}\dpd{f^\beta}{\tensor{x}{^\nu}}
% +\tensor{\hat{g}}{_4_4}\dpd{f^4}{\tensor{x}{^\mu}}\dpd{f^4}{\tensor{x}{^\nu}}
% \\
% &=\tensor{\hat{g}}{_\alpha_\beta}\dpd{f^\alpha}{\tensor{x}{^\mu}}\dpd{f^\beta}{\tensor{x}{^\nu}}
% +\psi\tensor{A}{_\alpha}\dpd{g^\alpha}{\tensor{x}{^\mu}}\pdif{_\nu}h
% +\psi\tensor{A}{_\beta}\dpd{g^\beta}{\tensor{x}{^\nu}}\pdif{_\mu}h
% +\psi\pdif{_\mu}h\pdif{_\nu}h\\
% &=\tensor{g}{_\alpha_\beta}\dpd{g^\alpha}{\tensor{x}{^\mu}}\dpd{g^\beta}{\tensor{x}{^\nu}}
% +\psi\left(\tensor{A}{_\alpha}\dpd{g^\alpha}{\tensor{x}{^\mu}}+\pdif{_\mu}h\right)
% \left(\tensor{A}{_\alpha}\dpd{g^\alpha}{\tensor{x}{^\nu}}+\pdif{_\nu}h\right)
% \end{split}
% \end{equation}
% \begin{equation}
% \begin{split}
% \tensor*{\hat{g}}{*^\prime_\mu_4}&=\tensor{\hat{g}}{_a_b}\dpd{f^a}{\tensor{x}{^\mu}}\dpd{f^b}{\tensor{x}{^4}}\\
% &=\tensor{\hat{g}}{_\alpha_\beta}\dpd{f^\alpha}{\tensor{x}{^\mu}}\dpd{f^\beta}{\tensor{x}{^4}}
% +\tensor{\hat{g}}{_\alpha_4}\dpd{f^\alpha}{\tensor{x}{^\mu}}\dpd{f^4}{\tensor{x}{^4}}
% +\tensor{\hat{g}}{_4_\beta}\dpd{f^4}{\tensor{x}{^\mu}}\dpd{f^\beta}{\tensor{x}{^4}}
% +\tensor{\hat{g}}{_4_4}\dpd{f^4}{\tensor{x}{^\mu}}\dpd{f^4}{\tensor{x}{^4}}\\
% &=\psi\tensor{A}{_\alpha}\dpd{g^\alpha}{\tensor{x}{^\mu}}\pdif{_4}h+\psi\pdif{_\mu}h\pdif{_4}h\\
% &=\psi\pdif{_4}h\left(\tensor{A}{_\alpha}\dpd{g^\alpha}{\tensor{x}{^\mu}}+\pdif{_\mu}h\right)
% \end{split}
% \end{equation}
% \begin{equation}
% \begin{split}
% \tensor*{\hat{g}}{*^\prime_4_4}&=\tensor{\hat{g}}{_a_b}\dpd{f^a}{\tensor{x}{^4}}\dpd{f^b}{\tensor{x}{^4}}\\
% &=\tensor{\hat{g}}{_\alpha_\beta}\dpd{f^\alpha}{\tensor{x}{^4}}\dpd{f^\beta}{\tensor{x}{^4}}
% +\tensor{\hat{g}}{_\alpha_4}\dpd{f^\alpha}{\tensor{x}{^4}}\dpd{f^4}{\tensor{x}{^4}}
% +\tensor{\hat{g}}{_4_\beta}\dpd{f^4}{\tensor{x}{^4}}\dpd{f^\beta}{\tensor{x}{^4}}
% +\tensor{\hat{g}}{_4_4}\dpd{f^4}{\tensor{x}{^4}}\dpd{f^4}{\tensor{x}{^4}}\\
% &=\psi\left(\pdif{_4}h\right)^2
% \end{split}
% \end{equation}
% Um konsistent zu bleiben muss $\pdif{_4}h = 1$ also sind die erlaubten
% transformationen von der Form
% \begin{equation}
% \tensor{g}{_\mu_\nu}\to\tensor{g}{_\alpha_\beta}\dpd{g^\alpha}{\tensor{x}{^\mu}}\dpd{g^\beta}{\tensor{x}{^\nu}}\,,\quad
% \tensor{A}{_\alpha}\to\tensor{A}{_\alpha}\dpd{g^\alpha}{\tensor{x}{^\mu}}+\pdif{_\mu}h\\
% \psi\to\psi
% \end{equation}
%  
%  In Verallgemeinerung der vierdimensionalen Geodätischen
%  \begin{equation}
%  0=\tensor{\hat{U}}{^m}\tensor{\nabla}{_m}
%  \tensor{\hat{U}}{_n}=\dod{}{\lambda}
%  \tensor{\hat{U}}{_n}+\tensor{\hat{U}}{^m}\tensor*{\hat{\Gamma}}{^\ell_m_n}\tensor{\hat{U}}{_\ell}
%  \end{equation}
%  % Erhaltungsgröße zu killingtensor d4
%  \begin{equation}
%  \begin{split}
%   0&=\tensor{\hat{U}}{^m}\tensor{\nabla}{_m} \tensor{\hat{U}}{_5}\\
%  &=\dod{}{\lambda}\tensor{\hat{U}}{_5}
%  +\tensor{\hat{U}}{^m}\tensor*{\hat{\Gamma}}{*^n_5_m} \tensor{\hat{U}}{_n}\\
%  &=\dod{}{\lambda}\tensor{\hat{U}}{_5}
%  +\frac{1}{2}\tensor{\hat{U}}{^m}\tensor{\hat{g}}{^n^a}\left(\tensor{\hat{g}}{_a_m_{,5}}
%  +\tensor{\hat{g}}{_a_5_{,m}}
%  -\tensor{\hat{g}}{_m_5_{,a}}
%  \right)\tensor{\hat{U}}{_n}\\
%   &=\dod{}{\lambda}\tensor{\hat{U}}{_5}
%  +\frac{1}{2}\tensor{\hat{U}}{^m}\tensor{\hat{U}}{^a}\left(
%  \tensor{\hat{g}}{_a_5_{,m}}
%  -\tensor{\hat{g}}{_m_5_{,a}}
%  \right)\\
%   &=\dod{}{\lambda}\tensor{\hat{U}}{_5}\\
%  \end{split}
%  \end{equation}
%  Der zweite Term verschwindet als Spur eines Produkt eines symmetrischen mit
%  einem antisymmetrischen Tensors.
\chapter{Vorhersagen}
\begin{itemize}
  \item Lorentz Kraft
  \item Einstein Gleichungen
  \item MW Gleichungen 
  \item Eichinvarianz (unter Diffeomorphismen für g)
  \item Eichinvarianz (unter Eichtrafos für MW-Gl.)
  \item Dilaton (Bewegungsgleichung ist nicht Klein Gordon Gleichung!)
\end{itemize}
Die Kaluza Klein Theorie weicht von der Einstein-Maxwell Theorie ab. Die
Kaluza-Klein-Theorie ist also falsifizierbar. 
Dilaton muss massiv sein um geringe Abweichungen zu erklären.

\chapter{Im Kontext moderner Theorien}
Yang-Mills
%Duff\footnote{M. J. Duff, GR11 Konferenz, Stockholm 1989.}
%\begin{quote}
% Kaluza-Klein is dead, long
% live Kaluza-Klein!
%\end{quote}
\section{Diskussion}
%TODO warum einstein und nicht lovelock in 5d?
\section{TODOs}
\begin{enumerate}

  \item Yang-Mills
  \item proper time
  \item Quantisierung der Ladung
  \item $\delta$ definition so anpassen dass mit Chap3 konsistent
  \item Geodäten -> Lorentzkraft
  \item Chap4
  \item Kovariante vs parielle Ableitung in Lagrangeformalismus
  \item Hoch/runterziehen von indices
  \item Probleme:
  \begin{enumerate}
   \item Fermionenmassen
   \item quantisized mass
   \end{enumerate}


    
\end{enumerate}
evtl. zu viel:
\begin{enumerate}
   \item Homogene Räume -> Normalteiler etc.
     \item Fermionen
\end{enumerate}
\input{07_Moderne_Formulierung}
\input{A_Appendix}
\bibliographystyle{alpha} 
%\nocite{*}
\bibliography{bibfile}


\end{document}