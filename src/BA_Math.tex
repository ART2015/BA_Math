\documentclass[a4paper,oneside]{scrreprt}

%%% PACKAGES + MODIFICATIONS%%%
 
%language and encoding
\usepackage[utf8]{inputenc}
\usepackage[T1]{fontenc} 
\usepackage[english,ngerman]{babel}
\usepackage[autostyle=true,german=quotes]{csquotes}
 
%math
\usepackage{amsmath}
\usepackage{amsthm,thmtools}
\usepackage{amssymb}
\usepackage{amsfonts} 
\usepackage{mathtools}
\usepackage{commath}
\usepackage{dsfont}		% double stroke characters
\usepackage{braket}
\usepackage{mdframed}
\usepackage{eufrak}
\usepackage{tensor}

%science
\usepackage{units}
\usepackage{bpchem}	% type­set chem­i­cal names, for­mu­lae, etc

%layout + style
\usepackage{sectsty}
\usepackage[automark,headsepline]{scrlayer-scrpage} 						% headings
\renewcommand*{\headfont}{\normalfont}										% nicht kursive Kopfzeile
\usepackage{setspace}														% set space be­tween lines
\usepackage[font=small,labelfont=bf,labelsep=endash,format=plain]{caption}	%
% change style of captions
\usepackage[section]{placeins}												% de­fines a \FloatBar­rier com­mand, be­yond which floats may not pass.
\usepackage[protrusion=true,expansion=true]{microtype}						% sublim­i­nal re­fine­ments to­wards ty­po­graph­i­cal per­fec­tion
\usepackage{enumerate} % better way to config enumerates
\usepackage{pdflscape} % landscape mode
\usepackage{afterpage}\addtokomafont{caption}{\small\linespread{1}\selectfont}					% Ändert Schriftgröße und Zeilenabstand bei captions

%graphics
\usepackage{graphicx}
\usepackage{xcolor}
\usepackage{subfig}		% subfigures
\usepackage{wrapfig}	% pro­duces fig­ures which text can flow around
\usepackage[export]{adjustbox}%

%tikz
\usepackage{tikz}
\usepackage{tikz-cd}
\usetikzlibrary{trees}
\usetikzlibrary{decorations.pathmorphing}
\usetikzlibrary{decorations.markings}
\usetikzlibrary{positioning,arrows,shapes}

\tikzset{
    partial ellipse/.style args={#1:#2:#3}{
        insert path={+ (#1:#3) arc (#1:#2:#3)}
    }
}

%tables
\usepackage{tabularx}	% tab­u­lars with ad­justable-width columns
\usepackage{multirow}	% create tab­u­lar cells span­ning mul­ti­ple rows
\usepackage{booktabs}	% publi­ca­tion qual­ity ta­bles in LaTeX
\usepackage{array}
\usepackage{dcolumn}	% align on the dec­i­mal point of num­bers in tab­u­lar columns

%hyperref
\usepackage[pdftex]{hyperref}
\hypersetup{
	pdftitle={Kaluza-Klein-Theorie},
	pdfsubject={Cosmology},
	pdfkeywords={Kaluza, Klein, Differential Geometrie},
	pdfauthor={Michael Ruf},
	pdfcreator={Michael Ruf},
	pdfproducer={Michael Ruf},
	bookmarksnumbered=true, % bookmarks are numbered
	bookmarksopen=true,     % show bookmarks at start of pdf viewer
	bookmarksopenlevel=2,   % level to which the bookmarks are opened
	bookmarksdepth=2,       % depth of bookmarks 
	unicode=true,           % non-Latin characters in pdf viewer's bookmarks
	pdftoolbar=false,       % show pdf viewer's toolbar?
	pdfmenubar=true,        % show pdf viewer's menu?
	pdffitwindow=false,     % window fit to page when opened
	pdfstartview={FitH},    % fits the width of the page to the window
	pdfnewwindow=true,      % links in new window
	pdfborder={0 0 1},		% no border for links
	colorlinks=false,		% false: black links; true: colored links
	linkcolor=section_color,         % color of internal links (change box color with
	% linkbordercolor)
	citecolor=green,        % color of links to bibliography
	filecolor=magenta,      % color of file links
	urlcolor=blue			% color of external links
}
\usepackage{nameref}

%general stuff
\usepackage{cite}
\usepackage{glossaries}	% create glos­saries and lists of acronyms

%\usepackage[nohyperlinks]{acronym}
\setcounter{tocdepth}{1} 

\newcommand{\HRule}{\rule{\linewidth}{0.5mm}}			% line for title page
\newcommand*\EmptyPage{\newpage\null\thispagestyle{empty}\newpage}

\addto\extrasenglish{
	\renewcommand{\chapterautorefname}{Chapter}
	\renewcommand{\sectionautorefname}{Chapter}
	\renewcommand{\subsectionautorefname}{Chapter}
 	\renewcommand{\subsubsectionautorefname}{Chapter}
}

\addto\extrasngerman{
	\renewcommand{\sectionautorefname}{Kapitel}
	\renewcommand{\subsectionautorefname}{Kapitel}
 	\renewcommand{\subsubsectionautorefname}{Kapitel}
}
\newcommand{\const}{\ensuremath{\mathrm{const.}}}		


% general math

\newcommand{\transpose}{^\top}
\DeclareMathOperator{\tr}{Tr}
\DeclareMathOperator{\difD}{D}
\DeclareMathOperator{\diag}{diag}
\DeclareMathOperator{\re}{Re}
\DeclareMathOperator{\im}{Im}
\DeclareMathOperator{\id}{id}
\DeclareMathOperator{\vol}{vol}

% vector calculus

\DeclareMathOperator{\Div}{div}
\DeclareMathOperator{\Rot}{rot}
\DeclareMathOperator{\Grad}{grad}
\DeclareMathOperator{\Un}{U}
\DeclareMathOperator{\Or}{O}
\DeclareMathOperator{\SpUn}{SU}
\DeclareMathOperator{\SpOr}{SO}

% sets

\newcommand{\Reals}{\ensuremath{\mathbb{R}}}
\newcommand{\Complex}{\ensuremath{\mathbb{C}}}
\newcommand{\Integers}{\ensuremath{\mathbb{Z}}}
\newcommand{\Sphere}{\ensuremath{\mathbb{S}}}

% relativity

%\newcommand{\cSym}[3]{\ensuremath{\begin{Bmatrix} #1 \\ #2 #3 \end{Bmatrix}}}
\newcommand{\cSym}[3]{\ensuremath{\tensor*{\Gamma}{^{#1}_{#2}_{#3}}}}
\newcommand{\csym}[3]{\ensuremath{[#1 #2,\, #3]}}
\newcommand{\affin}[3]{\ensuremath{\Gamma^{#1}_{#2 #3}}}
\newcommand{\fourint}{\int\dif{}^4 x\,}
\newcommand{\lagrangian}{\mathcal{L}}
\newcommand{\liedif}[2]{\ensuremath{\mathcal{L}_{#1}#2}}
\newcommand{\pdif}[1]{\tensor{\partial}{#1}}
\newcommand{\cdif}[1]{\tensor{\partial}{#1}}

% area functions
\DeclareMathOperator{\artanh}{artanh}
\DeclareMathOperator{\arsinh}{arsinh}
\DeclareMathOperator{\arcosh}{arcosh}

% miscellaneous

\newcommand{\imI}{\ensuremath{\mathrm{i}}}
\newcommand{\landauO}{\mathcal{O}}
\newcommand{\name}[1]{\textsc{#1}}
\newcommand\grad{\ensuremath{^\circ}}
\let\originalleft\left  %fix bracket spacing when using \left( \right)
\let\originalright\right
\renewcommand{\left}{\mathopen{}\mathclose\bgroup\originalleft}
\renewcommand{\right}{\aftergroup\egroup\originalright}
\renewcommand{\vec}{\mathbf}
\newcommand\gmal{.}

%new environments
 
\newtheorem{theorem}{Theorem}
\newtheorem{lemma}{Lemma}
\newtheorem{satz}{Satz}
\newtheorem{proposition}{Proposition}
\newtheorem{korollar}{Korollar}

\theoremstyle{definition}
\newtheorem{definition}{Definition}
\newtheorem{beispiel}{Beispiel}

\theoremstyle{remark}
\newtheorem{bemerkung}{Bemerkung}

\newenvironment{tabulars}[1]{\renewcommand*{\arraystretch}{2}\tabular{#1}}{\endtabular}		% stretched table

%%% DOCUMENT %%%
\newglossary[slg]{symbolslist}{syi}{syg}{Symbolverzeichnis}

\makeglossaries

\newglossaryentry{symb:Pi}{
name=$\pi$,
description={Die Kreiszahl.},
sort=symbolpi, 
type=symbolslist
}

\newglossaryentry{latex}
{
    name=latex, 
    description={Is a mark up specially suited 
    for scientific documents}
}
 
\newglossaryentry{maths} 
{
    name=mathematics,
    description={Mathematics is what mathematicians do}
}


\begin{document}
\selectlanguage{ngerman}
\onehalfspacing

\hypersetup{pageanchor=false} %stop page numbering (hyperref) to prevent for double page numers


\begin{titlepage}
\begin{center}
	\HRule \\[0.4cm]
	{ \huge \bfseries Kaluza Klein Theorie}\\
	\HRule \\[0.5cm]
	\begin{flushright}
  		\Large \textbf{Michael Ruf}\\[1cm]
  	\end{flushright}
  \vspace*{\fill}	
%  \includegraphics{logo.eps}\\
  \vspace*{\fill}
  \normalsize
  \textsc{Mathematisches Institut} \\
  \textsc{Albert-Ludwigs-Universität} \\
  \textsc{Freiburg im Breisgau} \\[1cm]
  
  \large Freiburg im Breisgau \\
  Mai 2016
\end{center}
\end{titlepage} 



\begin{titlepage}
\begin{center}
	\HRule \\[0.4cm]
	{ \huge \bfseries Kaluza-Klein-Theorie}\\
	\HRule \\[2cm]
	
	\textsc{\LARGE Bachelorarbeit}\\[1.5cm]
	
	\large Autor \\
  	\Large Michael \textsc{Ruf}\\[1.5cm]
  	
  	\large Betreuerin\\
  	\Large Prof. Dr. Katrin \textsc{Wendland}
  	\vfill
	\normalsize
	\textsc{Mathematisches Institut} \\
	\textsc{Albert-Ludwigs-Universität} \\
	\textsc{Freiburg im Breisgau} \\[2cm]
  
	\large Freiburg im Breisgau \\
	2. Februar 2015
\end{center}
\end{titlepage} 



\EmptyPage

\chapter*{Erklärung}
\vspace{2.5cm}
Hiermit versichere ich, die eingereichte Bachelorarbeit selbständig verfasst und 
keine anderen als die von mir angegebenen Quellen und Hilfsmittel benutzt zu haben. 
Wörtlich oder inhaltlich verwendete Quellen wurden entsprechend den anerkannten Regeln 
wissenschaftlichen Arbeitens (lege artis) zitiert. Ich erkläre weiterhin, dass die 
vorliegende Arbeit noch nicht anderweitig als Bachelorarbeit eingereicht wurde. 
\\[3.5cm]
\begin{tabular}{p{7cm}p{.5cm}l}
\dotfill \\ 
Ort, Datum
\end{tabular}
\vspace{1,5 cm} 
\begin{tabular}{p{7cm}p{.5cm}l}
\dotfill \\ 
Michael Ruf 
\end{tabular}
\thispagestyle{empty}

\pagenumbering{roman}
\setcounter{page}{1}
\cfoot[- \textit{\pagemark} -]{- \textit{\pagemark} -}

\tableofcontents
\newpage

\pagenumbering{arabic} 
\hypersetup{pageanchor=true} %start page numbering again
\setcounter{page}{1}
\cfoot[\pagemark]{\pagemark}

   
\chapter{Einleitung}
\section{Historischer Hintergrund}
Die Idee, dass die Welt in ihren Grundfesten geometrischer Natur ist
existiert ist nicht neu.
Beispielsweise führte Keppler die Abstände der Planetenbahnen auf
verschachtelte platonische Körper zurück. Auch manche Schulen der griechischen
Philosophie glaubten, dass Geometrie neben den Zahlen, das sei was das Universum
im wesentlichen ausmacht.
Selbsversändlich hatten sie dabei weder Differentialgeometrie noch die 
Relatitätstheorie im Sinn.  
Im laufe der Jahre löste sich die physikalischen Theorie immer weiter von dieser Idee. 
Zu Zeiten Kaluzas alle Kräfte durch Geometrie beschrieben
 

Mit der allgemeinen Relativitätstheorie gelang es Albert Einstein 1915
zumindest die Gravitation als eine von vier elementaren Kräften durch Geometrie
zu beschreiben. Die verbleibenden Kräfte, die schwache, die starke und die elektromagnetische Kraft
blieben grundsätzlich unangetastet. Es liegt Nahe, besonders da die Theorien
viele Ähnlichkeiten aufweisen, dass auch die anderen Kräfte durch eine ähnliche
Theorie beschreiben werden können.

Die neue Theorie soll dabei der alten 
möglichst ähnlich sein, also eine minimale Erweiterung der allgemeinen
Relativitätstheorie darstellen.
Da die Einsteingleichungen keine
zusätzlichen Freiheitsgrade enthalten, führt man heuristisch zusätzliche
Dimensionen ein, behält aber die Struktur der Gleichungen.
Da makroskopische Zusatzdimensionen aus verschiedenen Gründen ausgeschlossen
sind müssen die Zusatzdimensionen kompakt sein.
Zusammenfassend zeichnet sich eine solche Theorie durch folgende Punkte aus:
\begin{enumerate} 
\item Die Physik lässt sich vollständig durch Geometrie beschreiben.
\item Die uns zugängliche, vierdimensionale Raumzeit ist eingebettet in einen
höherdimensionalen Raum, welcher kompakte Zusatzdimensionen enthält.
\item Die Theorie ist eine minimale Erweiterung der allgemeinen
Relativitätstheorie.
\item Die Projektion auf vier Dimensionen liefert die uns bekannten Gesetze, mit
möglicherweise kleinen Abweichungen, welche sich auf den von beobachteten Skalen
nicht zeigen.
\end{enumerate}
\section{Konventionen}
Bereits Rechnungen in der "`gewöhnlichen"' Relativitätstheorie haben die Tendenz
unübersichtlich zu werden, das Einführen zusätzlicher Dimensionen hilft dabei
natürlich kaum.
Es ist deshalb an dieser Stelle nützlich einige Konventionen festzulegen. Im
Folgenden bezeichnet $n$ die Anzahl der Zusatzdimensionen.\footnote{Im Weiteren
meist $n=1$}

Die Metrik soll stets die Signatur $(1,3)$, bzw. $(1,3+n)$, insbesondere sind
die Zusatzdimensionen. Die meisten Rechnungen werden in lokalen Koordinaten
durchgeführt.
Wir beginnen die Indizierung bei 0, wobei die 0. Komponente mit der Zeit identifiziert wird.
Griechische Indices laufen über ersten vier Raum-Zeit Koordinaten
$\mu=0,\ldots,3\,$, lateinische über alle inklusive der
Zusatzkomponenten \footnote{Verwirrung bezüglich der gängigen Konvention
mit lateinischen Indices die Raumkomponenten zu benennen sollte dabei nicht
aufkommen.}, $i=0,\ldots,3+n\,$. Weiter verwenden wir die Einsteinsche
Summenkonvention, d.h. über paare von Indizes wird implizit summiert.

Im Zusammenhang mit Zusatzdimensionen tauchen Größen auf, die sowohl ein 4, als
auch ein $4+n$ dimensionales Pendant besitzen. Um diese von einander zu
unterscheiden, kennzeichnen wir die $4+n$ dimensionale Version mit einem
Zirkumflex.
Beispielsweise bezeichnen wir den $4+n$ dimensionalen Ricci-Tensor mit
$\tensor{\hat{R}}{_i_j}$. Mit $g$ bezeichnen wir die Determinante der Metrik.
\subsection*{Geometrisierte Einheiten}
Wir verwenden geometrisierte Einheiten, d.h. Einheiten in denen
$8\pi G=c=1$. Es verbleibt nur noch eine Längendimension.
\chapter{Mathematische Grundlagen}
Die meisten KK-Theorien machen in der einen oder anderen Form Annahmen über
Symmetrien der Raumzeit-Manigfalitgeit. Dies ist nötig um die Komplexität der
auftretenden Ausdrücke zu reduzieren und nicht zuletzt auch dafür diese mit der
physikalischen Realität in Verbindung zu setzen.
In diesem Kapitel werden wir die mathematischen Grundlage zur Formulierung von
Symmetrien in der Physik legen. Zunächst wiederholen wir einige Grundkonzepte
der Differentialgeometrie.
\section{Differentialgeometrie}
Christoffel Symbole, Riemann-Tensor, Ricci-Tensor, Geodätische. Faserbündel
\begin{figure}
\centering
\vspace*{0cm}
\begin{tikzpicture}[baseline]
   \draw[dashed] (1.3,-1.33) [partial ellipse= 90:270:0.5cm and 1cm];
   \draw[dashed] (-1.3,-1.33) [partial ellipse=90:-90:0.5cm and 1cm];
   \draw[thick, red,dashed] (-0,-1.5) [partial ellipse=270:90:0.4cm and 1cm];
   \node[fill=white] at (0.7,-1.2) {$p$};
  \node at (0.4,-1.5){\textbullet};
   \node at (1.8,-1.3){\textbullet};
   \node at (-1.8,-1.3){\textbullet};
\fill[fill=gray,fill opacity = 0.2]  (-3.5,0) -- (0, 2.5)  -- (3.5,0);
\fill[fill=gray,fill opacity = 0.2]  (-3.5,0) -- (0, -2.5)  -- (3.5,0);
\draw[fill=gray,fill opacity = 0.2] (-3.5,0) .. controls (-3.5,2) and (-1.5,2.5) .. (0,2.5);
\draw[xscale=-1,fill=gray,fill opacity = 0.2] (-3.5,0) .. controls (-3.5,2) and (-1.5,2.5) .. (0,2.5);
\draw[rotate=180,fill=gray,fill opacity = 0.2] (-3.5,0) .. controls (-3.5,2) and (-1.5,2.5) .. (0,2.5);
\draw[yscale=-1,fill=gray,fill opacity = 0.2] (-3.5,0) .. controls (-3.5,2) and (-1.5,2.5) .. (0,2.5);

\draw (-2,.2) .. controls (-1.5,-0.3) and (-1,-0.5) .. (0,-.5) .. controls (1,-0.5) and (1.5,-0.3) .. (2,0.2);

\draw[fill=white] (-1.75,0) .. controls (-1.5,0.3) and (-1,0.5) .. (0,.5) .. controls (1,0.5) and (1.5,0.3) .. (1.75,0);
\draw[fill=white] (-1.75,0) .. controls (-1.5,-0.3) and (-1,-0.5) .. (0,-.5) .. controls (1,-0.5) and (1.5,-0.3) .. (1.75,0);
  \draw[thick, red] (-0,-1.5) [partial ellipse=90:-90:0.4cm and 1cm];
   \draw (1.3,-1.33) [partial ellipse=90:-90:0.5cm and 1.cm];

    \draw (-1.3,-1.33)  [partial ellipse=270:90:0.5cm and 1cm];

    \draw[dashed,  blue!80!black,thick] (-0,0) [partial ellipse=-180:0:3.5cm and 1.5cm];
    \draw[thick,  blue!80!black] (-0,0) [partial ellipse=-110:-70:3.5cm and 1.5cm];
\end{tikzpicture}
\quad
\begin{tikzpicture}[baseline]	
	\node (A) at (-1.45,0.35) {};
	\node (B) at (-0.33,3.85){};
	\node (C) at (-0.3683,-3.85) {};
	\node (D) at (-0.97,1.24) {};
	\node (E) at (-0.11,0.34) {};

	\node (H) at (-0.7,-0.6){};
	\node (I) at (-0.7,-2.65){};
	\node (J) at (-0.7,-2.65){};
	\node (K) at (2.3,0.5){};
	
	\node (F) at (-1.5,3) {};
	\node (G) at (0.71,3.39){};
	\node (L) at (-1.2,2.3){};
	
	\node (i1) at (-1.1,0.6) { };
	\node (i2) at (-2,2.5)  { };
	\node (i3) at (-1.5,3.4) { };
	\node (i4) at (-0.5,2.2) { };
	\node (i5) at (-2.1,-2) { };
	\node (i6) at (-0.4,-3.7) { };
	\node (i7) at (2.2,0.5) { };
	\node (i8) at (0.7,3.7) { };
	\node (i9) at (0.3,3.2) { };
	\node (i10) at (0.35,2) { };
	\node (i11) at (-0.55,-0.1) {};
	\node (i12) at (-1.45,-2) { };
	\node (i13) at (-0.7,-2.85) { };
	\node (i14) at (0.15,-2) { };
	
	
	\draw[dashed, ultra thick] (F.center)
	to[out=230, in =130,looseness=1.6] (L.center) 
	to[out=180+130, in =180+250,looseness=0.8] (D.center)
	to[out =250,in=235] (E.center)
	to[out =235+180,in=180-210,looseness=0.6] (G.center);
	
	\draw[red,ultra thick,dashed] (I.center) to[out = 180+190,in=180-170] (C.center);
	
	\draw[fill=gray,draw=none,fill opacity=0.2]
	(A.center)
	to[out= -235,in=190,looseness=1.5] (B.center) 
	to[out=10, in = 90,looseness=1.2] (K.center)
	to[out=-90, in = 0,looseness=1.] (C.center)
	to[out=180, in = 235,looseness=1.3] (A.center)
	to[out=235+180, in = 250,looseness=1.2] (D.center)
	to[out=250+180, in = 250+180,looseness=1.3] (E.center)
	to[out=250, in = 235+180,looseness=0.9] (H.center)
	to[out=235, in = 180,looseness=1.2] (I.center)
	to[out=0, in = 180-235,looseness=1.2] (H.center)
	--cycle
	;

	\draw[thick,gray] (i1.center)
	to[out= 130, in = 90+180] (i2.center)
	to[out =90,in=180,looseness=0.7] (i3.center)
	to[out=0,in=90] (i4.center)
	to[out=180+90,in=180-120] (i1.center)
	to[out=-120,in=90] (i5.center)
	to[out=-90,in=180] (i6.center)
	to[out=0,in=-90] (i7.center)
	to[out=90,in=0,looseness=0.9] (i8.center)
	to[out=180,in=90] (i9.center)
	to[out=-90,in=90] (i10.center)
	to[out=-90,in=180-120] (i11.center)
	to[out=-120,in=90] (i12.center)
	to[out=-90,in=180] (i13.center)
	to[out=0,in=-90] (i14.center)
	to[out=90,in=180+120] (i11.center);
	to[out=130,in=180+130,dashed] (i1.center);
	
	\draw[dashed,thick,gray] (i11.center) to[out=130,in=180+130] (i1.center);
	\draw[ultra thick](A.center)
	to[out= -235,in=190,looseness=1.5] (B.center) 
	to[out=10, in = 90,looseness=1.2] (K.center)
	to[out=-90, in = 0,looseness=1.] (C.center)
	to[out=180, in = 235,looseness=1.3] (A.center)
	to[out=235+180, in = 250,looseness=1.2] (D.center)
	to[out=250+180, in = 250+180,looseness=1.3] (E.center)
	to[out=250, in = 235+180,looseness=0.9] (H.center)
	to[out=235, in = 180,looseness=1.2] (I.center)
	to[out=0, in = 180-235,looseness=1.2] (H.center);
	
	\draw[ultra thick] (F.center) to[out=230, in =180-210,looseness=1.3] (G.center);
	\draw[red,ultra thick] (I.center) to[out = -130,in=170] (C.center);
\end{tikzpicture}
\caption{Torrus und Kleinstsche Flasche sind Beispiele
für $\Sphere^1$-Faserbündel über $\Sphere^1$.}
\end{figure}
\subsection{Äußeres Kalkül}
Äußere Ableitung, Hodge Dual
\section{Symmetrien}
Wir beginnen mit einem einfachen Beispiel für eine Symmetrie, dass uns auch im
Folgenden als Anschauungsobjekt dienen wird.
\begin{beispiel}[Symmetrien eines Zylinders]
Wir betrachten einen Zylinder 
\begin{equation}
Z=\Reals\times\Sphere^1=\left\{(x,y,z)\in\Reals^3\,\Big|\,x^2+y^2=1\right\}\,,
\end{equation}
als Teilmenge des $\Reals^3$. 
Offensichtlich besitzt dieser zwei Symmetrien. Zum Einen wird er durch
Drehung um die $z$-Achse in sich selbst überführt (Rotationsinvarianz), zum
Anderen können wir den Ursprung entlang der $z$-Achse beliebig wählen
(Translationsinvarianz).
\end{beispiel}
Eine formale Beschreibung solcher kontinuierlicher Symmetrien bieten die
so genannten \emph{Lie-Gruppen}, Gruppen die zusätzlich eine Differenzierbare
Struktur besitzen.
\begin{definition}[Lie-Gruppe]
Sei $G$ sowohl eine glatte Manigfaltigkeit als auch eine Gruppe mit
Multiplikation $m:(g,h)\mapsto g\cdot h$ und Inversion $i:g\mapsto g^{-1}$.
Sind die Abbildungen $m$ und $i$ glatt so heißt $G$ \emph{Lie-Gruppe}.
\end{definition}
\begin{bemerkung}
Der Zylinder $Z$ besitzt offensichtlich auch eine Spiegelsymmetrie bezüglich der
Koordinatenachsen. Dabei
handelt es sich aber um eine diskrete Symmetrien, welche sich nicht durch Lie-Gruppen beschreiben lassen.
\end{bemerkung}
\begin{beispiel}[$SO(n)$]% TODO
\end{beispiel}
In unserem Fall ist die Lie-Gruppe die die Operationen beschreibt die Gruppe der
Drehungen um die $z$-Achse. Diese wird mit $SO(2)$ bezeichnet, eine Darstellung
ergit sich beispielsweise durch Matrizen der Form
\begin{equation}
R(\alpha)=
\begin{pmatrix}
\cos\alpha&\sin\alpha&0\\
-\sin\alpha&\cos\alpha&0\\
0&0&1
\end{pmatrix}
\end{equation}
%TODO zusammenhang SO(2) und R/Z
Das diese Gruppe eine differenzierbare Struktur besitzt ist klar da die
Komponenten der Matrizen differenzierbar sind.
Gruppen können auf Mengen operieren.  Um die Diskussion allgemeiner zu gestalten definieren wir
Operationen allgemeiner Gruppen. 
Eine sinnvolle Operation einer Gruppe sollte mit den Gruppenoperationen
kompatibel sein. 
%TODO wedge product äußere Ableitung 
\begin{definition}[Gruppenwirkung]
Sei $G$ eine Gruppe, $X$ eine Menge. Eine Abbildung
\begin{equation}
\Phi:G\times X\to X\,,\quad (g,x)\mapsto\Phi_g(x)
\end{equation}
heißt \emph{Gruppenwirkung} von $G$ auf $X$, falls die folgende Eigenschaften
erfüllt sind
\begin{enumerate}
  \item \emph{Identität}: für das neutrale Element $e\in G$ gilt
  $\displaystyle\Phi_e=\id_X$
  \item \emph{Verträglichkeit}: $\displaystyle\Phi_{gh}=\Phi_g\circ\Phi_h$
\end{enumerate}
$X$ heißt dann auch $G$-Menge.
\end{definition}
Statt $\Phi_g(x)$ schreiben wir im folgenden kurz $g\gmal x$.
Ist die Gruppe $G$ eine Lie-Gruppe, $X=M$ eine Mannigfaltigkeit und $\Phi_g$
glatt, so spricht man von einer \emph{Lie-Gruppenwirkung}. $M$ heißt dann
auch $G$-Mannigfaltigkeit.
\begin{definition}
Sei $X$ eine $G$-Menge. Die Wirkung von $G$ heißt:
\begin{enumerate}
  \item \emph{eigentlich}, falls unter der Abbildung $\Gamma:
  G\times X\to X\times X\,,(g,x)\mapsto (g \gmal x,x)$, Urbilder kompakter
  Mengen kompakt sind
  \item \emph{(Fixpunkt-)frei}, falls nur die Identität Fixpunkte besitzt, d.h.
  aus $g\gmal x=x$, folgt $g=e$.
\end{enumerate}
\end{definition}
\begin{beispiel}[Wirkung von $\Reals/\Integers$ auf einem Zylinder]
Sei $G=(\Reals/\Integers,+)$, $Z$ wie oben. Eine Wirkung von $G$ auf $Z$ ist
erklärt durch
\begin{equation}
t.p= \begin{pmatrix}
\cos\left( 2\pi t\right)&\sin\left( 2\pi t\right)&0\\
-\sin\left( 2\pi t\right)&\cos\left( 2\pi t\right)&0\\
0&0&1
\end{pmatrix}p\,,\quad p\in Z\subset \Reals^3
\end{equation}
Wie man sich leicht klar macht ist die Wirkung glatt, frei und eigentlich.
\end{beispiel}
\begin{figure}[!htbp]
\centering
\begin{tikzpicture}
\draw[-latex] (2,1)-- (-2,-1)  ;
\draw[-latex]  (-2,1)--(2,-1)  ;
\draw[fill=white,draw=none] (-1,4) -- (-1,0) arc (180:360:1cm and 0.5cm) -- (1,4) arc (-180:0:-1cm and 0.5cm) ;
\draw[fill=white,draw=none] (0,4) ellipse (1cm and 0.5cm);
\draw[-,thick, dashed] (0,-0.5) -- (0,4) ;
\draw[dashed] (2,1)-- (-2,-1)  ;
\draw[dashed]  (-2,1)--(2,-1)  ;
\draw[-,thick] (0,-1) -- (0,-0.5) ;
\node at (0,0) {\tiny\textbullet};
\node at (0,4) {\tiny\textbullet};
\draw[densely dashed] (-1,2) arc (180:0:1cm and 0.5cm);
\draw[fill=gray,fill opacity = 0.1,draw=none] (0,4) ellipse (1cm and 0.5cm);
\draw[fill=gray,fill opacity = 0.2] (-1,4) -- (-1,0) arc (180:360:1cm and 0.5cm) -- (1,4) arc (-180:0:-1cm and 0.5cm) ;
\draw[densely dashed] (-1,0) arc (180:0:1cm and 0.5cm);
\draw(-1,4) arc (180:0:1cm and 0.5cm);
\draw[densely dashed]  (-1,2) arc (-180:0:1cm and 0.5cm);
\draw[thick, red, -latex] (0,2) [partial ellipse=-60:-140:1cm and 0.5cm];
\node at (2.2,4) {$Z=\mathbb{R}\times\mathbb{S}^1$};
\node at (2+0.3,-1-0.2) {$y$};
\node at (-2+0.3,-1-0.2) {$x$};
\node at (0.3,5) {$z$};
\node at (0.5,1.25) {$p$};
\draw[-latex,thick] (0,4) -- (0,5) ;
\node at (0.51,1.57){\textbullet};
\end{tikzpicture}
\caption{Wirkung von $\Reals/\Integers$ auf dem Zylinder $Z$.}
\end{figure}
\section{Orbiträume}
Die Wirkung einer Gruppe definiert in natürlicher
Weise Äquivalenzklassen auf einer Mannigfaltigkeit, welche gerade diejenigen
Elemente enthalten, die sich durch $G$-Wirkung ineinander überführt werden.
Eine solche Äquivalenzklasse nennen wir Bahn oder Orbit. 
\begin{definition}[Orbit]
Sei $X$ eine $G$-Menge, $x\in X$, dann heißt die Menge
\begin{equation}
G \gmal x=\{g \gmal x\,|\,g\in G\}
\end{equation}
\emph{Orbit} von $x$. Als \emph{Orbitraum} bezeichnen wir die Menge aller Orbits
\begin{equation}
M/G=\{[x]=G.x\,|\,x\in M\}\,,
\end{equation}
versehen mit der Quotiententopologie, der gröbsten Topologie für die die
kanonische Projektion $x\mapsto [x]$ stetig ist.
\end{definition}
\begin{beispiel}[Orbits von $\Reals/\Integers$ auf $Z$]
Wenn wir wieder das Beispiel $G=\Reals/\Integers$, $M=Z$ heranziehen, so sind die Orbits gegeben durch
\begin{equation}
G.(0,0,z)=\left\{(\sin (2\pi t),\sin (2\pi t),z)\,,t \in
[0,1)\right\}\cong\Sphere^1\,.
\end{equation}
Der Orbitraum $M/G$ selbst ist offensichtlich diffeomorph zu $\Reals$. Der
Zylinder lässt sich also lokal als Produkt von Orbitraum und Orbit darstellen.
\end{beispiel}
\begin{figure}[!htbp]
\centering
\begin{tikzpicture}
\draw[-latex] (2,1)-- (-2,-1)  ;
\draw[-latex]  (-2,1)--(2,-1)  ;

\draw[fill=white,draw=none] (-1,4) -- (-1,0) arc (180:360:1cm and 0.5cm) -- (1,4) arc (-180:0:-1cm and 0.5cm) ;
\draw[fill=white,draw=none] (0,4) ellipse (1cm and 0.5cm);
\draw[-,thick, dashed] (0,-0.5) -- (0,4) ;
\draw[dashed] (2,1)-- (-2,-1)  ;
\draw[dashed]  (-2,1)--(2,-1)  ;
\draw[-,thick] (0,-1) -- (0,-0.5) ;
\node at (0,0) {\tiny\textbullet};
\node at (0,4) {\tiny\textbullet};
\draw[densely dashed,red,thick] (-1,2) arc (180:0:1cm and 0.5cm);
\draw[fill=gray,fill opacity = 0.1,draw=none] (0,4) ellipse (1cm and 0.5cm);
\draw[fill=gray,fill opacity = 0.2] (-1,4) -- (-1,0) arc (180:360:1cm and 0.5cm) -- (1,4) arc (-180:0:-1cm and 0.5cm) ;
\draw[densely dashed] (-1,0) arc (180:0:1cm and 0.5cm);
\draw(-1,4) arc (180:0:1cm and 0.5cm);
\draw[red,thick]  (-1,2) arc (-180:0:1cm and 0.5cm);
\draw[thick, blue!80!black] (0.51,-0.4)-- (0.51,3.55);
\node at (2.2,4) {$Z=\mathbb{R}\times\mathbb{S}^1$};
\node[red] at (1.7,2) {$G.p$};
\node[ blue!80!black] at (0.6,-0.9) {$M/G$};
\node at (2+0.3,-1-0.2) {$y$};
\node at (-2+0.3,-1-0.2) {$x$};
\node at (0.3,5) {$z$};
\node at (0.75,1.25) {$p$};
\draw[-latex,thick] (0,4) -- (0,5) ;
\node at (0.51,1.57){\textbullet};
;\end{tikzpicture}
\caption{Orbits der Wirkung der Gruppe $\Reals/\Integers$ auf $Z$.}
\end{figure}

Interessant sind insbesondere Fälle, in denen der Orbitraum selbst wieder eine
Mannigfaltigkeit darstellt. Dass dabei auch Probleme auftreten können, zeigt
folgendes 
\begin{beispiel}[Ein nicht Hausdorffscher Quotient
\cite{abraham1978foundations}] Sei $G=(\Reals,+)$, $M=\Reals$ und $G$ wirke auf
$M$ durch $t \gmal x=e^tx$.
Der Quotient $M/G$ enthält die drei Äquivalenzklassen $[-1],[0],[1]$, da die
Abbildung das Vorzeichen nicht ändert.
Die Quotiententopologie $\tau$ lässt sich explizit angeben:
\begin{equation}
\tau =\big\{\emptyset,\{[-1]\},\{[1]\},\{[-1],[1]\},M/G\big\}\,.
\end{equation}
Offensichtlich ist die einzige Menge, die $[0]$ enthält $M/G$. Die Elemente
$[0],[1]$ lassen sich damit nicht durch offene Mengen trennen, d.h. $(M/G,\tau)$
ist nicht Hausdorff.
\begin{figure}[!htbp]
\centering
\begin{tikzpicture}
\draw [dashed](-3,0)--(-2,0);
\draw [dashed](3,0)--(2,0);
\draw (-2,0)--(-0.3,0);
\draw (2,0)--(0.3,0);
\node at (-1,-2){\textbullet};
\node at (0,-2){\textbullet};
\node at (1,-2){\textbullet};
\node at (4,0){$M=\mathbb{R}$};
\node at (4,-2){$M/G$};
%\draw[<-] (-1,-3)--(-0.8,0);
%\draw (-1,-3)--(-2.5,-2.5);
%\draw (1,-3)--(0,0) --(2,0) --cycle;
%\draw (0,-3)--(0,0)  --cycle;
\node at (0,0){\textbullet};
\node at (-1,-2.5){$[-1]$};
\node at (0,-2.5){$[0]$};
\node at (1,-2.5){$[1]$};
\node at (0,.5){$G.0$};
\node at (-1.3,.5){$G.(-1)$};
\node at (1.3,.5){$G.1$};
\node at (-0.3,0){$)$};
\node at (0.3,0){$($};
\end{tikzpicture}
\caption{Konstruktion eines nicht Hausdorffschen Quotienten.}
\end{figure}
\end{beispiel}
Tatsächlich scheitert die Konstruktion daran, dass die Wirkung nicht eigentlich
ist, beispielsweise ist das Urbild $\Gamma^{-1}\big([0,1]\times
\{1\}\big)=(-\infty,0]\times \{1\}$ nicht kompakt.
Vielmehr lässt sich zeigen dass der Quotient genau dann Hausdorff ist, wenn die
Gruppe frei wirkt.
%TODO REF

Wir geben nun eine Charakterisierung, die solche pathologische
Fälle ausschließt.
% \begin{proposition}
% Sei $M$ eine $G$-Manigfaltigkeit, $R:=\big\{(g,g.x)|g\in G x\in M\big\}$ dann
% ist $R$ genau dann eine abgeschlossene Untermanigfaltigkeit von $M\times M$,
% wenn $M/G$ eine glatte Maigfaltigkeitsstruktur besitzt, sodass $\pi:M\to M/G$
% eine Submersion ist.
% \end{proposition}
% \begin{proof}
% Siehe R.Abraham "`Foundations of Mechanics"' \cite{abraham1978foundations}.
% \end{proof}
% \begin{theorem}[Slice Theorem]
% Ist $G$ eigentlich, so besitzt $M/G$ eine
% \end{theorem}
% Das Slice Theorem liefert, das falls $G$ kompakt und fixpunktfrei ist $M/G$ eine
% Manigfaltigkeitstruktur besitzt, bzw. sogar $M\to M/G$ ein $G$-Hauptfaserbündel
% ist.
% %http://math.stackexchange.com/questions/1315445/quotient-manifold-theorem-provides-a-fibrations
% Da die Gruppenwirkung differenzierbar ist lasst sich auch die Orbitkarte
% \begin{equation}
% \sigma_x:G\to M\,,\quad x\mapsto g \gmal x
% \end{equation}
% bei in der Identität differenzierbar. Man erhält so zu jedem $x\in M$ einen
% Vektor insgesammt erhält man ein Vektorfeld $V$ auf $M$.
%TODO Typen von Orbits. Maximale Orbits.
\begin{theorem}[Quotient Manifold Theorem \cite{lee2003smooth}]
Sei $G$ eine Lie-Gruppe, die glatt, frei und eigentlich auf einer glatten
Mannigfaltigkeit $M$ wirkt, dann ist der Orbitraum eine topologische
Mannigfaltigkeit der Dimension $\dim M/G=\dim M -\dim G$ und es existiert eine
eindeutige glatte Struktur sodass $\pi:M\to  M/G$ eine glatte Submersion ist.
\end{theorem}
\begin{korollar}
Sei $G$ eine Lie-Gruppe, die glatt, frei und eigentlich auf einer glatten
Mannigfaltigkeit $M$ wirkt. Dann ist $\pi:M\to M/G$ ein glattes Hauptfaserbündel,
mit Basis $M/G$ und typischer Faser $G$.
\end{korollar}
\section[Integration inv Fkt]{Integration auf Faserbündeln}
\begin{theorem}[Verallgemeinerter Fubini]\label{th:fubini}
Sei $M,N$, $m$ bzw. $n$-dimensionale, orientierbare Mannigfaltigkeiten, mit
$m\geq n$.
Sei $\varphi:M\to N$ glatt, $\omega$ eine $(m-n)$-Form auf $M$, sowie $\eta$
eine $n$-Form auf $N$. Sei $f:M\to \Reals$ Lebesgue integrabel, dann existiert
\begin{equation}
F(p)=\int_{\varphi^{-1}(p)}f\omega\,,
\end{equation}
das Integral entlang der Faser, für fast alle $p\in N$ und $F$ ist messbar. Es
gilt
\begin{equation}
\int_{M}f\omega\wedge\varphi^*\eta=\int_{N}F \eta\,.
\end{equation}
% Seien $M$, $N$ orientierbare Manigfaltigkeiten, $\pi$ die Projektion auf $M$
% bzw $N$, $\eta$, $\omega$ differentialformen auf $M$ bzw. $N$, $h:M\times N\to
% \Reals$ glatt, dann gilt
% \begin{equation}
% \int_{M\times N} h\pi_1^\star\eta\wedge\pi_2^\star\omega
% =\int_M g\eta\,,\quad g(p):=\int_N h(p,\cdot)\omega\,.
% \end{equation}
\end{theorem}
\begin{proof}
Siehe Sulantke und Wintgen
%Polynomial Convexity Edgar Lee Stout
\end{proof}
\begin{bemerkung}
Bei \autoref{th:fubini} handelt es sich um eine Verallgemeinerung des aus der
Analysis bekannten Satzes von Fubini. 
Sei $M=\Reals^m$, $N=\Reals^n$, und 
\begin{align*}
  \pi :M &\to N\\
  (x_1,\dots,x_n,y_1,\dots,y_k) &\mapsto (x_1,\dots,x_n)\,.
\end{align*}
Wir definieren Differentialformen auf $N$ bzw. $M$ durch 
$\omega=\dif{x}^1\wedge\cdots\wedge\dif{x}^n$ und
$\eta=\dif{x}^{n+1}\wedge\cdots\wedge\dif{x}^m$.
Nach\autoref{th:fubini}gilt
\begin{equation}
\begin{split}
\int_{\Reals^m} f\dif{}^m x &= \int_{\Reals^m}
f\dif{x}^1\wedge\cdots\wedge\dif{x}^n\wedge\dif{x}^{n+1}\wedge\cdots\wedge\dif{x}^m\\
&= \int_{\Reals^m}f\,\omega\wedge\pi^*\eta\\
&= \int_{\Reals^n}\left(\int_{\pi^{-1}(x)}f\,\omega\right)\eta\\
&= \int_{\Reals^n}\left(\int_{\{x\}\times\Reals^{(m-n)}}f\,\omega\right)\eta\,,
\end{split}
\end{equation}
was der bekannten Formel entspricht.
\end{bemerkung}
Ein interessantes Resultat ergibt sich wenn wir Funktionen betrachten, die
entlang der Fasern $\pi^{-1}(x)$ konstant sind.
\begin{definition}
Sei $X$ eine $G$-Menge, $Y$ eine Menge, $f:X\to Y$ mit
\begin{equation}
f(g\gmal p)=f(p)\,,\quad \forall g\in G\,,p\in E\,,
\end{equation}
dann heißt $f$ $G$-invariant.
\end{definition}
Sei $G$ eine kompakte Lie-Gruppe dann gilt $\vol(G):=\int_G\dif h<\infty$
%TODO Beweis
. 
Wir betrachten ein $G$-Hauptfaserbündel $\pi: E\to M$,
ausgestattet mit einer $G$-invarianten {(pseudo-)riemannschen} Metrik $g$. 
Die Abbildung $f:E\to \Reals$ sei $G$-invariant und integrierbar. Dann gilt nach
\autoref{th:fubini}
\begin{equation}
\begin{split}
\int_E f\sqrt{g}\dif{}\hat{x}&=\int_{M}\int_{G}f\sqrt{g}\dif^{}h \dif x\\
&=\int_{M}f\sqrt{g}\int_{G}\dif^{}h \dif x\\
&=\vol(G)\int_{M}f\sqrt{g} \dif x\\
\end{split}
\end{equation}
Die Integration über den Totalraum $E$ kann also mit einer Integration über die
Basis $M$ identifiziert werden.
% Eine
% Tivialisierung einer offenen Menge $U\subset M=E/G$, ist ein Homoemorphismus
% \begin{equation}
% \Psi:\pi^{-1}(U)\to U\times G\,.
% \end{equation}
% Wir definieren eine Abbildung $\tilde{f}: M\to\mathrm{R}$
% \begin{equation}
% \tilde{f}\left(x\right):=\left(f\circ \Psi^{-1}\right)(x,e)\,.
% \end{equation}
% % Diese ist wohldefiniert, denn
% % \begin{equation}
% % \begin{split}
% % f\left(\Psi^{-1}(x,h)\right)&=f\left(\Psi^{-1}(x,h.e)\right)\\
% % &=f\left(h.\Psi^{-1}(x,e)\right)\\
% % &=f\left(\Psi^{-1}(x,e)\right)\,.
% % \end{split}
% % \end{equation}
% Damit gilt
% \begin{equation}
% \begin{split}
% \int_{\pi^{-1}(U)}f(x)\dif x&=
% \int_{\Psi\left(\pi^{-1}(U)\right)}\left[f\circ\Psi^{-1}\right](y,h)\dif y
% \dif h\\
% &=
% \int_{U\times G}\left[f\circ\Psi^{-1}\right](y,h)\dif y
% \dif h\\
% &=\int_G\int_{U}\tilde{f}\left(y\right)\dif y
% \dif h\\
% &=\vol (G)\int_{U}\tilde{f}\left(y\right)\dif y\,,\label{eq:Intprod}
% \end{split}
% \end{equation}
% Eine anschauliche Betrachtung $\tilde{f}$ als Mittelung der makroskopischen Funktion $f$ über die Zusatzdimension auffassen.
% Phsikalisch liegt ein solcher Fall vor, falls die Auflösung einer Messung zu
% gering ist um die kompakte Dimension zu beobachten. Bisher konnten kompakte
% Zusatzdimensionen nicht beobachtet werden, neue Erkentnisse bringt
% möglicherweise der 2015 gestarteten Lauf des Large Hadron Coliders am CERN. Die
% lokalen Trivialisierungen überdecken $M$, damit lassen sich auch Integrale auf dem gesammten Raum auf
% Integrale von der Form \eqref{eq:Intprod} zurückführen insbesondere gilt für die
% $G$-invariante Funktion $f\sqrt{g}$
% \begin{equation}
% \int_{E}f(x)\sqrt{g(x)}\dif x=\vol
% (G)\int_{M}\tilde{f}\left(y\right)\sqrt{\tilde{g}(y)}\dif y
% \end{equation}
% wobei $\tilde{g}(y)=\left[g\circ\Psi^{-1}\right](y,h)$. Im folgenden lassen wir
% die Tilden weg und identifizieren die Ausdrücke stillschweigend miteinander.
% Wir beenden dieses Kapitel mit einem Beispiel, der sogenannten
%\emph{ADM-decomposition}.
\begin{beispiel}[ADM-decomposition]
 
 \end{beispiel}
\section{Variationsrechnung}
Variationsprinzipien stellen eine elegante Möglichkeit dar, physikalische
Theorien zu formulieren. Die Physik wird dabei durch einen Satz
von Parametern bzw. Funktionen beschrieben, die in gewisser Weise "`optimal"'
sind.
Wir orientieren uns in diesem Kapitel grob an den Ausführungen von \name{R.Wald}
\cite[s.454 ff]{wald2010general}.
Im folgenden bezeichne \emph{Feld} allgemein ein Tensorfeld, wobei wir den
Grad nicht angeben, sowie \emph{Feldkonfiguration} ein Tupel solcher
Feldern. 

Wir wollen nun einen Formalismus entwickeln, der es erlaubt ein Funktional, nach
einer Funktion abzuleiten, um dadurch Extremalbedingungen in
Analogie zur klassischen Analysis zu formulieren.  
Im Folgenden ist $X$ einen Banachraum, von Funktionen von einer Mannigfaltigkeit
$M$ in die reellen Zahlen, Typischerweise $X\subseteq C^\infty(M)$.
Weiter sei ein Funktional $I:X\to\Reals$ gegeben, dass uns ermöglicht, Problemen
der Form
\begin{equation}
I[f]\to \mathrm{min.}
\end{equation}
zu formulieren. In der endlichen Analysis gilt für Lösungen analoger
Probleme durch $\pd{I}{f}=0$.
Wie aber soll dies auf den Fall unendlichdimensionaler Variablen $f$ übertragen
werden? 
Angenommen $f$ ist eine Funktion, die $I$ minimiert, sei $h\in X$ eine
"`Störung"' dieser Funktion, dann gilt
\begin{equation}
I[f]\leq I[f+\varepsilon h]\,,
\end{equation} 
für $\varepsilon$ hinreichend klein. Fasst man $I[f+\varepsilon h]$ als Funktion
von $\varepsilon$, so kann das Problem auf gewöhnliche Differentialrechnung
zurück gespielt werden.
Das resultierende Objekt ist die so genannte Funktionalableitung nach $h$.
Insbesondere in der physikalischen Literatur wird auf
eine formale Definition der "`Ableitung nach einer Funktion"' häufig verzichtet, wir
wollen deshalb an dieser Stelle versuchen, eine möglichst genaue Definition der
Funktionalableitung zu geben.
% TODO in der physikalischen Litereatur wird auf eine formale definition der
% Funktionalableitung meist gänzlich verzichtet
\begin{definition}[Funktionalableitung] 
Sei $X$ ein Banachraum, $I:X\to \Reals$, eine Funktional. Falls 
\begin{equation}
\delta I(f)[h]:=\left[\dod{}{\alpha}I[f+\alpha
h]\right]_{\alpha=0}\label{eq:varderdef}
\end{equation}
für alle in $h\in X$, so heißt $I$ in $f$ \emph{(Gâteaux-)differnzierbar} und
$\delta I(f)$ \emph{Gâteaux-Differential} von $I$ in $f$. Falls $X$ ein
Funktionenraum ist so heißt $\delta I(f)$ auch \emph{Variationsableitung} oder
\emph{erste Variation}.
% für alle $T\in \mathfrak{T}^{m}_n$ und glatt ist, so heißt $\delta F(S)$
% \emph{Funktionalableitung} oder \emph{1.\ Variation} von $F$ in $S$. 
% Sei $X$ ein Banachraum, $I:X\to
% \Reals$ ein Funktional. Existiert ein lineares Funktional
% \begin{equation}
% \begin{split}
% \delta I(f):X&\to \Reals\\
% \end{split}
% \end{equation}
% sodass
% \begin{equation}
% \lim_{\varepsilon\to 0}\frac{I[f+\varepsilon g]-I[f]-\delta
% I(f)(g)}{\varepsilon}=0\,,
% \end{equation}
% für alle $g\in X,$ so heißt $I$ in $f$ differnzierbar und $\delta I(f)$
% Variationsableitung von $I$ in $f$.
% 
% die lineare Abbildung 
% \begin{equation}
% \begin{split}
% \delta I(f):X&\to \Reals\\
% g&\mapsto \dod{}{\varepsilon} I[f+\varepsilon g]\bigg|_{\varepsilon=0}
% \end{split}
% \end{equation}
% Variationsableitung von $I$ in $f$.
\end{definition}
% \begin{bemerkung}[$\delta
% F(S)$ ist ein Tensorfeld]
% Nach Zorn (Characterization of Analytic Functions in Banach Spaces) ist $\delta
% F(f)$ eine lineare, glatte Abbildung von $X$
% nach $\Reals$, d.h. es existiert ein $\frac{\delta F}{\delta f}\in X^*$, dass
% z.B. falls $X=L^p$, nach dem Rietz'schen Darstellungssatz auch in integralform
% geschrieben werden kann:
% \begin{equation}
% \begin{split}
% \delta F(f)[g]&=\left\langle\frac{\delta F}{\delta f},g\right\rangle\\
% &=\int_M\frac{\delta F}{\delta f} g\dif x\,.
% \end{split}
% \end{equation}
% % Dabei bezeichnet $C:\mathfrak{T}^n_n\to C^\infty(M)$ die totale Kontraktion d.h.
% % z.B. $A=\tensor{A}{^i_j}\dif \tensor{x}{^j}\otimes\tensor{\partial}{_j}\in
% % \mathfrak{T}^1_1$$C(A)=\tensor{A}{^i_i}$. Wir schreiben dann auch 
% % \begin{equation}
% % \frac{\delta F}{\delta S}:=\chi\,.
% % \end{equation}$f:\mathfrak{T}^m_n\to L^1(M)$
% % \begin{equation}
% % \begin{split}
% % \delta F[S] [T]&=\int_M \left[\dod{}{\alpha}f(S+\alpha T)\right]_{\alpha=0}\dif
% % x
% % \end{split}
% % \end{equation}
% \end{bemerkung}
\begin{lemma}[Eigenschaften der Variationsableitung]
Sei $X$ ein Banachraum, $F,G:X\to \Reals$ Funktionale,  $f\in X $, $g\in
C^1(\Reals)$ $a,b\in\Reals$, dann gilt:
\begin{enumerate}
  \item Linearität: $\delta (aF+bG)(f)=a\delta F (f)+b\delta G(f)$
  \item Kettenregel: $\delta (g\circ F)(f)=g^\prime(F[f])\cdot \delta F(f)$
  \item Leibnizregel: $\delta (F\cdot G)(f)=F[f]\cdot \delta G(f)+G[f]\cdot
  \delta F(f)$
\end{enumerate}
\end{lemma}
\begin{proof}
Per Definition lassen sich alle Eigenschaften auf die Eigenschaften, der
Gewöhnlichen Ableitung zurück spielen.
\end{proof}
Im Weiteren wollen wir allgemeiner Funktionale betrachten, die von Tensorfeldern
$T$ abhängen, innerhalb einer Koordinatenumgebung können wir diese durch
glatte Koordinatenfunktionen $\tensor*{T}{^{i_1,\dots
i_n}_{j_1,\dots,j_m}}\in C^\infty(M)$ beschreiben. Wir betrachten dann
Funktionale der Komponentenfunktionen.
% \begin{lemma}[Jacobi-Formel]
% Betrachte $\det: \matfrak{T}^1_1(\Reals)\to C^\infty$
% \end{lemma}
% \begin{beispiel}
% \begin{enumerate}
%   \item
%   \begin{equation}
% 	  \begin{split}
% 	  \det:{\mathfrak{T}}^0_2|_p\to \Reals
% 	  \end{split}
%   \end{equation}
% \end{enumerate}
% \end{beispiel}
% \begin{definition}[Funktionalableitung] 
% \label{def:Funktionalableitung}
% Sei $M$ eine Mannigfaltigkeit,
% $X\subseteq C^\infty(M)$ ein Funktionenraum. Sei $I:X\to \Reals$ ein Funktional,
% $\left(f_\lambda\right)_{\lambda\in \Reals}\subseteq X$, eine einparametrige
% Familie von Funktionen, die differenzierbar von $\lambda$ abhängt.
% Angenommen $\lambda\mapsto I\left[f_\lambda\right]$ ist für alle solche Familien
% in $\lambda=0$ differenzierbar und es existiert eine Distribution $g$, sodass
%  \begin{equation}
%  \dod{}{\lambda}\Big|_{\lambda=0}I[f_\lambda]
%  =\int_M g\dod{}{\lambda}\Big|_{\lambda=0}f_\lambda\dif x\,
%  \label{eq:Vardiff1}
%  \end{equation}
%  für alle Familien $f_\lambda$, dann heißt
%  \begin{equation}
% \frac{\delta{I}}{\delta{f}}\bigg|_{f_0}:=g
%  \end{equation}
%  \emph{Variationsableitung} von $I$ in $f_0$.
% \end{definition}
% \begin{bemerkung}
% Der Begriff der Variationsableitung lässt sich auf natürliche Weise auf 
% Funktionale von Feldern oder allgemeiner Feldkonfigurationen erweitern wenn wir
% in \autoref{def:Funktionalableitung} Funktion durch Feld bzw. Feldkonfiguration
% ersetzen, sowie  in \eqref{eq:Vardiff1} die Tensorindizes der
% Felder $g$ und $f_\lambda$ kontrahieren.
% \end{bemerkung}
%TODO was ist delta genau? produktregel etc.??
%TODO Variationsableitung erfüllt Produktregel etc.
\subsection{Der Lagrangeformalismus}
Einen wichtigen Spezialfall stellen lokale Funktionale dar, die von der Form 
%\footnote{"`Lokal"' bezieht sich anschaulich darauf, dass der Wert des
%Integranden in $x\in M$ lediglich von diesem Punkt abhängt.} Funktion
\begin{equation}
I[f]=\int_M L\left(f(x),\tensor{\nabla}{_i}
f(x),\tensor{\nabla}{_i}\tensor{\nabla}{_j} f(x),\dots,x\right)\dif x\,.
\end{equation}
Wir definieren kritische bzw. stationäre Punkte von Funktionalen als diejenigen
Punkte an denen die Funktionalableitung verschwindet. 
\begin{definition}
Sei $F:X\to\Reals$ ein Funktional, eine Funktion $f$ heißt \emph{stationärer
Punkt} von $I$ falls
\begin{equation}
\delta {I}(f)=0\,,
\end{equation} 
bzw. $\delta {I}(f)[h]=0$ für $h\in X$.
\end{definition}
\begin{lemma}
Sei $f$ stationärer Punkt eines lokalen Funktionals $I$,
dann erfüllt $f$ die \emph{Euler-Lagrange-Gleichung}
\begin{equation}
0=\dpd{L}{f}-\nabla_i\left[\dpd{L}{\left(\tensor{\nabla}{_i}f\right)}\right]
+\nabla_i\nabla_j\left[\dmd{L}{2}{\left(\tensor{\nabla}{_i}f\right)}{}{\left(\tensor{\nabla}{_j}f\right)}{}
\right]-\dots\,.
\end{equation}
\end{lemma}
\begin{proof}
Sei $I$ ein lokales Funktional, dann gilt
\begin{equation}
\begin{split}
\delta {I}(f)[h]&=\int_M \left[\dod{}{\varepsilon}L\Big(f+\varepsilon
h,\tensor{\nabla}{_i} f+\varepsilon\tensor{\nabla}{_i} h,\dots,x\Big)
\right]_{\varepsilon=0}
\dif x\\
&=\int_M
\left[\dpd{L}{f}h+\dpd{L}{\left(\tensor{\nabla}{_i}f\right)}\nabla_i
h+\dots\right] \dif x\\
&=\int_M
\left\{\dpd{L}{f}h-\nabla_i\left[\dpd{L}{\left(\tensor{\nabla}{_i}f\right)}\right]
h+\dots\right\} \dif x+\int_{\partial
M}\dpd{L}{\left(\tensor{\nabla}{_i}f\right)}h n_i\dif \sigma \,.
\end{split}
\end{equation}
dabei ist $\dif \sigma $ die induzierte Volumenform auf dem Rand und 
das Vektorfeld $n_i$ ist normal zum Rand
$\partial M$ der Mannigfaltigkeit, wir nehmen dabei an, dass die
Mannigfaltigkeit orientierbar ist, d.h.\ ein solches Normalenfeld existiert.
Im folgenden wollen wir solche Variationen betrachten die auf dem Rand
verschwinden, d.h. $h|_{\partial M}\equiv 0$. Damit ergibt sich 
\begin{equation}
\begin{split}
\delta {I}(f)[g]&=\int_M
\left\{\dpd{L}{f}h-\nabla_i\left[\dpd{L}{\left(\tensor{\nabla}{_i}f\right)}\right]
h+\dots\right\} \dif x \\
&=\int_M
\left\{\dpd{L}{f}-\nabla_i\left[\dpd{L}{\left(\tensor{\nabla}{_i}f\right)}\right]
+\dots\right\}h \dif x \,.
\end{split}
\end{equation}
Mit dem Lemma der Variationsrechnung und $\delta I(f) [h]=0$ für alle $h\in X$
folgt
\begin{equation}
0=\dpd{L}{f}-\nabla_i\left[\dpd{L}{\left(\tensor{\nabla}{_i}f\right)}\right]
+\dots\,.
\end{equation}
\end{proof}
Die \emph{Euler-Lagrange-Gleichung} ist von zentraler Bedeutung unter anderem in der theoretischen Mechanik, 
spielt aber auch in Feldtheorien eine wichtige Rolle.
Der Formalismus eine Funktion als Extremum eines Funktionals zu wählen,
ist unter dem Namen \emph{Lagrange-Formalismus} bekannt und lässt sich auf
natürliche Weise auf Funktionale Mehrerer Funktionen, d.h.\ Tensorfelder bzw.
Feldkonfigurationen verallgemeinern.
% 
% \subsection{Der Lagrangeformalismus}
% Einen wichtigen Spezialfall stellen Funktionale dar, die durch eine
% lokale\footnote{"`Lokal"' bezieht sich anschaulich darauf, dass der Wert des
% Integranden in $x\in M$ lediglich von diesem Punkt abhängt.} Funktion
% ${L:\Reals^3\to \Reals}$ charakterisiert werden, d.h. \footnote{Hier
% und im folgenden bezeichnet $\partial_i$ eine Differentiation nach der $i$-ten
% Koordinate.}
% \begin{equation}
% I[f]=\int_M L\left(f(x),\partial_i f(x),x\right)\dif x\,.
% \end{equation}
% Die Funktion $L$ heißt dabei auch \emph{Lagrange-Funktion} oder auch
% \emph{Lagrangian}.
% Die Variation des Funktionals $I$ führt in diesem Fall auf
% \begin{equation}
% \begin{split}
% \dod{I}{\lambda}
% &=\int_M\dod{}{\lambda}L(f_\lambda,\partial_i
% f_\lambda,x)\dif x\\
% &=\int_M\dpd{L}{ f}\dpd{ f_\lambda}{\lambda}
% +\dpd{L}{(\partial_i f)}\dpd{(\partial_i f_\lambda)}{\lambda}\dif x\\
% &=\int_M\dpd{L}{ f}\dpd{ f_\lambda}{\lambda}
% +\dpd{L}{(\partial_i f)}\partial_i\left(\dpd{
% f_\lambda}{\lambda}\right)\dif x\\
% &=\int_M\left\{\dpd{L}{ f}
% -\partial_i\left[\dpd{L}{(\partial_i f)}\right]\right\}\dpd{
% f_\lambda}{\lambda}\dif x +\int_{\partial
% M}\dpd{L}{(\partial_i f)}\dpd{
% f_\lambda}{\lambda}n_i\dif\sigma\,,
% \end{split}
% \end{equation}
% dabei haben wir im dritten Schritt die Vertauschbarkeit
% der partiellen Ableitungen nach dem Satz von Schwarz, sowie im letzten
% partielle Integration verwendet. Das Vektorfeld $n_i$ ist normal zum Rand
% $\partial M$ der Mannigfaltigkeit, wir nehmen dabei an, dass die
% Mannigfaltigkeit orientierbar ist, d.h.\ ein solches Normalenfeld existiert.
% %TODO partielle Integration auf MF
% Häufig beschränkt man sich auf kompakte Variationen, d.h. für alle $\lambda$ ist
% $f_\lambda(x)=f_0(x)$, bzw. $\pd{ f_\lambda}{\lambda}=0$ auf $\partial M$. Die
% Integration über den Rand verschwindet damit und es ergibt sich
% \begin{equation}
% \begin{split}
% \dod{I}{\lambda}
% &=\int_M\left\{\dpd{L}{ f}
% -\partial_i\left[\dpd{L}{(\partial_i f)}\right]\right\}\dpd{
% f_\lambda}{\lambda}\dif x
% \end{split}
% \end{equation}
% und somit per Definition 
% \begin{equation}
% \frac{\delta I}{\delta f}=\dpd{L}{ f}
% -\partial_i\left[\dpd{L}{(\partial_i f)}\right]\,.
% \end{equation}
% % Geht man nun davon aus, dass eine bestimmte Funktion $f$ das Funktional
% % $I$ minimiert, d.h. $\od{I}{\lambda}=0$ für eine beliebige Variation, 
% % so muss nach dem Lemma der Variationsrechnung
% % TODO stationär
% Geht man nun davon aus, dass eine bestimmte Funktion $f$ das Funktional
% $I$ minimiert, d.h. $\frac{\delta I}{\delta f}=0$ so ergibt sich die
% \emph{Euler-Lagrange-Gleichung}
% \begin{equation}
% \dpd{L}{f}
% -\partial_i\left[\dpd{L}{(\partial_i f)}\right]=0\,.
% \end{equation}
% Diese ist von zentraler Bedeutung unter anderem in der theoretischen
% Formulierung der klassischen Mechanik, spielt aber auch in Feldtheorien eine
% wichtige Rolle.
% Der Formalismus eine Funktion als Extremum eines Funktionals zu wählen,
% ist unter dem Namen \emph{Lagrange-Formalismus} bekannt und lässt sich auf
% natürliche Weise auf Feldkonfigurationen $\Psi$ verallgemeinern. 
Das Funktional ist dann von der Form
\begin{equation}
I[\Psi]=\int_M L\left(\Psi,\nabla_i \Psi,x\right)\dif x
\end{equation}
und die resultierenden Euler-Lagrange-Gleichungen lauten
\begin{equation}
\dpd{L}{\Psi}
-\nabla_i\left[\dpd{L}{(\nabla_i\Psi)}\right]=0\,,
\end{equation}
wobei dieser Ausdruck als eine Gleichung pro
unabhängige Feldkomponente zu interpretieren ist. Das folgende Beispiel soll
erläutern wie der Formalismus anzuwenden ist.
 \begin{beispiel}[Spin-1 Felder] \label{bsp:Spinone}
Als Beispiel betrachten wir ein Feld, welches durch eine
1-Form $A$ beschrieben wird, ein solches Feld beschreibt physikalische Teilchen
mit Spin-1.
In lokalen Koordinaten wird $A$ durch Funktionen $A_0,\dots,A_3$ beschreiben. Nehmen wir an die Lagrange
Funktion $L$ sei gegeben durch \footnote{Diese muss nicht "`geraten"'
werden, vielmehr handelt es sich um den allgemeinste solche Funktion die
gewissen Symmetrien genügt. }
\begin{equation}
L\left(A_\alpha(x),\nabla_\beta
A_\alpha(x),x\right)=-\frac{1}{4}\tensor{F}{_\mu_\nu}\tensor{F}{^\mu^\nu}\,.
\end{equation}
mit
$\tensor{F}{_\mu_\nu}=\tensor{\nabla}{_\mu}\tensor{A}{_\nu}-\tensor{\nabla}{_\nu}\tensor{A}{_\mu}$.
Unabhängigkeit der Felder bedeutet
\begin{equation}
\pd{\tensor{F}{_\mu_\nu}}{(\tensor{\nabla}{_\alpha}\tensor{A}{_\beta})}
=\tensor*{\delta}{^\alpha_\mu}\tensor*{\delta}{^\beta_\nu}\,.
\end{equation}
Damit können wir $L$ nach $\tensor{\nabla}{_\mu}\tensor{A}{_\nu}$
differenzieren:
\begin{equation}
\begin{split}
\dpd{L}{(\tensor{\nabla}{_\alpha}\tensor{A}{_\beta})}
&=\dpd{\tensor{F}{_\mu_\nu}}{(\tensor{\nabla}{_\alpha}\tensor{A}{_\beta})}\tensor{F}{^\mu^\nu}
+\dpd{\tensor{F}{^\mu^\nu}}{(\tensor{\nabla}{_\alpha}\tensor{A}{_\beta})}\tensor{F}{_\mu_\nu}
\\
&=\dpd{\tensor{F}{_\mu_\nu}}{(\tensor{\nabla}{_\alpha}\tensor{A}{_\beta})}\tensor{F}{^\mu^\nu}
+\dpd{(\tensor{g}{^\mu^\delta}\tensor{g}{^\mu^\gamma}\tensor{F}{_\delta_\gamma})}{(\tensor{\nabla}{_\alpha}\tensor{A}{_\beta})}\tensor{F}{_\mu_\nu}
\\
&=\dpd{\tensor{F}{_\mu_\nu}}{(\tensor{\nabla}{_\alpha}\tensor{A}{_\beta})}\tensor{F}{^\mu^\nu}
+\dpd{\tensor{F}{_\delta_\gamma}}{(\tensor{\nabla}{_\alpha}\tensor{A}{_\beta})}\tensor{g}{^\mu^\delta}\tensor{g}{^\mu^\gamma}\tensor{F}{_\mu_\nu}
\\
&=2\dpd{\tensor{F}{_\mu_\nu}}{(\tensor{\nabla}{_\alpha}\tensor{A}{_\beta})}\tensor{F}{^\mu^\nu}\\
&=2\tensor*{\delta}{^\alpha_\mu}\tensor*{\delta}{^\beta_\nu}\tensor{F}{^\mu^\nu}\\
&=2\tensor{F}{^\alpha^\beta}\,.
\end{split}
\end{equation}
Mit $\dpd{L}{\tensor{A}{_\alpha}}=0$ lauten die Euler-Lagrange-Gleichungen
\begin{equation}
0=\nabla_\alpha\left[\dpd{L}{\left(\tensor{\nabla}{_\alpha}f\right)}\right]
=-2\tensor{\nabla}{_\alpha}\tensor{F}{^\alpha^\beta}\,.
\end{equation}
in Kapitel ??? werden wir sehen das es sich hierbei gerade um die
Maxwell-Gleichungen im Vakuum handelt.
%TODO 
\end{beispiel}
% \begin{equation}
% \begin{split}
% \dod{S}{\lambda}
% &=\int_M\left\{\dpd{L}{\Psi}
% -\nabla_i\left[\dpd{L}{(\nabla_i\Psi)}\right]\right\}\dpd{\Psi}{\lambda}\dif
% x\,.
% \end{split}
% \end{equation}
% Identifizieren wir die Felder mit ihren lokalen Darstellungen im $\Reals^m$ und
% bezeichnet $d=\dim M$ die Dimension der Mannigfaltigkeit, so können wir
% $L$ als Funktion
% \begin{equation}
% L:\Reals^m\times\Reals^{m\cdot d}\times \Reals^d\to\Reals 
% \end{equation}
% auffassen. Damit ist klar was unter den Ausdrücken 
% \begin{equation}
% \dpd{L}{\Psi},\dpd{L}{(\nabla_i\Psi)},\dots
% \end{equation}
% zu verstehen ist. Die Variation des Funktionals $I$ führt auf 
% \begin{equation}
% \begin{split}
% \dod{I}{\lambda}
% &=\int_M\dod{L}{\lambda}\dif x\\
% &=\int_M\dpd{L}{\Psi}\dpd{\Psi}{\lambda}
% +\dpd{L}{(\nabla_i\Psi)}\dpd{(\nabla_i\Psi)}{\lambda}\dif x\\
% &=\int_M\dpd{L}{\Psi}\dpd{\Psi}{\lambda}
% +\dpd{L}{(\nabla_i\Psi)}\nabla_i\left[\dpd{\Psi}{\lambda}\right]\dif
% x\\
% &=\int_M\left\{\dpd{L}{\Psi}
% -\nabla_i\left[\dpd{L}{(\nabla_i\Psi)}\right]\right\}\dpd{\Psi}{\lambda}\dif x
% +\int_{\nabla
% M}\dpd{L}{(\nabla_i\Psi)}\dpd{\Psi}{\lambda}n_i\dif\sigma\,,
% \end{split}
% \end{equation}
% dabei wurde im zweiten Schritt die Kettenregel im dritten die Vertauschbarkeit
% der partiellen Ableitungen und im letzten partielle Integration verwendet. Die
% Integration über den Rand der Mannigfaltigkeit $\nabla M$ wird typischerweise zu null angenommen, was sich beispielsweise dadurch begründen
% lässt, dass die Variation $\dpd{\Psi}{\lambda}$ auf dem Rand verschwindet. Dann
% ergibt sich
% \begin{equation}
% \begin{split}
% \dod{S}{\lambda}
% &=\int_M\left\{\dpd{L}{\Psi}
% -\nabla_i\left[\dpd{L}{(\nabla_i\Psi)}\right]\right\}\dpd{\Psi}{\lambda}\dif
% x\,.
% \end{split}
% \end{equation}
% Geht man nun davon aus, dass eine bestimmte Konfiguration $\Psi$ das Funktional
% $S$ minimiert, d.h. $\dod{S}{\lambda}=0$ für eine beliebige Variation.
% Dann muss nach dem Lemma der Variationsrechnung 
% \begin{equation}
% \dpd{L}{\Psi}
% -\nabla_i\left[\dpd{L}{(\nabla_i\Psi)}\right]=0
% \end{equation}

\begin{definition}[Tensordichte]
Eine Tensordichte $\mathcal{T}$ vom Gewicht $w$
\end{definition}
Wirkungsintegral, Lagrangedichte, Lokalität
\begin{beispiel}[Levi-Civita-Symbol]
\end{beispiel}
\begin{lemma}[Jacobi Formel] 
Sei $A\in \Reals^{n\times n}$ eine Matrix mit $\det(A)\neq 0$, dann gilt 
\begin{equation}
\dpd{\det A}{A_{ij}}=\det A ({A^{-1}})_{ij}\,.
\end{equation}
\end{lemma}
\subsection{Randterme}
% TODO Volumenformen 
% \begin{definition}[Variationsableitung]
% Sei $\Omega$ ein normierter Vektorraum, $I:\Omega\to \Reals$ ein Funktional,
% das auf einer offenen Umgebung von $x\in \Omega$ definiert ist $h\in \Omega$,
% dann heißt der Ausdruck
% \begin{equation}
% \delta_h I=\lim_{\varepsilon\to 0} \frac{I(x+\varepsilon h)-I(x)}{\varepsilon}
% \end{equation}
% \emph{G\^ateaux Ableitung} von $I$, bzw. \emph{erste Variation} von $I$ nach
% $h$, sofern er existiert.
% % es wird nach einem Feld gesucht nicht nach einem "`optimalen"' Weg, quasi
% % unendlichdimensionale Form
% % https://en.wikipedia.org/wiki/Lagrangian_system
% \end{definition}
% \begin{beispiel}[Lokale Funktionale]
% Sei $I$ ein Lokales Funktional 
% \begin{equation}
% I(f)=\int_M \mathcal{I}[f(x),\nabla_i f(x),x]\dif x
% \end{equation}
% \begin{equation}
% \begin{split}
% \mathcal{I}[f(x)+\varepsilon h(x),\nabla_i
% f(x)+\varepsilon\nabla_i h,x]
% &=\dpd{\mathcal{I}}{f}h+\dpd{\mathcal{I}}{(\nabla_i
% f)}\nabla_i h
% \end{split}
% \end{equation}
%\end{beispiel}
% \begin{lemma}[Variationsprinzip]
% Extremum $\implies\delta_h I = 0$
% \end{lemma}
% \subsubsection{Skalarfelder (Spin-0)}
% \begin{equation}
% \mathcal{L}\left[\phi(x),\nabla_\alpha\phi(x),x\right]=
% \end{equation}
% \subsubsection{Vektorfelder (Spin-1)}
% \begin{equation}
% \mathcal{L}\left[A_\alpha(x),\nabla_\beta
% A_\alpha(x),x\right]=-\frac{1}{4}\tensor{F}{_\mu_\nu}\tensor{F}{^\mu^\nu}\,.
% \end{equation}
%https://en.wikipedia.org/wiki/Euler%E2%80%93Lagrange_equation
%https://en.wikipedia.org/wiki/Functional_derivative#cite_note-2

% Fibred manifold vs principal bundle (spezialfall\ldots)
% In topology, the words fiber (Faser in German) and fiber space (gefaserter Raum)
% appeared for the first time in a paper by Seifert in 1932
% Seifert, H. (1932). "Topologie dreidimensionaler geschlossener Räume". Acta
% Math. (in French) 60: 147–238. doi:10.1007/bf02398271.

\chapter{Maxwell-Einstein Theorie \label{chap:EMW}}
Dieser Teil soll die physikalischen Theorien vorstellen, die der
Kaluza-Klein-Theorie zu Grunde liegen. Dazu zählt neben der
in ihrer heutigen Formulierung auf Maxwell zurückgehenden
Elektrodynamik, Einsteins allgemeine Relativitätstheorie.
Deutlich älter ist die Elektrodynamik, deren fundamentales Gesetz, die Konstanz
der Lichtgeschwindigkeit, die Grundlage der speziellen Relativitätstheorie
darstellt.
Die Ähnlichkeit beider Theorien lässt hoffen, dass eine
übergeordnete Theorie beide als Spezialfälle enthält.
\section{Elektrodynamik}
Die Elektrodynamik beschreibt das Wechselspiel von elektrischen und
magnetischen Feldern. Die Phänomenologie lässt sich mittels zweier glatter Vektorfelder
${\vec{E},\vec{B}:\Reals^3\times\Reals\to \Reals^3}$ beschreiben.
Diese Felder erfüllen die auf Maxwell zurückgehenden Gleichungen
\begin{equation}
\begin{alignedat}{2}
\Div\vec{B} &= 0    \,,  & \qquad \Div\vec{E} &= \rho
\,,\\
\Rot\vec{E}+\dpd{\vec{B}}{t}&=0\,,& \qquad\Rot\vec{B}-\dpd{\vec{E}}{t}&
=\vec{j}\,.
\end{alignedat}
\end{equation}
Die links stehenden Gleichungen werden als \emph{homogene}, die rechten als
\emph{inhomogene} Maxwell-Gleichungen bezeichnet. Die Ladungsdichte $\rho$,
sowie die Stromdichte $\vec{j}$ können als "`Quellen"' der Felder angesehen
werden. Zusätzlich muss eine weitere Gleichung hinzugefügt werden, die die
Dynamik von Testteilchen\footnote{Ein Testteilchen ist ein idealisiertes Objekt, welches selbst die wirkenden Felder nicht beeinflusst.} beschreibt
\begin{equation}
m\ddot{\vec{x}}=\vec{F}(\vec{x},\dot{\vec{x}},t)
=q\left[\vec{E}(\vec{x},t)+\dot{\vec{x}}\times\vec{B}(\vec{x},t)\right]\,,
\end{equation}
dabei ist $\vec{F}$ die so genannte \emph{Lorentzkraft} und $q$, $m$ die
Ladung bzw. Masse des Teilchens.
Es lässt sich mittels des Helmholtz-Theorems und der Form der Maxwell-Gleichung
die Existenz von \emph{Potentialen} $\vec{A}$, $\Phi$ folgern, sodass
\begin{equation}
\vec{B}=\Rot \vec{A}\,,\quad
\vec{E}=-\Grad\Phi+\dpd{\vec{A}}{t}\,.\label{eq:Vecpot}
\end{equation}
Es ist von Vorteil, eine Beschreibung mit Hilfe der Differentialgeometrie
zu wählen.
Die Potentiale werden als Komponenten
$\tensor{A}{^\mu}=(\Phi,\vec{A})\transpose$ einer 1-Form
$A=\tensor{A}{_\mu}\dif\tensor{x}{^\mu}$ aufgefasst.
Der Elektromagnetische Feldstärketensor ist definiert als
\begin{equation}
F:=\dif
A\,,
\end{equation}
bzw. in lokalen Koordinaten
\begin{equation}
\tensor{F}{_\mu_\nu}=\partial_\mu\tensor{A}{_\nu}-\partial_\nu\tensor{A}{_\mu}\,.
\end{equation}
Die physikalischen Felder lassen sich aus dem Feldstärketensor wiederum durch
\begin{equation}
\vec{E}_{i}=\tensor{F}{_0_i}\,,\quad
\vec{B}_i=\frac{1}{2}\tensor{\varepsilon}{_i_j_k}\tensor{F}{^j^k}
\end{equation}
zurückgewinnen, wie man leicht mit Hilfe der Definition von $F$ und
\eqref{eq:Vecpot} nachrechnet. Führt man auch eine Viererstromdichte
$\tensor{J}{^\nu}=(\rho,\vec{j})\transpose$ ein,
reduzieren sich die Maxwell-Gleichungen zu zwei
Gleichungen
\begin{equation}
\tensor{\varepsilon}{^\alpha^\beta^\gamma^\delta}
\pdif{_{\alpha}}\tensor{F}{_\gamma_{\delta}}=0\,,\quad
\pdif{_\mu}\tensor{F}{^\mu^\nu}=\tensor{J}{^\nu}\,.
\end{equation}
Die homogene Gleichung ist dabei trivial, da sie bereits aus der Form des
Feldstärketensors folgt.
%d^2=0 oder 
% https://en.wikipedia.org/wiki/Yang%E2%80%93Mills_theory
% ^ Zusammenhang Jakobi und Maxwellgleichung\ldots
In koordinatenfreier Notation lauten die Gleichungen schließlich 
\begin{equation}
\dif F = 0\,,\quad\quad \dif\,(\star F)= J\,.
\end{equation}
% MW2= Jakobi?
% Die Lagrange Dichte für das Elektromagnetische Feld im Vakuum $\tensor{J}{^\mu}=0$
% \begin{equation}
% \mathcal{L}\left[\partial_\mu
% A(x),x\right]=-\frac{1}{4}\tensor{F}{_\mu_\nu}\tensor{F}{^\mu^\nu} \,.
% \end{equation}
% Wobei Beiträge die die Dynamik von Spin-$\nicefrac{1}{2}$ beschreiben nicht
% berücksichtigt wurden.
Wie in \autoref{bsp:Spinone} gezeigt wurde, erfüllen die Felder die (Vakuum-)
Maxwell-Gleichungen\footnote{Der Fall $J^\mu\neq 0$ erfordert zusätzlich das
Einführen des Elektronfeldes, worauf wir der Einfachheit halber hier
verzichten.}, sofern sie das zum Wirkungsintegral
\begin{equation}
S[A]=-\frac{1}{4}\int_M\tr\left(\star F\wedge F\right)\dif x
\end{equation}
minimieren.
\subsection{Eichtransformationen}
In den Maxwell-Gleichungen taucht das Feld $A$ nie explizit, sondern stets in
der durch $F$ vorgegebenen Kombination auf. Man macht sich leicht klar, das $F$
invariant unter Transformationen $A\to A+\dif\Lambda$ mit $\Lambda\in
C^\infty(\Reals^4)$ einer glatten Funktion. Diese zusätzliche Unbestimmtheit
bezeichnet man als \emph{Eichfreibeit}, bzw. Transformationen der Form als
Eichtransformationen.
%TODO U1 symmetrie
%TODO relativistische Lorentzkraft
%TODO QED in gekrümmtem Raum
\section{Allgemeine Relativitätstheorie}
\subsection{Einsteinsche Feldgleichungen}
Wie die Elektrodynamik, wird auch die allgemeine Relativitätstheorie durch einen
Satz von Differentialgleichungen beschrieben. Die so genannten
\emph{Einsteinschen Feldgleichungen} lauten in lokalen Koordinaten\footnote{Der Term proportional
zur kosmologischen Konstante $\Lambda$ hat über kleine Distanzen vernachlässigbaren Einfluss und wird im Folgenden nicht berücksichtigt.}
\begin{equation}
\tensor{R}{_\mu_\nu} - \frac{R}{2}\, \tensor{g}{_\mu_\nu}
+\Lambda\, \tensor{g}{_\mu_\nu}
=\tensor{T}{_\mu_\nu}\,.
\end{equation}
Der Tensor $\tensor{T}{_\mu_\nu}$ heißt \emph{Energie-Impuls-Tensor} und
enthält Information über die Materie, d.h. über die Felder im Raum. Eine
besonders einfache Form nehmen die Gleichungen im Vakuum an, denn dann gilt
$\tensor{T}{_\mu_\nu}=0$.
Bilden wir die Spur der Einsteingleichungen, so finden wir
\begin{equation}
0=R- 2 R =-R\,,
\end{equation}
die Skalarkrümmung $R$ verschwindet.
Einsetzen in die Einsteingleichungen liefert wiederum die Vakuumgleichung
\begin{equation}
\tensor{R}{_\mu_\nu}=0\,.
\end{equation}
\subsection{Variationsprinzip}
% TODO So formulieren Das S: Rank2sym->R problem S-> Max und
% |_\alphS(g+\alphah)=0 für alle Testfunktionen. Rechenregeln
Eine elegante Ableitung der Vakuumgleichungen ergibt sich mithilfe eines auf
Hilbert zurückgehenden Variationsprinzips.
Die Idee lautet: wähle die Metrik $g$ als kritischen
Punkt eines lokalen Funktionales, welches nur vom Riemannschen Krümmungstensor
abhängt.
Das einfachste solche Funktional ist die so genannte
\emph{Einstein-Hilbert-Wirkung}
\begin{equation}
S[g]=\int_{M}R[g]\,\Omega = \int_{M}\sqrt{-g}R[g] \dif{}^4x \,.
\end{equation}
Die zugehörige Lagrange Funktion ist gegeben als
\begin{equation}
L(g)=\sqrt{-g}R[g]\,.
\end{equation}
Bei der Variation dieses Funktionales treten verschiedene Terme auf. Wir
betrachten zunächst den Volumenfaktor $\sqrt{-g}$, mit Jacobis Formel und der
Kettenregel
%TODO REF
\begin{equation}
\begin{split}
\delta \sqrt{-g}
&= -\frac{1}{2\sqrt{-g}}  \left(\sqrt{-g}\tensor{g}{_\mu_\nu}
\delta\tensor{g}{^\mu^\nu} \right)\\
&= -\frac{1}{2}\sqrt{-g}\tensor{g}{_\mu_\nu}
\delta\tensor{g}{^\mu^\nu} \\
\end{split}
\end{equation}
Um die Variation des Krümmungskalar zu berechnen, wählen wir zweckmäßigerweise
ein Riemannsches Normalkoordinatensystem\footnote{Ein Koordinatensystem in dem
$\cSym{\rho}{\mu}{\sigma}=0$ im betrachteten Punkt}, sodass gilt
\begin{equation}
\tensor{R}{^\rho_\mu_\nu_\sigma}=\tensor{\partial}{_\nu}\cSym{\rho}{\mu}{\sigma}
-\tensor{\partial}{_\mu}\cSym{\rho}{\nu}{\sigma}\,.\\
\end{equation}
Betrachtet die Variation des Ricci-Tensors nach der Metrik so gilt
\begin{equation}
\begin{split}
\delta \tensor{R}{_\mu_\nu}
&=\delta \tensor{R}{^\rho_\mu_\rho_\nu}\\
&=\delta\tensor{\partial}{_\rho} \cSym{\rho}{\mu}{\nu}
-\delta\tensor{\partial}{_\mu} \cSym{\rho}{\rho}{\nu}\\
&=\tensor{\partial}{_\rho}\delta \cSym{\rho}{\mu}{\nu}
-\tensor{\partial}{_\mu}\delta \cSym{\rho}{\rho}{\nu}\\
&=\tensor{\nabla}{_\rho}\delta \cSym{\rho}{\mu}{\nu}
-\tensor{\nabla}{_\mu}\delta \cSym{\rho}{\rho}{\nu}\,,
\end{split}
\end{equation}
d.h. die Variation liefert eine totale Divergenz.
Weiter berechnen wir
\begin{equation}
\begin{split}
\delta R &=\delta \left(\tensor{g}{^\mu^\nu}\tensor{R}{_\mu_\nu}\right)\\
&=\tensor{R}{_\mu_\nu}\delta\tensor{g}{^\mu^\nu}
+\tensor{g}{^\mu^\nu}\delta\tensor{R}{_\mu_\nu}\,.
\end{split}
\end{equation}
Damit ist die Variation des EH-Funktionales gegeben als
\begin{equation}
\begin{split}
\delta S
&=\int\left(
R\delta\sqrt{-g}+\sqrt{-g}\delta R\right)\dif{}^4x\\
&=\int \left[
\frac{1}{2}\sqrt{-g}\tensor{g}{_\mu_\nu}\delta\tensor{g}{^\mu^\nu}
R+\sqrt{-g}\left(\tensor{R}{_\mu_\nu}\delta\tensor{g}{^\mu^\nu}
+\tensor{g}{^\mu^\nu}\delta\tensor{R}{_\mu_\nu}\right)\right]\dif{}^4x\\
&=\int \sqrt{-g}\left(
\frac{R}{2}\tensor{g}{_\mu_\nu}
+\tensor{R}{_\mu_\nu}\right)\delta\tensor{g}{^\mu^\nu}\dif{}^4x
+\int \sqrt{-g}\tensor{g}{^\mu^\nu}\delta\tensor{R}{_\mu_\nu}\dif{}^4x
\,.
\end{split}
\end{equation}
Das zweite Integral liefert als totale Divergenz nur einen Oberflächenterm.
Damit verbleibt
\begin{equation}
\begin{split}
\delta S
&=\int \sqrt{-g}\left(
\frac{R}{2}\tensor{g}{_\mu_\nu}
+\tensor{R}{_\mu_\nu}\right)\delta\tensor{g}{^\mu^\nu}\dif{}^4x\,.
\end{split}
\end{equation}
Die Extremalbedingung $\delta S=0$ für beliebige $\delta\tensor{g}{^\mu^\nu}$
impliziert
\begin{equation}
\tensor{R}{_\mu_\nu}+\frac{R}{2}\tensor{g}{_\mu_\nu}=0\, .
\end{equation}
Die Einsteingleichungen lassen sich also aus dem Variationsprinzip herleiten.
\subsection{Skalar-Tensor Theorien}
Brans-Dicke-Theorie
%TODO Overduin and Wesson 1997a).
\section{Gemeinsamkeiten und Unterschiede}
Unterschiede:
\begin{itemize}
  \item Lorentz-Kraft muss postuliert werden. Reine ART enthält keine Kräfte,
  alle Teilchen bewegen sich frei auf Geodäten.
  \item Ladung hat zwei Vorzeichen, d.h. es kann mittels umgekehrter Ladung
  entschieden werden ob man frei fällt oder nicht
  \item Die Einsteingleichungen sind nichtlinear und zweiter Ordnung die
  Maxwell-Gleichungen linear und erster Ordnung
  \item Kraft vs Geometrie
\end{itemize}
Gemeinsamkeiten:
\begin{itemize}
  \item Diffeomorphismen-Invarianz/Eichinvarianz
  \item In der nichtrelativistischen Näherung $1/r^2$ Gesetz.
  \item Masseloses bosonisches Vermittlerteilchen.
  \item Lagrangeformulierung
\end{itemize}
Insgesamt lässt sich hoffen das eine vereinheitlichte Theorie so formuliert
werden kann das sie einerseits die Ähnlichkeiten erklärt und auf gemeinsame
Ursachen zurückführt, andererseits aber auch Gründe für die Unterschiede gibt
bzw. diese möglicherweise ausräumt.

\chapter{Kaluzas erste Schritte}
Wie bereits in den vorherigen Kapiteln klar wurde, weißt die Struktur von
Elektromagnetismus und allgemeiner Relativitätstheorie deutliche Parallelen auf.
Dies wurde bereits in den ersten Jahren nach der Entwicklung der
allgemeinen Relativitätstheorie untersucht und mündete
unter anderem in Theorien von \name{H. Weyl}\cite{weyl1918gravitation} und
\name{G. Nordström} \cite{nordstrom1914moglichkeit}.

Dem Finnen \name{Nordström} gebührt dabei die Ehre, als Erster den
Elektromagnetismus als Phänomen einer fünfdimensionalen Raumzeit zu deuten.
Dabei baute seine Theorie auf der \name{Nordström}schen Gravitationstheorie
auf, welche allerdings verworfen wurde, da Voraussagen macht, welche nicht in
Einklang mit den Beobachtungen stehen\footnote{Beispielsweise drastische
Abweichungen in der gravitative Zeitverschiebung und der Periheldrehung des
Merkur.} .
Nicht zuletzt deshalb fand die Theorie seinerzeit wenig Beachtung, ist heute
allerdings weiterhin von theoretischem Interesse.
\name{Kaluza} bezieht sich vor allem auf \name{Weyl}s Arbeit, die aber ihrem
Charakter nach deutlich verschieden von der \name{Kaluza}-Theorie ist.
\section{Kaluza-Theorie}
\name{Kaluza} erkennt die Verwandtschaft der Theorien an der Form der
Christoffelsymbolen. Dazu bemerkt er\cite{kaluza1921unitatsproblem}:
%zu welcher er bei dem Vortrag \enquote{zum Unitätsproblem der Physik}
\begin{quote}
Die Rotorform der elektromagnetischen Feldkomponenten
$\tensor{F}{_\mu_\nu}$, noch mehr aber das
unverkennbare formale Entsprechen im Bau der Gravitations- 
und der elektromagnetischen Gleichungen fordern förmlich
die Vermutung heraus, die Feldkomponenten könnten so etwas wie
verstümmelte Dreizeigergrößen [Christoffelsymbole] sein.
\end{quote}
Er bezieht sich dabei darauf das die Gestalt der Christoffelsymbole erster Art
\begin{equation}
\tensor*{\Gamma}{_\lambda_\mu_\nu}=\frac{1}{2}
\left(\tensor{\partial}{_\nu} \tensor{g}{_\lambda_\mu} +
\tensor{\partial}{_\mu}
\tensor{g}{_\lambda_\nu} - \tensor{\partial}{_\lambda} 
\tensor{g}{_\mu_\nu}\right)\,,
\end{equation}
welche formal identisch zu 
\begin{equation}
\tensor{F}{_\mu_\nu}=\partial_\mu\tensor{A}{_\nu}-\partial_\nu\tensor{A}{_\mu}\,.
\end{equation}
wären, falls einer der Terme vom Typ $\tensor{\partial}{_\nu}
\tensor{g}{_\lambda_\mu}$ wegfiele.

Um die Gravitation mit der Elektrodynamik zu vereinigen werden aber zunächst 
zusätzliche Freiheitsgrade benötigt, die bereits in $\tensor{g}{_\mu_\nu}$
enthaltenen sind für die Gravitation verbraucht.
Statt des üblichen Minkowski Raums $\Reals^4$ (bzw. einer entsprechenden
Riemannschen Manigfaltigkeit), führen wir eine zusätzliche fünfte Raumdimension
ein sodass wir lokal homöomorph zu $\Reals^5$ sind. 
Die neue Koordinate benennen wir $\tensor{x}{^4}$\footnote{Kaluza
bezeichnet die Zusatzkomponente mit $x^0$, was bei uns für die Zeitkomponente
vorbehalten ist.}.
Dadurch stehen zusätzlich fünf
unabhängige Freiheitsgrade $\tensor{\hat{g}}{_4_i}$ zur Verfügung. Da eine
fünften Raumdimension nicht beobachtet wird, fordert Kaluza zusätzlich das die
Metrik nicht von der neuen Koordinate abhängt
\begin{equation}
\tensor{\partial}{_4} \tensor{\hat{g}}{_i_j}=0\,,\label{eq:Zylinderbed}
\end{equation}
die so genannte \emph{Zylinderbedingung}.
%TODO Erklären wieso die so heißt
Es mag jetzt zunächst so erscheinen als ob die Zusatzdimension dadurch überhaupt
keinen Einfluss mehr auf die Physik hätte. Das dem nicht so ist zeigt sich im
Folgenden.

Kommen wir zu den Christoffelsymbolen zurück, die 
Ausgangspunkt unserer Überlegungen waren. In fünf Dimensionen haben diese die
Gestalt
\begin{equation}
\tensor*{\hat{\Gamma}}{_i_j_k}=\frac{1}{2}
\left(\tensor{\partial}{_i}
\tensor{\hat{g}}{_k_j}+\tensor{\partial}{_j} \tensor{\hat{g}}{_k_i} -
\tensor{\partial}{_k} \tensor{\hat{g}}{_i_j}\right)\, .
\end{equation}
% \begin{equation}
% \tensor*{\Gamma}{_\lambda_\mu_\nu}=\frac{1}{2}
% \left(\tensor{\partial}{_\lambda} \tensor{g}{_\mu_\nu} + \tensor{\partial}{_\mu} \tensor{g}{_\lambda_\nu} -
% \tensor{\partial}{_\nu} \tensor{g}{_\lambda_\mu}\right)
% \end{equation}
Mit der Zylinderbedingung \eqref{eq:Zylinderbed} findet man 
\begin{equation}
\begin{split}
\tensor*{\hat{\Gamma}}{_4_\mu_\nu}&=\frac{1}{2}
	\left(\tensor{\partial}{_4}\tensor{\hat{g}}{_\mu_\nu} 
+ 	\tensor{\partial}{_\mu}\tensor{\hat{g}}{_\nu_4} 
- 	\tensor{\partial}{_\nu}\tensor{\hat{g}}{_4_\mu}\right)\\
&=\frac{1}{2}
\left(\tensor{\partial}{_\mu}
\tensor{\hat{g}}{_4_\nu} - \tensor{\partial}{_\nu}
\tensor{\hat{g}}{_4_\mu}\right)\,,
\end{split}
\end{equation}
also tatsächlich die gewünschte \enquote{Rotorform}. 
Für die verbleibenden unabhängigen Komponenten ergibt sich analog
\begin{align}
\tensor*{\hat{\Gamma}}{_\mu_\nu_4}&=\frac{1}{2}
\left(\tensor{\partial}{_\mu}
\tensor{\hat{g}}{_4_\nu}+\tensor{\partial}{_\nu}
\tensor{\hat{g}}{_4_\mu}\right)\,,\\
\tensor*{\hat{\Gamma}}{_4_\mu_4}&=-\tensor*{\hat{\Gamma}}{_\mu_4_4}=\frac{1}{2}\tensor{\partial}{_\mu}
\tensor{\hat{g}}{_4_4}\,,\\
\tensor*{\hat{\Gamma}}{_4_4_4}&= 0\,.
\end{align}
Es liegt nahe die neue
Feldkomponente
$\tensor{\hat{g}}{_4_\mu}$ mit
dem Viererpotential $ $ zu verknüpfen
\begin{equation}
\tensor{\hat{g}}{_4_\mu}=2\alpha\tensor{A}{_\mu}\,.
\end{equation}
Die Konstante $\alpha$ wird dabei erst später angepasst.
 Weiter benötigen wir neben der antisymmetrisieren
Ableitung $\tensor{F}{_\mu_\nu}=\tensor{\partial}{_{[\mu}}\tensor{A}{_{\nu]}}$, auch die
symmetrisierte Form $\tensor{\partial}{_{(\mu}}\tensor{A}{_{\nu)}}$, um alle
auftretenden Terme abzudecken. Dazu führen wir das \emph{Nebenfeld}
\begin{equation}
\tensor{H}{_\mu_\nu}:=\tensor{\partial}{_{(\mu}}\tensor{A}{_{\nu)}}=\partial_\mu\tensor{A}{_\nu}+\partial_\nu\tensor{A}{_\mu}
\end{equation}
ein. Zuletzt identifizieren wir die Größe $\tensor{\hat{g}}{_4_4}$  mit einem
Skalar\footnote{Bei Kaluza als \emph{Eckpotential} bezeichnet.}
\begin{equation}
\phi:=\frac{1}{2}\tensor{\hat{g}}{_4_4}\,.
\end{equation}
Die unabhängigen Christoffelsymbole sind damit von der Gestalt
% \begin{align}
% \tensor*{\hat{\Gamma}}{_\lambda_\mu_\nu}&=\tensor*{\Gamma}{_\lambda_\mu_\nu}\,,\\
% \tensor*{\hat{\Gamma}}{_\mu_\nu_4}&=-\alpha \tensor{H}{_\mu_\nu}\, ,\\
% \tensor*{\hat{\Gamma}}{_4_\mu_\nu}&=\alpha\tensor{F}{_\mu_\nu}\,,\\
% \tensor*{\hat{\Gamma}}{_4_4_\mu}&=-\tensor*{\hat{\Gamma}}{_4_\mu_4}
% =\tensor{\partial}{_\mu}\phi\,.
% \end{align}
% \begin{equation}
%     \tensor*{\hat{\Gamma}}{_\lambda_\mu_\nu}=\tensor*{\Gamma}{_\lambda_\mu_\nu}\,,
%     \quad\tensor*{\hat{\Gamma}}{_\mu_\nu_4}=-\alpha \tensor{H}{_\mu_\nu}\,,
%     \quad\tensor*{\hat{\Gamma}}{_4_\mu_\nu}=\alpha\tensor{F}{_\mu_\nu}\,,
%    \quad\tensor*{\hat{\Gamma}}{_4_4_\mu}=-\tensor*{\hat{\Gamma}}{_4_\mu_4}=\tensor{\partial}{_\mu}\phi\,.
% \end{equation}
\begin{equation}
  \begin{alignedat}{2}
    \tensor*{\hat{\Gamma}}{_\lambda_\mu_\nu}&=\tensor*{\Gamma}{_\lambda_\mu_\nu}\,,
    & \qquad \tensor*{\hat{\Gamma}}{_4_\mu_\nu}&=\alpha\tensor{F}{_\mu_\nu},\\
    \tensor*{\hat{\Gamma}}{_\mu_\nu_4}&=-\alpha \tensor{H}{_\mu_\nu}\,,&
    \qquad\quad\tensor*{\hat{\Gamma}}{_4_4_\mu}&=-\tensor*{\hat{\Gamma}}{_4_\mu_4}=\tensor{\partial}{_\mu}\phi\,.
  \end{alignedat}
\end{equation}
\begin{lemma}
\label{lemma:Christrel}
Die Christoffelsymbole erfüllen die Relation
\begin{equation}
\pdif{_m}\left(\tensor*{\hat{\Gamma}}{_i_k_\ell}+\tensor*{\hat{\Gamma}}{_k_\ell_i}+\tensor*{\hat{\Gamma}}{_\ell_i_k}\right)
=\tensor*{\hat{\Gamma}}{_m_i_k_{,\ell}}+\tensor*{\hat{\Gamma}}{_m_k_\ell_{,i}}+\tensor*{\hat{\Gamma}}{_m_\ell_i_{,k}}\,.
\end{equation}
\end{lemma}
\begin{proof}
\begin{equation*}
\begin{split}
\pdif{_m}\left(\tensor*{\hat{\Gamma}}{_i_k_\ell}+\tensor*{\hat{\Gamma}}{_k_\ell_i}+\tensor*{\hat{\Gamma}}{_\ell_i_k}\right)
&=\tensor*{\hat{\Gamma}}{_i_k_\ell_{,m}}+\tensor*{\hat{\Gamma}}{_k_\ell_i_{,m}}+\tensor*{\hat{\Gamma}}{_\ell_i_k_{,m}}\\
&=\frac{1}{2}
\left( \tensor{\hat{g}}{_\ell_i_{,km}} + \tensor{\hat{g}}{_\ell_k_{,im}}
-\tensor{\hat{g}}{_i_k_{,\ell m}}\right)\\
&\phantom{=}+\frac{1}{2}\left( \tensor{\hat{g}}{_i_\ell_{,km}} +
\tensor{\hat{g}}{_i_k_{,\ell m}} -\tensor{\hat{g}}{_k_\ell_{,im}}\right)\\
&\phantom{=}+\frac{1}{2}\left( \tensor{\hat{g}}{_k_\ell_{,im}} +
\tensor{\hat{g}}{_k_i_{,\ell m}} -\tensor{\hat{g}}{_i_\ell_{,km}}\right)\\
&=\frac{1}{2}
\left( \tensor{\hat{g}}{_\ell_i_{,km}} + \tensor{\hat{g}}{_\ell_k_{,im}}
+\tensor{\hat{g}}{_i_k_{,\ell m}}\right)\\
\end{split}
\end{equation*}
Durch Nulladdition von $\tensor{\hat{g}}{_k_m_{,i\ell}}$, $
\tensor{\hat{g}}{_i_m_{,\ell k}}$ und
$\tensor{\hat{g}}{_i_m_{,\ell k}}$ erhält man
\begin{equation}
\begin{split}
\pdif{_m}\left(\tensor*{\hat{\Gamma}}{_i_k_\ell}+\tensor*{\hat{\Gamma}}{_k_\ell_i}+\tensor*{\hat{\Gamma}}{_\ell_i_k}\right)
&=\frac{1}{2}
\left( \tensor{\hat{g}}{_k_m_{,i\ell}} + \tensor{\hat{g}}{_k_i_{,m\ell}}
-\tensor{\hat{g}}{_m_i_{,k\ell}}\right)\\
&\phantom{=}+\frac{1}{2}\left( \tensor{\hat{g}}{_\ell_m_{,ki}} +
\tensor{\hat{g}}{_\ell_k_{,mi}} -\tensor{\hat{g}}{_m_k_{,\ell i}}\right)\\
&\phantom{=}+\frac{1}{2}\left( \tensor{\hat{g}}{_i_m_{,\ell k}} +
\tensor{\hat{g}}{_i_\ell_{,mk}} -\tensor{\hat{g}}{_m_\ell_{,ik}}\right)\\
&=\tensor*{\hat{\Gamma}}{_m_i_k_{,\ell}}+\tensor*{\hat{\Gamma}}{_m_k_\ell_{,i}}+\tensor*{\hat{\Gamma}}{_m_\ell_i_{,k}}
\,. \qedhere
\end{split}
\end{equation}
\end{proof}
Lemma~\ref{lemma:Christrel} impliziert dann für $m=4$, zusammen mit der
Zylinderbedingung
\begin{equation}
0=\tensor*{\hat{\Gamma}}{_4_i_k_{,\ell}}+\tensor*{\hat{\Gamma}}{_4_k_\ell_{,i}}+\tensor*{\hat{\Gamma}}{_4_\ell_i_{,k}}\,.
\label{eq:fdchristrel}
\end{equation}
Setzt man $\ell = \lambda$, $i=\nu$, $k = \mu$, erhält man weiter 
\begin{equation}
\begin{split}
0&=\tensor*{\hat{\Gamma}}{_4_\nu_\mu_{,\lambda}}
+\tensor*{\hat{\Gamma}}{_4_\mu_\lambda_{,\nu}}
+\tensor*{\hat{\Gamma}}{_4_\lambda_\nu_{,\mu}}\\
&=\alpha\left(\tensor{F}{_\nu_\mu_{,\lambda}}
+\tensor{F}{_\mu_\lambda_{,\nu}}
+\tensor{F}{_\lambda_\nu_{,\mu}}\right)\,,
\end{split}
\end{equation}
die homogenen Maxwell-Gleichungen \eqref{eq:MaxwellHom}. Alle weiteren
Relationen, welche sich aus \eqref{eq:fdchristrel} ableiten lassen sind
trivial
\footnote{Zum einen erhält man falls zwei Indizes gleich vier sind
$0=\tensor{\phi}{_{,\nu\mu}}-\tensor{\phi}{_{,\mu\nu}}$, was aber nach dem
Satz von Schwarz für beliebige glatte $\phi$ erfüllt ist. Falls alle
Indizes gleich vier sind, verschwindet auch die rechte Seite der Gleichung.}.
%für
% $\ell = 4$, $i=\nu$, $k = \mu$ erhält man weiter
% \begin{equation}
% \begin{split}
% 0&=\tensor*{\hat{\Gamma}}{_4_\nu_\mu_{,4}}
% +\tensor*{\hat{\Gamma}}{_4_\mu_4_{,\nu}}
% +\tensor*{\hat{\Gamma}}{_4_4_\nu_{,\mu}}\\
% &=\tensor{\phi}{_{,\nu\mu}}-\tensor{\phi}{_{,\mu\nu}}
% \end{split}
% \end{equation}
% Was nach dem Satz von Schwartz erfüllt ist. Wenn mindestens zwei Indizes 4 sind
% ist die Gleichung mit der Zylinderbedingung ref erfüllt.

Wir wollen nun weiter überprüfen, welche Gleichungen sich aus den
fünfdimensionalen Analogien der Feld- und Geodätengleichungen ergeben.
Kaluza führte seine Berechnungen in der linearisierten Theorie durch, d.h. 
die auftretenden Energien sind klein.
%TODO wie klein
In der linearisierten Theorie gilt insbesondere
\begin{align}
\Gamma^2&\approx 0\tag{N1}\label{eq:N1}\,,\\
R&\approx 0\tag{N2}\label{eq:N2}\,,
\end{align}
wobei $\Gamma^2$ einen beliebigen, quadratischen Ausdruck in den
Christoffelsymbolen bezeichne.
Zur Berechnung des Ricci Tensors
$\tensor{\hat{R}}{_i_j}$ ist die folgende Formel nützlich:
\begin{equation}
\tensor{\hat{R}}{_i_j}=\tensor{\partial}{_l}\tensor*{\hat{\Gamma}}{^l_i_j}
-\tensor*{\hat{\Gamma}}{^m_i_l}\tensor*{\hat{\Gamma}}{^l_j_m}
-\tensor{\nabla}{_j}\left[\tensor{\partial}{_i}\left(\log\sqrt{-\hat{g}}\right)\right]\,
.\end{equation}
Insbesondere gilt aufgrund der Zylinderbedingung und der Näherungen
\eqref{eq:N1}`
%TODO wie schauts mit RNKS aus sind die resultierenden gleichungen tensoriell?
\begin{equation}
\tensor{\hat{R}}{_4_j}=\tensor{\partial}{_\ell}\tensor*{\hat{\Gamma}}{^\ell_4_j}
=\tensor{\partial}{^\ell}\tensor*{\hat{\Gamma}}{_4_j_\ell}
=\tensor{\partial}{^\lambda}\tensor*{\hat{\Gamma}}{_4_j_\lambda}\,.
\end{equation}
% \begin{equation}
% \tensor{\hat{R}}{_4_j}=\tensor{\partial}{_\lambda}\tensor*{\hat{\Gamma}}{^\lambda_4_j}
% -\tensor*{\hat{\Gamma}}{^m_4_l}\tensor*{\hat{\Gamma}}{^l_j_m}\,
% .\end{equation}
Damit berechnen wir für die zusätzlichen Komponenten von $\tensor{\hat{R}}{_i_j}$ 
% \begin{equation}
% \begin{split}
% \tensor{\hat{R}}{_4_4}
% &=\tensor{\partial}{_\lambda}\tensor*{\hat{\Gamma}}{^\lambda_4_4}
% -\tensor*{\hat{\Gamma}}{^m_4_l}\tensor*{\hat{\Gamma}}{^l_4_m}\\
% &=\tensor{\partial}{_\lambda}\tensor*{\hat{\Gamma}}{^\lambda_4_4}
% -\tensor*{\hat{\Gamma}}{^\mu_4_\lambda}\tensor*{\hat{\Gamma}}{^\lambda_4_\mu}
% -\tensor*{\hat{\Gamma}}{^\mu_4_4}\tensor*{\hat{\Gamma}}{^4_4_\mu}
% -\tensor*{\hat{\Gamma}}{^4_4_\lambda}\tensor*{\hat{\Gamma}}{^\lambda_4_4}
% \\
% &=-\tensor{\partial}{_\lambda}\tensor{\partial}{^\lambda}\phi
% -\alpha^2\tensor{F}{^\mu_\lambda}\tensor{F}{^\lambda_\mu}
% +\tensor{\partial}{^\mu}\phi\tensor{\partial}{_\mu}\phi
% +\tensor{\partial}{_\lambda}\phi\tensor{\partial}{^\lambda}\phi\\
% &=-\alpha^2\tensor{F}{_\mu_\nu}\tensor{F}{^\mu^\nu}-\square\phi
% +2\tensor{\partial}{^\mu}\phi\tensor{\partial}{_\mu}\phi
% \end{split}
% \end{equation}
% \begin{equation}
% \begin{split}
% \tensor{\hat{R}}{_4_\nu}
% &=\tensor{\partial}{_\lambda}\tensor*{\hat{\Gamma}}{^\lambda_4_\nu}
% -\tensor*{\hat{\Gamma}}{^m_4_l}\tensor*{\hat{\Gamma}}{^l_\nu_m}\\
% &=\tensor{\partial}{_\lambda}\tensor*{\hat{\Gamma}}{^\lambda_4_\nu}
% -\tensor*{\hat{\Gamma}}{^\mu_4_\lambda}\tensor*{\hat{\Gamma}}{^\lambda_\nu_\mu}
% -\tensor*{\hat{\Gamma}}{^\mu_4_4}\tensor*{\hat{\Gamma}}{^4_\nu_\mu}
% -\tensor*{\hat{\Gamma}}{^4_4_\lambda}\tensor*{\hat{\Gamma}}{^\lambda_\nu_4}\\
% &=-\alpha\tensor{\partial}{_\lambda}\tensor{F}{^\lambda_\nu}
% +\alpha\tensor{F}{^\mu_\lambda}\tensor*{\Gamma}{^\lambda_\nu_\mu}
% +\alpha\tensor{\partial}{^\mu}\phi\tensor{H}{_\nu_\mu}
% -\alpha\tensor{\partial}{_\lambda}\phi\tensor{F}{^\lambda_\nu}\\
% &=-\alpha\left(\tensor{\partial}{^\mu}\tensor{F}{_\mu_\nu}
% +\tensor*{\Gamma}{_\lambda_\mu_\nu}\tensor{F}{^\mu^\lambda}
% +2\tensor{\partial}{^\mu}\phi\tensor{\partial}{_\nu}\tensor{A}{_\mu}\right)
% \end{split}
% \end{equation}
\begin{equation}
\tensor{\hat{R}}{_4_4}
=\tensor{\partial}{^\lambda}\tensor*{\hat{\Gamma}}{_4_4_\lambda}
=\tensor{\partial}{^\lambda}\tensor{\partial}{_\lambda}\phi=\square\phi\,, 
\end{equation}
für die 44 Komponente, sowie 
\begin{equation}
\tensor{\hat{R}}{_4_\nu}
=\tensor{\partial}{^\lambda}\tensor*{\hat{\Gamma}}{_\lambda_4_\nu}
=-\alpha\tensor{\partial}{^\lambda}\tensor{F}{_\lambda_\nu}
=:-\alpha\tensor{J}{_\nu}\,.
\end{equation}
Trivialerweise gilt zudem $\tensor{\hat{R}}{_\mu_\nu}=\tensor{R}{_\mu_\nu}$. Die
, auf 5 Dimensionen verallgemeinerten, Feldgleichungen sind unter Näherung
\eqref{eq:N2} und mit Annahme verschwindender Kosmologischer Konstante
\begin{equation}
\kappa\tensor{\hat{T}}{_i_j}=\tensor{\hat{R}}{_i_j}\,.
\end{equation}
Damit ergeben sich zusätzlich zu den vierdimensionalen Einsteinschen
Feldgleichungen
\begin{equation}
\kappa\tensor{\hat{T}}{_4_4}=-\square\phi, \quad
\kappa\tensor{\hat{T}}{_4_\mu}=-\alpha\tensor{J}{_\mu}\,.\label{eq:ZusatzFG}
\end{equation}
Die $\tensor{\hat{T}}{_4_4}$ Komponente entspricht also der Masse des Feldes
$\phi$, $\tensor{\hat{T}}{_4_\mu}$ ist proportional zur Stromdichte.
 Wir definieren die Fünfergeschwindigkeit $\tensor{\hat{u}}{_i}$
\begin{equation}
\tensor{\hat{u}}{_i}:=\dod{\tensor{x}{_i}}{\hat{\tau}}\,,
\quad\dif\hat{\tau}^2=\tensor{\hat{g}}{_i_j}\dif\tensor{x}{^i}\dif\tensor{x}{^j}\,.
\end{equation}
Im Weiteren machen wir zusätzlich die Näherung, dass alle
Geschwindigkeiten\footnote{Im Fall von $\tensor{u}{_4}$ spricht man
besser von kleiner spezifischer Ladung.} klein sind, sprich
\begin{equation}
\tensor{\hat{u}}{_1},\tensor{\hat{u}}{_2},\tensor{\hat{u}}{_3},\tensor{\hat{u}}{_4}\ll 1
\,,\quad\tensor{\hat{u}}{_0}\approx 1\, .\tag{N3}\label{eq:N3}
\end{equation}
Wir beschreiben unser System durch Staub, d.h. wir vernachlässigen den Druck. 
Den zugehörigen Energie-Impuls-Tensor $\tensor{T}{_\mu_\nu}$
verallgemeinern wir auf fünf Dimensionen
\begin{equation}
\tensor{\hat{T}}{_i_j}=\mu_0\tensor{\hat{u}}{_i}\tensor{\hat{u}}{_j}\, ,
\end{equation}
wobei $\mu_0$ die Ruhemassendichte des Staubs bezeichnet. Für die
Zusatzkomponenten findet man 
\begin{equation}
\tensor{\hat{T}}{_4_\mu}
=\mu_0\tensor{\hat{u}}{_4}\tensor{\hat{u}}{_\mu}\approx\mu_0\tensor{\hat{u}}{_4}\tensor{u}{_\mu}\,,
\label{eq:StaubEMTens}
\end{equation}
da wegen \eqref{eq:N3} auch $\dif\hat{\tau}\approx\dif\tau$ gilt.
Dabei ist $\tensor{u}{_\mu}$ die gewöhnliche Vierergeschwindigkeit.
Setzt man jetzt \eqref{eq:ZusatzFG} und \eqref{eq:StaubEMTens} gleich so erhält
man
\begin{equation}
-\alpha\tensor{J}{_\mu}\approx\kappa\mu_0\tensor{\hat{u}}{_4}\tensor{u}{_\mu}
\,,
\end{equation}
und damit mit $\tensor{J}{_\mu}=\varrho_0\tensor{u}{_\mu}$
\begin{equation}
\varrho_0=-\frac{\kappa\mu_0}{\alpha}\tensor{\hat{u}}{_4}\,.\label{eq:murhorel}
\end{equation}
Wenden wir uns den Geodäten zu. Die verallgemeinerte Geodätengleichung
lautet
\begin{equation}
\od{\tensor{\hat{u}}{^\ell}}{\hat{\tau}}=-\tensor*{\hat{\Gamma}}{^\ell_m_n}\tensor{\hat{u}}{^m}\tensor{\hat{u}}{^n}
\, ,
\end{equation}
Da wir uns für die Bahnen in den uns zugänglichen vier Raumzeitdimensionen
interessieren ist, sind natürlich vor allem diese Komponenten interessant.
Hierbei gilt
\begin{equation}
\begin{split}
\dod{\tensor{\hat{u}}{^\lambda}}{\hat{\tau}}
+\tensor*{\hat{\Gamma}}{^\lambda_\mu_\nu}\tensor{\hat{u}}{^\mu}\tensor{\hat{u}}{^\nu}
&=
-\tensor*{\hat{\Gamma}}{^\lambda_m_n}\tensor{\hat{u}}{^m}\tensor{\hat{u}}{^n}
+\tensor*{\hat{\Gamma}}{^\lambda_\mu_\nu}\tensor{\hat{u}}{^\mu}\tensor{\hat{u}}{^\nu}\\
&=
-\tensor*{\hat{\Gamma}}{^\lambda_4_4}\tensor{\hat{u}}{^4}\tensor{\hat{u}}{^4}
-\tensor*{\hat{\Gamma}}{^\lambda_4_\nu}\tensor{\hat{u}}{^4}\tensor{\hat{u}}{^\nu}
-\tensor*{\hat{\Gamma}}{^\lambda_\mu_4}\tensor{\hat{u}}{^\mu}\tensor{\hat{u}}{^4}
-\tensor*{\hat{\Gamma}}{^\lambda_\mu_\nu}\tensor{\hat{u}}{^\mu}\tensor{\hat{u}}{^\nu}
+\tensor*{\hat{\Gamma}}{^\lambda_\mu_\nu}\tensor{\hat{u}}{^\mu}\tensor{\hat{u}}{^\nu}
\\
&=
\tensor{\partial}{_\lambda}\phi\left(\tensor{\hat{u}}{^4}\right)^2
+\alpha\tensor{F}{^\lambda_\nu}\tensor{\hat{u}}{^4}\tensor{\hat{u}}{^\nu}
+\alpha\tensor{F}{^\lambda_\mu}\tensor{\hat{u}}{^\mu}\tensor{\hat{u}}{^4}\\
&=
\tensor{\partial}{_\lambda}\phi\left(\tensor{\hat{u}}{^4}\right)^2
+2\alpha\tensor{F}{^\lambda_\nu}\tensor{\hat{u}}{^4}\tensor{\hat{u}}{^\nu}\,.
\end{split}
\end{equation}
Nach Voraussetzung ist $\tensor{\hat{u}}{^4}$ klein und damit
näherungsweise
\begin{equation}
\dod{\tensor{u}{^\lambda}}{\tau}
+\tensor*{\Gamma}{^\lambda_\mu_\nu}\tensor{u}{^\mu}\tensor{u}{^\nu}
=2\alpha\tensor{F}{^\lambda^\nu}\tensor{\hat{u}}{^4}\tensor{\hat{u}}{_\nu}\,.
\label{eq:KaluzaGeo}
\end{equation}
Damit sind wir am Ziel, denn der Parameter $\alpha$ ist weiterhin eine
freie Größe, wir setzen 
 $\alpha=\sqrt{\frac{\kappa}{2}}\approx 3,06\cdot 10^{-14}$ und
erhalten vermöge \eqref{eq:murhorel}
\begin{equation}
2\alpha\tensor{\hat{u}}{^4}=-
\frac{\varrho_0}{\mu_0}\,.
\end{equation}
Die Tatsache, dass $\alpha$ klein ist, legitimiert
die zuvor gemachten Näherungen im Nachhinein, denn Terme die quadratisch in den
Christoffelsymbolen sind sind auch quadratisch in $\alpha$.
Setzt man nun in \eqref{eq:KaluzaGeo} ein, so erhält man
\begin{equation}
\dod{\tensor{u}{^\lambda}}{\tau}
+\tensor*{\Gamma}{^\lambda_\mu_\nu}\tensor{u}{^\mu}\tensor{u}{^\nu}
=-\frac{\varrho_0}{\mu_0}\tensor{F}{^\lambda^\nu}\tensor{u}{_\nu}\, ,
\end{equation}
die Formel für die Geodäten mit Massendichte $\mu_0$ und Ladungsdichte
$\varrho_0$, unter Einwirkung der Lorentzkraft! Es verbleibt eine
Untersuchung der 4. Komponente
\begin{equation}
\begin{split}
\dod{\tensor{\hat{u}}{^4}}{\hat{\tau}}
&=
-\tensor*{\hat{\Gamma}}{^4_m_n}\tensor{\hat{u}}{^m}\tensor{\hat{u}}{^n}\\
&=
-\tensor*{\hat{\Gamma}}{^4_4_\nu}\tensor{\hat{u}}{^4}\tensor{\hat{u}}{^\nu}
-\tensor*{\hat{\Gamma}}{^4_\mu_4}\tensor{\hat{u}}{^\mu}\tensor{\hat{u}}{^4}
-\tensor*{\hat{\Gamma}}{^4_\mu_\nu}\tensor{\hat{u}}{^\mu}\tensor{\hat{u}}{^\nu}\\
&=
-\tensor*{\hat{\Gamma}}{^4_4_\nu}\tensor{\hat{u}}{^4}\tensor{\hat{u}}{^\nu}
-\tensor*{\hat{\Gamma}}{^4_\mu_4}\tensor{\hat{u}}{^\mu}\tensor{\hat{u}}{^4}
-\alpha\tensor{H}{_\mu_\nu}\tensor{\hat{u}}{^\mu}\tensor{\hat{u}}{^\nu}\\
&\approx
-\alpha\tensor{H}{_0_0}\\
&=
-2\alpha\tensor{A}{_0}\\
\end{split}
\end{equation}
aufgrund der Näherung ist $\dod{\tensor{\hat{u}}{^4}}{\hat{\tau}}\approx 0$ und
damit $\alpha$ konstant
\begin{equation}
\begin{split}
\tensor{\partial}{_0}\left(\frac{\varrho_0}{\mu_0}\right)
=-2\alpha\tensor{\partial}{_0}u_4=
4\alpha^2\tensor{A}{_0}
\end{split}
\end{equation}
\begin{bemerkung}
In den physikalisch relevanten Gleichungen taucht das Nebenfeld
$\tensor{H}{_\mu_\nu}$ nicht mehr auf. Eine Abhängigkeit der Gleichungen stünde
allerdings auch im Widerspruch zur Eichinvarianz.
\end{bemerkung}
\begin{bemerkung}
Wir verzichten darauf Indices mit der Metrik zu heben bzw. zu senken, da dies
nicht Wohldefiniert ist:
\begin{equation}
\tensor*{\delta}{*_\mu^\nu}=\tensor{\hat{g}}{_\mu_a}\tensor{\hat{g}}{^a^\nu}
=\tensor{g}{_\mu_\sigma}\tensor{g}{^\sigma^\nu}+\tensor{\hat{g}}{_\mu_5}\tensor{\hat{g}}{^5^\nu}
\end{equation}
\begin{equation}
\tensor{g}{_\mu_\sigma}\tensor{g}{^\sigma^\nu}
=\tensor*{\delta}{*_\mu^\nu}-\tensor{\hat{g}}{_\mu_5}\tensor{\hat{g}}{^5^\nu}\neq\tensor*{\delta}{*_\mu^\nu}
\end{equation}
\end{bemerkung}
\subsection{Erfolg}
Die Ableitung Kaluzas hat Charme. Insbesondere, dass er die
Lorentzkraft den fünfdimensionalen Einsteingleichungen folgern kann ist sehr 
befriedigend, da weniger Annahmen gemacht werden müssen als in der klassischen
Theorie. Die Annahme das die Spezifische Ladung klein ist ist schon für relativ
generische Objekte wie beispielsweise ein Elektron verletzt: die spezifische
Elementarladung, d.h. das Verhältnis eines Elektrons zu seiner Masse, beträgt nämlich
$\frac{e}{m}\approx1.76\cdot10^{11}\unitfrac{C}{kg}$.
Letztlich ist die Theorie damit mehr ein Spezialfall für "`schwach"' geladene
Objekte. Auch ist das Prozedere sehr konstruktiv und an vielen Stellen müssen
Näherungen gemacht werden.
Insgesamt ist durch Kaluzas Arbeit allerdings der Grundstein für die
Vereinheitlichung der Kräfte gelegt. Eine modernere Formulierung die sich unter
anderem an der Arbeiten von Klein, Jordan und Thiry orientiert wird im nächsten
Kapitel vorgestellt.

%
%
% Um weitere Freiheitsgrade zu erhalten führen wir eine weitere Raumdimension ein
% (Weitere Zeitdimensionen bereiten Probleme), da nur 3
% raumdimensionen beobachtet werden muss es einen Grund dafür geben das diese
% nicht sichtbar sind. Wir erweitern die Theorie deshalb um eine zusätzliche
% \emph{kompakte} Raumdimension mit ``Ausdehnung'' auf einer uns nicht
% zugänglichen Skala ($r\sim l\textsubscript{Plank}$ = Welche Länge? = Welche
% Energie Vgl. mit zugänglicher Energieskala am LHC).
% Wir nehmen an das die kompakte Dimension durch $\Sphere^1$ beschrieben wird, die
% Raumzeitmanigfaltigkeit als lokal(?) homoömorph zu $\Reals^4\times\Sphere^1$
% ist.
% Wir nehmen an das auch in dieser 5 dimensionalen Raumzeit die
% Einsteingleichungen ihre gültigkeit behalten. Allerdings erhalten wir nun
% anstatt der 10 Gleichungen 15 also 5 zusätzliche Freiheitsgrade (???)

% \section{Kleins Ansatz}
% 
% 
% 
% \section{Kaluzas Erweiterung}
% \section{Probleme}
% \section{Faserbündel}
% \section{Yang-Mills}
% \section{Flache Raumzeit}
% Falls die 4-Raumzeit flach ist also die betrachtete Manigfaltigkeit
% $M=\Reals^4\times\Sphere^1$ (Achtung hier braucht man eigendlich Faserbündel!!!)
% ergibt sich als Metrik mit $z=\left(x,\theta\right)$
% \begin{equation}
% \tensor{g}{_i_j}\dif\tensor{z}{^i}\dif\tensor{z}{^j}
% =\tensor{\eta}{_\mu_\nu}\dif\tensor{x}{^\mu}\dif\tensor{x}{^\nu}+r^2\dif\theta^2
% \end{equation}
% Untersuchung von Störungen dieser Metrik
% \begin{equation}
% \tensor{\hat{g}}{_i_j}\dif\tensor{z}{^i}\dif\tensor{z}{^j}
% =e^{-\phi/3}\left[e^{\phi}\left(\dif\theta+\kappa\tensor{A}{_\mu}\dif\tensor{x}{^\mu}\right)^2
% +\tensor{g}{_\mu_\nu}\dif\tensor{x}{^\mu}\dif\tensor{x}{^\nu}\right]
% \end{equation}

%\begin{equation}
%\tensor{\hat{g}}{_{MN}}\dif\tensor{z}{^M}\dif\tensor{z}{^N}
%=e^{-\phi/3}\left[e^{\phi}\left(\dif\theta+\kappa\tensor{A}{_\mu}\dif\tensor{x}{^\mu}\right)^2
% +\tensor{g}{_\mu_\nu}\dif\tensor{x}{^\mu}\dif\tensor{x}{^\nu}\right]
%\end{equation}
% Entwicklung in Fourrierreihen
% \begin{align*}
% \tensor{g}{_\mu_\nu}(z)&=\sum_{n=-\infty}^\infty
% \tensor*{g}{*^{(n)}*_\mu*_\nu}\left(x\right)e^{\imI n \theta}\\
% \tensor{A}{_\mu}(z)&=\sum_{n=-\infty}^\infty
% \tensor*{A}{*^{(n)}*_\mu}\left(x\right)e^{\imI n \theta}\\
% \phi(z)&=\sum_{n=-\infty}^\infty
% \phi^{(n)}\left(x\right)e^{\imI n \theta}
% \end{align*}
% \section{Zylinderbedingung}
% \section{Nordströms Theorie} 
% Bereits vor Einstein formulierte Gunnar Nordström 1912, eine rein geometrische Theorie der Gravitation (Zusammenhang?)
% 
% TODO ist dtheta ein killing vektorfeld?
\chapter{Moderne Formulierung}
Wir diskutieren in diesem Kapitel nun eine Formulierung der KK-Theorie in 
einer zeitgemäßen Formulierung. Die Lie-Gruppe $G$ sei kompakt und wirke auf
$E$ durch Isometrien. Wir diskutieren zunächst den Fall einer fündimensionalen
Manigfaltigkeit $E$.
Damit muss die Lie-Gruppe $\dim G=1$ eindimensional sein. Alle kompakten
eindimensionalem Manigfaltigkeiten sind diffeomorph zu $\Sphere^1$. Ohne
Einschränkung sei deshalb im Folgenden $E$ ein $\Sphere^1$-Faserbündel.
Die Wirkung der Gruppe $G$ erzeugt einen Fluss 
\begin{equation}
\Phi_t(p):=t.p
\end{equation}
sowie ein assoziertes Vektorfeld 
\begin{equation}
V(p):=\left[\dod{}{t}\Phi_t(p)\right]_{t=1}\,.
\end{equation}
weiter nehmen wir an das $g(V,V)\neq 0$. 
Die zu $V$ duale Form bezeichnen wir mit $\alpha$. 
Per Konstruktion gilt
\begin{equation}
\mathcal{L}_{V}g=0
\end{equation}
%TODO mit def von L
\begin{equation}
\pi^*\bar{g}=g
\end{equation}
Die physikalische 4 dimensionale Raumzeit ist damit $(E/G,\bar{g})$ gegeben.
Der eindimensionale Fall ist besonders schön da insbesondere nur eine kompakte 
Cinfty MF existiert. Jede Die Orbits sind dann zwingend periodisch. Dies muss 
schon für $d=2$ nicht mehr der Fall sein wie das Beispiel $\mathbb{T}^2$ zeigt.
\section{Horizontale und vertikale Räume}
Den Kern der Bündelprojektion nennen wir vertikalen Raum $\Verp
E:=\ker\{\pi_{*p}\}$. Anschaulich enthält $\Verp$ also gerade die Felder die
entlang der Faser verlaufen.
\begin{bemerkung}[$\Verp E$ ist $G$ invariant]
Zunächst bemerkt man, dass
$\pi\circ\Phi_g=\pi$ und damit $\pi_*\circ{\Phi_g}_*=\pi_*$. Sei $X_p\in \Verp$, dann gilt für 
$f\circ \Phi_g = f$ 
\begin{equation}
\begin{split}
{\Phi_g}_*\left(X_p(f)\right)&=X_{\Phi_g(p)}(f\circ\Phi_g)\\
&=X_{\Phi_g(p)}(f)
\end{split}
\end{equation}
Also ${\Phi_g}_*V_p=V_{g.p}$
\end{bemerkung}
Ein Zusammenhang auf $E$ ist eine Zerlegung 
\begin{equation}
\Tanp E=\Verp E\,\oplus\, \Horp E
\end{equation}
die stetig von $p$ abhängt und ${\Phi_g}_*\Horp=\mathrm{H}_{g.p}$.
Ist eine $G$-invariante Metrik auf $\Tan E$ gegeben, so ergibt sich durch  
\begin{equation}
\Horp:=\Verp^\perp=\left\{X\in \Tanp E\,\big|\, g_p(X,V)=0\,,\,\forall\, V\in
\Verp\right\}\,
\end{equation}
eine natürliche Definition des horizontalen Bündels.
Insegsammt erhalten wir eine Zerlegung des Tangentialbündels in ein vertikales
und ein horizontales Bündel
\begin{equation}
\Tan E=\Ver E\,\oplus\, \Hor E\,.
\end{equation}

\begin{equation}
\Ver E\cong \Tan G\cong M\times \mathfrak{g}\,,\quad
\Hor E\cong \Tan  (E/G)
\end{equation}
Die Horizontalen Felder sind diejenigen die physikalische Größen beschreiben. 
\section{Eigenschaften der Klein-Metrik}
Wie man leicht nachrechnet gilt
\begin{equation}
\begin{split}
\widehat{G}=\begin{pmatrix}B& C\\
C\transpose& H\end{pmatrix}
&=
\begin{pmatrix}1& A\\0& 1\end{pmatrix}
\begin{pmatrix}G& 0\\0& H
\end{pmatrix}
\begin{pmatrix}1& 0\\A\transpose& 1\end{pmatrix}\,.
\end{split}
\end{equation}
mit Blöcken 
\begin{equation}
G=B-CH^{-1}C\transpose\,,\quad A=CH^{-1}\,.
\end{equation}
Damit gilt weiter
\begin{equation}
\det(\fived{G})=\det(G)\det(H)
\end{equation}
Die Transformation lässt sich leicht umkehren
\begin{equation}
\begin{pmatrix}1& A\\0& 1\end{pmatrix}^{-1}=\begin{pmatrix}1& -A\\0&
1\end{pmatrix}\,.
\end{equation}
% TODO: damit misst das feld quasi das volumen der Zusatzdimension
Mit diesen Relationen ergibt sich eine alternative Darstellung des
Linienelements:
\begin{equation}
\begin{split}
\dif \fived{s}^2&=\dif \vec{x}\transpose G
\dif \vec{x}\\
&=\dif \vec{x}\transpose
\begin{pmatrix}1& -A\\0& 1\end{pmatrix}
\begin{pmatrix}G& 0\\0& H
\end{pmatrix}
\begin{pmatrix}1& 0\\-A\transpose& 1\end{pmatrix}
\dif \vec{x}\\
&=\tensor{g}{_\mu_\nu}\dif \tensor{x}{^\mu}\dif\tensor{x}{^\nu}
+\tensor{h}{_a_b}\left(\dif\tensor{x}{^a}-\tensor*{A}{^a_\mu}\dif\tensor{x}{^\mu}\right)
\left(\dif\tensor{x}{^b}-\tensor*{A}{^b_\mu}\dif\tensor{x}{^\mu}\right)\,.
\end{split}
\end{equation}
Da diese Darstellung im Wesentlichen vom Typ ist wie sie von Klein verwendet
wurde soll wird sie im Folgenden als Klein-Darstellung bzw. eine Metrik vom Typ
als Klein-Metrik bezeichnen.
%Berechnungen angelehnt an \cite{Coquereaux:1990qs} \cite{williams2015field}.
Im fünfdimensionalen Fall können wir die Metrik schreiben als
\begin{equation}
\tensor{\fived{g}}{_\mu_\nu}=\tensor{g}{_\mu_\nu}+\psi\tensor*{A}{_\mu}\tensor*{A}{_\nu}\,,\quad
\tensor{\fived{g}}{_4_\mu}=\psi\tensor*{A}{_\mu}\,,\quad
\tensor{\fived{g}}{_4_4}=\psi
\end{equation}
In Koordinatenfreier Darstellung
\begin{equation}
\fived{g}=g+\psi\,\omega\otimes\omega\,.
\end{equation} 
mit der 
1-Form $\omega$
\begin{equation}
\omega=\dif\tensor{x}{^4}+\tensor{A}{_\mu}\dif\tensor{x}{^\mu}\,.
\end{equation}
Damit ist $M=\ker \omega$, mit anderen Worten $\omega$ ist eine
Zusammenhangsform auf $E$
\subsection{Inverse}
Wie sich leicht nachprüfen lässt ist die Inverse der Klein-Metrik gegeben als
\begin{equation}
\tensor{\fived{g}}{^\mu^\nu}=\tensor{g}{^\mu^\nu}\,,\quad
\tensor{\fived{g}}{^4^\mu}=-\tensor{A}{^\mu}\,,\quad
\tensor{\fived{g}}{^4^4}=\tensor*{A}{_\mu}\tensor*{A}{^\mu}+\frac{1}{\psi}\,.
\end{equation}
% TODO damit kann entweder mit ghut oder g indices nach oben gezogen werden.
% Inkonsistent? ghut4mu?

% \subsection{Determinante}
% Die Determinante wird wie üblich zur Berechnung von Volumenelementen benötigt.
% Bei der Berechnung ist folgende Formel für Matrizen $A,B,C,D$ hilfreich:
% \begin{equation}
% \begin{split}
% \det\begin{pmatrix}A& B\\ C& D\end{pmatrix}&=\det\left[
% \begin{pmatrix}1& B\\0& D\end{pmatrix}\begin{pmatrix}A-BD^{-1}C& 0\\DC^{-1
% }& 1\end{pmatrix} \right]\\
% &= \det(D) \det\left(A - B D^{-1}
% C\right)\,.
% \end{split}
% \end{equation}
% %TODO evtl. Beweis
% Damit ergibt sich 
% \begin{equation}
% \begin{split}
%  \fived{g}&=
%  \det\begin{pmatrix}\tensor{g}{_\mu_\nu}+\psi\tensor{A}{_\mu}\tensor{A}{_\nu}
%  &\psi \tensor{A}{_\mu}\\
%  \psi \tensor{A}{_\mu}	 & \psi
%  \end{pmatrix}\\
%  &=\psi
%  \det\left(\tensor{g}{_\mu_\nu}+\psi\tensor{A}{_\mu}\tensor{A}{_\nu}
%  -\psi\tensor{A}{_\mu}\psi^{-1}\psi\tensor{A}{_\nu}\right)\\
%  &=\psi \det\left(\tensor{g}{_\mu_\nu}\right)\\
%  &=\psi g\,.
% \end{split}
% \end{equation}
% Die Determinante der fünfdimensionalen Metrik ist also proportional zur 4
% dimensionalen. 
% \begin{bemerkung}
% Die entsprechenden Ausdrücke für die Kaluza-Metrik sind deutlich komplizierter.
% \end{bemerkung}
% 
% 
% \section{Die fünfdimensionale Raumzeit}
% Konstruktion der MFG: Siehe "`Coquereau Geometry of Multidimensional
% Universes"'\cite{coquereaux1983geometry}.
% Die Metrik ist von der Form
\section{Identifikation der Größen}
Wir wollen die Objekte $A$ und $\psi$ als Eichpotential bzw. Skalarfeld auf
$M/G$ auffassen, dazu müssen wir zeigen das sie entsprechend transformieren.
Da die Lie-Ableitung der Metrik nach $V=\partial_4$ verschwindet wissen wir,
dass die Metrik invariant unter infinitesimalen Diffeomorphismen $\xi$ sein muss.
Betrachte die Abbildung $f:M\to M$ 
\begin{equation}
\tensor{x}{^\mu}\mapsto
\tensor{x}{^\mu}\,,\quad\tensor{x}{^4}\mapsto\tensor{x}{^4}+
\xi\left(\tensor{x}{^\mu}\right)
\end{equation}
mit $\xi\in C^\infty\left(\Reals\right)$ beliebig. 
%TODO kommt von killingvector del4
Pullback $\fived{g}^\prime:=f^*\fived{g}$
%TODO Pullback in Komponenten
\begin{equation}
\begin{split}
\tensor*{\fived{g}}{*^\prime_\mu_\nu}&=\tensor{\fived{g}}{_a_b}\dpd{f^a}{\tensor{x}{^\mu}}\dpd{f^b}{\tensor{x}{^\nu}}\\
&=\tensor{\fived{g}}{_\alpha_\beta}\dpd{f^\alpha}{\tensor{x}{^\mu}}\dpd{f^\beta}{\tensor{x}{^\nu}}
+\tensor{\fived{g}}{_\alpha_4}\dpd{f^\alpha}{\tensor{x}{^\mu}}\dpd{f^4}{\tensor{x}{^\nu}}
+\tensor{\fived{g}}{_4_\beta}\dpd{f^4}{\tensor{x}{^\mu}}\dpd{f^\beta}{\tensor{x}{^\nu}}
+\tensor{\fived{g}}{_4_4}\dpd{f^4}{\tensor{x}{^\mu}}\dpd{f^4}{\tensor{x}{^\nu}}
\\
&=\tensor{\fived{g}}{_\mu_\nu}
+\psi\tensor{A}{_\mu}\pdif{_\nu}\xi
+\psi\tensor{A}{_\nu}\pdif{_\mu}\xi
+\psi\pdif{_\mu}\xi\pdif{_\nu}\xi\\
&=\tensor{g}{_\mu_\nu}
+\psi\left(\tensor{A}{_\mu}+\pdif{_\mu}\xi\right)
\left(\tensor{A}{_\nu}+\pdif{_\nu}\xi\right)
\end{split}
\end{equation}
\begin{equation}
\begin{split}
\tensor*{\fived{g}}{*^\prime_\mu_4}&=\tensor{\fived{g}}{_a_b}\dpd{f^a}{\tensor{x}{^\mu}}\dpd{f^b}{\tensor{x}{^4}}\\
&=\tensor{\fived{g}}{_\alpha_\beta}\dpd{f^\alpha}{\tensor{x}{^\mu}}\dpd{f^\beta}{\tensor{x}{^4}}
+\tensor{\fived{g}}{_\alpha_4}\dpd{f^\alpha}{\tensor{x}{^\mu}}\dpd{f^4}{\tensor{x}{^4}}
+\tensor{\fived{g}}{_4_\beta}\dpd{f^4}{\tensor{x}{^\mu}}\dpd{f^\beta}{\tensor{x}{^4}}
+\tensor{\fived{g}}{_4_4}\dpd{f^4}{\tensor{x}{^\mu}}\dpd{f^4}{\tensor{x}{^4}}\\
&=\psi\left(\tensor{A}{_\mu}+\pdif{_\mu}\xi\right)
\end{split}
\end{equation}
\begin{equation}
\begin{split}
\tensor*{\fived{g}}{*^\prime_4_4}&=\tensor{\fived{g}}{_a_b}\dpd{f^a}{\tensor{x}{^4}}\dpd{f^b}{\tensor{x}{^4}}\\
&=\tensor{\fived{g}}{_\alpha_\beta}\dpd{f^\alpha}{\tensor{x}{^4}}\dpd{f^\beta}{\tensor{x}{^4}}
+\tensor{\fived{g}}{_\alpha_4}\dpd{f^\alpha}{\tensor{x}{^4}}\dpd{f^4}{\tensor{x}{^4}}
+\tensor{\fived{g}}{_4_\beta}\dpd{f^4}{\tensor{x}{^4}}\dpd{f^\beta}{\tensor{x}{^4}}
+\tensor{\fived{g}}{_4_4}\dpd{f^4}{\tensor{x}{^4}}\dpd{f^4}{\tensor{x}{^4}}\\
&=\psi
\end{split}
\end{equation}
Das Transformationsverhalten der Größen lässt sich zusammenfassen als 
\begin{equation}
\tensor{g}{_\mu_\nu}\to\tensor{g}{_\mu_\nu}\,,\quad
\tensor{A}{_\mu}\to\tensor{A}{_\mu}+\pdif{_\mu}\xi\,,\quad
\psi\to\psi
\end{equation}
Also entsprechen infinitesimalen Diffeomorphismen auf $M$ Eichtransformationen.
Weiter ist zu Zeigen das $A$ als Differentialform $M/G$ interpretiert werden,
um dies einzusehen studieren wir das Verhalten unter Koordinatenwechseln $f:M\to
M$
\begin{equation}
\tensor{x}{^\mu}\mapsto
h(\tensor{x}{^\mu})\,,\quad\tensor{x}{^4}\mapsto\tensor{x}{^4}
\end{equation}
mit einem Diffeomorphismus $h$
\begin{equation}
\begin{split}
\tensor*{\fived{g}}{*^\prime_\mu_\nu}&=\tensor{\fived{g}}{_a_b}\dpd{f^a}{\tensor{x}{^\mu}}\dpd{f^b}{\tensor{x}{^\nu}}\\
&=\tensor{\fived{g}}{_\alpha_\beta}\dpd{f^\alpha}{\tensor{x}{^\mu}}\dpd{f^\beta}{\tensor{x}{^\nu}}
+\tensor{\fived{g}}{_\alpha_4}\dpd{f^\alpha}{\tensor{x}{^\mu}}\dpd{f^4}{\tensor{x}{^\nu}}
+\tensor{\fived{g}}{_4_\beta}\dpd{f^4}{\tensor{x}{^\mu}}\dpd{f^\beta}{\tensor{x}{^\nu}}
+\tensor{\fived{g}}{_4_4}\dpd{f^4}{\tensor{x}{^\mu}}\dpd{f^4}{\tensor{x}{^\nu}}\\
&=\tensor{\fived{g}}{_\alpha_\beta}\dpd{h^\alpha}{\tensor{x}{^\mu}}\dpd{h^\beta}{\tensor{x}{^\nu}}\\
&=\tensor{g}{_\alpha_\beta}\dpd{h^\alpha}{\tensor{x}{^\mu}}\dpd{h^\beta}{\tensor{x}{^\nu}}
+\psi\tensor{A}{_\alpha}\dpd{h^\alpha}{\tensor{x}{^\mu}}
\tensor{A}{_\beta}\dpd{h^\beta}{\tensor{x}{^\nu}}
\end{split}
\end{equation}
\begin{equation}
\begin{split}
\tensor*{\fived{g}}{*^\prime_\mu_4}&=\tensor{\fived{g}}{_a_b}\dpd{f^a}{\tensor{x}{^\mu}}\dpd{f^b}{\tensor{x}{^4}}\\
&=\tensor{\fived{g}}{_\alpha_\beta}\dpd{f^\alpha}{\tensor{x}{^\mu}}\dpd{f^\beta}{\tensor{x}{^4}}
+\tensor{\fived{g}}{_\alpha_4}\dpd{f^\alpha}{\tensor{x}{^\mu}}\dpd{f^4}{\tensor{x}{^4}}
+\tensor{\fived{g}}{_4_\beta}\dpd{f^4}{\tensor{x}{^\mu}}\dpd{f^\beta}{\tensor{x}{^4}}
+\tensor{\fived{g}}{_4_4}\dpd{f^4}{\tensor{x}{^\mu}}\dpd{f^4}{\tensor{x}{^4}}\\
&=\psi\tensor{A}{_\alpha}\dpd{h^\alpha}{\tensor{x}{^\mu}}
\end{split}
\end{equation}
\begin{equation}
\begin{split}
\tensor*{\fived{g}}{*^\prime_4_4}&=\tensor{\fived{g}}{_a_b}\dpd{f^a}{\tensor{x}{^4}}\dpd{f^b}{\tensor{x}{^4}}\\
&=\tensor{\fived{g}}{_\alpha_\beta}\dpd{f^\alpha}{\tensor{x}{^4}}\dpd{f^\beta}{\tensor{x}{^4}}
+\tensor{\fived{g}}{_\alpha_4}\dpd{f^\alpha}{\tensor{x}{^4}}\dpd{f^4}{\tensor{x}{^4}}
+\tensor{\fived{g}}{_4_\beta}\dpd{f^4}{\tensor{x}{^4}}\dpd{f^\beta}{\tensor{x}{^4}}
+\tensor{\fived{g}}{_4_4}\dpd{f^4}{\tensor{x}{^4}}\dpd{f^4}{\tensor{x}{^4}}\\
&=\psi
\end{split}
\end{equation}
\begin{equation}
\tensor{g}{_\mu_\nu}\to\tensor{g}{_\alpha_\beta}\dpd{h^\alpha}{\tensor{x}{^\mu}}\dpd{h^\beta}{\tensor{x}{^\nu}}\,,\quad
\tensor{A}{_\mu}\to\tensor{A}{_\alpha}\dpd{h^\alpha}{\tensor{x}{^\mu}}\,,\quad
\psi\to\psi
\end{equation}
Diffeomorphismus-Invarianz der Klein Metrik impliziert also sowohl die
Eichinvarianz des Viererpotentials $\tensor{A}{_\mu}$, als auch das korrekte
Transformationsverhalten unter vierdimensionalen Koordinatenwechseln.
Dabei transformieren $\psi$, $\tensor{A}{_\mu}$ und $\tensor{g}{_\mu_\nu}$ als
Skalar, Vektor, bzw. Tensor.
%TODO sind noch weitere Trafos erlaubt?
\section{Krümmung}
\begin{equation}
\fived{R}=R-\frac{1}{4}\psi\tensor{F}{_\mu_\nu}\tensor{F}{^\mu^\nu}
+\frac{1}{2\psi^2}(\pdif{_\mu}\psi)(\pdif{^\mu}\psi)
-\frac{1}{\psi}\square\psi
\end{equation}
Wir führen zweckmäßigerweise ein Feld
$\sigma=\ln\frac{\psi}{2}\psi$ ein, womit sich der Term nochmals
vereinfacht \footnote{Dies stellt keine Einschränkung dar, da $\psi$ positiv sein muss, damit $\fived{g}$ das gleiche Vorzeichen hat wie $g$. (Bei der Zusatzdimension handelt es sich um
eine Raumdimension)}:
\begin{equation}
\fived{R}=R-\frac{1}{4}e^{2\sigma}\tensor{F}{_\mu_\nu}\tensor{F}{^\mu^\nu}
-2e^{-\sigma}\square e^{\sigma}
\end{equation}
\section{Die Wirkung in fünf Dimensionen}
Kleins Idee war es das Wirkungsprinzip nach Hilbert auf fünf Dimensionen zu
erweitern. Folglich lautet das Wirkungsintegral
\begin{equation}
\fived{S}=\int_{E}\sqrt{-\fived{g}}\fived{R}\dif{^5}
x\,.
\end{equation}
Da die Metrik $G$-invariant ist der Integrand offensichtlich ebenfalls
$G$-Invariant und wir können ??? anwenden. Damit gilt:
\begin{equation}
\begin{split}
\fived{S}&=2\pi
r\int_{M}\sqrt{-g}\psi^{1/2}\fived{R}\dif{^4}x\,,
\end{split}
\end{equation}
wobei wir mit $r=\frac{1}{2\pi}\vol(G)$ den Radius des Orbits bezeichnen.
Die Lagrangedichte auf $M$ ist also gegeben durch
\begin{equation}
\begin{split}
\mathcal{L}&=\sqrt{-g}e^{\sigma}\left(R-\frac{1}{4}e^{2\sigma}\tensor{F}{_\mu_\nu}\tensor{F}{^\mu^\nu}
-2e^{-\sigma}\square e^{\sigma}\right)\\
&=\sqrt{-g}e^{\sigma}\left(R-\frac{1}{4}e^{2\sigma}\tensor{F}{_\mu_\nu}\tensor{F}{^\mu^\nu}\right)+2\sqrt{-g}\square
e^{\sigma}\,.\label{eq:Lagrange1}
\end{split}
\end{equation}
Der letzte Term liefert als totale Divergenz keinen Beitrag.
\subsection{Konforme Transformation}
Da wir den Lagrangian gerne in einer Form vorliegen hätten, die dem
EH-Lagrangian enspricht müssen wir den Vorfaktor $e^{\sigma}$ loswerden. 
Wir führen dazu eine konforme Transformation durch
\begin{equation}
\tensor*{\fived{g}}{*^\star*_i*_j}=e^{2\tau}\tensor{\fived{g}}{_i_j}\,.
\end{equation}
Dies impliziert sofort
\begin{equation}
\tensor*{g}{*^\star*_\mu*_\nu}=e^{2\tau}\tensor{g}{_\mu_\nu}\,,\quad\sigma^\star
=\sigma+\tau\,,\quad \sqrt{-g}=e^{-4\tau}\sqrt{-g^\star}\,.
\end{equation}
Die Komponenten des elektrischen Feldstärketensors ist in der konformen Metrik
gegeben als
\begin{equation}
\tensor*{F}{*^\star_\mu_\nu}=\tensor{F}{_\mu_\nu}\,,\quad\tensor*{F}{*^\star^\mu^\nu}
=\tensor*{\fived{g}}{*^\star*^\mu*^\alpha}\tensor*{\fived{g}}{*^\star*^\nu*^\beta}\tensor{F}{_\alpha_\beta}
=e^{-4\tau}\tensor{F}{^\mu^\nu}\,.
\end{equation}
Wendet man die Formel für das Transformationsverhalten des Krümmungsskalars an,
so findet man schließlich
\begin{equation}
R=e^{2\tau}\left[R^\star-6(\tensor*{\partial}{^\star_\mu}\tau)(\tensor{\partial}{^\star^\mu}\tau)
-6\square^\star\tau\right]\,,
\end{equation}
beziehungsweise
\begin{equation}
\begin{split}
\sqrt{-g}\phi
\fived{R}
&=e^{\sigma-2\tau}\sqrt{-g^\star}\left[\fived{R}^\star
-6(\tensor*{\partial}{^\star_\mu}\tau)(\tensor{\partial}{^\star^\mu}\tau)
-6\square^\star\tau\right]\,.
\end{split}
\end{equation}
% TODO Einfluss konformer Trafos
Setzt man $\tau = \frac{1}{2}\sigma$\footnote{Die daraus resultierende
Relation $\sigma^\star=\frac{3}{2}\sigma$ kann durch eine Skalierung von
$\sigma$ behandelt werden.}, so ergibt sich
\begin{equation}
\begin{split}
\sqrt{-g}\phi
\fived{R}&=\sqrt{-g^\star}\left[R^\star
-\frac{3}{2}(\tensor*{\partial}{^\star_\mu}\sigma)(\tensor{\partial}{^\star^\mu}\sigma)
-3\square^\star\sigma\right]\,.
\end{split}
\end{equation}
Setzt man in \eqref{eq:Lagrange1} ein erhält man
\begin{equation}
\begin{split}
\mathcal{L}&=\sqrt{-g^\star}\left[R^\star
+\frac{3}{2}(\tensor*{\partial}{^\star_\mu}\sigma)(\tensor{\partial}{^\star^\mu}\sigma)
-3\square^\star\sigma\right]
+\sqrt{-g}e^{\sigma}\left(-\frac{1}{4}e^{2\sigma}\tensor{F}{_\mu_\nu}\tensor{F}{^\mu^\nu}\right)\\
&=\sqrt{-g^\star}\left[R^\star
-\frac{3}{2}(\tensor*{\partial}{^\star_\mu}\sigma)(\tensor{\partial}{^\star^\mu}\sigma)
-\frac{1}{4}e^{3\sigma}\tensor*{F}{^\star_\mu_\nu}\tensor*{F}{^\star^\mu^\nu}\right]-3\sqrt{-g^\star}\square^\star\sigma\,.
\end{split}
\end{equation}
Der Term mit der totalen Divergenz produziert einen nicht beitragenden Randterm
und kann deshalb fallen gelassen werden. 
%TODO Randterme
Da die konform Transformierte Metrik die selben Eigenschaften wie die Metrik
selbst besitzt sind die Probleme $S[g]\to\text{min}$ und
$S\left[g^\star\right]\to\text{min}$ äquivalent, wir lassen deshalb wir im
Folgenden die Sterne an den Größen weg. 
Zudem führen wir Bezeichnungen die den Lagrangian in eine gebräuchlich Form
bringen:
\begin{equation}
\begin{split}
\mathcal{L}
&=\sqrt{-g}\left[R
-\frac{1}{2}(\tensor{\partial}{_\mu}\sigma)(\tensor{\partial}{^\mu}\sigma)
-\frac{1}{4}\tensor*{H}{_\mu_\nu}\tensor*{F}{^\mu^\nu}\right]\,.
\label{eq:Lagrange2}
\end{split}
\end{equation}
Dabei bezeichnet die
Größe $H$ den elektromagnetischen Verschiebungstensor\footnote{In Analogie zu
den Verschiebungsfeldern $\vec{D},\vec{H}$ der klassischen Elektrodynamik.
}
%TODO Insbesondere handelt es sich nicht um das Nebenfeld
% $\tensor{H}{_\mu_\nu}$.
\begin{equation}
\tensor*{H}{_\mu_\nu}:=e^{\sqrt{3}\sigma}\tensor*{F}{_\mu_\nu}\,,
\end{equation}
wobei der Term $e^{\sqrt{3}\sigma}$ als variable Perimitivität $\mu(\sigma)$
interpretiert wird. 
% Ist es konsistent nach den Komponenten einzeln zu variieren?
\subsection{Bewegungsgleichungen}
In bekannter Manier impliziert die Form der Lagrangedichte 
Die Einsteingleichungen:
\begin{equation}
\tensor{G}{_\mu_\nu}=\tensor*{T}{*^{\sigma}_\mu_\nu}+\tensor*{T}{*^{A}*_\mu*_\nu}
\end{equation}
Weiter erhält man für die Variation nach den $A$ 
Das Feld $A$ ist zyklisch, taucht also nicht selbst in der Lagrangedichte auf.
Die resultierende Bewegungsgleichungen lautet
\begin{equation}
0=\tensor{\nabla}{_\alpha}
\left[\dpd{\mathcal{L}}{\left(\tensor{\nabla}{_\alpha}\tensor{A}{_\beta}\right)}\right]
=\tensor{\nabla}{_\alpha}\left[-\frac{1}{4}e^{\sqrt{3}\sigma}\dpd{\left(\tensor{F}{_\mu_\nu}\tensor{F}{^\mu^\nu}\right)}{\left(\tensor{\nabla}{_\alpha}\tensor{A}{_\beta}\right)}\right]
=-\tensor{\nabla}{_\alpha}\tensor{H}{^\alpha^\beta}
 \end{equation}
 Für das Vektorpotential $A$, sowie
 % Ableitung nach DA im MW teil herleinten
 \begin{equation}
0=\tensor{\nabla}{_\alpha}\left[\dpd{\mathcal{L}}{\left(\tensor{\partial}{_\alpha}\sigma\right)}\right]
-\dpd{\mathcal{L}}{\sigma}\\
=\square \sigma
-\frac{\sqrt{3}}{4}\tensor*{H}{_\mu_\nu}\tensor*{F}{^\mu^\nu}
\label{eq:dymdilat}
 \end{equation}
 für das Skalarfeld $\sigma$. 
 % Lässt sich das durch redifinition zur Klein Gordon Gleichung hinbiegen?
\section{Erhaltungrößen}
Da nach Konstruktion $\partial_4$ ein Killing Vektorfeld, damit erhalten wir
eine Erhaltungsgröße
\begin{equation}
\begin{split}
K&=\delta^\mu_4\fived{U}_\mu\\
&=\fived{g}_{4\mu}\fived{U}^\mu\\
&=\psi\left(A_\nu\fived{U}^\nu+\fived{U}^4\right)\\
&=\psi\left(A_\nu U^\nu+U^4\right)\dod{s}{\tau}\\
\end{split}
\end{equation}
Die vierergeschwindigkeit identifizert man mit
\begin{equation}
U^a=\dod{x^a}{\tau}=\dod{s}{\tau}\hat{U}^a
\end{equation}
Es ist günstig die Größen
\begin{equation}
\Delta=A_\nu U^\nu+U^4\,,\quad
\fived{\Delta}=\Delta\dod{s}{\tau}\,,
\end{equation}
einzuführen.
\begin{equation}
\begin{split}
K
&=\psi\hat{\Delta}\\
\end{split}
\end{equation}
Weiter gilt dann 
\begin{equation}
\left(\dod{\tau}{s}\right)^2=1+\psi\fived{\Delta}^2=1+K^2\psi^{-1}
\end{equation}
%TODO Tachionen
 \section{Geodäten und die Lorentzkraft}
In lokalen Koordinaten lautet die Geodätengleichung
\begin{equation}
\dod[2]{x^a}{s}+\fived{\Gamma}^{a}_{bc}\dod{x^b}{s}\dod{x^c}{s}=0
\end{equation}
Zunächst bemerken wir, dass das Linienelement  
$\dif s^2=\dif \tau^2+\psi(A_\nu\dif x^\nu+\dif x^4)^2$ 
verschieden von der Eigenzeit $\dif \tau^2$ ist.
Parametrisieren wir die Kurve nach der Eigenzeit $\tau$, so erhalten
wir 
\begin{equation}
\dod[2]{x^a}{\tau}+\fived{\Gamma}^{a}_{bc}\dod{x^b}{\tau}\dod{x^c}{\tau}
=f(\tau)\dod{x^a}{\tau}\,.
\end{equation}
Für die ersten vier Komponenten erhalten wir
\begin{equation}
\begin{split}
0=\dod[2]{x^\mu}{\tau}
+\fived{\Gamma}^{\mu}_{\nu\lambda}\dod{x^\nu}{\tau}\dod{x^\lambda}{\tau}
+2\fived{\Gamma}^{\mu}_{\nu 4}\dod{x^\nu}{\tau}\dod{x^4}{\tau}
+\fived{\Gamma}^{\mu}_{44}\dod{x^4}{\tau}\dod{x^4}{\tau}
-f(\tau)\dod{x^a}{\tau}\\
\end{split}
\end{equation}
Umstellen liefert wiederum
\begin{equation}
\begin{split}
\dod[2]{x^\mu}{\tau}
+\Gamma^{\mu}_{\nu\lambda}\dod{x^\nu}{\tau}\dod{x^\lambda}{\tau}
=&\left(\fived{\Gamma}^{\mu}_{\nu\lambda}-\Gamma^{\mu}_{\nu\lambda}\right)
\dod{x^\nu}{\tau}\dod{x^\lambda}{\tau}
-2\fived{\Gamma}^{\mu}_{\nu
4}\dod{x^\nu}{\tau}\dod{x^4}{\tau} \\
&-\fived{\Gamma}^{\mu}_{44}\left(\dod{x^4}{\tau}\right)^2
+f(\tau)\dod{x^\mu}{\tau}\,.\label{eq:geod4d}
\end{split}
\end{equation}
Geodäten der fünfdimensionalen Raumzeit sind also im Algemeinen nicht 
geodäten in der vierdimensionalen Raumzeit. Die
auftretenden Christoffelsymbole der Klein Metrik lauten
\begin{align}
\fived{\Gamma}^{\mu}_{\nu\lambda}-\Gamma^{\mu}_{\nu\lambda}
&=\frac{1}{2}g^{\mu\alpha}\left(\psi
A_{(\nu}F_{\lambda)\alpha}+A_{\nu}A_{\lambda}\partial_{\alpha}\psi \right)\,,
\\
\fived{\Gamma}^{\mu}_{\nu 4}
&=\frac{1}{2}g^{\mu\alpha}\left(\psi F_{\nu\alpha}-A_\nu\partial_\alpha
\psi\right)\,,\\
\fived{\Gamma}^{\mu}_{4 4}
&=-\frac{1}{2}g^{\mu\alpha}\partial_{\alpha}\psi\,.
\end{align}
Eine Berechnung findet sich beispielsweise in \cite{williams2015field} und wird
zweckmäßigerweise mithilfe eines Computers durchgeführt. In der
vierdimensionalen Raumzeit wirkt die  rechte Seite von
\eqref{eq:geod4d} wie eine Kraft $F^\mu$ 
\begin{equation}
\begin{split}
F^\mu=&\frac{1}{2}g^{\mu\alpha}\Bigg[\left(\psi
A_{\nu}F_{\lambda\alpha}+A_{\nu}A_{\lambda}\partial_{\alpha}\psi
\right)U^\nu U^\lambda\\
&+2\left(A_\nu\partial_\alpha\psi-\psi F_{\nu\alpha}\right)U^\nu
U^4+\partial_{\alpha}\psi \left(U^4\right)^2\Bigg]+f(\tau)U^\mu\\
=&\frac{1}{2}g^{\mu\alpha}\Bigg[\psi
(A_\nu U^\nu-2U^4) F_{\lambda\alpha}U^\lambda +
\Delta ^2\partial_\alpha\psi\Bigg]+f(\tau)U^\mu\,.
\end{split}
\end{equation}
Im \enquote{klassischen} Grenzfall
\begin{equation}
A_\nu U^\nu\to 0\,,\quad
\partial_\mu\phi\to 0\,\quad
\end{equation}
verbleibt nur ein Beitrag zur Kraft
\begin{equation}
\begin{split}
F^\mu=-\psi g^{\mu\alpha}F_{\nu\alpha}U^\nu U^4\,.
\end{split}
\end{equation}
Dieser Ausdruck motiviert die Identifikation $U^4=\frac{q}{m}$.
\begin{equation}
\begin{split}
F^\mu=F^\mu_{\text{L}}+\frac{1}{2}g^{\mu\alpha}\Bigg[\psi A_\nu
U^\nu F_{\lambda\alpha}U^\lambda + \Delta
^2\partial_\alpha\psi\Bigg]+h(\psi)U^\mu
\end{split}
\end{equation}
\subsection{Das Kaluza-Klein Wunder}
Setzt man in \eqref{eq:Lagrange2}, $\sigma=\const=0$  so erhält man 
\begin{equation}\label{eq:Lagrange2}
\begin{split}
\mathcal{L}
&=\sqrt{-g}\left[R-\frac{1}{4}\tensor*{F}{_\mu_\nu}\tensor*{F}{^\mu^\nu}\right]\,,
\end{split}
\end{equation}
die Lagrangedichte der klassischen
Maxwell-Einstein Theorie. 
Das Ergebnis ist beachtlich: Im fünfdimensionalen Vakuum sind die Einstein- und
Maxwell-Gleichungen enthalten.
Mit anderen Worten die Theorien Einsteins
und Maxwells lassen sich in einer einzigen Theorie vereinigen, die von 
überraschender Einfachheit ist. Dieser Umstand ist auch als
\emph{Kaluza-Klein Wunder} bekannt.
% 
% \begin{equation}  \begin{split}
%   C&=
%   \tensor{\fived{g}}{_\mu_\nu}\tensor{\fived{U}}{^\mu}\tensor{\fived{U}}{^\nu}
%   +\tensor{\fived{g}}{_4_4}\tensor{\fived{U}}{^4}\tensor{\fived{U}}{^4}\\\
%  &=\tensor{\fived{g}}{_\mu_\nu}\tensor{\fived{U}}{^\mu}\tensor{\fived{U}}{^\nu}
%  +\psi Q^2\\
%   &=\tensor{g}{_\mu_\nu}\tensor{\fived{U}}{^\mu}\tensor{\fived{U}}{^\nu}
%   -\psi\left(\tensor{A}{_\mu}\tensor{\fived{U}}{^\mu}\right)^2
%  +\psi Q^2\\
%  \end{split}
%  \end{equation}
%  
%  
%   Die Geodätengleichung hat die Form
%  \begin{equation}
%  0=\tensor{\fived{U}}{^m}\tensor{\fived{\nabla}}{_m} \tensor{\fived{U}}{_n}
%  \end{equation}
%  Betrachten wir speziell die vierdimensionalen Komponenten $n=\nu$ so finden wir
% \begin{equation}
% \begin{split}
% 0&=\tensor{\fived{U}}{^m}\tensor{\fived{\nabla}}{_m} \tensor{\fived{U}}{_\nu}\\
% &=\tensor{\fived{U}}{^m}\tensor{\partial}{_m} \tensor{\fived{U}}{_\nu}
%  +\tensor*{\fived{\Gamma}}{_m_\ell_\nu}
%  \tensor{\fived{U}}{^m}\tensor{\fived{U}}{^\ell}\\
%  &=\dod{}{\lambda} \tensor{\fived{U}}{_\nu}
% +\tensor*{\fived{\Gamma}}{_\mu_\lambda_\nu}
%  \tensor{\fived{U}}{^\mu}\tensor{\fived{U}}{^\lambda}
%  +2\tensor*{\fived{\Gamma}}{_\mu_4_\nu}
%  \tensor{\fived{U}}{^4}\tensor{\fived{U}}{^\mu}
%  +\tensor*{\fived{\Gamma}}{_4_4_\nu}
%  \tensor{\fived{U}}{^4}\tensor{\fived{U}}{^4}\\
%   &=\dod{}{\lambda} \tensor{\fived{U}}{_\nu}
% +\tensor*{\fived{\Gamma}}{_\mu_\lambda_\nu}
%  \tensor{\fived{U}}{^\mu}\tensor{\fived{U}}{^\lambda}
%  +2Q\tensor*{\fived{\Gamma}}{_\mu_4_\nu}
%  \tensor{\fived{U}}{^\mu}
%  +Q^2\tensor*{\fived{\Gamma}}{_4_4_\nu}
%  \\
%    &=\dod{}{\lambda} \tensor{\fived{U}}{_\nu}
% +\tensor*{\Gamma}{_\mu_\lambda_\nu}
%  \tensor{\fived{U}}{^\mu}\tensor{\fived{U}}{^\lambda}
%  +\psi Q\tensor*{F}{_\mu_\nu}
%  \tensor{\fived{U}}{^\mu}
%  +Q^2\partial_\nu\psi\\
% \end{split}
% \end{equation}
% Insegammt findet man 
% \begin{equation}
% m\tensor{\ddot{x}}{^\mu}+m\tensor*{\Gamma}{^\mu_\nu_\lambda}
% \tensor{\dot{x}}{^\nu}\tensor{\dot{x}}{^\lambda}=
% \psi mQ\tensor*{F}{_\mu_\nu}
%  \tensor{\dot{x}}{^\mu}
%  +mQ^2\partial_\nu\psi
% \end{equation}
% Es liegt nahe $mQ=q$ zu setzen
% \begin{equation}
% m\tensor{\ddot{x}}{^\mu}+m\tensor*{\Gamma}{^\mu_\nu_\lambda}
% \tensor{\dot{x}}{^\nu}\tensor{\dot{x}}{^\lambda}=
% \psi q\tensor*{F}{_\mu_\nu}
%  \tensor{\fived{U}}{^\mu}
%  \tensor{\fived{U}}{^\mu}
%  +\frac{q^2}{m}\partial_\nu\psi
% \end{equation}
% % Dies kann wie folgt interpretiert werden: 
% % \begin{itemize}
% %   \item $-Q\tensor*{F}{_\mu_\nu}
% %  \tensor{U}{^\mu}$ beschreibt die gewöhnliche Lorentzkraft modifiziert mit d
% %  \item  $-Q^2\partial_\nu\psi$ beschreibt eine Kraft die durch ein Potential
% %  $Q^2\psi$ ausgeübt wird.
% % \end{itemize}
% Setzt man $\psi=1$ so erhält man die bekannten Gleichungen der Maxwell Theorie.
% \begin{bemerkung}
% Will man die Normierung $\tensor{U}{_\mu}\tensor{U}{^\mu}={0,-1}$ aufrecht
% erhalten so muss
% $\tensor{\fived{U}}{_m}\tensor{\fived{U}}{^m}=\tensor{U}{_\mu}\tensor{U}{^\mu}+Q^2={Q^2,Q^2-1}$
% gelten. Die Vektoren in fünf Dimensionen sind also je nach Ladung raumartig!
% Dies führt zu fragen bezüglich Kausalität
% \end{bemerkung}
\section{Die Rolle des skalaren Felds}
% Radius des Internen space
  Klein maß dem zusätzlichen Freiheitsgrad der durch das Skalarfeld
  $\sigma$\footnote{Bzw. $\phi$ oder $\psi$.} gegeben ist, keine physikalische
  Bedeutung bei.
  Dementsprechend setzte er $\psi=\const=1$ .
 Dadurch vereinfacht sich die Lagrangedichte, zu der der Einstein-Maxwell
 Theorie.
 Allerdings impliziert die dynamische Gleichung
 \eqref{eq:dymdilat}
  \begin{equation}
\tensor*{F}{_\mu_\nu}\tensor*{F}{^\mu^\nu}=0\,,
 \end{equation}
 der Beitrag der elektrische Beitrag zu \eqref{eq:Lagrange2} verschwindet also.
 Dies führt die Konstruktion ad absurdum. Einziger Ausweg ist die $\tensor{g}{_4_4}$-Komponente
\emph{a priori} konstant zu setzen und nicht zu variieren. Dies trübt
die Allgemeinheit der Theorie, da dadurch diese Komponente gegenüber den anderen
ausgezeichnet ist.
Einziger Ausweg ist das Skalarfeld als physikalische Größe Zu betrachten, diese 
Interpretation wurde erstmals durch Yordan und Thiery vorgenommen.
Zusätzliche skalare Felder tauchen häufig in Theorien auf die kompakte
Zusatzdimensionen enthalten.
Sie werden häufig mit einem Teilchen, dem \emph{Dilaton} identifiziert. Zusätzliche Teilchen stellen an sich kein
Probleme dar, einige ungelöste Fragestellungen der Physik (dunkle
Materie/Energie) benötigen sie sogar um beantwortet zu werden. 
\section{Quantisierung der Ladung}
Die Kompaktheit von $\Sphere^1$ impliziert die Quantisierung der Ladung (Klein)
% \fived{R}^\star
% =e^{-\tau}\left(\fived{R}+3\tensor{\nabla}{_\mu}\left(\tensor{\fived{g}}{^\mu^\nu}\tensor{\partial}{_\nu}\tau\right)
% -\frac{3}{2}\tensor{\fived{g}}{^\mu^\nu}\tensor{\partial}{_\mu}\tau\tensor{\partial}{_\nu}\tau\right)\,,\quad\sqrt{-\fived{g}^\star}=e^{5\varphi}\sqrt{-\fived{g}}
% \end{equation}
% \begin{equation}
% \begin{split}
% \mathcal{L}^\star
% &=\fived{R}^\star\sqrt{-\fived{g}^\star}\\
% &=e^{3\varphi}\sqrt{-\fived{g}}\left(\fived{R}+\frac{16}{3}e^{-\frac{3}{2}\varphi}\square
% e^{\frac{3}{2}\varphi}\right)\\
% &=e^{3\varphi}\sqrt{-g}\phi\left(R-\frac{1}{4}\phi^3\tensor{F}{_\mu_\nu}\tensor{F}{^\mu^\nu}+\frac{16}{3}e^{-\frac{3}{2}\varphi}\square
% e^{\frac{3}{2}\varphi}\right)\\
% &=e^{3\varphi}\sqrt{-g}\phi\left(R-\frac{1}{4}\phi^3\tensor{F}{_\mu_\nu}\tensor{F}{^\mu^\nu}+12\pdif{_\mu}\varphi\pdif{^\mu}\varphi+8\square
% \varphi\right)\\
% &=e^{3\varphi}\sqrt{-g}\phi\left(R
% -\frac{1}{4}\phi^3\tensor{F}{_\mu_\nu}\tensor{F}{^\mu^\nu}
% +12\pdif{_\mu}\varphi\pdif{^\mu}\varphi
% +8e^{2\varphi}\square^\star\varphi+24e^{2\varphi}\pdif{_\mu}\varphi\pdif{^\mu}\varphi\right)
% \end{split}
% \end{equation}
% Es liegt nahe $\phi=e^{-3\varphi}$ zu setzen.
% \begin{equation}
% \begin{split}
% \fived{R}^\star\sqrt{-\fived{g}^\star}
% &=\sqrt{-g}\left(R+12\pdif{_\mu}\varphi\pdif{^\mu}\varphi+8\square
% \varphi\right)
% \end{split}
% \end{equation}
% $\sigma=\sqrt{24}\varphi$
% \begin{equation}
% \begin{split}
% \fived{R}^\star\sqrt{-\fived{g}^\star}
% &=\sqrt{-g}\left(R+\frac{1}{2}\pdif{_\mu}\sigma\pdif{^\mu}\sigma\right)
% \end{split}
% \end{equation}
% Coqueraux Jordan Thiery
% % Im Folgenden lassen wir den Term $2\pi
% % r$ weg, da er keinen Einfluss auf das Variationsprinzip hat. Die Lagrangedichte
% % ist damit
% % \begin{equation}
% % \begin{split}
% % \mathcal{L}=\sqrt{-g}\left(\phi
% % R-\frac{1}{4}\phi^3\tensor{F}{_\mu_\nu}\tensor{F}{^\mu^\nu}\right)
% % &=\sqrt{-g}\left(\phi
% % \mathcal{L}\textsubscript{g}+\phi^3\mathcal{L}\textsubscript{EM}\right)
% % \end{split}
% % \end{equation}
% % Die Tatsache dass sich die Lagrangedichte in einen Term der die Raumkrümmung
% % enthält und einen Elektromagnetischen Anteil aufspaltet ist auch als
% % Kaluza-Klein Wunder\footnote{"`Kaluza-Klein miracle"'} bekannt. Insbesondere
% % erhält man für $\phi=1$ die Lagrangedichte für ein System das sowohl einstein
% % als auch Maxwellgleichungen erfüllt. Da der Lagrangian keine Kinetischen Terme
% % in $\varphi$ enthält folgt
% % \begin{equation}
% % 0=\dpd{\mathcal{L}}{\phi}=\sqrt{-g}\left(R-\frac{3}{4}\phi^2\tensor{F}{_\mu_\nu}\tensor{F}{^\mu^\nu}\right)
% % \end{equation}
% % Weiter erhält man für die Variation nach den $A$ 
% % Das Feld $A$ ist Zyklisch, taucht als nicht selbst in der Lagrangedichte auf.
% % Die resultierende Erhaltungsgleichung lautet
% % \begin{equation}
% % 0=\tensor{\nabla}{_\alpha}\left(\dpd{\mathcal{L}}{\left(\tensor{\nabla}{_\alpha}\tensor{A}{_\beta}\right)}\right)
% =\tensor{\nabla}{_\alpha}\left(\phi^3\dpd{\tensor{F}{_\mu_\nu}\tensor{F}{^\mu^\nu}}{\left(\tensor{\nabla}{_\alpha}\tensor{A}{_\beta}\right)}\right)
% =4\tensor{\nabla}{_\alpha}\left(\phi^3\tensor{F}{^\alpha^\beta}\right)
% \end{equation}
% \begin{equation}
% \tensor{G}{_\mu_\nu}=\phi^2\tensor*{T}{*^{\text{M}}*_\mu*_\nu}
% \end{equation}
% Oder umformuliert
% \begin{equation}
% \tensor{\nabla}{_\alpha}\tensor{F}{^\alpha^\beta}
% =-\frac{3}{\phi}\tensor{F}{^\alpha^\beta}\tensor{\partial}{_\alpha}\phi
% \end{equation}
% Bzw:
% \begin{equation}
% \tensor{J}{^\beta}
% =-\frac{3}{\phi\sqrt{-g}}\tensor{F}{^\alpha^\beta}\tensor{\partial}{_\alpha}\phi
% =-3\left(-\fived{g}\right)^{-\nicefrac{1}{2}}\tensor{F}{^\alpha^\beta}\tensor{\partial}{_\alpha}\phi
% \end{equation}
% \begin{equation}
% \begin{split}
% \fived{R}&=e^{-2\varphi}\left(R+\frac{16}{3}e^{-\frac{3}{2}\varphi}\square
% e^{\frac{3}{2}\varphi}\right)\\
% &=e^{-2\varphi}\left(R+8\square\varphi+12\pdif{_\mu}\varphi\pdif{^\mu}\varphi\right)
% \end{split}
% \end{equation}
% \begin{equation}
% \begin{split}
% \phi\sqrt{-g}R
% &=\phi
% e^{-\varphi}\sqrt{\fived{g}}\left(\fived{R}+8\square\varphi+12\pdif{_\mu}\varphi\pdif{^\mu}\varphi\right)\\
% \end{split}
% \end{equation}
% $\varphi=\ln \phi$
% \begin{equation}
% \begin{split}
% \phi\sqrt{-g}R
% &=\sqrt{\fived{g}}\left(\fived{R}
% +8{\square}\ln\phi 
% +\frac{12}{\phi^2}\pdif{_\mu}\phi\pdif{^\mu}\phi\right)
% \end{split}
% \end{equation}
% \begin{equation}
% \begin{split}
% \phi\sqrt{-g}R
% &=\sqrt{\fived{g}}\left(\fived{R}
% +8\frac{1}{\phi}\square\phi 
% +\frac{4}{\phi^2}\pdif{_\mu}\phi\pdif{^\mu}\phi\right)
% \end{split}
% \end{equation}
% \begin{equation}
% \begin{split}
% \phi\sqrt{-g}R
% &=\sqrt{\fived{g}}\left(\fived{R}
% +8\frac{1}{\phi}\square\phi 
% +16\pdif{_\mu}\Lambda\pdif{^\mu}\Lambda\right)
% \end{split}
% \end{equation}
% \begin{equation}
% \begin{split}
% 0&=\frac{\delta\mathcal{L}}{\delta\tensor{g}{_\mu_\nu}}\\
% &=\phi\frac{\delta\mathcal{L}\textsubscript{EH}}{\delta\tensor{g}{_\mu_\nu}}
% +\phi^3\frac{\delta\mathcal{L}\textsubscript{EM}}{\delta\tensor{g}{_\mu_\nu}}\\
% &=
% \phi\sqrt{-g}\left[\frac{1}{2}\tensor{g}{^\mu^\nu}
% R+\tensor{R}{^\mu^\nu}\right]
% \end{split}
% \end{equation}
% \begin{split}
% \tensor*{\fived{g}}{*^\prime_\mu_\nu}&=\tensor{\fived{g}}{_a_b}\dpd{f^a}{\tensor{x}{^\mu}}\dpd{f^b}{\tensor{x}{^\nu}}\\
% &=\tensor{\fived{g}}{_\alpha_\beta}\dpd{f^\alpha}{\tensor{x}{^\mu}}\dpd{f^\beta}{\tensor{x}{^\nu}}
% +\tensor{\fived{g}}{_\alpha_4}\dpd{f^\alpha}{\tensor{x}{^\mu}}\dpd{f^4}{\tensor{x}{^\nu}}
% +\tensor{\fived{g}}{_4_\beta}\dpd{f^4}{\tensor{x}{^\mu}}\dpd{f^\beta}{\tensor{x}{^\nu}}
% +\tensor{\fived{g}}{_4_4}\dpd{f^4}{\tensor{x}{^\mu}}\dpd{f^4}{\tensor{x}{^\nu}}
% \\
% &=\tensor{\fived{g}}{_\alpha_\beta}\dpd{f^\alpha}{\tensor{x}{^\mu}}\dpd{f^\beta}{\tensor{x}{^\nu}}
% +\psi\tensor{A}{_\alpha}\dpd{g^\alpha}{\tensor{x}{^\mu}}\pdif{_\nu}h
% +\psi\tensor{A}{_\beta}\dpd{g^\beta}{\tensor{x}{^\nu}}\pdif{_\mu}h
% +\psi\pdif{_\mu}h\pdif{_\nu}h\\
% &=\tensor{g}{_\alpha_\beta}\dpd{g^\alpha}{\tensor{x}{^\mu}}\dpd{g^\beta}{\tensor{x}{^\nu}}
% +\psi\left(\tensor{A}{_\alpha}\dpd{g^\alpha}{\tensor{x}{^\mu}}+\pdif{_\mu}h\right)
% \left(\tensor{A}{_\alpha}\dpd{g^\alpha}{\tensor{x}{^\nu}}+\pdif{_\nu}h\right)
% \end{split}
% \end{equation}
% \begin{equation}
% \begin{split}
% \tensor*{\fived{g}}{*^\prime_\mu_4}&=\tensor{\fived{g}}{_a_b}\dpd{f^a}{\tensor{x}{^\mu}}\dpd{f^b}{\tensor{x}{^4}}\\
% &=\tensor{\fived{g}}{_\alpha_\beta}\dpd{f^\alpha}{\tensor{x}{^\mu}}\dpd{f^\beta}{\tensor{x}{^4}}
% +\tensor{\fived{g}}{_\alpha_4}\dpd{f^\alpha}{\tensor{x}{^\mu}}\dpd{f^4}{\tensor{x}{^4}}
% +\tensor{\fived{g}}{_4_\beta}\dpd{f^4}{\tensor{x}{^\mu}}\dpd{f^\beta}{\tensor{x}{^4}}
% +\tensor{\fived{g}}{_4_4}\dpd{f^4}{\tensor{x}{^\mu}}\dpd{f^4}{\tensor{x}{^4}}\\
% &=\psi\tensor{A}{_\alpha}\dpd{g^\alpha}{\tensor{x}{^\mu}}\pdif{_4}h+\psi\pdif{_\mu}h\pdif{_4}h\\
% &=\psi\pdif{_4}h\left(\tensor{A}{_\alpha}\dpd{g^\alpha}{\tensor{x}{^\mu}}+\pdif{_\mu}h\right)
% \end{split}
% \end{equation}
% \begin{equation}
% \begin{split}
% \tensor*{\fived{g}}{*^\prime_4_4}&=\tensor{\fived{g}}{_a_b}\dpd{f^a}{\tensor{x}{^4}}\dpd{f^b}{\tensor{x}{^4}}\\
% &=\tensor{\fived{g}}{_\alpha_\beta}\dpd{f^\alpha}{\tensor{x}{^4}}\dpd{f^\beta}{\tensor{x}{^4}}
% +\tensor{\fived{g}}{_\alpha_4}\dpd{f^\alpha}{\tensor{x}{^4}}\dpd{f^4}{\tensor{x}{^4}}
% +\tensor{\fived{g}}{_4_\beta}\dpd{f^4}{\tensor{x}{^4}}\dpd{f^\beta}{\tensor{x}{^4}}
% +\tensor{\fived{g}}{_4_4}\dpd{f^4}{\tensor{x}{^4}}\dpd{f^4}{\tensor{x}{^4}}\\
% &=\psi\left(\pdif{_4}h\right)^2
% \end{split}
% \end{equation}
% Um konsistent zu bleiben muss $\pdif{_4}h = 1$ also sind die erlaubten
% transformationen von der Form
% \begin{equation}
% \tensor{g}{_\mu_\nu}\to\tensor{g}{_\alpha_\beta}\dpd{g^\alpha}{\tensor{x}{^\mu}}\dpd{g^\beta}{\tensor{x}{^\nu}}\,,\quad
% \tensor{A}{_\alpha}\to\tensor{A}{_\alpha}\dpd{g^\alpha}{\tensor{x}{^\mu}}+\pdif{_\mu}h\\
% \psi\to\psi
% \end{equation}
%  
%  In Verallgemeinerung der vierdimensionalen Geodätischen
%  \begin{equation}
%  0=\tensor{\fived{U}}{^m}\tensor{\nabla}{_m}
%  \tensor{\fived{U}}{_n}=\dod{}{\lambda}
%  \tensor{\fived{U}}{_n}+\tensor{\fived{U}}{^m}\tensor*{\fived{\Gamma}}{^\ell_m_n}\tensor{\fived{U}}{_\ell}
%  \end{equation}
%  % Erhaltungsgröße zu killingtensor d4
%  \begin{equation}
%  \begin{split}
%   0&=\tensor{\fived{U}}{^m}\tensor{\nabla}{_m} \tensor{\fived{U}}{_5}\\
%  &=\dod{}{\lambda}\tensor{\fived{U}}{_5}
%  +\tensor{\fived{U}}{^m}\tensor*{\fived{\Gamma}}{*^n_5_m} \tensor{\fived{U}}{_n}\\
%  &=\dod{}{\lambda}\tensor{\fived{U}}{_5}
%  +\frac{1}{2}\tensor{\fived{U}}{^m}\tensor{\fived{g}}{^n^a}\left(\tensor{\fived{g}}{_a_m_{,5}}
%  +\tensor{\fived{g}}{_a_5_{,m}}
%  -\tensor{\fived{g}}{_m_5_{,a}}
%  \right)\tensor{\fived{U}}{_n}\\
%   &=\dod{}{\lambda}\tensor{\fived{U}}{_5}
%  +\frac{1}{2}\tensor{\fived{U}}{^m}\tensor{\fived{U}}{^a}\left(
%  \tensor{\fived{g}}{_a_5_{,m}}
%  -\tensor{\fived{g}}{_m_5_{,a}}
%  \right)\\
%   &=\dod{}{\lambda}\tensor{\fived{U}}{_5}\\
%  \end{split}
%  \end{equation}
%  Der zweite Term verschwindet als Spur eines Produkt eines symmetrischen mit
%  einem antisymmetrischen Tensors. 
\chapter{Vorhersagen}
\begin{itemize}
  \item Lorentz Kraft
  \item Einstein Gleichungen
  \item MW Gleichungen 
  \item Eichinvarianz (unter Diffeomorphismen für g)
  \item Eichinvarianz (unter Eichtrafos für MW-Gl.)
  \item Dilaton (Bewegungsgleichung ist nicht Klein Gordon Gleichung!)
\end{itemize}
Die Kaluza Klein Theorie weicht von der Einstein-Maxwell Theorie ab. Die
Kaluza-Klein-Theorie ist also falsifizierbar. 
Dilaton muss massiv sein um geringe Abweichungen zu erklären.

\chapter{Im Kontext moderner Theorien}
Yang-Mills
%Duff\footnote{M. J. Duff, GR11 Konferenz, Stockholm 1989.}
%\begin{quote}
% Kaluza-Klein is dead, long
% live Kaluza-Klein!
%\end{quote}
\section{Diskussion}
%TODO warum einstein und nicht lovelock in 5d?
\section{TODOs}
\begin{enumerate}

  \item Yang-Mills
  \item proper time
  \item Quantisierung der Ladung
  \item $\delta$ definition so anpassen dass mit Chap3 konsistent
  \item Geodäten -> Lorentzkraft
  \item Chap4
  \item Kovariante vs parielle Ableitung in Lagrangeformalismus
  \item Hoch/runterziehen von indices
  \item Probleme:
  \begin{enumerate}
   \item Fermionenmassen
   \item quantisized mass
   \end{enumerate}


    
\end{enumerate}
evtl. zu viel:
\begin{enumerate}
   \item Homogene Räume -> Normalteiler etc.
     \item Fermionen
\end{enumerate}
\input{07_Moderne_Formulierung} 
\input{A_Appendix}
\bibliographystyle{alpha} 
%\nocite{*}
\bibliography{bibfile}
\clearpage  
\printglossary[type=symbolslist]
\printglossary
\end{document}