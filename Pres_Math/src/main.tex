\documentclass[10pt,oneside,a4paper]{scrartcl} 

%language and encoding
\usepackage[utf8]{inputenc}
\usepackage[T1]{fontenc}
%\usepackage{lmodern}
\usepackage[english,ngerman]{babel}

%graphics
\usepackage{graphicx}
\usepackage{xcolor}

%math
\usepackage{amsmath}
\usepackage{amssymb}
\usepackage{amsfonts}
\usepackage{amsthm}
\usepackage{thmtools}
\usepackage{mathtools}
\usepackage{tensor}
\usepackage{commath}
\usepackage{dsfont}		% double stroke characters
\usepackage{braket}
\usepackage{mdframed}	% framed environments that can split at page bound­aries
\usepackage{eufrak}

%science
\usepackage{units}
\usepackage{bpchem}	% type­set chem­i­cal names, for­mu­lae, etc

%layout + style
\definecolor{section_color}{rgb}{0.35,0.0,0}								% colors
\definecolor{MyGray}{rgb}{0.96,0.97,0.98}

\usepackage[bottom]{footmisc}
\usepackage[automark,headsepline]{scrlayer-scrpage} 						% headings
\renewcommand*{\headfont}{\normalfont}										% nicht kursive Kopfzeile
\usepackage{setspace}														% set space be­tween lines
\usepackage[font=small,labelfont=bf,labelsep=endash,format=plain]{caption}	% change style of captions
\usepackage[section]{placeins}												% de­fines a \FloatBar­rier com­mand, be­yond which floats may not pass.
\usepackage[protrusion=true,expansion=true]{microtype}						% sublim­i­nal re­fine­ments to­wards ty­po­graph­i­cal per­fec­tion
\addtokomafont{caption}{\small\linespread{1}\selectfont}					% Ändert Schriftgröße und Zeilenabstand bei captions
\usepackage{enumerate} % better way to config enumerates
\usepackage{pdflscape} % landscape mode
\usepackage{afterpage}
% styles for math environments
\declaretheoremstyle[														
  spaceabove = 6pt,
  spacebelow = 6pt,
  headfont = \color{section_color}\sffamily\bfseries,
  notefont = \mdseries,
  notebraces = {(}{)},
  bodyfont = \normalfont,
  postheadspace = 1em,
  qed = ,
]{mythmstyle} %theorems
\declaretheoremstyle[														
  spaceabove = 6pt,
  spacebelow = 6pt,
  headfont = \color{section_color}\sffamily\bfseries,
  notefont = \mdseries,
  notebraces = {(}{)},
  bodyfont = \normalfont,
  postheadspace = 1em,
  qed = ,
]{myrmstyle} %remarks

\renewcommand{\labelitemi}{\scriptsize$\blacksquare$} 
\renewcommand{\labelenumi}{\textsf{\textbf{\arabic{enumi}}}.}
% \renewcommand{\labelitemi}{\color{section_color}\scriptsize$\blacksquare$} 
% \renewcommand{\labelenumi}{\color{section_color}\textsf{\textbf{\arabic{enumi}}}.}

%tikz
\usepackage{tikz}
\usetikzlibrary{decorations.markings,arrows.meta}

\tikzset{
	->-/.style={decoration={
			markings,
			mark=at position .85 with {\arrow{Latex}}},postaction={decorate}},
	-<-/.style={decoration={
			markings,
			mark=at position .15 with {\arrow{Latex[reversed]}}},postaction={decorate}},
}
%tables
\usepackage{tabularx}	% tab­u­lars with ad­justable-width columns
\usepackage{multirow}	% create tab­u­lar cells span­ning mul­ti­ple rows
\usepackage{booktabs}	% publi­ca­tion qual­ity ta­bles in LaTeX
\usepackage{array}
\usepackage{dcolumn}	% align on the dec­i­mal point of num­bers in tab­u­lar columns

%hyperref
\usepackage[pdftex]{hyperref}
\hypersetup{
	pdftitle={General Relativity},
	pdfsubject={General Relativity},
	pdfkeywords={general,relativity,space,time,minkowksi},
	pdfauthor={Michael Ruf and Benjamin Rottler},
	pdfcreator={Michael Ruf and Benjamin Rottler},
	pdfproducer={Michael Ruf and Benjamin Rottler},
	bookmarksnumbered=true, % bookmarks are numbered
	bookmarksopen=true,     % show bookmarks at start of pdf viewer
	bookmarksopenlevel=2,   % level to which the bookmarks are opened
	bookmarksdepth=3,       % depth of bookmarks 
    unicode=true,           % non-Latin characters in pdf viewer's bookmarks
    pdftoolbar=false,       % show pdf viewer's toolbar?
    pdfmenubar=true,        % show pdf viewer's menu?
    pdffitwindow=false,     % window fit to page when opened
    pdfstartview={FitH},    % fits the width of the page to the window
    pdfnewwindow=true,      % links in new window
    pdfborder={0 0 1},		% no border for links
    colorlinks=false,		% false: black links; true: colored links
    linkcolor=section_color,         % color of internal links (change box color with
    % linkbordercolor)
    citecolor=green,        % color of links to bibliography
    filecolor=magenta,      % color of file links
    urlcolor=blue			% color of external links
}
\usepackage{nameref}

%general stuff
\usepackage{cite}
\usepackage{glossaries}	% create glos­saries and lists of acronyms


%%% NEW COMMANDS %%%

\newcommand\grad{\ensuremath{^\circ}}
\newcommand{\imI}{\ensuremath{\mathrm{i}}}
\newcommand{\Reals}{\ensuremath{\mathbb{R}}}
\newcommand{\Complex}{\ensuremath{\mathbb{C}}}
\newcommand{\Sphere}{\ensuremath{\mathbb{S}}}
\newcommand{\transpose}{^\top}
\newcommand{\cSym}[3]{\ensuremath{\begin{Bmatrix} #1 \\ #2 #3 \end{Bmatrix}}}
\newcommand{\csym}[3]{\ensuremath{[#1 #2,\, #3]}}
\newcommand{\affin}[3]{\ensuremath{\Gamma^{#1}_{#2 #3}}}
\newcommand{\landauO}{\mathcal{O}}
\newcommand{\name}[1]{\textsc{#1}}
\newcommand{\fourint}{\int\dif{}^4 x\,}
\DeclareMathOperator{\tr}{Tr}
\DeclareMathOperator{\re}{Re}
\DeclareMathOperator{\im}{Im}
\DeclareMathOperator{\id}{id}
\DeclareMathOperator{\Div}{div}
\let\originalleft\left  %fix bracket spacing when using \left( \right)
\let\originalright\right
\renewcommand{\left}{\mathopen{}\mathclose\bgroup\originalleft}
\renewcommand{\right}{\aftergroup\egroup\originalright}
\renewcommand{\vec}{\mathbf}
\def\mathunderline#1#2{\color{#1}\underline{{\color{black}#2}}\color{black}}
\newcommand{\liedif}[2]{\ensuremath{\mathcal{L}_{#1}#2}}
\newcommand{\lagrangian}{\mathcal{L}}
\DeclareMathOperator{\diag}{diag}
\newcommand{\const}{\text{const.}}
\DeclareMathOperator{\difD}{D}

%new environments
\newenvironment{thm}[1][]{%
  \definecolor{shadethmcolor}{rgb}{.9,.9,.95}%
  \definecolor{shaderulecolor}{rgb}{0.0,0.0,0.4}%
  %\setlength{\shadeboxrule}{1.5pt}%
  \begin{thms}[#1]\hspace*{1mm}%
}{\end{thms}}
\newtheorem{theorem}{Theorem}
% \declaretheorem[
%   style = mythmstyle,
%   name = Theorem,
%   shaded = {
%     bgcolor = MyGray,
%     padding = 2mm,
%     textwidth = 0.98\textwidth
% }
% ]{theorem}
\theoremstyle{definition}
\newtheorem{definition}{Definition}
\theoremstyle{remark}
\newtheorem{remark}{Remark}
\newtheorem{sidenote}{Aside}
\newtheorem{example}{Example}
% \declaretheorem[
%   style = mythmstyle,
%   name = Definition,
%   numberlike=theorem,
%   shaded = {
%     bgcolor = MyGray,
%     padding = 2mm,
%     textwidth = 0.98\textwidth
% }
% ]{definition}
% \declaretheorem[
%   style = myrmstyle,
%   name = Remark,
%   numberlike=theorem,
%   shaded = {
%     bgcolor = white,
%     padding = 2mm,
%     textwidth = 0.98\textwidth
% }
% ]{remark}
% \declaretheorem[
% style = myrmstyle,
% name = Aside,
% numberlike=theorem,
% shaded = {
% bgcolor = white,
% padding = 2mm,
% textwidth = 0.98\textwidth
% }
% ]{sidenote}
% \declaretheorem[
%   style = myrmstyle,
%   name = Example,
%   numberlike=theorem,
%   shaded = {
%     bgcolor = white,
%     padding = 2mm,
%     textwidth = 0.98\textwidth
% }
% ]{example}
% 
% \makeatletter
% \newtoks\FTN@ftn
% \def\pushftn{%
%  \let\@footnotetext\FTN@ftntext\let\@xfootnotenext\FTN@xftntext
%   \let\@xfootnote\FTN@xfootnote}
% \def\popftn{%
%  \global\FTN@ftn\expandafter{\expandafter}\the\FTN@ftn}
% \long\def\FTN@ftntext#1{%
%   \edef\@tempa{\the\FTN@ftn\noexpand\footnotetext
%                     [\the\csname c@\@mpfn\endcsname]}%
%   \global\FTN@ftn\expandafter{\@tempa{#1}}}%
% \long\def\FTN@xftntext[#1]#2{%
%   \global\FTN@ftn\expandafter{\the\FTN@ftn\footnotetext[#1]{#2}}}
% \def\FTN@xfootnote[#1]{%
%    \begingroup
%      \csname c@\@mpfn\endcsname #1\relax
%      \unrestored@protected@xdef\@thefnmark{\thempfn}%
%    \endgroup
%    \@footnotemark\FTN@xftntext[#1]}


%\newtheorem*{sidenote}{Sidenote}
%\newtheorem*{example}{Example}
\newenvironment{tabulars}[1]{\renewcommand*{\arraystretch}{2}\tabular{#1}}{\endtabular}		% stretched table

%%% DOCUMENT %%%

\begin{document}
\onehalfspacing
\selectlanguage{ngerman}
\section{Inhalt}
\begin{itemize}
  \item Evolution of the scale factor
  \item singularities
  \item Big Bang
  \item Red shifts and distances
  \item Our universe
\end{itemize}
Literatur
\begin{itemize}
  \item  Carroll (8.4, 8.5, 8.7)
  \item  Schutz (Chapter 12)
\end{itemize}
\section{Hauptteil}
\begin{equation}
\dif s^2=\dif
t^2-a(t)^2\tensor{\gamma}{_i_j}\dif\tensor{x}{^i}\dif\tensor{x}{^j}
\end{equation}
\begin{equation}
\tensor{\gamma}{_r_r}=\frac{1}{1-\kappa r^2}\,,
\quad\tensor{\gamma}{_\theta_\theta}=r^2\,,
\quad\tensor{\gamma}{_\varphi_\varphi}=r^2\sin^2\theta 
\end{equation}
Wdh(Tafel):
Friedmann-Gl.:
$H$ Hubble Parameter
$a$ scale factor
metric

aquivalent zu Einsteingleichungen

> **Definition 1** A spacetime $(M,g)$ is *stationary* if there exists a timelike Killing field $K$, i.e. a vector field $K$ such that $\langle K,K\rangle<0$ and $\mathcal{L}_Kg=0$. 

We shall show that Definition 1 implies the existence of local coordinates for which $g_{\mu\nu}$ is independent of time. 

Choose a spacelike hypersurface $\Sigma$ of $M$ and consider the integral curves of $K$ passing through $\Sigma$. In $\Sigma$ we choose arbitrary coordinates and introduce local coordinates of $M$ as follows: If $p=\phi_t(p_0)$, where $p_0\in\Sigma$ and $\phi_t$ is the flow of $K$, then the Lagrange coordinates of $p$ are $(t,\vec{x}(p_0))$. In terms of these coordinates, we have $$K=\frac{\partial}{\partial t}$$ and $\mathcal{L}_Kg=0$ implies
$$\partial_tg_{\mu\nu}=0$$
We call such coordinates *adapted* to the Killing field.
Timelike Killing vectors are associated with conservation of energy, and space-like Killing vectors with the conservation of momentum quantities. But the energy-momentum tensor is always 'conserved' - well in GR, this goes to it's divergence being zero. And , the FRW metric does not possess a time-like Killing vector - so energy is not conserved. It possess space-like ones, and momentum is thus conserved.

%Reference
% https://www.physicsforums.com/threads/killing-tensor-vector-very-basic-context-question-about-consevation.790924/
%TODO killing tensoren schon definiert? Ansonsten Zeigen das konservierung gilt
\section{Wiederholung}
Robertson–Walker Metrik
\begin{equation}
\dif s^2=\dif
t^2-a(t)^2\tensor{\gamma}{_i_j}\dif\tensor{x}{^i}\dif\tensor{x}{^j}
\end{equation}
$a=$ Skalenfaktor. $\gamma$ Metrik auf räumlichen Anteil
\begin{equation}
\tensor{\gamma}{_r_r}=\frac{1}{1-\kappa r^2}\,,
\quad\tensor{\gamma}{_\theta_\theta}=r^2\,,
\quad\tensor{\gamma}{_\varphi_\varphi}=r^2\sin^2\theta 
\end{equation}
Skalarkrümmung
\begin{equation}
R=6\left[\frac{\ddot{a}}{a}+\left(\frac{\dot{a}}{a}\right)^2+\frac{\kappa}{a^2}\right]
\end{equation}
Friedmann-Gleichungen
\begin{align}
H^2&=\left(\frac{\dot{a}}{a}\right)^2
=\frac{8\pi G}{3}\rho-\frac{\kappa}{a^2}
\\
\frac{\ddot{a}}{a}
&=-\frac{4\pi G}{3}\left(\rho+3p\right)
\end{align}
Zustandsparameter
\begin{equation}
w_i=\frac{p_i}{\rho_i}
\end{equation}
Dichteparameter
\begin{equation}
\Omega_i=\frac{8\pi G}{3H^2}\rho_i
\end{equation}
\section{Entwicklung des Skalenfaktors und Urknall}
Krümmung wird ebenfalls mit Energiedichte identifiziert
Zur Lösung der Friedmann Gleichungen wird benötigt:
$\kappa$ sowie Zustandsgleichung $p(\rho)$. Allgemein: numerische Lösung.
Vereinfacht nimmt man an dass sich die Energiedichten gemäß eines Potenzgesetzes
entwickeln.
\begin{equation}
\rho_i=\hat{\rho}_i a^{-n_i}
\end{equation}
Einflüsse sind stark Zeitabhängig $a\to 0$ Vakuum vernachlässigbar $a\to \infty$
Materie und Krümmung vernachlässigbar.
\begin{table}
\centering
\begin{tabular}{lr}
\toprule
Materie& 3\\
Strahlung&4\\
Krümmung&2\\
Vakuum&0\\
\bottomrule
\end{tabular}
\end{table}
Zustandsparameter 
\begin{equation}
w_i=\frac{p_i}{\rho_i}=\frac{1}{3}n_i-1
\end{equation}

\begin{equation}
H^2=\frac{8\pi G}{3}\sum \rho_i
\end{equation}
\begin{equation}
1=\sum \Omega_i
\end{equation}
\begin{equation}
\begin{split}
\dot{H}&=\frac{\ddot{a}}{a}-\left(\frac{\dot{a}}{a}\right)^2\\
&=-\frac{4\pi G}{3}\left(\rho+3p\right)-H^2\\
&=-\frac{4\pi G}{3}\sum 1\rho_i+3p_i+2\rho_i\\
&=-\frac{4\pi G}{3}\sum \left(1+w_i\right)\rho_i\leq 0\\
\end{split}
\end{equation}
$\dot{H}\leq 0$ 
\subsection{Spezialfall nur eine Materieform}
Friedmann Gleichung mit nur einem Term ($\rho\propto a^{-n}$):
\begin{equation}
a=\beta a^{1-n/2}
\end{equation}
Fall 1: $n\neq 0$
\begin{equation}
a=\beta t^{2/n}
\end{equation}
Fall 2: Vakuum $n =0$  $H=const$
\begin{equation}
a=\exp(Ht)
\end{equation}
Kein Urknall?
Statische Lösung
Kosmologische Konstante 
Was ist eine Singularität?
Lösungen der Verschiedenen Modelle.
$H$vs$a$
Die Theorie sagt zwar eine Singularität voraus aber: $t\to 0$ $a\to 0$
$\rho\to\infty$ damit muss Quantengravitation zur Beschreibung herangezogen
werden.
\section{Rotverschiebung und Entfernungen}
Die RW Metrik besitzt klarerweise keine zeitartigen Killing Vektorfelder.
Die Energie ist i.A. auch nicht erhalten (Materie bleibt gleich, Strahlung
veringert sich (Rotverschiebung),Vakuum erhöht sich ) (Raum maximal symmetrisch)
$\implies$ Konzept von Killing-Vektorfeldern muss erweitert werden um
Erhaltungsgrößen zu erhalten.
\begin{theorem}
Sei
$T=\tensor{T}{_{\mu_1\dots\mu_n}}\dif\tensor{x}{^{\mu_1}}\otimes\cdots\otimes\dif\tensor{x}{^{\mu_n}}
\in\mathfrak{T}^{0,n}$
ein Tensorfeld mit 
\begin{equation}
\tensor{\nabla}{_{(\nu}}\tensor{T}{_{\mu_1\dots\mu_n)}}=0\,,
\end{equation}
dann ist die Größe
\begin{equation}
\bar{T}
:=\tensor{T}{_{\mu_1\dots\mu_n}}\tensor{V}{^{\mu_1}}\cdots\tensor{V}{^{\mu_n}}
\end{equation}
entlang von Geodätischen mit Vierergeschwindigkeit $V$ konstant.
\end{theorem}
\begin{proof}
\begin{equation}
\begin{split}
\nabla_V\bar{T}
&=\tensor{V}{^\nu}\tensor{\nabla}{_\nu}\left(\tensor{T}{_{\mu_1\dots\mu_n}}
\tensor{V}{^{\mu_1}}\cdots\tensor{V}{^{\mu_n}}\right)\\
&=\tensor{V}{^{\mu_1}}\cdots\tensor{V}{^{\mu_n}}
\tensor{V}{^\nu}\tensor{\nabla}{_\nu}\tensor{T}{_{\mu_1\dots\mu_n}}\\
&=\tensor{V}{^{\mu_1}}\cdots\tensor{V}{^{\mu_n}}
\tensor{V}{^\nu}\tensor{\nabla}{_{(\nu}}\tensor{T}{_{\mu_1\dots\mu_n)}}\\
&=0\\
\end{split}
\end{equation}
\end{proof}
Betrachte den Tensor
\begin{equation}\label{FLRWKillingTensor}
K_{\mu\nu}
=a^{2}\left(U_{\mu}U_{\nu}-g_{\mu\nu}\right),
\end{equation}
wobei $U$, die Vierergeschwindigkeit eines mitbewegten (comoving)
Teilchens.
Der Tensor ist ein Killing Tensor der RW Metrik, d.h.
\begin{equation}
\nabla_{(\sigma}K_{\mu\nu)}=0.
\end{equation}
\begin{proof}
Skizze:
\begin{itemize}
  \item Gehe in Ruhesystem des Teilchens s.d.\
  $\tensor{U}{_\mu}=a\delta_{\mu}^{0}$
  \item Berechne explizit $\tensor{\nabla}{_\sigma}\tensor{K}{_\mu_\nu}$
  mithilfe der Christoffelsymbole der RW- Metrik
\end{itemize}
\end{proof}

Erhaltungsgröße
\begin{equation}
K^2:=\tensor{K}{_\mu_\nu}\tensor{V}{^\mu}\tensor{V}{^\nu}=a^2\left[\tensor{V}{_\mu}\tensor{V}{^\mu}
+\left(\tensor{U}{_\mu}\tensor{V}{^\mu}\right)^2\right]
\end{equation}
\subsection{Massive Teilchen ($\tensor{V}{_\mu}\tensor{V}{^\mu}=-1$)}
\begin{equation}
\left(V^0\right)^2=1+|\vec{V}|^2\,,\quad\tensor{V}{_\mu}\tensor{U}{^\mu}=-V^0
\end{equation}
$|\vec{V}|^2=\tensor{V}{^i}\tensor{V}{_i}$
Einsetzen
\begin{equation}
|\vec{V}|=\frac{K}{a}
\end{equation}
Teilchen verlieren bei der Expansion Geschwindikeit, Materie kühlt ab.
\subsection{Strahlung/massenlose Teilchen
($\tensor{V}{_\mu}\tensor{V}{^\mu}=0$)} $\tensor{V}{_\mu}\tensor{V}{^\mu}=0$ also
$\omega=-\tensor{V}{_\mu}\tensor{U}{^\mu}=-\frac{K}{a}$
\begin{equation}
\frac{\omega\textsubscript{obs}}{\omega\textsubscript{em}}
=\frac{a\textsubscript{em}}{a\textsubscript{obs}}
\end{equation}
Wenn das Universum expandiert,
$a\textsubscript{obs}>a\textsubscript{em}\implies\omega\textsubscript{obs}<\omega\textsubscript{em}$,
ergo rotverschoben.
\section{Unser Universum}
Alter
There is a growing consensus among cosmologists that the universe is flat and
will continue to expand foreve
\section{Sonstiges}
Im Ruhesystem des Teilchens gilt:
\begin{equation}K_{\mu\nu}=a^{4}\delta_{\mu}^{0}\delta_{\nu}^{0}
-a^{2}g_{\mu\nu}.\end{equation}
\begin{align*}
\tensor{\nabla}{_\sigma}\tensor{K}{_\mu_\nu}&
=a^4\tensor{\nabla}{_\sigma}
\left(\tensor*{\delta}{_\mu^0}\tensor*{\delta}{_\nu^0}\right)
+\left(2a^{2}\tensor*{\delta}{_\mu^0}\tensor*{\delta}{_\nu^0}
-\tensor{g}{_\mu_\nu}\right)\cdot 2a\tensor{\nabla}{_\sigma}a\\
&=-a^{4}
\left(\tensor*{\Gamma}{^0_\sigma_\mu}\tensor*{\delta}{_\nu^0}
+\tensor*{\Gamma}{^0_\sigma_\nu}\tensor*{\delta}{_\mu^0}\right)
+h_{\mu\nu}\cdot 2a\partial_{\sigma}a\\
&=-a^{2}
\left(\Gamma_{0\sigma\mu}\delta_{\nu}^{0}
+\Gamma_{0\sigma\nu}\delta_{\mu}^{0}\right)
+2a\dot{a}h_{\mu\nu}\delta_{\sigma}^{0},
\end{align*}
where $h_{\mu\nu}=g_{\mu\nu}
-2a^{2}\delta_{\mu}^{0}\delta_{\nu}^{0}$.
On calculating the Christoffel symbols
\begin{align*}
\tensor*{\Gamma}{_0_0_0}&=\frac{1}{2}\tensor{\partial}{_0}\tensor{g}{_0_0}
=a\dot{a}=\frac{\dot{a}}{a}\tensor{g}{_0_0};\\
\tensor*{\Gamma}{_0_0_i}&=0\\
\tensor*{\Gamma}{_0_i_j}&=-\frac{1}{2}\tensor{\partial}{_0}\tensor{g}{_i_j}
=-\frac{\dot{a}}{a}g_{ij}
\end{align*}
we see that they can be put down in the form
\begin{equation}\Gamma_{0\mu\nu}=\frac{\dot{a}}{a}
\big[2a^2\delta_{\mu}^{0}\delta_{\nu}^{0}
-g_{\mu\nu}\big]=
-\frac{\dot{a}}{a}h_{\mu\nu}.\end{equation}
Then
\begin{equation}\nabla_{\sigma}K_{\mu\nu}
=a\dot{a}\big[
2\delta_{\sigma}^{0}h_{\mu\nu}
-\delta_{\mu}^{0}h_{\sigma\nu}
-\delta_{\nu}^{0}h_{\sigma\mu}\big],\end{equation}
and on symmetrization we obtain the desired generalization of the Killing equation
\begin{equation}\nabla_{(\sigma}K_{\mu\nu)}=0.\end{equation}

% \section{Robertson Walker Metric}
% Konforme Faktoren.
% Invarianten.
% \footnote{This is a very long footnote, isn't it?}
% Annahme:
% \begin{itemize}
% 	\item Isotropie
% 	\item Homogenit�t
% \end{itemize}
% Metrik:
% \begin{equation}
% \dif s^2=-\dif t^2+R^2(t)\dif \sigma^2
% \end{equation}
% Maximal symmetrischer Raum, nicht maximal symmetrische Raumzeit
% \begin{equation}
% \dif \sigma^2:=\gamma_{ij}\dif u^i\dif u^j
% \end{equation}
% wobei maximal symmetrisch bedeutet das der Ricci Tensor proportional zur Metrik $\gamma_{ij}$ ist
% \begin{equation}
% 	R_{ij}=\kappa\gamma_{ij}
% \end{equation}
% \begin{equation}
% 	R=R\indices{^i_i}=\kappa\gamma^{ij}\gamma_{ij}=\kappa\tr\gamma=3\kappa
% \end{equation}
% Radialsymmetrie impliziert das das Linienelement die Form
% \begin{equation}
% 	\dif \sigma^2=e^{2\alpha(r)}\dif r^2+e^{2\beta(r)}r^2 \dif \Omega^2
% \end{equation}
% besitzen muss. Variablenwechsel $\overline{r}=e^{2\beta(r)}r$ liefert
% \begin{equation}
% \dif \overline{r}^2 =\left(2\od{\beta}{r}+1\right)e^{2\beta(r)}\dif r
% \end{equation}
% Mit $e^{2\tilde{\beta}(\overline{r})}:=\left(2\od{\beta}{r}+1\right)e^{2\beta(r)}$
% \begin{equation}
% \dif \sigma^2=e^{2\tilde{\beta}(\overline{r})}\dif \overline{r}^2+\overline{r}^2 \dif \Omega^2
% \end{equation}
% Dies impliziert zusammen mit (R=gamma) die Gleichungen
% \begin{align}
% 	\kappa e^{2\tilde{\beta}(\overline{r})}=R_{rr}&=\\
% 	\kappa R_{\vartheta\vartheta}&=
% \end{align}
% Die entsprechende $R_{\varphi\varphi}$ Gleichung ist dann aufgrund der Form der Metrik ebenfalls erf�llt. Die Gleichungen zusammen f�hren auf
% \begin{equation}
% 	\beta=-\frac{1}{2}\ln\left(1-kr^2\right)
% \end{equation}
% implizieren. Wenn man wieder in das Linienelement einsetzt erh�lt man die \emph{Robertson-Walker Metrik}
% \begin{equation}
% \dif s^2=-\dif t^2+R^2(t)\left(\frac{\dif {\bar{r}}^2}{1-k\bar{r}^2}+\overline{r}^2\dif \Omega^2\right)
% \end{equation}
\end{document}